\documentclass{article}

% Document extensibility %
%
% Disables native paragraph indentation
\usepackage{parskip} 
%
% Provides further bullet options for lists
\usepackage{enumitem}

% Mathematical symbol and statement packages %
%
% Necessary
\usepackage{amsmath}
\usepackage{amssymb}
%
% Extensive fraction notation
\usepackage{xfrac}
%
% Generic mathematical commands
% Notable: \degree, \celcius
\usepackage{gensymb}
%
% Variable vector notation (arrow above variable)
\usepackage{esvect}
%
% Multiline boxed equations
\usepackage{empheq}

% Graphic packages %
%
% Diagrams and illustrations
\usepackage{tikz}
%
% Image insertion
\usepackage{graphicx}

% Document content %
%
% Change title of table of contents
% \renewcommand{\contentsname}{Title}

\begin{document}

% Command `\hr` to insert horizontal rules
\newcommand{\hr}{\par\noindent\rule{\textwidth}{0.4pt}}

\section{Exercise 2}

\subsection{Procedure 1}

\subsection{Procedure 2}

\subsection{Procedure 3}

\begin{tabular}{ | c | c | c | c | }
    \textbf{Molecular Test} & \textbf{Materials Tested} & \textbf{Positive Result} & \textbf{Negative Result} \\
    \hline
    Protein & Egg white & Light purple & Transparent \\
    Starch & Starch solution & Darker brown & Lighter brown \\
    Sugar & Glucose solution & Orange & Blue \\
    Lipid & Vegetable oil & Stained & Dry
\end{tabular}

\subsection{Procedure 4}

\begin{tabular}{ | c | c | c | c | c | }
    & \textbf{Protein test} & \textbf{Sugar test} & \textbf{Starch test} & \textbf{Lipid test} \\
    \hline
    Food & Slightly purple & Murky orange & Black & Stained
\end{tabular}

\textbf{QUESTION}

What organic molecules are found in the food?

\textbf{HYPOTHESIS}

Based on my intuition after looking at the food sample, I believe all of the steps will prove positive.

\textbf{EXPERIMENT TO TEST HYPOTHESIS (SUMMARY)}

In order to test our group's hypothesis, we performed each of the chemical tests to the food sample in sample tubes (similarly to the previous procedure) and observed any changes after ~5 minutes.

\textbf{RESULTS}

The protein test resulted in a very faint shade of purple, while slightly visually foggy.

The sugar test turned from an initially blue, to murky orange color.

The starch test led to the food to become of a darker black color.

The lipid test resulted in the paper being stained.

\textbf{CONCLUSION}

Based on the results found after procedure three, it can be found that the food sample proved positive for the sugar, starch, and lipid test, whilst negative for the protein test. When compared with my initial hypothesis, I was incorrect for the protein test.

\section{Exercise 3}

\subsection{Procedure 1}

\begin{tabular}{ | c | c | }
    \textbf{Tube} & \textbf{Bubble Column (cm)} \\
    \hline
    + & 10cm \\
    - & 0cm
\end{tabular}

\subsection{Procedure 2}

\begin{tabular}{ | c | c | }
    \textbf{Tube} & \textbf{Bubble Column (cm)} \\
    \hline
    $ 4\degree $ C & 10cm \\
    $ 37\degree $ C & 7cm \\
    $ 100\degree $ C & 0cm
\end{tabular}

\textbf{QUESTIONS}
\begin{enumerate}[label=\arabic*.]
    \item
        \begin{enumerate}[label=\alph*)]
            \item $ 4\degree $ C
            \item $ 100\degree $ C
            \item Temperature
        \end{enumerate}
    \item
        \begin{enumerate}[label=\arabic*)]
            \item The type and size of tube
            \item The amount of potato juice
            \item The time spent for each tube from start to end of experimentation
        \end{enumerate}
    \item The lowest temperature tube allowed the most work based on our experiment displaying the most amount of foam.
    \item The highest temperature tube inactivated the enzyme which is based on our experiment displaying the tube with the least amount of foam.
    \item As the liver contains more catalase, the reaction speed is faster compared to the potato.
\end{enumerate}

\end{document}
