\documentclass{article}

% Document extensibility %
%
% Disables native paragraph indentation
\usepackage{parskip} 
%
% Provides further bullet options for lists
\usepackage{enumitem}
\usepackage{framed}

% Mathematical symbol and statement packages %
%
% Necessary
\usepackage{amsmath}
\usepackage{amssymb}
%
% Extensive fraction notation
\usepackage{xfrac}
%
% Generic mathematical commands
% Notable: \degree, \celcius
\usepackage{gensymb}
%
% Variable vector notation (arrow above variable)
\usepackage{esvect}
%
% Multiline boxed equations
\usepackage{empheq}
%
% SI Unit
\usepackage{siunitx}
\usepackage{physunits}
%
% More intuitive arrays/matrices
\usepackage{array}
%
% Linear Equations
\usepackage{systeme}
%
% Boxes!
\usepackage{mdframed}
%
% Matrix Notation
\usepackage{bm}

% Graphic packages %
%
% Diagrams and illustrations
\usepackage{tikz}
\usetikzlibrary{positioning}
%
% Image insertion
\usepackage{graphicx}

% LaTeX Commands
%
% Argument Parser
\usepackage{xparse}

% Document content %
%
% Change title of table of contents
% \renewcommand{\contentsname}{Title}

\title{Assignment \#1}
\author{Corey Mostero - 2566652}
\date{8 September 2023}

\begin{document}

% Command `\hr` to insert horizontal rules
\newcommand{\hr}{\par\noindent\rule{\textwidth}{0.4pt}}

% Command to box and center math equations
\newcommand{\bc}[1]{
	\begin{equation*}
		\begin{boxed}
			{#1}
		\end{boxed}
	\end{equation*}
}

% Command for single line equations with a condition
\newcommand{\cond}[2]{
	\ifmmode
		{#1} \quad {#2}
	\else
		$$ {#1} \quad {#2} $$
	\fi
}

% Matrix and Vector notation
\newcommand{\matr}[1]{
	\ifmmode \bm{#1}
	\else \textit{\textbf{#1}}
	\fi
}
\newcommand{\vect}[1]{
	\ifmmode \mathbf{#1}
	\else \textbf{#1}
	\fi
}

% Laplace
\NewDocumentCommand{\lap}{o}{
	\IfNoValueTF{#1}
		{ \mathcal{L} }
		{ \mathcal{L} \left\{ {#1} \right\} }
}
\NewDocumentCommand{\ilap}{o}{
	\IfNoValueTF{#1}
		{ \mathcal{L}^{-1} }
		{ \mathcal{L}^{-1} \left\{ {#1} \right\} }
}

\maketitle
\newpage

\tableofcontents

\section{}
Translate the following numbers to BOTH binary AND unsigned hexadecimal

\subsection{65}
\begin{itemize}
	\item[\textbf{Binary}] 0100 0001
	\item[\textbf{Hexadecimal}] 41
\end{itemize}

\subsection{409}
\begin{itemize}
	\item[\textbf{Binary}] 0001 1001 1001
	\item[\textbf{Hexadecimal}] 199
\end{itemize}

\subsection{16385}
\begin{itemize}
	\item[\textbf{Binary}] 0100 0000 0000 0001
	\item[\textbf{Hexadecimal}] 4001
\end{itemize}

\section{}
What are the unsigned decimal AND hexadecimal representations of each of the following binary numbers?

\subsection{0011 0101 1101 1010}
\begin{itemize}
	\item[\textbf{Decimal}] 13786
	\item[\textbf{Hexadecimal}] 35DA
\end{itemize}

\subsection{1100 1110 1010 0011}
\begin{itemize}
	\item[\textbf{Decimal}] 52899
	\item[\textbf{Hexadecimal}] CEA3
\end{itemize}

\subsection{1111 1110 1101 1011}
\begin{itemize}
	\item[\textbf{Decimal}] 65243
	\item[\textbf{Hexadecimal}] FEDB
\end{itemize}

\section{}
What is the binary representation of the following hexadecimal numbers?

\subsection{A4693FBC}
1010 0100 0110 1001 0011 1111 1011 1100

\subsection{B697C7A1}
1011 0110 1001 0111 1100 0111 1010 0001

\subsection{2B3D9461}
0010 1011 0011 1101 1001 0100 0110 0001

\section{}
What is the 16-bit hexadecimal representation of each of the following signed decimal integers?

\subsection{-21}

\begin{align*}
	& \rightarrow \text{ 0000 0000 0001 0101 } \\
	& \rightarrow \text{ 1111 1111 1110 1010 } \\
	& \rightarrow \text{ 1111 1111 1110 1011 } \\
	& \rightarrow \text{ FFEB } \\
\end{align*}

\subsection{-45}

\begin{align*}
	& \rightarrow \text{ 0000 0000 0010 1101 } \\
	& \rightarrow \text{ 1111 1111 1101 0010 } \\
	& \rightarrow \text{ 1111 1111 1101 0011 } \\
	& \rightarrow \text{ FFD3 } \\
\end{align*}

\section{}
The following 16-bit hexadecimal numbers represent signed integers. Convert each to decimal.

\subsection{6BF9}

$ (6 \times 16^{3}) + (11 \times 16^{2}) + (15 \times 16^{1}) + (9 \times 16^{0}) = 27641 $

\subsection{C123}

\begin{align*}
	& \rightarrow \text{ 1100 0001 0010 0011 }, \text{ MSB = 1 $ \therefore $ negative } \\
	& \rightarrow \text{ 0011 1110 1101 1100 } \\
	& \rightarrow \text{ 0011 1110 1101 1101 } \\
	& \rightarrow -16093
\end{align*}

\section{}
What is the 8-bit binary (two’s-complement) representation of each of the following signed decimal integers?

\subsection{-72}

\begin{align*}
	& \rightarrow \text{ 0100 1000 } \\
	& \rightarrow \text{ 1011 0111 } \\
	& \rightarrow \text{ 1011 1000 }
\end{align*}

\subsection{-98}

\begin{align*}
	& \rightarrow \text{ 0110 0010 } \\
	& \rightarrow \text{ 1001 1101 } \\
	& \rightarrow \text{ 1001 1110 }
\end{align*}

\subsection{-26}

\begin{align*}
	& \rightarrow \text{ 0001 1010 } \\
	& \rightarrow \text{ 1110 0101 } \\
	& \rightarrow \text{ 1110 0110 }
\end{align*}

\section{}
What is the sum of each pair of hexadecimal integers?

\subsection{6B4 + 3FE}

\begin{align*}
	4 + E & = 2, \text{ carry } = 1 \\
	B + F + 1 & = B, \text{ carry } = 1 \\
	6 + 3 + 1 & = A
\end{align*}
$ \text{ 0xAB2 } \equiv 2738 $

\subsection{A49 + 6BD}
\begin{align*}
	9 + D & = 6, \text{ carry = 1 } \\
	4 + B + 1 & = 0, \text{ carry = 1 } \\
	A + 6 + 1 & = 1, \text{ carry = 1 }
\end{align*}
$ \text{ 0x1106 } \equiv 4358 $

\end{document}
