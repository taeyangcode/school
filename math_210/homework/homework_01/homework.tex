\documentclass{article}

% Document extensibility %
%
% Disables native paragraph indentation
\usepackage{parskip}
%
% Provides further bullet options for lists
\usepackage{enumitem}

% Mathematical symbol and statement packages %
%
% Necessary
\usepackage{amsmath}
\usepackage{amssymb}
%
% Extensive fraction notation
\usepackage{xfrac}
%
% Generic mathematical commands
% Notable: \degree, \celcius
\usepackage{gensymb}
%
% Variable vector notation (arrow above variable)
\usepackage{esvect}
%
% Multiline boxed equations
\usepackage{empheq}
%
% SI Unit
\usepackage{siunitx}
\usepackage{physunits}
%
% More intuitive arrays/matrices
\usepackage{array}
%
% Linear Equations
\usepackage{systeme}
%
% Boxes!
\usepackage{mdframed}
%
% Matrix Notation
\usepackage{bm}

% Graphic packages %
%
% Diagrams and illustrations
\usepackage{tikz}
\usetikzlibrary{positioning}
%
% Image insertion
\usepackage{graphicx}

% LaTeX Commands
%
% Argument Parser
\usepackage{xparse}

% Document content %
%
% Change title of table of contents
% \renewcommand{\contentsname}{Title}

\title{Math 210 Homework \#1}
\date{29 September 2023}
\author{Corey Mostero - 2566652}

\begin{document}

% Command `\hr` to insert horizontal rules
\newcommand{\hr}{\par\noindent\rule{\textwidth}{0.4pt}}

% Command to box and center math equations
\newcommand{\bc}[1]{
	\begin{equation*}
		\begin{boxed}
			{#1}
		\end{boxed}
	\end{equation*}
}

% Command for single line equations with a condition
\newcommand{\cond}[2]{
	\ifmmode
	{#1} \quad {#2}
	\else
	$$ {#1} \quad {#2} $$
	\fi
}

% Matrix and Vector notation
\newcommand{\matr}[1]{
	\ifmmode \bm{#1}
	\else \textit{\textbf{#1}}
	\fi
}
\newcommand{\vect}[1]{
	\ifmmode \mathbf{#1}
	\else \textbf{#1}
	\fi
}

% Laplace
\NewDocumentCommand{\lap}{o}{
	\IfNoValueTF{#1}
	{ \mathcal{L} }
	{ \mathcal{L} \left\{ {#1} \right\} }
}
\NewDocumentCommand{\ilap}{o}{
	\IfNoValueTF{#1}
	{ \mathcal{L}^{-1} }
	{ \mathcal{L}^{-1} \left\{ {#1} \right\} }
}

\newcommand{\boldalph}[0]{ \textbf{\alph*)} }

\maketitle
\newpage

\tableofcontents

\section{Section 1.1}

\subsection{29}

How many rows appear in a truth table for each of these compound propositions?

The amount of rows a complete truth table requires is found by noting the relation between the amount of variables, and the possible values for each variable (in this case binary). The anonymous function can be written as such:
\begin{equation}
	x \mapsto 2^{x} \label{truth_table_rows}
\end{equation}

\begin{enumerate}[label = \boldalph]
	\item $ p \implies \neg p $
	      \begin{align*}
		       & \left( x \mapsto 2^{x} \right)\left( 1 \right) \\
		       & = 2^{(1)}                                      \\
		       & = 2
	      \end{align*}

	\item $ \left( p \lor \neg r \right) \land \left( q \lor \neg s \right) $
	      \begin{align*}
		       & \left( x \mapsto 2^{x} \right)\left( 4 \right) \\
		       & = 2^{(4)}                                      \\
		       & = 16
	      \end{align*}

	\item $ q \lor p \lor \neg s \lor \neg r \lor \neg t \lor u $
	      \begin{align*}
		       & \left( x \mapsto 2^{x} \right)\left( 6 \right) \\
		       & = 2^{(6)}                                      \\
		       & = 64
	      \end{align*}

	\item $ \left( p \land r \land t \right) \iff \left( q \land t \right) $
	      \begin{align*}
		       & \left( x \mapsto 2^{x} \right)\left( 4 \right) \\
		       & = 2^{(4)}                                      \\
		       & = 16
	      \end{align*}
\end{enumerate}

\subsection{30}

How many rows appear in a truth table for each of these compound propositions?

The same anonymous function \eqref{truth_table_rows} can be utilized to found the amount of rows.

\begin{enumerate}[label = \boldalph]
	\item $ \left( q \implies \neg p \right) \lor \left( \neg p \implies \neg q \right) $
	      \begin{align*}
		       & \left( x \mapsto 2^{x} \right) \left( 2 \right) \\
		       & = 2^{(2)}                                       \\
		       & = 4
	      \end{align*}

	\item $ \left( p \lor \neg t \right) \land \left( p \lor \neg s \right) $
	      \begin{align*}
		       & \left( x \mapsto 2^{x} \right) \left( 3 \right) \\
		       & = 2^{(3)}                                       \\
		       & = 8
	      \end{align*}

	\item $ \left( p \implies r \right) \lor \left( \neg s \implies \neg t \right) \lor \left( \neg u \implies v \right) $
	      \begin{align*}
		       & \left( x \mapsto 2^{x} \right) \left( 6 \right) \\
		       & = 2^{(6)}                                       \\
		       & = 64
	      \end{align*}

	\item $ \left( p \land r \land s \right) \lor \left( q \land t \right) \lor \left( r \land \neg t \right) $
	      \begin{align*}
		       & \left( x \mapsto 2^{x} \right) \left( 5 \right) \\
		       & = 2^{(5)}                                       \\
		       & = 32
	      \end{align*}
\end{enumerate}

\subsection{31}

Construct a truth table for each of these compound propositions.

\begin{enumerate}[label = \boldalph]
	\item $ p \land \neg p $ \\
	      \begin{tabular}{ | c | c | c | }
		      $ p $ & $ \neg p $ & $ p \land \neg p $ \\
		      \hline
		      T     & F          & F                  \\
		      F     & T          & F                  \\
	      \end{tabular}

	\item $ p \lor \neg p $ \\
	      \begin{tabular}{ | c | c | c | }
		      $ p $ & $ \neg p $ & $ p \lor \neg p $ \\
		      \hline
		      T     & F          & T                 \\
		      F     & T          & T                 \\
	      \end{tabular}

	\item $ \left( p \lor \neg q \right) \implies q $ \\
	      \begin{tabular}{ | c | c | c | c | c | }
		      $ p $ & $ q $ & $ \neg q $ & $ \left( p \lor \neg q \right) $ & $ \left( p \lor \neg q \right) \implies q $ \\
		      \hline
		      T     & T     & F          & T                                & T                                           \\
		      T     & F     & T          & T                                & F                                           \\
		      F     & T     & F          & F                                & T                                           \\
		      F     & F     & T          & T                                & F                                           \\
	      \end{tabular}

	\item $ \left( p \lor q \right) \implies \left( p \land q \right) $ \\
	      \begin{tabular}{ | c | c | c | c | c | }
		      $ p $ & $ q $ & $ \left( p \lor q \right) $ & $ \left( p \land q \right) $ & $ \left( p \lor q \right) \implies \left( p \land q \right) $ \\
		      \hline
		      T     & T     & T                           & T                            & T                                                             \\
		      T     & F     & T                           & F                            & F                                                             \\
		      F     & T     & T                           & F                            & F                                                             \\
		      F     & F     & F                           & F                            & T                                                             \\
	      \end{tabular}

	\item $ \left( p \implies q \right) \iff \left( \neg q \implies \neg p \right) $ \\
	      \begin{tabular}{ | c | c | c | c | c | c | c | }
		      $ p $ & $ q $ & $ \left( p \implies q \right) $ & $ \neg q $ & $ \neg p $ & $ \left( \neg q \implies \neg p \right) $ & $ \left( p \implies q \right) \iff \left( \neg q \implies \neg p \right) $ \\
		      \hline
		      T     & T     & T                               & F          & F          & T                                         & T                                                                          \\
		      T     & F     & F                               & T          & F          & F                                         & T                                                                          \\
		      F     & T     & T                               & F          & T          & T                                         & T                                                                          \\
		      F     & F     & T                               & T          & T          & T                                         & T                                                                          \\
	      \end{tabular}

	\item $ \left( p \implies q \right) \implies \left( q \implies p \right) $ \\
	      \begin{tabular}{ | c | c | c | c | c | }
		      $ p $ & $ q $ & $ \left( p \implies q \right) $ & $ \left( q \implies p \right) $ & $ \left( p \implies q \right) \implies \left( q \implies p \right) $ \\
		      \hline
		      T     & T     & T                               & T                               & T                                                                    \\
		      T     & F     & F                               & T                               & T                                                                    \\
		      F     & T     & T                               & F                               & F                                                                    \\
		      F     & F     & T                               & T                               & T                                                                    \\
	      \end{tabular}

\end{enumerate}

\subsection{38}

Construct a truth table for $ \left( \left( p \implies q \right) \implies r \right) \implies s $.

\begin{tabular}{ | c | c | c | c | c | c | c | }
	$ p $ & $ q $ & $ r $ & $ s $ & $ \left( p \implies q \right) $ & $ \left( \left( p \implies q \right) \implies r \right) $ & $ \left( \left( p \implies q \right) \implies r \right) \implies s $ \\
	\hline
	T     & T     & T     & T     & T                               & T                                                         & T                                                                    \\
	T     & T     & T     & F     & T                               & T                                                         & F                                                                    \\
	T     & T     & F     & T     & T                               & F                                                         & T                                                                    \\
	T     & T     & F     & F     & T                               & F                                                         & T                                                                    \\
	T     & F     & T     & T     & F                               & T                                                         & T                                                                    \\
	T     & F     & T     & F     & F                               & T                                                         & F                                                                    \\
	T     & F     & F     & T     & F                               & T                                                         & T                                                                    \\
	T     & F     & F     & F     & F                               & T                                                         & F                                                                    \\
	F     & T     & T     & T     & T                               & T                                                         & T                                                                    \\
	F     & T     & T     & F     & T                               & T                                                         & F                                                                    \\
	F     & T     & F     & T     & T                               & F                                                         & T                                                                    \\
	F     & T     & F     & F     & T                               & F                                                         & T                                                                    \\
	F     & F     & T     & T     & T                               & T                                                         & T                                                                    \\
	F     & F     & T     & F     & T                               & T                                                         & F                                                                    \\
	F     & F     & F     & T     & T                               & F                                                         & T                                                                    \\
	F     & F     & F     & F     & T                               & F                                                         & T                                                                    \\
\end{tabular}

\subsection{39}

Construct a truth table for $ \left( p \iff q \right) \iff \left( r \iff s \right) $.

\begin{tabular}{ | c | c | c | c | }
	$ p $ & $ q $ & $ r $ & $ s $ \\
	\hline
	T     & T     & T     & T     \\
	T     & T     & T     & F     \\
	T     & T     & F     & T     \\
	T     & T     & F     & F     \\
	T     & F     & T     & T     \\
	T     & F     & T     & F     \\
	T     & F     & F     & T     \\
	T     & F     & F     & F     \\
	F     & T     & T     & T     \\
	F     & T     & T     & F     \\
	F     & T     & F     & T     \\
	F     & T     & F     & F     \\
	F     & F     & T     & T     \\
	F     & F     & T     & F     \\
	F     & F     & F     & T     \\
	F     & F     & F     & F     \\
\end{tabular}

\subsection{40}

Explain, without using a truth table, why $ \left( p \lor \neg q \right) \land \left( q \lor \neg r \right) \land \left( r \lor \neg p \right) $ is true when $ p $, $ q $, and $ r $ have the same truth value and it is false otherwise.

The proposition $ \left( x \lor \neg y \right) $ is seen an expression of implication. Therefore the expression can be rewritten as $ \left( p \implies q \right) \land \left( q \implies r \right) \land \left( r \implies p \right) $. If $ x \implies y $ and $ y \implies z $, we can simplify the proposition to $ \left( p \iff r \right) \land \left( q \iff p \right) \land \left( r \iff q \right) $. It can now be observed that the compound proposition is true if and only if $ p \equiv q \equiv r $.

\subsection{41}

Explain, without using a truth table, why $ \left( p \lor q \lor r \right) \land \left( \neg p \lor \neg q \lor \neg r \right) $ is true when at least one of $ p $, $ q $, and $ r $ is true and at least one is false, but is false when all three variables have the same truth value.

\subsection{42}

\end{document}
