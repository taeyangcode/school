\documentclass{article}

% Document extensibility %
%
% Disables native paragraph indentation
\usepackage{parskip}
%
% Provides further bullet options for lists
\usepackage{enumitem}

% Mathematical symbol and statement packages %
%
% Necessary
\usepackage{amsmath}
\usepackage{amssymb}
%
% Extensive fraction notation
\usepackage{xfrac}
%
% Generic mathematical commands
% Notable: \degree, \celcius
\usepackage{gensymb}
%
% Variable vector notation (arrow above variable)
\usepackage{esvect}
%
% Multiline boxed equations
\usepackage{empheq}
%
% SI Unit
\usepackage{siunitx}
\usepackage{physunits}
%
% More intuitive arrays/matrices
\usepackage{array}
%
% Linear Equations
\usepackage{systeme}
%
% Boxes!
\usepackage{mdframed}
%
% Matrix Notation
\usepackage{bm}

% Graphic packages %
%
% Diagrams and illustrations
\usepackage{tikz}
\usetikzlibrary{positioning}
%
% Image insertion
\usepackage{graphicx}

% LaTeX Commands
%
% Argument Parser
\usepackage{xparse}

% Document content %
%
% Change title of table of contents
% \renewcommand{\contentsname}{Title}

\begin{document}

% Command `\hr` to insert horizontal rules
\newcommand{\hr}{\par\noindent\rule{\textwidth}{0.4pt}}

% Command to box and center math equations
\newcommand{\bc}[1]{
	\begin{equation*}
		\begin{boxed}
			{#1}
		\end{boxed}
	\end{equation*}
}

% Command for single line equations with a condition
\newcommand{\cond}[2]{
	\ifmmode
	{#1} \quad {#2}
	\else
	$$ {#1} \quad {#2} $$
	\fi
}

% Matrix and Vector notation
\newcommand{\matr}[1]{
	\ifmmode \bm{#1}
	\else \textit{\textbf{#1}}
	\fi
}
\newcommand{\vect}[1]{
	\ifmmode \mathbf{#1}
	\else \textbf{#1}
	\fi
}

% Laplace
\NewDocumentCommand{\lap}{o}{
	\IfNoValueTF{#1}
	{ \mathcal{L} }
	{ \mathcal{L} \left\{ {#1} \right\} }
}
\NewDocumentCommand{\ilap}{o}{
	\IfNoValueTF{#1}
	{ \mathcal{L}^{-1} }
	{ \mathcal{L}^{-1} \left\{ {#1} \right\} }
}

\tableofcontents

\section{5.1 Weak Induction}

\section{5.2 Strong Induction}

\subsection{Example}

Prove that any positive integer can be written as a sum of $ 2^k $, where $ k $'s are distinct.

\textbf{Proof}:
\begin{enumerate}
	\item Basis step:
	      \begin{equation*}
		      1 = 2^0, 2 = 2^1, 3 = 2^0 + 2^1
	      \end{equation*}
	\item Inductive step:

	      We assume that $ k $ can be written as a sum of distinct powers of 2.

	      Where $ 1 \leq k \leq n $, we want to show that $ n + 1 $ is a sum of distinct powers of 1.

	      Consider the following cases
	      \begin{enumerate}
		      \item $ n = \sum 2^r $, where $ r $ is some distinct positive integers. Then $ n + 1 = 1 + \sum 2^r = 2^0 + \sum 2^r $ is a sum of distinct powers of 2.

		      \item $ n = 1 + \sum 2^s $, where $ s $ is some distinct positive integers. Then $ n + 1 = 1 + \sum 2^s + 1 = 2 + \sum 2^s = 2 \left( 1 + \sum 2^{s - 1} \right) $. Since $ n \geq 1 $, $ \frac{ n + 1 }{ 2 } = 1 + \sum 2^{s - 1} $, then $ \frac{ n + 1 }{ 2 } $ is a sum of distinct powers of 2.
	      \end{enumerate}
\end{enumerate}

\end{document}
