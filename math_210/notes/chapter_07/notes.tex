\documentclass{article}

% Document extensibility %
%
% Disables native paragraph indentation
\usepackage{parskip}
%
% Provides further bullet options for lists
\usepackage{enumitem}

% Mathematical symbol and statement packages %
%
% Necessary
\usepackage{amsmath}
\usepackage{amssymb}
%
% Extensive fraction notation
\usepackage{xfrac}
%
% Generic mathematical commands
% Notable: \degree, \celcius
\usepackage{gensymb}
%
% Variable vector notation (arrow above variable)
\usepackage{esvect}
%
% Multiline boxed equations
\usepackage{empheq}
%
% SI Unit
\usepackage{siunitx}
\usepackage{physunits}
%
% More intuitive arrays/matrices
\usepackage{array}
%
% Linear Equations
\usepackage{systeme}
%
% Boxes!
\usepackage{mdframed}
%
% Matrix Notation
\usepackage{bm}

% Graphic packages %
%
% Diagrams and illustrations
\usepackage{tikz}
\usetikzlibrary{positioning}
%
% Image insertion
\usepackage{graphicx}

% LaTeX Commands
%
% Argument Parser
\usepackage{xparse}

% Document content %
%
% Change title of table of contents
% \renewcommand{\contentsname}{Title}

\begin{document}

% Command `\hr` to insert horizontal rules
\newcommand{\hr}{\par\noindent\rule{\textwidth}{0.4pt}}

% Command to box and center math equations
\newcommand{\bc}[1]{
	\begin{equation*}
		\begin{boxed}
			{#1}
		\end{boxed}
	\end{equation*}
}

% Command for single line equations with a condition
\newcommand{\cond}[2]{
	\ifmmode
	{#1} \quad {#2}
	\else
	$$ {#1} \quad {#2} $$
	\fi
}

% Matrix and Vector notation
\newcommand{\matr}[1]{
	\ifmmode \bm{#1}
	\else \textit{\textbf{#1}}
	\fi
}
\newcommand{\vect}[1]{
	\ifmmode \mathbf{#1}
	\else \textbf{#1}
	\fi
}

% Laplace
\NewDocumentCommand{\lap}{o}{
	\IfNoValueTF{#1}
	{ \mathcal{L} }
	{ \mathcal{L} \left\{ {#1} \right\} }
}
\NewDocumentCommand{\ilap}{o}{
	\IfNoValueTF{#1}
	{ \mathcal{L}^{-1} }
	{ \mathcal{L}^{-1} \left\{ {#1} \right\} }
}

\tableofcontents

\section{7.1}

\subsection{8}

What is the probability that a five-card poker hand contains the ace of hearts?
\begin{align*}
	\mathbb{P}(A \heartsuit) & = \frac{ |E| }{ |S| }                                                                                                                                            \\
	                         & = \frac{ C(51, 4) }{ C(52, 5) }                                                                                                                                  \\
	                         & = \frac{ 51 \cdot 50 \cdot 49 \cdot 48 }{ 1 \cdot 2 \cdot 3 \cdot 4 } \cdot \frac{ 1 \cdot 2 \cdot 3 \cdot 4 \cdot 5 }{ 52 \cdot 51 \cdot 50 \cdot 49 \cdot 48 } \\
	                         & = \frac{ 5 }{ 52 }
\end{align*}

\subsection{Birthday Problem}

Given a group of $ n $ people, what is the probability of at least two people having the same birthday?

\section{7.2}

$ \mathbb{P} \colon S \rightarrow [0, 1] $ is called probability if
\begin{enumerate}
	\item $ \forall s \in S, 0 \leq \mathbb{P}(S) \leq 1 $
	\item $ \sum_{s \in S} \mathbb{P}(s) = 1 $
\end{enumerate}

\textbf{Disjoint} (mutually exclusive)
\[ \mathbb{P}(E \cup F) = \mathbb{P}(E) + \mathbb{P}(F) \]
\begin{align*}
	\mathbb{P}(E \cup F) & = \mathbb{P}(E) + \mathbb{P}(F) - \mathbb{P}(E \cap F) \\
	\mathbb{P}(E \cap F) & = 0
\end{align*}
\textbf{Independent}
\[ \mathbb{P}(E \cap F) = \mathbb{P}(E) \cdot \mathbb{P}(F) \]

\subsection{Random Variable}

\begin{align*}
	X \colon S    & \rightarrow \mathbb{R}                    \\
	\mathbb{P}(X) & \rightarrow [0, 1]                        \\
	S             & \rightarrow \mathbb{R} \rightarrow [0, 1]
\end{align*}

\subsection{Conditional Probability}

\begin{align*}
	\mathbb{P}(E \mid F) & = \frac{ \mathbb{P}(E \cap F) }{ \mathbb{P}(F) }
\end{align*}
Probably of $ E $ if $ F $ happened. (after $ F $ happened)

\end{document}
