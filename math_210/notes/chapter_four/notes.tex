\documentclass{article}

% Document extensibility %
%
% Disables native paragraph indentation
\usepackage{parskip}
%
% Provides further bullet options for lists
\usepackage{enumitem}

% Mathematical symbol and statement packages %
%
% Necessary
\usepackage{amsmath}
\usepackage{amssymb}
%
% Extensive fraction notation
\usepackage{xfrac}
%
% Generic mathematical commands
% Notable: \degree, \celcius
\usepackage{gensymb}
%
% Variable vector notation (arrow above variable)
\usepackage{esvect}
%
% Multiline boxed equations
\usepackage{empheq}
%
% SI Unit
\usepackage{siunitx}
\usepackage{physunits}
%
% More intuitive arrays/matrices
\usepackage{array}
%
% Linear Equations
\usepackage{systeme}
%
% Boxes!
\usepackage{mdframed}
%
% Matrix Notation
\usepackage{bm}

% Graphic packages %
%
% Diagrams and illustrations
\usepackage{tikz}
\usetikzlibrary{positioning}
%
% Image insertion
\usepackage{graphicx}

% LaTeX Commands
%
% Argument Parser
\usepackage{xparse}

% Document content %
%
% Change title of table of contents
% \renewcommand{\contentsname}{Title}

\begin{document}

% Command `\hr` to insert horizontal rules
\newcommand{\hr}{\par\noindent\rule{\textwidth}{0.4pt}}

% Command to box and center math equations
\newcommand{\bc}[1]{
	\begin{equation*}
		\begin{boxed}
			{#1}
		\end{boxed}
	\end{equation*}
}

% Command for single line equations with a condition
\newcommand{\cond}[2]{
	\ifmmode
	{#1} \quad {#2}
	\else
	$$ {#1} \quad {#2} $$
	\fi
}

% Matrix and Vector notation
\newcommand{\matr}[1]{
	\ifmmode \bm{#1}
	\else \textit{\textbf{#1}}
	\fi
}
\newcommand{\vect}[1]{
	\ifmmode \mathbf{#1}
	\else \textbf{#1}
	\fi
}

\newcommand{\modulus}{\text{mod }}

% Laplace
\NewDocumentCommand{\lap}{o}{
	\IfNoValueTF{#1}
	{ \mathcal{L} }
	{ \mathcal{L} \left\{ {#1} \right\} }
}
\NewDocumentCommand{\ilap}{o}{
	\IfNoValueTF{#1}
	{ \mathcal{L}^{-1} }
	{ \mathcal{L}^{-1} \left\{ {#1} \right\} }
}

% GCD & LCM
% \newcommand{\gcd}{\text{gcd}}
\newcommand{\lcm}{\text{lcm}}

\tableofcontents

\section{Section 4.1}

\subsection{Theorem 1}

\textbf{Proof}:
\begin{align*}
	a \vert b & \implies b = a \cdot m \\
	a \vert c & \implies c = a \cdot n
\end{align*}
For some $ m, n \in \mathbb{Z} \implies m + n \in \mathbb{Z} $. Then $ b + c = am + an = a(m + n) \therefore a \vert b + c $.

\subsection{Division Algorithm}

The function \textbf{div} is called the division algorithm.

\begin{equation}
	\text{div}(a, d) = a \text{ div } d = \left\lfloor \frac{ a }{ d } \right\rfloor
\end{equation}
\begin{equation}
	\text{div}: \mathbb{Z} \times \mathbb{Z}^{+} \implies \mathbb{Z}
\end{equation}
The function receives a dividend and divisor and produces the quotient.

\subsection{Modulus Algorithm}

The function \textbf{mod} is called the modulus algorithm.

\begin{equation}
	\text{mod}(a, d) = a \text{ mod } d = a - \left\lfloor \frac{ a }{ d } \right\rfloor
\end{equation}
where $ a = d \cdot q + r $.
\begin{equation}
	\text{mod}: \mathbb{Z} \times \mathbb{Z}^{+} \implies \mathbb{Z}
\end{equation}
The function receives a dividend and divisor and produces the remainder.
\begin{equation}
	a \equiv b \left( \text{mod } m \right) \iff m \vert \left( a - b \right)
\end{equation}

\subsection{Remarks}

\begin{enumerate}
	\item
	      \begin{equation*}
		      \mathbb{Z}_{m} \left( Z \text{ mod } m \right)
	      \end{equation*}
	      \begin{equation*}
		      Z_{m} = \left\{ 0_m, 1_m, 2_m, \cdots, (m - 1)_m \right\}
	      \end{equation*}
	      where $ 0_m $ is a set
	      \begin{itemize}
		      \item $ r \in 0_m $ if $ r \equiv 0 \left( \text{mod } m \right) $
		      \item $ r \in 1_m $ if $ r \equiv 1 \left( \text{mod } m \right) $
	      \end{itemize}
\end{enumerate}

\section{Section 4.2}

\subsection{Theorem 1}

Let $ b $ be an integer greater than 1. Then if $ n $ is a positive integer, it can be expressed uniquely in the form
\begin{equation}
	n = a_kb^k + a_{k - 1}b^{k - 1} + \cdots + a_1b + a_0,
\end{equation}
where $ k $ is a nonnegative integer, $ a_0, a_1, \cdots, a_k $ are nonnegative integers less than $ b $, and $ a_k \neq 0 $.

\subsection{Example}

When $ b = 10, a_i \in \left\{ 0, 1, 2, 3, 4, 5, 6, 7, 8, 9 \right\} $
\begin{equation*}
	7254887 = 7 \cdot 10^6 + 2 \cdot 10^5 + 5 \cdot 10^4 + 4 \cdot 10^3 + 8 \cdot 10^2 + 8 \cdot 10^1 + 7 \cdot 10^0
\end{equation*}

\section{Section 4.3}

\subsection{Prime Factorization}

\begin{align*}
	24 & = 2 \cdot 2 \cdot 2 \cdot 3 = 2^3 \cdot 3^1 \cdot 5^0 \\
	36 & = 2 \cdot 2 \cdot 3 \cdot 3 = 2^2 \cdot 3^2 \cdot 5^0 \\
	60 & = 2 \cdot 2 \cdot 3 \cdot 5 = 2^2 \cdot 3^1 \cdot 5^1
\end{align*}
\begin{align*}
	\text{gcd} & = 2^{\min} \cdot 3^{\min} \cdot 5^{\min} \\
	\text{gcd} & = 2^{2} \cdot 3^{1} \cdot 5^{0}
\end{align*}

\subsection{Greatest Common Divisor}

$ d = \gcd(a, b) $ if $ d \geq x $ for all $ x $ such that $ x \mid a \& x \mid b $.

\subsection{Least Common Multiple}

$ m = \text{lcm}(a, b) $ if $ m \leq y $ for all $ y $ such that $ a \mid y \& b \mid y $.

\subsection{Example 4.3.9}

$ x = -1 $ is a solution to $ x^m + 1 = 0 $ if $ m $ is odd.
\begin{align*}
	x^m + 1 & = \left( x + 1 \right) \left( x^{m - 1} - x^{m - 2} + x^{m - 3} - x^{m - 4} + \cdots - x + 1 \right) \\
	a^m + 1 & = \left( a + 1 \right) \left( a^{m - 1} - a^{m - 2} + \cdots - a + 1 \right)
\end{align*}
\begin{align*}
	a \text{ is great than } 1 & \implies a + 1 > 1       \\
	x \text{ is at least } 3   & \implies a + 1 < a^m + 1
\end{align*}
$ \therefore 1 < a + 1 < a^m + 1 $

\subsection{4.3.40}

Using the method followed in Example 17, express the greatest common divisor of each of these pairs of integers as a linear combination of these integers.

\begin{enumerate}[label = \textbf{\alph*)}]
	\item
	\item
	\item
	\item
	\item
	\item
	\item 2002, 2339
	      \begin{enumerate}
		      \item Show that $ \gcd(x, y) = 1 $.
		      \item Find $ x, y, \in \mathbb{Z} $, such that $ 2002x + 2339y = 1 $.
	      \end{enumerate}
	      \begin{itemize}
		      \item By the Euclidean Algorithm
		            \begin{align*}
			            2339 & = 2002 \cdot 1 + 337 \\
			            2002 & = 337 \cdot 6 + 317  \\
			            337  & = 317 \cdot 1 + 20   \\
			            317  & = 20 \cdot 15 + 17   \\
			            20   & = 17 \cdot 1 + 3     \\
			            17   & = 3 \cdot 5 + 2      \\
			            3    & = 2 \cdot 1 + 1      \\
			            2    & = 1 \cdot 2
		            \end{align*}
		            \begin{align*}
			            1 & = 3 - 2 \cdot 1                                       \\
			              & = 3 - (16 - 3 \cdot 5)                                \\
			              & = 3 \cdot 6 - 17                                      \\
			              & = (20 - 17 \cdot 1) \cdot 6 + (-1) \cdot 17           \\
			              & = 20 \cdot 6 + (-7) \cdot 17                          \\
			              & = 20 \cdot 6 + (-7)(317 - 20 \cot 15)                 \\
			              & = 20 \cdot 111 + (-7) \cdot 317                       \\
			              & = (337 - 317 \cdot 10) \cdot 111 + (-7) \cdot 317     \\
			              & = 317 \cdot 111 + (-118) \cdot 317                    \\
			              & = 317 \cdot 111 + (-118)(2002 - 337 \cdot 6)          \\
			              & = 317 \cdot 819 + (-118) \cdot 2002                   \\
			              & = (2339 - 2002 \cdot 1) \cdot 819 + (-118) \cdot 2002 \\
			              & = 2339 \cdot 819 + (-937) \cdot 2002                  \\
			            x & = -937, y = 819
		            \end{align*}
	      \end{itemize}
	\item
	\item
\end{enumerate}

\section{Section 4.4}

\subsection{Theorem 1}

\begin{align*}
	ax + b          & = c                    \\
	a^{-1} \cdot ax & = c - b \cdot a^{-1}   \\
	x               & = (c - b) \cdot a^{-1}
\end{align*}
\begin{align*}
	4x + 2       & \equiv 1 (\modulus 9)   \\
	4x           & \equiv -1 (\modulus 9)  \\
	4x           & \equiv 8 (\modulus 9)   \\
	7 \cdot ( 4x & \equiv 8 (\modulus 9) ) \\
	x            & \equiv 56 (\modulus 9)  \\
	x            & \equiv 2 (\modulus 9)
\end{align*}

\subsection{Chinese Remainder Theorem}

\begin{align*}
	x & \equiv a_1 (\modulus m_1) \\
	x & \equiv a_2 (\modulus m_2) \\
	x & \equiv a_3 (\modulus m_3) \\
\end{align*}
Where $ x $ is the same and $ \modulus m_i $ is pair-wise relatively prime.

Let $ m = m_1, m_2, \cdots m_n $
\begin{equation}
	M_i = \frac{ m }{ m_i }
\end{equation}
$ y_i $ as inverse of $ M_i \modulus m_i $
\begin{equation}
	x = a_1M_1y_1 + a_2M_2y_2 + \cdots + a_nM_ny_n
\end{equation}

\subsubsection{Example}

\begin{alignat*}{1}
	x & \equiv 1 ( \modulus 5 ) \\
	x & \equiv 2 ( \modulus 7 ) \\
	x & \equiv 3 ( \modulus 4 )
\end{alignat*}
\begin{align*}
	m   & = 5 \cdot 7 \cdot 4 = 140 \\
	M_1 & = \frac{ 140 }{ 5 } = 28  \\
	M_2 & = \frac{ 140 }{ 7 } = 20  \\
	M_2 & = \frac{ 140 }{ 4 } = 35
\end{align*}
\begin{align*}
	y_1 \cdot 28 & = 1(\modulus 5) = 2 \\
	y_2 \cdot 20 & = 1(\modulus 7) = 6 \\
	y_3 \cdot 35 & = 1(\modulus 4) = 3
\end{align*}
\begin{align*}
	x & = 1 \cdot 28 \cdot 2 + 2 \cdot 20 \cdot 6 + 3 \cdot 35 \cdot 3 \\
	  & = 611 = 51(\modulus 140)
\end{align*}

\subsection{Method of Back Substitution}

\begin{alignat*}{1}
	x & \equiv 1 ( \modulus 5 ) \\
	x & \equiv 2 ( \modulus 7 ) \\
	x & \equiv 3 ( \modulus 4 )
\end{alignat*}
\begin{alignat*}{2}
	 &          & x & \equiv 1(\modulus 5) \\
	 & \implies & x & = 5t + 1
\end{alignat*}
Then $ x \equiv 2(\modulus 7) $ becomes
\begin{alignat*}{2}
	 &          & 5t + 1     & \equiv 2(\modulus 7)         \\
	 &          & 3 \cdot 5t & \equiv 1(\modulus 7) \cdot 3 \\
	 &          & t          & \equiv 3(\modulus 7)         \\
	 & \implies & t          & = 7s + 3                     \\
	 & \implies & c          & = 5(7s + 3) + 1              \\
	 &          &            & = 355 + 16
\end{alignat*}
Then $ x \equiv 3(\modulus 4) $ becomes
\begin{align*}
	355 + 16 & \equiv 3(\modulus 4)          \\
	355      & \equiv 3(\modulus 4)          \\
	35       & \equiv 3(\modulus 4)          \\
	s        & \equiv 9(\modulus 4)          \\
	s        & \equiv 1(\modulus 4)          \\
	s        & = 4r + 1                      \\
	x        & = 35(4r + 1) + 16 = 140r + 51
\end{align*}

\subsection{Fermat's Little Theorem}

\subsubsection{Example}

\begin{align*}
	a   & = 7                      \\
	p   & = 13 \implies p - 1 = 12 \\
	121 & = 12 \cdot 10 + 1        \\
	7^{p - 1} \equiv 1 \modulus 13
\end{align*}
\begin{align*}
	 & 7^{121} \modulus 13                                      \\
	 & \equiv 7^{12 \cdot 10 + 1} \modulus 13                   \\
	 & \equiv 7^{12 \cdot 10} \cdot 7 \modulus 13               \\
	 & \equiv \left( 7^{p - 1} \right)^{10} \cdot 7 \modulus 13 \\
	 & \equiv 1^{10} \cdot 7 \modulus 13                        \\
	 & = 7 \modulus 13
\end{align*}

\end{document}
