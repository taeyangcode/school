\documentclass{article}

% Document extensibility %
%
% Disables native paragraph indentation
\usepackage{parskip}
%
% Provides further bullet options for lists
\usepackage{enumitem}

% Mathematical symbol and statement packages %
%
% Necessary
\usepackage{amsmath}
\usepackage{amssymb}
%
% Extensive fraction notation
\usepackage{xfrac}
%
% Generic mathematical commands
% Notable: \degree, \celcius
\usepackage{gensymb}
%
% Variable vector notation (arrow above variable)
\usepackage{esvect}
%
% Multiline boxed equations
\usepackage{empheq}
%
% SI Unit
\usepackage{siunitx}
\usepackage{physunits}
%
% More intuitive arrays/matrices
\usepackage{array}
%
% Linear Equations
\usepackage{systeme}
%
% Boxes!
\usepackage{mdframed}
%
% Matrix Notation
\usepackage{bm}

% Graphic packages %
%
% Diagrams and illustrations
\usepackage{tikz}
\usetikzlibrary{positioning}
%
% Image insertion
\usepackage{graphicx}

% LaTeX Commands
%
% Argument Parser
\usepackage{xparse}

% Document content %
%
% Change title of table of contents
% \renewcommand{\contentsname}{Title}

\begin{document}

% Command `\hr` to insert horizontal rules
\newcommand{\hr}{\par\noindent\rule{\textwidth}{0.4pt}}

% Command to box and center math equations
\newcommand{\bc}[1]{
	\begin{equation*}
		\begin{boxed}
			{#1}
		\end{boxed}
	\end{equation*}
}

% Command for single line equations with a condition
\newcommand{\cond}[2]{
	\ifmmode
	{#1} \quad {#2}
	\else
	$$ {#1} \quad {#2} $$
	\fi
}

% Matrix and Vector notation
\newcommand{\matr}[1]{
	\ifmmode \bm{#1}
	\else \textit{\textbf{#1}}
	\fi
}
\newcommand{\vect}[1]{
	\ifmmode \mathbf{#1}
	\else \textbf{#1}
	\fi
}

% Laplace
\NewDocumentCommand{\lap}{o}{
	\IfNoValueTF{#1}
	{ \mathcal{L} }
	{ \mathcal{L} \left\{ {#1} \right\} }
}
\NewDocumentCommand{\ilap}{o}{
	\IfNoValueTF{#1}
	{ \mathcal{L}^{-1} }
	{ \mathcal{L}^{-1} \left\{ {#1} \right\} }
}

\tableofcontents

\section{Section 4.1}

\subsection{Theorem 1}

\textbf{Proof}:
\begin{align*}
	a \vert b & \implies b = a \cdot m \\
	a \vert c & \implies c = a \cdot n
\end{align*}
For some $ m, n \in \mathbb{Z} \implies m + n \in \mathbb{Z} $. Then $ b + c = am + an = a(m + n) \therefore a \vert b + c $.

\subsection{Division Algorithm}

The function \textbf{div} is called the division algorithm.

\begin{equation}
	\text{div}(a, d) = a \text{ div } d = \left\lfloor \frac{ a }{ d } \right\rfloor
\end{equation}
\begin{equation}
	\text{div}: \mathbb{Z} \times \mathbb{Z}^{+} \implies \mathbb{Z}
\end{equation}
The function receives a dividend and divisor and produces the quotient.

\subsection{Modulus Algorithm}

The function \textbf{mod} is called the modulus algorithm.

\begin{equation}
	\text{mod}(a, d) = a \text{ mod } d = a - \left\lfloor \frac{ a }{ d } \right\rfloor
\end{equation}
where $ a = d \cdot q + r $.
\begin{equation}
	\text{mod}: \mathbb{Z} \times \mathbb{Z}^{+} \implies \mathbb{Z}
\end{equation}
The function receives a dividend and divisor and produces the remainder.
\begin{equation}
	a \equiv b \left( \text{mod } m \right) \iff m \vert \left( a - b \right)
\end{equation}

\subsection{Remarks}

\begin{enumerate}
	\item
	      \begin{equation*}
		      \mathbb{Z}_{m} \left( Z \text{ mod } m \right)
	      \end{equation*}
	      \begin{equation*}
		      Z_{m} = \left\{ 0_m, 1_m, 2_m, \cdots, (m - 1)_m \right\}
	      \end{equation*}
	      where $ 0_m $ is a set
	      \begin{itemize}
		      \item $ r \in 0_m $ if $ r \equiv 0 \left( \text{mod } m \right) $
		      \item $ r \in 1_m $ if $ r \equiv 1 \left( \text{mod } m \right) $
	      \end{itemize}
\end{enumerate}

\section{Section 4.2}

\subsection{Theorem 1}

Let $ b $ be an integer greater than 1. Then if $ n $ is a positive integer, it can be expressed uniquely in the form
\begin{equation}
	n = a_kb^k + a_{k - 1}b^{k - 1} + \cdots + a_1b + a_0,
\end{equation}
where $ k $ is a nonnegative integer, $ a_0, a_1, \cdots, a_k $ are nonnegative integers less than $ b $, and $ a_k \neq 0 $.

\subsection{Example}

When $ b = 10, a_i \in \left\{ 0, 1, 2, 3, 4, 5, 6, 7, 8, 9 \right\} $
\begin{equation*}
	7254887 = 7 \cdot 10^6 + 2 \cdot 10^5 + 5 \cdot 10^4 + 4 \cdot 10^3 + 8 \cdot 10^2 + 8 \cdot 10^1 + 7 \cdot 10^0
\end{equation*}

\end{document}
