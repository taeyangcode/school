\documentclass{article}

% Document extensibility %
%
% Disables native paragraph indentation
\usepackage{parskip} 
%
% Provides further bullet options for lists
\usepackage{enumitem}

% Mathematical symbol and statement packages %
%
% Necessary
\usepackage{amsmath}
\usepackage{amssymb}
\usepackage{mathtools}
%
% Extensive fraction notation
\usepackage{xfrac}
%
% Generic mathematical commands
% Notable: \degree, \celcius
\usepackage{gensymb}
%
% Variable vector notation (arrow above variable)
\usepackage{esvect}
%
% Multiline boxed equations
\usepackage{empheq}
%
% SI Unit
\usepackage{siunitx}
\usepackage{physunits}
%
% More intuitive arrays/matrices
\usepackage{array}
%
% Linear Equations
\usepackage{systeme}
%
% Boxes!
\usepackage{mdframed}
%
% Matrix Notation
\usepackage{bm}

% Graphic packages %
%
% Diagrams and illustrations
\usepackage{tikz}
\usetikzlibrary{positioning}
%
% Image insertion
\usepackage{graphicx}

% LaTeX Commands
%
% Argument Parser
\usepackage{xparse}

% Document content %
%
% Change title of table of contents
% \renewcommand{\contentsname}{Title}

\begin{document}

% Command `\hr` to insert horizontal rules
\newcommand{\hr}{\par\noindent\rule{\textwidth}{0.4pt}}

% Command to box and center math equations
\newcommand{\bc}[1]{
	\begin{equation*}
		\begin{boxed}
			{#1}
		\end{boxed}
	\end{equation*}
}

% Command for single line equations with a condition
\newcommand{\cond}[2]{
	\ifmmode
		{#1} \quad {#2}
	\else
		$$ {#1} \quad {#2} $$
	\fi
}

% Matrix and Vector notation
\newcommand{\matr}[1]{
	\ifmmode \bm{#1}
	\else \textit{\textbf{#1}}
	\fi
}
\newcommand{\vect}[1]{
	\ifmmode \mathbf{#1}
	\else \textbf{#1}
	\fi
}

% Laplace
\NewDocumentCommand{\lap}{o}{
	\IfNoValueTF{#1}
		{ \mathcal{L} }
		{ \mathcal{L} \left\{ {#1} \right\} }
}
\NewDocumentCommand{\ilap}{o}{
	\IfNoValueTF{#1}
		{ \mathcal{L}^{-1} }
		{ \mathcal{L}^{-1} \left\{ {#1} \right\} }
}

\tableofcontents

\section{Chapter 1}

\subsection{Propositional Logic}

\subsection{Applications of Propositional Logic}

\subsection{Propositional Equivalences}

\subsection{Predicates and Quantifiers}

\subsection{Nested Quantifiers}

\subsection{Rules of Inference}

\subsection{Introduction to Proofs}

We can define rational numbers as
\begin{equation}
	\mathbb{Q} = \left\{ r | r = \frac{ p }{ q }, p, q, \in \mathbb{Z}, q \neq 0, gcd(p, q) = 1 \right\}
\end{equation}

\subsubsection{Basic Methods of Proofs}

\begin{enumerate}
	\item Directly
	\item Case by case
	\item Contradiction (unknown result, but know it is wrong) \& Contrapositive (known result)
		\begin{itemize}
			\item For contrapositive: To prove $ p \implies q $, we prove $ \neg q \implies \neg p $
			\item For contradiction: We assume the opposite of the result, then from the assumption, we draw contradiction
		\end{itemize}
\end{enumerate}

\subsubsection{Example}

Prove that the product of two odd integers is odd.

\textbf{Direct Proof}: Let $ a, b \in \mathbb{Z} $. Since $ a, b $ are odds, then by definition $ a = 2m + 1, b = 2n + 1 $, where $ m, n \in \mathbb{Z} $.

\begin{align*}
	a \cdot b & = (2m + 1)(2n + 1) \\
	a \cdot b & = 4mn + 2m + 2n + 1 \\
	a \cdot b & = 2(2mn + m + n) + 1 \\
	a \cdot b & = 2k + 1 \\
			  & \therefore a \cdot b \text{ is odd by definition}
\end{align*}

\subsubsection{Example}

Prove that $ n^2 + 1 \geq 2^n $ when $ n $ is a positive integer with $ 1 \leq n \leq 4 $.

\textbf{Case by Case Proof}

\subsubsection{Example}

Prove that if $ m + n $ and $ n + p $ are even integers, where $ m, n, $ and $ p $ are integers, then $ m + p $ is even. What kind of proof did you use?

\textbf{Proof}:

Case 1: When $ n $ is even.

Since $ m + n $ \& $ n + p $ are even, then $ m $ is even \& $ p $ is even. Thus, $ m + p $ is even.
\begin{align*}
	n & = 2k_1 \\
	m + n & = 2k_2 \\
	m & = 2k_2 - 2k_1 \\
	m & = 2(k_2 - k_1)
\end{align*}

\subsubsection{Example}

$ \forall a, b \in \mathbb{Z} $, if $ a \cdot b $ is even, then either $ a $ is even or $ b $ is even.

\textbf{Proof by Contrapositive}: The statement is equivalent to
\begin{align*}
	\neg ( a = 2k_1 \lor b = 2k_2 ) & \implies \neg ( a \cdot b = 2k_3 ) \\
	\neg ( a = 2k_1 ) \land \neg ( b = 2k_2 ) & \implies \neg ( a \cdot b = 2k_3 ) \\
	( a = 2k_1 + 1 ) \land ( b = 2k_2 + 1 ) & \implies ( a \cdot b = 2k_3 + 1 )
\end{align*}

\subsubsection{Example}

Prove that $ \sqrt{2} $ is irrational.

\textbf{Proof by Contadiction}: We assume that $ \sqrt{2} $ is rational. Then $ \sqrt{2} = \frac{ m }{ n } $, where $ m, n \in \mathbb{Z}, n \neq 0, gcd(m, n) = 1 $.
\begin{align*}
	2 & = \frac{ m^2 }{ n^2 } \\
	m^2 & = 2n^2
\end{align*}
Hence, $ m^2 $ is even $ \implies m $ is even $ \implies m^2 $ is a multiple of 4. Thus, $ n^2 $ is even $ \implies n $ is even $ \implies n^2 $ is a multiple of 2. $ \therefore gcd(m, n) \neq 1 $ contradicts with $ gcd(m, n) = 1 $.

\subsubsection{Example}

Prove that there is no largest positive real number.

\textbf{Proof by Contradiction}: Assume that there is a largest real number. Let $ x $ be the largest positive real number.

Consider $ x + 1 > x $. Then it contradicts with the assumption.

\subsubsection{Example}

Show that the equation has exactly one real solution.
\begin{equation*}
	2x + \cos(x) = 0
\end{equation*}

Start by showing there is at least one solution.

Let $ a = 0, f(x) = 2x + \cos(x) $, then $ f(a) = 1 > 0 $. When $ b = \pi $, then $ f(b) = 2(-\pi) + \cos(-\pi) = -2\pi - (-1) < 0 $.

Show at most one solution. By contradiction, assume we have at least two solutions: $ f(x_1) = 0, f(x_2) = 0 $.

By Rolle's Theorem: Show that $ f'(b) = 0 $. $ f'(b) = 2 - \sin(x) \neq 0 $.

\subsubsection{Example}

Prove that the arithmetic mean is greater than or equal to the geometric mean for the case when $ n = 2 $. That is, prove
\begin{equation*}
	\frac{ x + y }{ 2 } \geq \sqrt{ xy }, x \geq, y \geq 0
\end{equation*}
\begin{align*}
	\frac{ x + y }{ 2 } & \geq \sqrt{ xy } \\
	\left( \frac{ x + y }{ 2 } \right)^2 & \geq xy \\
	\frac{ x^2 + y^2 + 2xy }{ 4 } & \geq xy \\
	x^2 + y^2 + 2xy & \geq 4xy \\
	x^2 + y^2 - 2xy & \geq 0 \\
	(x - y)^2 & \geq 0
\end{align*}

\subsubsection{Existence and Uniqueness}

\begin{align*}
	\exists !x P(x) & \implies \text{ If $ P(x) $ is $ T $, then $ P(y) $ is $ F $ if $ y \neq x $. } \\
					& \iff P(x) = P(y) \text{ is true, then } x = y
\end{align*}
\textbf{Existence}: Consider $ x = \frac{ c - b }{ a } $ is well-defined since $ a, b, c \in \mathbb{R}, a \neq 0 $. Then $ ax + b = a \left( \frac{ c - b }{ a } \right) + b = c - b + b = c \therefore x = \frac{ c - b }{ a } $ is a solution.

\textbf{Uniqueness}: Let $ x_1, x_2 $ be two real numbers that are solutions to $ ax + b = c $. Then

\systeme{
	ax_1 + b = c,
	ax_2 + b = c
}

\subsection{Proof Methods and Strategy}

\end{document}
