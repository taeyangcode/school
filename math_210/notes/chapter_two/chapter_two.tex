\documentclass{article}

% Document extensibility %
%
% Disables native paragraph indentation
\usepackage{parskip} 
%
% Provides further bullet options for lists
\usepackage{enumitem}

% Mathematical symbol and statement packages %
%
% Necessary
\usepackage{amsmath}
\usepackage{amssymb}
%
% Extensive fraction notation
\usepackage{xfrac}
%
% Generic mathematical commands
% Notable: \degree, \celcius
\usepackage{gensymb}
%
% Variable vector notation (arrow above variable)
\usepackage{esvect}
%
% Multiline boxed equations
\usepackage{empheq}
%
% SI Unit
\usepackage{siunitx}
\usepackage{physunits}
%
% More intuitive arrays/matrices
\usepackage{array}
%
% Linear Equations
\usepackage{systeme}
%
% Boxes!
\usepackage{mdframed}
%
% Matrix Notation
\usepackage{bm}

% Graphic packages %
%
% Diagrams and illustrations
\usepackage{tikz}
\usetikzlibrary{positioning}
%
% Image insertion
\usepackage{graphicx}

% LaTeX Commands
%
% Argument Parser
\usepackage{xparse}

% Document content %
%
% Change title of table of contents
% \renewcommand{\contentsname}{Title}

\begin{document}

% Command `\hr` to insert horizontal rules
\newcommand{\hr}{\par\noindent\rule{\textwidth}{0.4pt}}

% Command to box and center math equations
\newcommand{\bc}[1]{
	\begin{equation*}
		\begin{boxed}
			{#1}
		\end{boxed}
	\end{equation*}
}

% Command for single line equations with a condition
\newcommand{\cond}[2]{
	\ifmmode
		{#1} \quad {#2}
	\else
		$$ {#1} \quad {#2} $$
	\fi
}

% Matrix and Vector notation
\newcommand{\matr}[1]{
	\ifmmode \bm{#1}
	\else \textit{\textbf{#1}}
	\fi
}
\newcommand{\vect}[1]{
	\ifmmode \mathbf{#1}
	\else \textbf{#1}
	\fi
}

% Laplace
\NewDocumentCommand{\lap}{o}{
	\IfNoValueTF{#1}
		{ \mathcal{L} }
		{ \mathcal{L} \left\{ {#1} \right\} }
}
\NewDocumentCommand{\ilap}{o}{
	\IfNoValueTF{#1}
		{ \mathcal{L}^{-1} }
		{ \mathcal{L}^{-1} \left\{ {#1} \right\} }
}

\tableofcontents

\section{Chapter 02 - Basic Structures}

\subsection{Sets}

\textbf{$ \in $}: belong to, is in

\subsubsection{Definition 2}

Two sets are equal if and only if they have the same elements. Therefore, if $ A $ and $ B $ are sets, then $ A $ and $ B $ are equal if and only if $ \forall x \left( x \in A \iff x \in B \right) $. We write $ A = B $ if $ A $ and $ B $ are equal sets.

\subsubsection{Definition 3}

The set $ A $ is also a subset of $ B $, and $ B $ is a superset of $ A $, if and only if every element of $ A $ is also an element of $ B $. We use the notation $ A \subseteq B $ to indicate that $ A $ is a subset of the set $ B $. If, instead, we want to stress that $ B $ is a superset of $ A $, we use the equivalent notation $ B \supseteq A $. (So, $ A \subseteq B $ and $ B \supseteq A $ are equivalent statements.)

\subsubsection{Definition 4}

Let $ S $ be a set. If there are exactly $ n $ distinct elements in $ S $ where $ n $ is a nonnegative integer, we say that $ S $ is a finite set and that $ n $ is the cardinality of $ S $. The cardinality of $ S $ is denoted by $ |S| $.

\subsubsection{Countable and Uncountable Sets}

\begin{itemize}
	\item Countable
		\begin{itemize}
			\item $ \mathbb{N} $
			\item $ \mathbb{Z} $
			\item $ \mathbb{Q} $
		\end{itemize}
	\item Uncountable
		\begin{itemize}
			\item $ \mathbb{R} $
			\item $ \mathbb{C} $
		\end{itemize}
\end{itemize}

Let $ S_0 = \left\{ x \right\} $, and $ S_1 = \left\{ \left\{ x \right\} \right\} $.
\begin{equation}
	S_0 \neq S_1
\end{equation}

\subsubsection{Example}

\begin{enumerate}
\item List the members of these sets.
	\begin{enumerate}[label = \textbf{\alph*)}]
		\item $ \left\{ x \mid x \text{ is a real number such that } x^2 = 1 \right\} $
			\begin{align*}
				S & = \left\{ x \in \mathbb{R} \mid x^2 = 1 \right\}
			\end{align*}
		\item $ \left\{ x \mid x \text{ is a positive integer less than 12} \right\} $
			\begin{align*}
				S & = \left\{ x \in \mathbb{R} \mid 0 \leq x < 12 \right\}
			\end{align*}
	\end{enumerate}
\end{enumerate}

\subsubsection{Definition 6}

Given a set $ S $, the power set of $ S $ is the set of all subsets of the set $ S $. The power set of $ S $ is denoted by $ \mathcal{P}(S) $.

\subsection{Set Operations}

\subsection{Functions}

\subsection{Sequences and Summations}

\subsection{Cardinality of Sets}

\subsection{Matrices}

\end{document}
