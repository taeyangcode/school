\documentclass{article}

% Document extensibility %
%
% Disables native paragraph indentation
\usepackage{parskip}
%
% Provides further bullet options for lists
\usepackage{enumitem}

% Mathematical symbol and statement packages %
%
% Necessary
\usepackage{amsmath}
\usepackage{amssymb}
%
% Extensive fraction notation
\usepackage{xfrac}
%
% Generic mathematical commands
% Notable: \degree, \celcius
\usepackage{gensymb}
%
% Variable vector notation (arrow above variable)
\usepackage{esvect}
%
% Multiline boxed equations
\usepackage{empheq}
%
% SI Unit
\usepackage{siunitx}
\usepackage{physunits}
%
% More intuitive arrays/matrices
\usepackage{array}
%
% Linear Equations
\usepackage{systeme}
%
% Boxes!
\usepackage{mdframed}
%
% Matrix Notation
\usepackage{bm}

% Graphic packages %
%
% Diagrams and illustrations
\usepackage{tikz}
\usetikzlibrary{positioning}
%
% Image insertion
\usepackage{graphicx}

% LaTeX Commands
%
% Argument Parser
\usepackage{xparse}

% Document content %
%
% Change title of table of contents
% \renewcommand{\contentsname}{Title}

\begin{document}

% Command `\hr` to insert horizontal rules
\newcommand{\hr}{\par\noindent\rule{\textwidth}{0.4pt}}

% Command to box and center math equations
\newcommand{\bc}[1]{
	\begin{equation*}
		\begin{boxed}
			{#1}
		\end{boxed}
	\end{equation*}
}

% Command for single line equations with a condition
\newcommand{\cond}[2]{
	\ifmmode
	{#1} \quad {#2}
	\else
	$$ {#1} \quad {#2} $$
	\fi
}

% Matrix and Vector notation
\newcommand{\matr}[1]{
	\ifmmode \bm{#1}
	\else \textit{\textbf{#1}}
	\fi
}
\newcommand{\vect}[1]{
	\ifmmode \mathbf{#1}
	\else \textbf{#1}
	\fi
}

% Laplace
\NewDocumentCommand{\lap}{o}{
	\IfNoValueTF{#1}
	{ \mathcal{L} }
	{ \mathcal{L} \left\{ {#1} \right\} }
}
\NewDocumentCommand{\ilap}{o}{
	\IfNoValueTF{#1}
	{ \mathcal{L}^{-1} }
	{ \mathcal{L}^{-1} \left\{ {#1} \right\} }
}

\tableofcontents

\section{Chapter 02 - Basic Structures}

\subsection{Sets}

\textbf{$ \in $}: belong to, is in

\subsubsection{Definition 2}

Two sets are equal if and only if they have the same elements. Therefore, if $ A $ and $ B $ are sets, then $ A $ and $ B $ are equal if and only if $ \forall x \left( x \in A \iff x \in B \right) $. We write $ A = B $ if $ A $ and $ B $ are equal sets.

\subsubsection{Definition 3}

The set $ A $ is also a subset of $ B $, and $ B $ is a superset of $ A $, if and only if every element of $ A $ is also an element of $ B $. We use the notation $ A \subseteq B $ to indicate that $ A $ is a subset of the set $ B $. If, instead, we want to stress that $ B $ is a superset of $ A $, we use the equivalent notation $ B \supseteq A $. (So, $ A \subseteq B $ and $ B \supseteq A $ are equivalent statements.)

\subsubsection{Definition 4}

Let $ S $ be a set. If there are exactly $ n $ distinct elements in $ S $ where $ n $ is a nonnegative integer, we say that $ S $ is a finite set and that $ n $ is the cardinality of $ S $. The cardinality of $ S $ is denoted by $ |S| $.

\subsubsection{Countable and Uncountable Sets}

\begin{itemize}
	\item Countable
	      \begin{itemize}
		      \item $ \mathbb{N} $
		      \item $ \mathbb{Z} $
		      \item $ \mathbb{Q} $
	      \end{itemize}
	\item Uncountable
	      \begin{itemize}
		      \item $ \mathbb{R} $
		      \item $ \mathbb{C} $
	      \end{itemize}
\end{itemize}

Let $ S_0 = \left\{ x \right\} $, and $ S_1 = \left\{ \left\{ x \right\} \right\} $.
\begin{equation}
	S_0 \neq S_1
\end{equation}

\subsubsection{Example}

\begin{enumerate}
	\item List the members of these sets.
	      \begin{enumerate}[label = \textbf{\alph*)}]
		      \item $ \left\{ x \mid x \text{ is a real number such that } x^2 = 1 \right\} $
		            \begin{align*}
			            S & = \left\{ x \in \mathbb{R} \mid x^2 = 1 \right\}
		            \end{align*}
		      \item $ \left\{ x \mid x \text{ is a positive integer less than 12} \right\} $
		            \begin{align*}
			            S & = \left\{ x \in \mathbb{R} \mid 0 \leq x < 12 \right\}
		            \end{align*}
	      \end{enumerate}
\end{enumerate}

\subsubsection{Definition 6}

Given a set $ S $, the power set of $ S $ is the set of all subsets of the set $ S $. The power set of $ S $ is denoted by $ \mathcal{P}(S) $.

\subsection{Set Operations}

\subsubsection{Definition 1}

Let $ A $ and $ B $ be sets. The union of the sets $ A $ and $ B $, denoted by $ A \cup B $, is the set that contains those elements that are either in $ A $ or in $ B $, or in both.
\begin{equation}
	A \cup B = \left\{ x \in U \mid ( x \in A ) \lor ( x \in B ) \right\}
\end{equation}

\subsubsection{Definition 2}

Let $ A $ and $ B $ be sets. The intersection of the sets $ A $ and $ B $, denoted by $ A \cap B $, is the set containing those elements in both $ A $ and $ B $.
\begin{equation}
	A \cap B = \left\{ x \in U \mid ( x \in A ) \land ( x \in B ) \right\}
\end{equation}

\subsubsection{Definition 3}

Two sets are disjoint if their intersection is the empty set.

\subsubsection{Definition 4}

Let $ A $ and $ B $ be sets. The difference of $ A $ and $ B $, denoted by $ A - B $, is the set containing those elements that are in $ A $ but not in $ B $. The difference of $ A $ and $ B $ is also called the complement of $ B $ with respect to $ A $.

\subsubsection{Definition 5}

Let $ U $ be the universal set. The complement of the set $ A $, denoted by $ \bar{A} $, is the complement of $ A $ with respect to $ U $. Therefore, the complement of the set $ A $ is $ U - A $.

\subsubsection{Proof}

Let $ A $, $ B $ be sets from $ U $. Show that $ A \subseteq B $ if and only if $ \overline{B} \subseteq \overline{A} $.

\textbf{Proof}:
\begin{itemize}
	\item For ``$ \implies $"

	      Given $ A \subseteq B $, need to show $ \overline{B} \subseteq \overline{A} $. Then $ \forall x \in A $, we have $ x \in B $.

	      By contrapositive we have
	      \begin{align*}
		      \neg \left( x \in B \right) & \implies \neg \left( x \in A \right)                           \\
		      x \notin B                  & \implies x \notin A, \quad \overline{B} \subseteq \overline{A} \\
	      \end{align*}
	\item For ``$ \impliedby $"

	      Given $ \overline{B} \subseteq \overline{A} $, we have $ \forall y \in \overline{B} $, $ y \in \overline{A} $ then the contrapositive is
	      \begin{align*}
		      \neg \left( y \in \overline{A} \right) & \implies \neg \left( y \in \overline{B} \right) \\
		      y \in A                                & \implies y \in B
	      \end{align*}
\end{itemize}

\subsubsection{Proof:}

Use the identities to show that $ \overline{ \left( A \cup B \right) } \cap \overline{ \left( B \cup C \right) } \cap \overline{ \left( A \cup C \right) } = \overline{A} \cap \overline{B} \cap \overline{C} $

\textbf{Proof}:
\begin{align*}
	\overline{A} \cap \overline{B} \cap \overline{C} & = \left( \overline{A} \cap \overline{B} \right) \cap \left( \overline{B} \cap \overline{C} \right) \cap \left( \overline{A} \cap \overline{C} \right) \\
	                                                 & = \overline{A} \cap \left( \overline{B} \cap \overline{B} \right) \cap \left( \overline{C} \cap \overline{C} \right) \cap \overline{A}                \\
	                                                 & = \overline{A} \cap \overline{B} \cap \overline{C}
\end{align*}

\subsubsection{Union}

The union is a collection of sets is the set that contains those elements that are member s of at least one set in the collection.

\begin{align*}
	A_1 \cup A_2 \cup \cdots \cup A_n & = \bigcup_{ i = 1 }^{ n } A_i
\end{align*}

\subsubsection{Intersection}

The intersection of a collection of sets is the set that contains those elements that are members of all the sets in the collection.

\begin{align*}
	A_1 \cap A_2 \cap \cdots \cap A_n & = \bigcap_{ i = 1 }^{ n } A_i
\end{align*}

\subsection{Functions}

\begin{itemize}
	\item Sets: $ A $, $ B $, $ C $, domain, codomain, range

	\item Relations: functions ($ f $, $ g $, $ h $)

	\item Elements: image, preimage
\end{itemize}

\subsubsection{Example}

\begin{align*}
	f_1 : \mathbb{R} \rightarrow \mathbb{R} \text{ as } y & = f_1(x) = 3x - 2 \\
	f_2 : \mathbb{R} \rightarrow \mathbb{R} \text{ as } y & = e^{x}           \\
	f_3 : \mathbb{R} \rightarrow \mathbb{R} \text{ as } y & = \sqrt{x}
\end{align*}
\begin{align*}
	f               & : A \rightarrow B                                           \\
	\text{range}(f) & = \left\{ y \in B \mid \forall x \in S \subseteq A \right\}
\end{align*}

\subsubsection{Properties}

\begin{enumerate}[label = \textbf{\arabic*)}]
	\item \label{function:property:1} $ f $ is injective – one-to-one \\
	\item \label{function:property:2} $ f $ is surjective – \\
	\item $ f $ is bijective if \ref{function:property:1} and \ref{function:property:2}
\end{enumerate}

Let $ f : A \rightarrow B $ be a function.

We say $ f $ is injective if
\begin{equation*}
	\left( x_1 \neq x_2 \implies f(x_1) \neq f(x_2) \right) \iff \left( f(x_1) = f(x_2) \implies x_1 = x_2 \right)
\end{equation*}
We say $ f $ is surjective if $ \forall y \in B $, $ \exists x \in A $ such that $ y = f(x) \iff \text{range}(f) = B $

We say $ f $ is bijective if $ f $ is injective and surjective.

\textbf{Monotonic Function}: We say $ f $ is increasing (strictly increasing) if $ x_1 > x_2 \implies f(x_1) > f(x_2) $

\subsubsection{Example 2.3.24}

Let $ f: \mathbb{R} \rightarrow \mathbb{R} $ and let $ f(x) > 0 $ for all $ x \in \mathbb{R} $. Show that $ f(x) $ is strictly increasing if and only if the function $ g(x) = \frac{ 1 }{ f(x) } $ is strictly decreasing.
\begin{align*}
	 & \rightarrow \frac{ 1 }{ f(x_1) } < \frac{ 1 }{ f(x_2) }                                                                       \\
	 & \rightarrow f(x_1) \cdot f(x_2) \left( \frac{ 1 }{ f(x_1) } \right) < f(x_1) \cdot f(x_2) \left( \frac{ 1 }{ f(x_2) } \right) \\
	 & \rightarrow f(x_2) < f(x_1)
\end{align*}
where the first implication comes from the fact that $ f(x) > 0, \forall x \in \mathbb{R} $.

\subsubsection{Example 2.3.73.b}

Prove or disprove each of these statements about the floor and ceiling functions.

\begin{equation*}
	\left\lfloor 2x \right\rfloor = 2 \left\lfloor x \right\rfloor
\end{equation*}

Consider $ x = 1.6 $
\begin{align*}
	\left\lfloor 2x \right\rfloor  & = \left\lfloor 2 \cdot 1.6 \right\rfloor = \left\lfloor 3.2 \right\rfloor = 3 \\
	2 \left\lfloor x \right\rfloor & = 2 \left\lfloor 1.6 \right\rfloor = 2(1) = 2
\end{align*}
Hence $ \left\lfloor 2x \right\rfloor = 2 \left\lfloor x \right\rfloor $ for all $ x \in \mathbb{R} $ is false.

\subsection{Sequences and Summations}

\subsection{Cardinality of Sets}

\subsection{Matrices}

\end{document}
