\documentclass{article}

% Document extensibility %
%
% Disables native paragraph indentation
\usepackage{parskip} 
%
% Provides further bullet options for lists
\usepackage{enumitem}

% Mathematical symbol and statement packages %
%
% Necessary
\usepackage{amsmath}
\usepackage{amssymb}
%
% Extensive fraction notation
\usepackage{xfrac}
%
% Generic mathematical commands
% Notable: \degree, \celcius
\usepackage{gensymb}
%
% Variable vector notation (arrow above variable)
\usepackage{esvect}
%
% Multiline boxed equations
\usepackage{empheq}
%
% SI Unit
\usepackage{siunitx}
\usepackage{physunits}
%
% More intuitive arrays/matrices
\usepackage{array}
%
% Linear Equations
\usepackage{systeme}
%
% Boxes!
\usepackage{mdframed}
%
% Matrix Notation
\usepackage{bm}

% Graphic packages %
%
% Diagrams and illustrations
\usepackage{tikz}
\usetikzlibrary{positioning}
%
% Image insertion
\usepackage{graphicx}

% LaTeX Commands
%
% Argument Parser
\usepackage{xparse}

% Document content %
%
% Change title of table of contents
% \renewcommand{\contentsname}{Title}

\title{Week 03 Participation Assignment - Part 01}
\author{Corey Mostero - 2566652}
\date{15 September 2023}

\begin{document}

% Command `\hr` to insert horizontal rules
\newcommand{\hr}{\par\noindent\rule{\textwidth}{0.4pt}}

% Command to box and center math equations
\newcommand{\bc}[1]{
	\begin{equation*}
		\begin{boxed}
			{#1}
		\end{boxed}
	\end{equation*}
}

% Command for single line equations with a condition
\newcommand{\cond}[2]{
	\ifmmode
		{#1} \quad {#2}
	\else
		$$ {#1} \quad {#2} $$
	\fi
}

% Matrix and Vector notation
\newcommand{\matr}[1]{
	\ifmmode \bm{#1}
	\else \textit{\textbf{#1}}
	\fi
}
\newcommand{\vect}[1]{
	\ifmmode \mathbf{#1}
	\else \textbf{#1}
	\fi
}

% Laplace
\NewDocumentCommand{\lap}{o}{
	\IfNoValueTF{#1}
		{ \mathcal{L} }
		{ \mathcal{L} \left\{ {#1} \right\} }
}
\NewDocumentCommand{\ilap}{o}{
	\IfNoValueTF{#1}
		{ \mathcal{L}^{-1} }
		{ \mathcal{L}^{-1} \left\{ {#1} \right\} }
}

\maketitle
\newpage

\tableofcontents

\section{Part 01}

The purpose of this exercise is to prove that for any real number: $ a: \sqrt{ a^2 } = \left| a \right| $.

First, we recall that the absolute value of any real number is defined by
\begin{align*}
	|a| & =
	\left\{
		\begin{array}{@{}l@{}}
			a \text{ if $ a \geq 0 $, and } \\
			-a \text{ if $ a < 0 $. }
		\end{array}
	\right.
\end{align*}

\begin{enumerate}[label = \textbf{\alph*)}]
	\item Use the definition above to explain why for any real number $ a : \left| a \right| \geq 0 $. \\
		\textbf{Case by Case Proof}:
		\begin{itemize}
			\item Case 1: $ a \geq 0 $.
				\begin{align*}
					|a| & = a, \quad a \geq 0 \\
					|a| & = a \geq 0 \\
					|a| & \geq a
				\end{align*}
			\item Case 2: $ a < 0 $.
				\begin{align*}
					|a| & = -a, \quad a < 0 \implies -a > 0 \\
					|a| & = -a > 0, \quad -a > 0 \implies -a \geq 0 \\
					|a| & \geq 0
				\end{align*}
		\end{itemize}
	\item Again, using the definition, show that $ |a|^2 = a^2 $. \\
		\textbf{Case by Case Proof}:
		\begin{itemize}
			\item Case 1: $ a \geq 0 $.
				\begin{align*}
					|a|^2 & = a^2 \\
					|a| \cdot |a| & = a \cdot a, \quad |a| = a \geq 0 \\
					a \cdot a & = a \cdot a
				\end{align*}
			\item Case 2: $ a < 0 $.
				\begin{align*}
					|a|^2 & = a^2 \\
					|a| \cdot |a| & = a \cdot a, \quad |a| = -a > 0 \implies |a| = -a \geq 0 \\
					-a \cdot -a & = a \cdot a \\
					a \cdot a & = a \cdot a
				\end{align*}
		\end{itemize}
	\item Our next goal is to show that $ \sqrt{b} $ is unique. In other words, prove that if $ c $ and $ d $ are two real numbers such that $ c \geq 0 $, and $ d \geq 0 $, and $ b = c^2 = d^2 $, then $ c = d $.
		\begin{align*}
			c^2 & = d^2 \\
			c^2 - d^2 & = 0 \\
			(c + d)(c - d) & = 0 \\
			c & = \pm d \\
			|c| & = |d|, \quad |c| = c \geq 0, |d| = d \geq 0 \\
			c & = d
		\end{align*}
	\item Rewrite the definition for $ \sqrt{b} $ to define $ \sqrt{ a^2 } $
	\item Put together all the steps above to write a complete proof that $ \sqrt{ a^2 } = |a| $.
		\begin{align*}
			\sqrt{ b } & = c^2 \\
			\sqrt{ b } & = (\pm d)^2 \\
			c^2 & = |d|^2 \\
			\sqrt{ c^2 } & = \sqrt{ |d|^2 } \\
			\sqrt{ c^2 } & = \sqrt{ d^2 } \\
			\sqrt{ c^2 } & = |d|
		\end{align*}
\end{enumerate}


\end{document}
