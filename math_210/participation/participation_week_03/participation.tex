\documentclass{article}

% Document extensibility %
%
% Disables native paragraph indentation
\usepackage{parskip}
%
% Provides further bullet options for lists
\usepackage{enumitem}

% Mathematical symbol and statement packages %
%
% Necessary
\usepackage{amsmath}
\usepackage{amssymb}
%
% Extensive fraction notation
\usepackage{xfrac}
%
% Generic mathematical commands
% Notable: \degree, \celcius
\usepackage{gensymb}
%
% Variable vector notation (arrow above variable)
\usepackage{esvect}
%
% Multiline boxed equations
\usepackage{empheq}
%
% SI Unit
\usepackage{siunitx}
\usepackage{physunits}
%
% More intuitive arrays/matrices
\usepackage{array}
%
% Linear Equations
\usepackage{systeme}
%
% Boxes!
\usepackage{mdframed}
%
% Matrix Notation
\usepackage{bm}

% Code Blocks
\usepackage{listings}

% Graphic packages %
%
% Diagrams and illustrations
\usepackage{tikz}
\usetikzlibrary{positioning}
%
% Image insertion
\usepackage{graphicx}

% LaTeX Commands
%
% Argument Parser
\usepackage{xparse}

% Document content %
%
% Change title of table of contents
% \renewcommand{\contentsname}{Title}

\title{Week 03 Participation Assignment}
\author{Corey Mostero - 2566652}
\date{15 September 2023}

\begin{document}

% Command `\hr` to insert horizontal rules
\newcommand{\hr}{\par\noindent\rule{\textwidth}{0.4pt}}

% Command to box and center math equations
\newcommand{\bc}[1]{
	\begin{equation*}
		\begin{boxed}
			{#1}
		\end{boxed}
	\end{equation*}
}

% Command for single line equations with a condition
\newcommand{\cond}[2]{
	\ifmmode
	{#1} \quad {#2}
	\else
	$$ {#1} \quad {#2} $$
	\fi
}

% Matrix and Vector notation
\newcommand{\matr}[1]{
	\ifmmode \bm{#1}
	\else \textit{\textbf{#1}}
	\fi
}
\newcommand{\vect}[1]{
	\ifmmode \mathbf{#1}
	\else \textbf{#1}
	\fi
}

% Laplace
\NewDocumentCommand{\lap}{o}{
	\IfNoValueTF{#1}
	{ \mathcal{L} }
	{ \mathcal{L} \left\{ {#1} \right\} }
}
\NewDocumentCommand{\ilap}{o}{
	\IfNoValueTF{#1}
	{ \mathcal{L}^{-1} }
	{ \mathcal{L}^{-1} \left\{ {#1} \right\} }
}

\maketitle
\newpage

\tableofcontents

\section{Part 01}

The purpose of this exercise is to prove that for any real number: $ a: \sqrt{ a^2 } = \left| a \right| $.

First, we recall that the absolute value of any real number is defined by
\begin{align*}
	|a| & =
	\left\{
	\begin{array}{@{}l@{}}
		a \text{ if $ a \geq 0 $, and } \\
		-a \text{ if $ a < 0 $. }
	\end{array}
	\right.
\end{align*}

\begin{enumerate}[label = \textbf{\alph*)}]
	\item Use the definition above to explain why for any real number $ a : \left| a \right| \geq 0 $. \\
	      \textbf{Case by Case Proof}:
	      \begin{itemize}
		      \item Case 1: $ a \geq 0 $.
		            \begin{align*}
			            |a| & = a, \quad a \geq 0 \\
			            |a| & = a \geq 0          \\
			            |a| & \geq a
		            \end{align*}
		      \item Case 2: $ a < 0 $.
		            \begin{align*}
			            |a| & = -a, \quad a < 0 \implies -a > 0         \\
			            |a| & = -a > 0, \quad -a > 0 \implies -a \geq 0 \\
			            |a| & \geq 0
		            \end{align*}
	      \end{itemize}
	\item Again, using the definition, show that $ |a|^2 = a^2 $. \\
	      \textbf{Case by Case Proof}:
	      \begin{itemize}
		      \item Case 1: $ a \geq 0 $.
		            \begin{align*}
			            |a|^2         & = a^2                             \\
			            |a| \cdot |a| & = a \cdot a, \quad |a| = a \geq 0 \\
			            a \cdot a     & = a \cdot a
		            \end{align*}
		      \item Case 2: $ a < 0 $.
		            \begin{align*}
			            |a|^2         & = a^2                                                    \\
			            |a| \cdot |a| & = a \cdot a, \quad |a| = -a > 0 \implies |a| = -a \geq 0 \\
			            -a \cdot -a   & = a \cdot a                                              \\
			            a \cdot a     & = a \cdot a
		            \end{align*}
	      \end{itemize}
	\item Our next goal is to show that $ \sqrt{b} $ is unique. In other words, prove that if $ c $ and $ d $ are two real numbers such that $ c \geq 0 $, and $ d \geq 0 $, and $ b = c^2 = d^2 $, then $ c = d $.
	      \begin{align*}
		      c^2            & = d^2                                       \\
		      c^2 - d^2      & = 0                                         \\
		      (c + d)(c - d) & = 0                                         \\
		      c              & = \pm d                                     \\
		      |c|            & = |d|, \quad |c| = c \geq 0, |d| = d \geq 0 \\
		      c              & = d
	      \end{align*}
	\item Rewrite the definition for $ \sqrt{b} $ to define $ \sqrt{ a^2 } $
	      \begin{align*}
		      b        & = c^2                                                                                         \\
		      \sqrt{b} & = c                                                                                           \\
		      \sqrt{b} & = \pm d                                                                                       \\
		      \sqrt{b} & = \sqrt{ d^2 }, \sqrt{ (-d)^2 }                                                               \\
		      \sqrt{b} & = \sqrt{ d^2 } = \sqrt{ a^2 }, \quad \left\{ d \in \mathbb{R}, \in (-\infty, \infty) \right\}
	      \end{align*}
	\item Put together all the steps above to write a complete proof that $ \sqrt{ a^2 } = |a| $.
	      \begin{align*}
		      \sqrt{ a^2 } & = |a|
	      \end{align*}
	      \textbf{Case by Case Proof}:
	      \begin{itemize}
		      \item Case 1: $ a \geq 0 $
		            \begin{align*}
			            \sqrt{ a^2 } & = |a|                                 \\
			            a^2          & = |a|^2                               \\
			            a \cdot a    & = |a| \cdot |a|, \quad |a| = a \geq 0 \\
			            a \cdot a    & = a \cdot a
		            \end{align*}
		      \item Case 2: $ a < 0 $
		            \begin{align*}
			            \sqrt{ a^2 } & = |a|                                                                 \\
			            a^2          & = |a|^2                                                               \\
			            a \cdot a    & = |a| \cdot |a|, \quad |a| = a < 0 \implies -a > 0 \implies -a \geq 0 \\
			            a \cdot a    & = -a \cdot -a                                                         \\
			            a \cdot a    & = a \cdot a
		            \end{align*}
	      \end{itemize}
\end{enumerate}

\section{Part 02}

Let's consider the powersets of a finite set. Our goal is to ``calculate" how many elements are in the power set. That is the cardinality of the powerset. (We could define powerset from an infinite set).

If $ X = \left\{ x_1, x_2, \cdots, x_n \right\} $ is a finite set, we define $ \mathcal{P}(X) $, the powerset of $ X $, to be the set of all subsets of $ X $. For Example, if $ X = \left\{ a, b \right\} $, then $ \mathcal{P}(X) = \left\{ \emptyset, \left\{ a \right\}, \left\{ b \right\}, \left\{ a, b \right\} \right\} $ and thus $ \mathcal{P}(X) $ has 4 elements.

\begin{enumerate}[label = \textbf{\alph*)}]
	\item \label{part_02_a} If $ X = \left\{ a, b, c \right\} $, list all the members of $ \mathcal{P}(X) $. How many subsets does $ X $ have?
	      \begin{align*}
		      \mathcal{P}(X)   & = \left\{ \emptyset, \left\{ A, B, C \right\}, \left\{ A, B \right\}, \left\{ A, C \right\}, \left\{ B, C \right\}, \left\{ A \right\}, \left\{ B \right\}, \left\{ C \right\} \right\} \\
		      |\mathcal{P}(X)| & = 8
	      \end{align*}
	\item \label{part_02_b} Separate the list that you got in part \ref{part_02_a} into two columns. Place on the left column those subsets that contain $ c $ and place on the right column those that do not contain $ c $. \\
	      \begin{tabular}{ | c | c | }
		      Contains $ c $               & Does Not Contain $ c $    \\
		      \hline
		      $ \left\{ A, B, C \right\} $ & $ \emptyset $             \\
		      $ \left\{ A, C \right\} $    & $ \left\{ A, B \right\} $ \\
		      $ \left\{ B, C \right\} $    & $ \left\{ A \right\} $    \\
		      $ \left\{ C \right\} $       & $ \left\{ B \right\} $
	      \end{tabular}
	\item \label{part_02_c} Now, cross out $ c $ from each subset on the left column. What do you notice? \\
	      \begin{tabular}{ | c | c | }
		      Contains $ c $                                & Does Not Contain $ c $    \\
		      \hline
		      $ \left\{ A, B \right\} $                     & $ \emptyset $             \\
		      $ \left\{ A \right\} $                        & $ \left\{ A, B \right\} $ \\
		      $ \left\{ B \right\} $                        & $ \left\{ A \right\} $    \\
		      $ \left\{ \right\} \left( \emptyset \right) $ & $ \left\{ B \right\} $
	      \end{tabular}

	      Let $ S_0 $ be the set of all sets that contain $ C $
	      \begin{equation*}
		      S_0 = \left\{ x \mid x \in U \land C \in x \right\}
	      \end{equation*}
	      and $ S_1 $ be the power set of $ X $ that does not contain any sets containing $ C $. It can be seen that
	      \begin{equation*}
		      \mathcal{P}(X) - S_0 = S_1
	      \end{equation*}
	      or
	      \begin{equation*}
		      \mathcal{P}(X) \cap \overline{S_0} = S_1
	      \end{equation*}
	      This could be understood otherwise as: The intersection of the $ \mathcal{P}(X) $ (The set of all subsets of $ X $) and the complement of $ S_0 $ (The set of all subsets of $ U $ that \textbf{does not} contain $ C $) is the power set of $ X $ that does not contain $ C $.

	      Table representation: \\
	      \begin{tabular}{ | c | c | c | c | }
		      $ \mathcal{P}(X) $           & $ S_0 $                      & $ \overline{S_0} $        & $ \mathcal{P}(X) \cap \overline{S_0} $ \\
		      \hline
		      $ \left\{ A, B, C \right\} $ & $ \left\{ A, B, C \right\} $ & $ \cdots $                & $ \left\{ A, B \right\} $              \\
		      $ \left\{ A, B \right\} $    & $ \cdots $                   & $ \left\{ A, B \right\} $ & $ \left\{ A \right\} $                 \\
		      $ \left\{ A, C \right\} $    & $ \left\{ A, C \right\} $    & $ \cdots $                & $ \left\{ B \right\} $                 \\
		      $ \left\{ B, C \right\} $    & $ \left\{ B, C \right\} $    & $ \cdots $                & $ \emptyset $                          \\
		      $ \left\{ A \right\} $       & $ \cdots $                   & $ \left\{ A \right\} $    &                                        \\
		      $ \left\{ B	\right\} $       & $ \cdots $                   & $ \left\{ B \right\} $    &                                        \\
		      $ \left\{ C	\right\} $       & $ \left\{ C \right\} $       & $ \cdots $                &                                        \\
		      $ \emptyset $                & $ \cdots $                   & $ \emptyset $             &
	      \end{tabular}
	\item Repeat part \ref{part_02_a}, \ref{part_02_b}, and \ref{part_02_c} for $ X = \left\{ a, b, c, d \right\} $
	      \begin{enumerate}[label = \textbf{\alph*)}]
		      \item
		            \begin{align*}
			            \mathcal{P}(X) = & \left\{ \right.                                                                                                                          \\
			                             & \left\{ A, B, C, D \right\},                                                                                                             \\
			                             & \left\{ A, B, C \right\}, \left\{ A, B, D \right\}, \left\{ A, C, D \right\}, \left\{ B, C, D \right\}                                   \\
			                             & \left\{ A, B \right\}, \left\{ A, C \right\}, \left\{ A, D \right\}, \left\{ B, C \right\}, \left\{ B, D \right\}, \left\{ C, D \right\} \\
			                             & \left\{ A \right\}, \left\{ B \right\}, \left\{ C \right\}, \left\{ D \right\}, \emptyset                                                \\
			                             & \left. \right\}                                                                                                                          \\
			            |\mathcal{P}(X)| & = 16
		            \end{align*}
		      \item
		            \begin{tabular}{ | c | c | }
			            Contains $ C $                  & Does Not Contain $ C $       \\
			            \hline
			            $ \left\{ A, B, C, D \right\} $ & $ \left\{ A, B, D \right\} $ \\
			            $ \left\{ A, B, C \right\} $    & $ \left\{ A, B \right\} $    \\
			            $ \left\{ A, C, D \right\} $    & $ \left\{ A, D \right\} $    \\
			            $ \left\{ B, C, D \right\} $    & $ \left\{ B, D \right\} $    \\
			            $ \left\{ A, C \right\} $       & $ \left\{ A \right\} $       \\
			            $ \left\{ B, C \right\} $       & $ \left\{ B \right\} $       \\
			            $ \left\{ C, D \right\} $       & $ \left\{ D \right\} $       \\
			            $ \left\{ C \right\} $          & $ \emptyset $
		            \end{tabular}
		      \item
		            \begin{tabular}{ | c | c | }
			            Contains $ C $                                & Does Not Contain $ C $       \\
			            \hline
			            $ \left\{ A, B, D \right\} $                  & $ \left\{ A, B, D \right\} $ \\
			            $ \left\{ A, B \right\} $                     & $ \left\{ A, B \right\} $    \\
			            $ \left\{ A, D \right\} $                     & $ \left\{ A, D \right\} $    \\
			            $ \left\{ B, D \right\} $                     & $ \left\{ B, D \right\} $    \\
			            $ \left\{ A \right\} $                        & $ \left\{ A \right\} $       \\
			            $ \left\{ B \right\} $                        & $ \left\{ B \right\} $       \\
			            $ \left\{ D \right\} $                        & $ \left\{ D \right\} $       \\
			            $ \left\{ \right\} \left( \emptyset \right) $ & $ \emptyset $
		            \end{tabular}
	      \end{enumerate}
	\item For $ X = \left\{ x_1, x_2, \cdots, x_n \right\} $, guess the number of elements in the power set $ \mathcal{P}(X) $. \\
	      From previous examples, it can be seen that the number of elements in a set's $ X $ power set $ \mathcal{P}(X) $ is $ 2 $ raised to the cardinality of $ X $. In other words: \\
	      Let $ X $ be any set, $ \mathcal{P}(X) $ be the power set of $ X $, and $ | X | $ be the cardinality or number of elements in $ X $
	      \begin{equation*}
		      | \mathcal{P}(X) | = 2^{| X |}
	      \end{equation*}
\end{enumerate}

\end{document}
