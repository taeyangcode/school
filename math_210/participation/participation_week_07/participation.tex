\documentclass{article}

% Document extensibility %
%
% Disables native paragraph indentation
\usepackage{parskip}
%
% Provides further bullet options for lists
\usepackage{enumitem}

% Mathematical symbol and statement packages %
%
% Necessary
\usepackage{amsmath}
\usepackage{amssymb}
%
% Extensive fraction notation
\usepackage{xfrac}
%
% Generic mathematical commands
% Notable: \degree, \celcius
\usepackage{gensymb}
%
% Variable vector notation (arrow above variable)
\usepackage{esvect}
%
% Multiline boxed equations
\usepackage{empheq}
%
% SI Unit
\usepackage{siunitx}
\usepackage{physunits}
%
% More intuitive arrays/matrices
\usepackage{array}
%
% Linear Equations
\usepackage{systeme}
%
% Boxes!
\usepackage{mdframed}
%
% Matrix Notation
\usepackage{bm}

% Graphic packages %
%
% Diagrams and illustrations
\usepackage{tikz}
\usetikzlibrary{positioning}
%
% Image insertion
\usepackage{graphicx}

% LaTeX Commands
%
% Argument Parser
\usepackage{xparse}

% Document content %
%
% Change title of table of contents
% \renewcommand{\contentsname}{Title}

\title{Week 07 Participation Assignment}
\author{Corey Mostero - 2566652}
\date{13 October 2023}

\begin{document}

% Command `\hr` to insert horizontal rules
\newcommand{\hr}{\par\noindent\rule{\textwidth}{0.4pt}}

% Command to box and center math equations
\newcommand{\bc}[1]{
	\begin{equation*}
		\begin{boxed}
			{#1}
		\end{boxed}
	\end{equation*}
}

% Command for single line equations with a condition
\newcommand{\cond}[2]{
	\ifmmode
	{#1} \quad {#2}
	\else
	$$ {#1} \quad {#2} $$
	\fi
}

% Matrix and Vector notation
\newcommand{\matr}[1]{
	\ifmmode \bm{#1}
	\else \textit{\textbf{#1}}
	\fi
}
\newcommand{\vect}[1]{
	\ifmmode \mathbf{#1}
	\else \textbf{#1}
	\fi
}

% Laplace
\NewDocumentCommand{\lap}{o}{
	\IfNoValueTF{#1}
	{ \mathcal{L} }
	{ \mathcal{L} \left\{ {#1} \right\} }
}
\NewDocumentCommand{\ilap}{o}{
	\IfNoValueTF{#1}
	{ \mathcal{L}^{-1} }
	{ \mathcal{L}^{-1} \left\{ {#1} \right\} }
}

\maketitle
\newpage

\tableofcontents

\section{Week 07 Participation Assignment}

\subsection{} \label{a}

Show that the positive integers less than 11, except 1 and 10, can be split into pairs of integers such that each pair consists of integers that are inverses of each other modulo 11.
\begin{alignat*}{2}
	2 \cdot 6 & \equiv 12 & = 1(\bmod \ 11) \\
	3 \cdot 4 & \equiv 12 & = 1(\bmod \ 11) \\
	5 \cdot 9 & \equiv 45 & = 1(\bmod \ 11) \\
	7 \cdot 8 & \equiv 56 & = 1(\bmod \ 11)
\end{alignat*}

\subsection{} \label{b}

Show that if $ p $ is a prime, the only solutions of $ x^2 \equiv 1(\bmod(p)) $ are the integers $ x $ such that $ x \equiv 1(\bmod(p)) $ or $ x \equiv -1(\bmod(p)) $.
\begin{align*}
	x^2            & \equiv 1(\bmod(p))    \\
	x^2 - 1        & \equiv 0(\bmod(p))    \\
	(x - 1)(x + 1) & \equiv 0(\bmod(p))    \\
	\therefore{}   & p \mid (x - 1)(x + 1)
\end{align*}
\begin{align*}
	x - 1 & \equiv 0(\bmod(p))  \\
	x     & \equiv 1(\bmod(p))  \\
	x + 1 & \equiv 0(\bmod(p))  \\
	x     & \equiv -1(\bmod(p))
\end{align*}

\subsection{} \label{c}

Generalize the result in part \ref{a}; that is, show that if $ p $ is a prime, the positive integers less than $ p $, except $ 1 $ and $ p - 1 $, can be split into $ \frac{ p - 3 }{ 2 } $ pairs of integers such that each pair consists of integers that are inversed of each other.
\begin{align*}
	S & = \mathbb{Z}_p = \left\{ 1, 2, \cdots, p - 2, p - 1 \right\}
\end{align*}
From \ref{b} we can see that there is $ x $ that makes $ x \cdot x^{-1} \equiv 1(\bmod(p)) $. It can also be observed that the equivalences for $ x $ can be written as $ x = 1, p - 1 $. The set $ S $ can now be rewritten as
\begin{align*}
	S & = \left\{ 2, 3, \cdots, p - 3, p - 2 \right\}
\end{align*}
where we have $ p - 3 $ (cannot be $ p - 2 $ as primes cannot be even, and the result of an odd divided by an even is not an integer) positive integers, $ \therefore{} \frac{ p - 3 }{ 2 } $ pairs.

\subsection{}

From part \ref{c}, conclude that $ (p - 1)! \equiv -1(\bmod(p)) $ whenever $ p $ is prime.
\begin{align*}
	(p - 1)! & \equiv 1 \cdot 2 \cdots (p - 2) \cdot (p - 1)                                                                       \\
	(p - 1)! & \equiv 1 \cdot (2 \cdot 2^{-1}) \cdots \left[ (p - 2)(p - 2)^{-1} \right] \cdot (p - 1), \quad \text{using \ref{c}} \\
	(p - 1)! & \equiv 1 \cdot (1) \cdots \left[ 1 \right] \cdot (p - 1)                                                            \\
	(p - 1)! & \equiv 1 \cdot (p - 1)                                                                                              \\
	(p - 1)! & \equiv (p - 1)\bmod(p)                                                                                              \\
	(p - 1)! & \equiv (p \bmod(p)) - (1 \bmod(p))                                                                                  \\
	(p - 1)! & \equiv 0 - 1 \bmod(p)                                                                                               \\
	(p - 1)! & \equiv -1 \bmod(p)
\end{align*}

\subsection{} \label{e}

Suppose that $ a $ is not divisible by the prime $ p $. Show that no two of the integers $ 1 \cdot a, 2 \cdot a, \cdots, (p - 1) \cdot a $ are congruent modulo $ p $.

Let the two integers be $ x $ and $ y $, where $ 1 \leq x < y < p $, giving $ p \mid a(y - x) $. As $ a $ is not divisible by the prime $ p $, it must conclude that $ p \mid (y - x) $. Though as $ p $ is prime, and $ 1 \leq y - x < p $, this cannot be true by the definition of a prime number.

\subsection{} \label{f}

Conclude from part \ref{e} that the product of $ 1, 2, \cdots, p - 1 $ is congruent modulo $ p $ to the product of $ a, 2a, \cdots, (p - 1)a $. Use this to show that $ (p - 1)! \equiv a^{p - 1}(p - 1)!(\bmod(p)) $.

\ref{e} shows that no two integers $ 1 \cdot a, 2 \cdot a, \cdots, (p - 1) \cdot a $ are congruent modulo $ p $.
\begin{align*}
	a \cdot 2a \cdots (p - 1)a                   & = (1 \cdot 2 \cdots p - 1) \bmod(p) \\
	(1 \cdot 2 \cdots (p - 1)) \cdot (a^{p - 1}) & = (p - 1)!                          \\
	(p - 1)! \cdot a^{p - 1}                     & = (p - 1)!
\end{align*}

\subsection{}

Use Theorem 7 of Section 4.3 to show that from part \ref{f} that $ a^{p - 1} \equiv 1(\bmod{p}) $ if $ p \mid a $.
\begin{align*}
	(p - 1)!                      & \equiv -1 (\bmod(p))         \\
	(-1) \cdot a^{p - 1}          & \equiv -1(\bmod(p))          \\
	-1 \cdot (-1) \cdot a^{p - 1} & \equiv -1(\bmod(p)) \cdot -1 \\
	a^{p - 1}                     & \equiv 1(\bmod(p))
\end{align*}

\subsection{}

Use part \ref{c} to show that $ a^{p} \equiv a(\bmod(p)) $ for all integers $ a $.
\begin{itemize}
	\item Case 1: $ p \mid a $
	      \begin{equation*}
		      \forall a \in \mathbb{Z} (a^{p} \equiv 0 \bmod(p) \iff a(\bmod(p)) \equiv 0 \bmod(p))
	      \end{equation*}
	\item Case 2: $ p \nmid a $ (Fermat's Little Theorem)
	      \begin{align*}
		      a^{p - 1} & \equiv 1(\bmod(p)) \\
		      a^p       & \equiv a(\bmod(p))
	      \end{align*}
\end{itemize}

\end{document}
