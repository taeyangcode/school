\documentclass{article}

% Document extensibility %
%
% Disables native paragraph indentation
\usepackage{parskip}
%
% Provides further bullet options for lists
\usepackage{enumitem}

% Mathematical symbol and statement packages %
%
% Necessary
\usepackage{amsmath}
\usepackage{amssymb}
\usepackage{cancel}
%
% Extensive fraction notation
\usepackage{xfrac}
%
% Generic mathematical commands
% Notable: \degree, \celcius
\usepackage{gensymb}
%
% Variable vector notation (arrow above variable)
\usepackage{esvect}
%
% Multiline boxed equations
\usepackage{empheq}
%
% SI Unit
\usepackage{siunitx}
\usepackage{physunits}
%
% More intuitive arrays/matrices
\usepackage{array}
%
% Linear Equations
\usepackage{systeme}
%
% Boxes!
\usepackage{mdframed}
%
% Matrix Notation
\usepackage{bm}

% Graphic packages %
%
% Diagrams and illustrations
\usepackage{tikz}
\usetikzlibrary{positioning}
%
% Image insertion
\usepackage{graphicx}

% LaTeX Commands
%
% Argument Parser
\usepackage{xparse}

% Document content %
%
% Change title of table of contents
% \renewcommand{\contentsname}{Title}

\title{Week 08 Participation Assignment}
\author{Corey Mostero - 2566652}
\date{20 October 2023}

\begin{document}

% Command `\hr` to insert horizontal rules
\newcommand{\hr}{\par\noindent\rule{\textwidth}{0.4pt}}

% Command to box and center math equations
\newcommand{\bc}[1]{
	\begin{equation*}
		\begin{boxed}
			{#1}
		\end{boxed}
	\end{equation*}
}

% Command for single line equations with a condition
\newcommand{\cond}[2]{
	\ifmmode
	{#1} \quad {#2}
	\else
	$$ {#1} \quad {#2} $$
	\fi
}

% Matrix and Vector notation
\newcommand{\matr}[1]{
	\ifmmode \bm{#1}
	\else \textit{\textbf{#1}}
	\fi
}
\newcommand{\vect}[1]{
	\ifmmode \mathbf{#1}
	\else \textbf{#1}
	\fi
}

% Laplace
\NewDocumentCommand{\lap}{o}{
	\IfNoValueTF{#1}
	{ \mathcal{L} }
	{ \mathcal{L} \left\{ {#1} \right\} }
}
\NewDocumentCommand{\ilap}{o}{
	\IfNoValueTF{#1}
	{ \mathcal{L}^{-1} }
	{ \mathcal{L}^{-1} \left\{ {#1} \right\} }
}

\maketitle
\newpage

\section{}

If $ X = \left\{ x_1, x_2, \cdots, x_n \right\} $ is a finite set, we define $ \mathcal{P}(X) $, the powerset of $ X $, to be the set of all subsets of $ X $.

\subsection{} \label{a}

If $ X = \left\{ a, b, c \right\} $, list all the members of $ \mathcal{P}(X) $. How many subsets does $ X $ have?

\begin{align*}
	\mathcal{P}(X) & = \left\{ \right.          \\
	               & \left\{ a, b, c \right\}   \\
	               & \left\{ a, b \right\}      \\
	               & \left\{ a, c \right\}      \\
	               & \left\{ b, c \right\}      \\
	               & \left\{ a \right\}         \\
	               & \left\{ b \right\}         \\
	               & \left\{ c \right\}         \\
	               & \left\{ \emptyset \right\} \\
	               & \left. \right\}
\end{align*}
\begin{align*}
	\left| \mathcal{P}(X) \right| & = 8
\end{align*}

\subsection{} \label{b}

Separate the list that you got in part \ref{a} into two columns. Place on the left column those subsets that contain $ c $ and place on the right column those that do not contain $ c $.

\begin{tabular}{ | c | c | }
	\textbf{Contains}            & \textbf{Does not contain}      \\
	\hline
	$ \left\{ a, b, c \right\} $ & $ \left\{ a, b \right\} $      \\
	$ \left\{ a, c \right\}    $ & $ \left\{ a \right\} $         \\
	$ \left\{ b, c \right\}    $ & $ \left\{ b \right\} $         \\
	$ \left\{ c \right\}       $ & $ \left\{ \emptyset \right\} $ \\
\end{tabular}

\subsection{} \label{c}

Now, cross out $ c $ from each subset on the left column. What do you notice?

\begin{tabular}{ | c | c | }
	\textbf{Contains}                      & \textbf{Does not contain}      \\
	\hline
	$ \left\{ a, b, \bcancel{c} \right\} $ & $ \left\{ a, b \right\} $      \\
	$ \left\{ a, \bcancel{c} \right\}    $ & $ \left\{ a \right\} $         \\
	$ \left\{ b, \bcancel{c} \right\}    $ & $ \left\{ b \right\} $         \\
	$ \left\{ \bcancel{c} \right\}       $ & $ \left\{ \emptyset \right\} $ \\
\end{tabular}

\begin{tabular}{ | c | c | }
	\textbf{Contains}                    & \textbf{Does not contain}      \\
	\hline
	$ \left\{ a, b \right\} $            & $ \left\{ a, b \right\} $      \\
	$ \left\{ a \right\}    $            & $ \left\{ a \right\} $         \\
	$ \left\{ b \right\}    $            & $ \left\{ b \right\} $         \\
	$ \left\{ \emptyset \right\}       $ & $ \left\{ \emptyset \right\} $ \\
\end{tabular}

The set of each column ends up equivalent.

\subsection{}

Repeat part \ref{a}, \ref{b}, and \ref{c} for $ X = \left\{ a, b, c, d \right\} $

\subsubsection{}

\begin{align*}
	\mathcal{P}(X) & = \left\{ \right.           \\
	               & \left\{ a, b, c, d \right\} \\
	               & \left\{ a, b, c \right\}    \\
	               & \left\{ a, b, d \right\}    \\
	               & \left\{ a, c, d \right\}    \\
	               & \left\{ b, c, d \right\}    \\
	               & \left\{ a, b \right\}       \\
	               & \left\{ a, c \right\}       \\
	               & \left\{ a, d \right\}       \\
	               & \left\{ b, c \right\}       \\
	               & \left\{ b, d \right\}       \\
	               & \left\{ c, d \right\}       \\
	               & \left\{ a \right\}          \\
	               & \left\{ b \right\}          \\
	               & \left\{ c \right\}          \\
	               & \left\{ d \right\}          \\
	               & \left\{ \emptyset \right\}  \\
	               & \left. \right\}
\end{align*}
\begin{align*}
	\left| \mathcal{P}(X) \right| & = 16
\end{align*}

\subsubsection{}

\begin{tabular}{ | c | c | }
	\textbf{Contains}               & \textbf{Does not contain}      \\
	\hline
	$ \left\{ a, b, c, d \right\} $ & $ \left\{ a, b, d \right\} $   \\
	$ \left\{ a, b, c \right\} $    & $ \left\{ a, b \right\} $      \\
	$ \left\{ a, c, d \right\} $    & $ \left\{ a, d \right\} $      \\
	$ \left\{ b, c, d \right\} $    & $ \left\{ b, d \right\} $      \\
	$ \left\{ a, c \right\} $       & $ \left\{ a \right\} $         \\
	$ \left\{ b, c \right\} $       & $ \left\{ b \right\} $         \\
	$ \left\{ c, d \right\} $       & $ \left\{ d \right\}         $ \\
	$ \left\{ c \right\} $          & $ \left\{ \emptyset \right\} $ \\
\end{tabular}

\subsubsection{}

\begin{tabular}{ | c | c | }
	\textbf{Contains}                         & \textbf{Does not contain}      \\
	\hline
	$ \left\{ a, b, \bcancel{c}, d \right\} $ & $ \left\{ a, b, d \right\} $   \\
	$ \left\{ a, b, \bcancel{c} \right\} $    & $ \left\{ a, b \right\} $      \\
	$ \left\{ a, \bcancel{c}, d \right\} $    & $ \left\{ a, d \right\} $      \\
	$ \left\{ b, \bcancel{c}, d \right\} $    & $ \left\{ b, d \right\} $      \\
	$ \left\{ a, \bcancel{c} \right\} $       & $ \left\{ a \right\} $         \\
	$ \left\{ b, \bcancel{c} \right\} $       & $ \left\{ b \right\} $         \\
	$ \left\{ \bcancel{c}, d \right\} $       & $ \left\{ d \right\}         $ \\
	$ \left\{ \bcancel{c} \right\} $          & $ \left\{ \emptyset \right\} $ \\
\end{tabular}

\begin{tabular}{ | c | c | }
	\textbf{Contains}              & \textbf{Does not contain}      \\
	\hline
	$ \left\{ a, b, d \right\} $   & $ \left\{ a, b, d \right\} $   \\
	$ \left\{ a, b \right\} $      & $ \left\{ a, b \right\} $      \\
	$ \left\{ a, d \right\} $      & $ \left\{ a, d \right\} $      \\
	$ \left\{ b, d \right\} $      & $ \left\{ b, d \right\} $      \\
	$ \left\{ a \right\} $         & $ \left\{ a \right\} $         \\
	$ \left\{ b \right\} $         & $ \left\{ b \right\} $         \\
	$ \left\{  d \right\} $        & $ \left\{ d \right\}         $ \\
	$ \left\{ \emptyset \right\} $ & $ \left\{ \emptyset \right\} $ \\
\end{tabular}

\subsection{}

For $ X = \left\{ x_1, x_2, \cdots, x_n \right\} $, guess the number of elements in the powerset $ \mathcal{P}(X) $.

\begin{equation*}
	\left| \mathcal{P} \right| = 2^n
\end{equation*}

\subsection{}

I hope you can guess that the number of elements in the powerset $ \mathcal{P}(X) $ is $ 2^n $. That means $ \left| \mathcal{P}(X) \right| = 2^n $.

Now, use induction to prove this guess.

\textbf{Proof by induction}:

Where $ n \in \mathbb{W} $.

\begin{itemize}
	\item Basis step: $ n = 0 $
	      \begin{align*}
		      X                             & = \left\{ \emptyset \right\} \\
		      \mathcal{P}(X)                & = \left\{ \emptyset \right\} \\
		      \left| \mathcal{P}(X) \right| & = 1
	      \end{align*}
	      \begin{equation*}
		      \left| \mathcal{P}(X) \right| \equiv 2^0 = 1
	      \end{equation*}

	\item Inductive step:

	      Where $ k \in \mathbb{W} $,
	      \begin{equation*}
		      \left| X \right| = k \implies \left| \mathcal{P}(X) \right| = 2^k
	      \end{equation*}
	      Let the set $ Y $ be the set with cardinality $ \left| X \right| + 1 $.
	      \begin{equation*}
		      Y = \left\{ a_1, a_2, \cdots, a_k, a_{k + 1} \right\}
	      \end{equation*}
	      $ Y $ can also be defined as
	      \begin{equation*}
		      Y = \left\{ a_1, a_2, \cdots, a_{k - 1}, a_k \right\} \cup \left\{ a_{k + 1} \right\}
	      \end{equation*}
	      The first set in the redefinition of $ Y $ can be observed as the set $ X $; or the set not containing $ a_{k + 1} $. From our initial inductive step, we can say that the cardinality of the powerset of $ \mathcal{P} \left( \left\{ a_1, a_2, \cdots, a_{k - 1}, a_k \right\} \right) = 2^k $. Now considering the union portion of the set $ \left\{ a_{k + 1} \right\} $, it can be observed that it will simply be the powerset $ \mathcal{P} \left( \left\{ a_1, a_2, \cdots, a_{k - 1}, a_k \right\} \right) $ that includes $ a_{k + 1} $ within each element. It can observed that the cardinality would also be $ 2^k $ as the amount of elements in the powerset doesn't change, and only the contents of each element.
	      \begin{align*}
		      \therefore\ Y                 & = \left\{ a_1, a_2, \cdots, a_{k - 1}, a_k \right\} \cup \left\{ a_{k + 1} \right\} \\
		      \left| \mathcal{P}(Y) \right| & = 2^k + 2^k = 2^{k + 1}
	      \end{align*}
	      It is proven that a set with $ n $ elements, its powerset must have $ 2^n $ elements.
\end{itemize}

\end{document}
