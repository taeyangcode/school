\documentclass{article}

% Document extensibility %
%
% Disables native paragraph indentation
\usepackage{parskip} 
%
% Provides further bullet options for lists
\usepackage{enumitem}

% Mathematical symbol and statement packages %
%
% Necessary
\usepackage{amsmath}
\usepackage{amssymb}
%
% Extensive fraction notation
\usepackage{xfrac}
%
% Generic mathematical commands
% Notable: \degree, \celcius
\usepackage{gensymb}
%
% Variable vector notation (arrow above variable)
\usepackage{esvect}
%
% Multiline boxed equations
\usepackage{empheq}
%
% SI Unit
\usepackage{siunitx}
\usepackage{physunits}
%
% More intuitive arrays/matrices
\usepackage{array}
%
% Linear Equations
\usepackage{systeme}
%
% Boxes!
\usepackage{mdframed}
%
% Matrix Notation
\usepackage{bm}

% Graphic packages %
%
% Diagrams and illustrations
\usepackage{tikz}
\usetikzlibrary{positioning}
%
% Image insertion
\usepackage{graphicx}

% LaTeX Commands
%
% Argument Parser
\usepackage{xparse}

% Document content %
%
% Change title of table of contents
% \renewcommand{\contentsname}{Title}

\title{Week 01 Participation Assignment}
\author{Corey Mostero - 2566652}
\date{8 September 2023}

\begin{document}

% Command `\hr` to insert horizontal rules
\newcommand{\hr}{\par\noindent\rule{\textwidth}{0.4pt}}

% Command to box and center math equations
\newcommand{\bc}[1]{
	\begin{equation*}
		\begin{boxed}
			{#1}
		\end{boxed}
	\end{equation*}
}

% Command for single line equations with a condition
\newcommand{\cond}[2]{
	\ifmmode
		{#1} \quad {#2}
	\else
		$$ {#1} \quad {#2} $$
	\fi
}

% Matrix and Vector notation
\newcommand{\matr}[1]{
	\ifmmode \bm{#1}
	\else \textit{\textbf{#1}}
	\fi
}
\newcommand{\vect}[1]{
	\ifmmode \mathbf{#1}
	\else \textbf{#1}
	\fi
}

% Laplace
\NewDocumentCommand{\lap}{o}{
	\IfNoValueTF{#1}
		{ \mathcal{L} }
		{ \mathcal{L} \left\{ {#1} \right\} }
}
\NewDocumentCommand{\ilap}{o}{
	\IfNoValueTF{#1}
		{ \mathcal{L}^{-1} }
		{ \mathcal{L}^{-1} \left\{ {#1} \right\} }
}

% Comments
\newcommand{\cmt}[1]{}

\maketitle
\newpage

\tableofcontents

\section{Problem 1}

Use a truth table to verify the first De Morgan law: $ \neg ( p \land q ) \equiv \neg p \lor \neg q $

\begin{tabular}{ | c | c | c | c | c | c | c | }
	$ p $ & $ q $ & $ p \land q $ & $ \neg ( p \land q ) $ & $ \neg p $ & $ \neg q $ & $ \neg p \lor \neg q $ \\
	\hline
	T & T & T & F & F & F & F \\
	T & F & F & T & F & T & T \\
	F & T & F & T & T & F & T \\
	F & F & F & T & T & T & T
\end{tabular}

The truth values for the compound propositions $ \neg ( p \land q ) $ and $ \neg p \lor \neg q $ are the same, and are therefore logically equivalent thereby verifying the first De Morgan law.

\section{Problem 2}

Verify by showing either both sides are true, or both sides are false, for exactly the same combinations of truth values of the propositional variables in these expressions (except for 18).

\subsection{16}

Show that $ p \iff q $ and $ ( p \land q ) \lor ( \neg p \land \neg q ) $ are logically equivalent.

\subsubsection{Truth Table}

\begin{tabular}{ | c | c | c | c | c | c | c | c | }
	$ p $ & $ q $ & $ p \iff q $ & $ p \land q $ & $ \neg p $ & $ \neg q $ & $ \neg p \land \neg q $ & $ ( p \land q ) \lor ( \neg p \land \neg q ) $ \\
	\hline
	T & T & T & T & F & F & F & T \\
	T & F & F & F & F & T & F & F \\
	F & T & F & F & T & F & F & F \\
	F & F & T & F & T & T & T & T
\end{tabular}

\subsubsection{Proof}

Start by observing that the first compound proposition $ p \iff q $ is true only if $ ( p = T \land q = T ) \lor ( p = F \land q = F ) $.

Looking at the second compound proposition, the first proposition $ (p \land q) $ is only true when both $ p = T \land q = T $.

As for the second proposition $ ( \neg p \land \neg q ) $, this can only be true when both $ p = F \land q = F $.

The second compound proposition $ ( p \land q ) \lor ( \neg p \land \neg q ) $ can now be seen as true only when $ ( p = T \land q = T ) \lor ( p = F \land q = F ) $.

$ \therefore $ it can be concluded that each compound proposition is true if and only if: $ ( p = T \land q = T ) \lor ( p = F \land q = F ) $, proving the expression.

\subsection{17}

Show that $ \neg ( p \iff q ) $ and $ p \iff \neg q $ are logically equivalent.

\subsubsection{Truth Table}

\begin{tabular}{ | c | c | c | c | c | c | }
	$ p $ & $ q $ & $ p \iff q $ & $ \neg ( p \iff q ) $ & $ \neg q $ & $ p \iff \neg q $ \\
	\hline
	T & T & T & F & F & F \\
	T & F & F & T & T & T \\
	F & T & F & T & F & T \\
	F & F & T & F & T & F
\end{tabular}

\subsubsection{Proof}

Begin by observing the first compound proposition $ \neg ( p \iff q ) $. The expression $ p \iff q $ is \textbf{true} \textit{if and only if} $ ( ( p = T ) \land ( q = T ) ) \lor ( ( p = F ) \land ( q = F ) ) $. Including the negation connective, the LHS compound proposition is \textbf{false} under the same aforementioned truth values.

Now looking at the RHS compound proposition, we can see that it is \textbf{false} \textit{if and only if} $ p \equiv q $. In other words, the RHS compound proposition is \textbf{false} \textit{if and only if} $ ( ( p = T ) \land ( q = T ) ) \lor ( ( p = F ) \land ( q = F ) ) $.

$ \therefore $ both sides are verified to be \textbf{false} under the same combination of truth values: $ p \equiv q $, or more specifically $ ( ( p = T ) \land ( q = T ) ) \lor ( ( p = F ) \land ( q = F ) ) $.

\subsection{18}

Show that $ p \implies q $ and $ \neg q \implies \neg p $ are logically equivalent.

\begin{tabular}{ | c | c | c | c | c | c | }
	$ p $ & $ q $ & $ p \implies q $ & $ \neg q $ & $ \neg p $ & $ \neg q \implies \neg p $ \\
	\hline
	T & T & T & F & F & T \\
	T & F & F & T & F & F \\
	F & T & T & F & T & T \\
	F & F & T & T & T & T \\
\end{tabular}

\subsection{19}

Show that $ \neg p \iff q $ and $ p \iff \neg q $ are logically equivalent.

\begin{tabular}{ | c | c | c | c | c | c | }
	$ p $ & $ \neg p $ & $ q $ & $ \neg p \iff q $ & $ \neg q $ & $ p \iff \neg q $ \\
	\hline
	T & F & T & F & F & F \\
	T & F & F & T & T & T \\
	F & T & T & T & F & T \\
	F & T & F & F & T & F \\
\end{tabular}

\subsection{20}

Show that $ \neg ( p \oplus q ) $ and $ p \iff q $ are logically equivalent.

\begin{tabular}{ | c | c | c | c | c | }
	$ p $ & $ q $ & $ p \oplus q $ & $ \neg ( p \oplus q ) $ & $ p \iff q $ \\
	\hline
	T & T & F & T & T \\
	T & F & T & F & F \\
	F & T & T & F & F \\
	F & F & F & T & T \\
\end{tabular}

\subsection{21}

Show that $ \neg ( p \iff q ) $ and $ \neg p \iff q $ are logically equivalent.

\begin{tabular}{ | c | c | c | c | c | c | }
	$ p $ & $ q $ & $ p \iff q $ & $ \neg ( p \iff q ) $ & $ \neg p $ & $ \neg p \iff q $ \\
	\hline
	T & T & T & F & F & F \\
	T & F & F & T & F & T \\
	F & T & F & T & T & T \\
	F & F & T & F & T & F
\end{tabular}

\subsection{22}

Show that $ ( p \implies q ) \land ( p \implies r ) $ and $ p \implies ( q \land r ) $ are logically equivalent.

\begin{tabular}{ | c | c | c | c | c | c | }
	$ p $ & $ q $ & $ r $ & $ p \implies q $ & $ p \implies r $ & $ ( p \implies q ) \land ( p \implies r ) $ \\
	\hline
	T & T & T & T & T & T \\
	T & T & F & T & F & F \\
	T & F & T & F & T & F \\
	T & F & F & F & F & F \\
	F & T & T & T & T & T \\
	F & T & F & T & T & T \\
	F & F & T & T & T & T \\
	F & F & F & T & T & T \\
\end{tabular}

\begin{tabular}{ | c | c | c | c | c | }
	$ p $ & $ q $ & $ r $ & $ q \land r $ & $ p \implies ( q \land r ) $ \\
	\hline
	T & T & T & T & T \\
	T & T & F & F & F \\
	T & F & T & F & F \\
	T & F & F & F & F \\
	F & T & T & T & T \\
	F & T & F & F & T \\
	F & F & T & F & T \\
	F & F & F & F & T \\
\end{tabular}

\subsection{23}

Show that $ ( p \implies r ) \land ( q \implies r ) $ and $ ( p \lor q ) \implies r $ are logically equivalent.

\begin{tabular}{ | c | c | c | c | c | c | }
	$ p $ & $ q $ & $ r $ & $ p \implies r $ & $ q \implies r $ & $ ( p \implies r ) \land ( q \implies r ) $ \\
	\hline
	T & T & T & T & T & T \\
	T & T & F & F & F & F \\
	T & F & T & T & T & T \\
	T & F & F & F & T & F \\
	F & T & T & T & T & T \\
	F & T & F & T & F & F \\
	F & F & T & T & T & T \\
	F & F & F & T & T & T \\
\end{tabular}

\begin{tabular}{ | c | c | c | c | c | }
	$ p $ & $ q $ & $ r $ & $ p \lor q $ & $ ( p \lor q ) \implies r $ \\
	\hline
	T & T & T & T & T \\
	T & T & F & T & F \\
	T & F & T & T & T \\
	T & F & F & T & F \\
	F & T & T & T & T \\
	F & T & F & T & F \\
	F & F & T & F & T \\
	F & F & F & F & T \\
\end{tabular}

\subsection{24}

Show that $ ( p \implies q ) \lor ( p \implies r ) $ and $ p \implies ( q \lor r ) $ are logically equivalent.

\begin{tabular}{ | c | c | c | c | c | c | }
	$ p $ & $ q $ & $ r $ & $ p \implies q $ & $ p \implies r $ & $ ( p \implies q ) \lor ( p \implies r ) $ \\
	\hline
	T & T & T & T & T & T \\
	T & T & F & T & F & T \\
	T & F & T & F & T & T \\
	T & F & F & F & F & F \\
	F & T & T & T & T & T \\
	F & T & F & T & T & T \\
	F & F & T & T & T & T \\
	F & F & F & T & T & T \\
\end{tabular}

\begin{tabular}{ | c | c | c | c | c | }
	$ p $ & $ q $ & $ r $ & $ q \lor r $ & $ p \implies ( q \lor r ) $ \\
	\hline
	T & T & T & T & T \\
	T & T & F & T & T \\
	T & F & T & T & T \\
	T & F & F & F & F \\
	F & T & T & T & T \\
	F & T & F & T & T \\
	F & F & T & T & T \\
	F & F & F & F & T \\
\end{tabular}

\subsection{25}

Show that $ ( p \implies r ) \lor ( q \implies r ) $ and $ ( p \land q ) \implies r $ are logically equivalent.

\begin{tabular}{ | c | c | c | c | c | c | }
	$ p $ & $ q $ & $ r $ & $ p \implies r $ & $ q \implies r $ & $ ( p \implies r ) \lor ( q \implies r ) $ \\
	\hline
	T & T & T & T & T & T \\
	T & T & F & F & F & F \\
	T & F & T & T & T & T \\
	T & F & F & F & T & T \\
	F & T & T & T & T & T \\
	F & T & F & T & F & T \\
	F & F & T & T & T & T \\
	F & F & F & T & T & T \\
\end{tabular}

\begin{tabular}{ | c | c | c | c | c | }
	$ p $ & $ q $ & $ r $ & $ p \land q $ & $ ( p \land q ) \implies r $ \\
	\hline
	T & T & T & T & T \\
	T & T & F & T & F \\
	T & F & T & F & T \\
	T & F & F & F & T \\
	F & T & T & F & T \\
	F & T & F & F & T \\
	F & F & T & F & T \\
	F & F & F & F & T \\
\end{tabular}

\subsection{26}

Show that $ \neg p \implies ( q \implies r ) $ and $ q \implies ( p \lor r ) $ are logically equivalent.

\begin{tabular}{ | c | c | c | c | c | c | }
	$ p $ & $ q $ & $ r $ & $ \neg p $ & $ q \implies r $ & $ \neg p \implies ( q \implies r ) $ \\
	\hline
	T & T & T & F & T & T \\
	T & T & F & F & F & T \\
	T & F & T & F & T & T \\
	T & F & F & F & T & T \\
	F & T & T & T & T & T \\
	F & T & F & T & F & F \\
	F & F & T & T & T & T \\
	F & F & F & T & T & T \\
\end{tabular}

\begin{tabular}{ | c | c | c | c | c | }
	$ p $ & $ q $ & $ r $ & $ p \lor r $ & $ q \implies ( p \lor r ) $ \\
	\hline
	T & T & T & T & T \\
	T & T & F & T & T \\
	T & F & T & T & T \\
	T & F & F & T & T \\
	F & T & T & T & T \\
	F & T & F & F & F \\
	F & F & T & T & T \\
	F & F & F & F & T \\
\end{tabular}

\subsection{27}

Show that $ p \iff q $ and $ ( p \implies q ) \land ( q \implies p ) $ are logically equivalent.

\begin{tabular}{ | c | c | c | }
	$ p $ & $ q $ & $ p \iff q $ \\
	\hline
	T & T & T \\
	T & F & F \\
	F & T & F \\
	F & F & T \\
\end{tabular}

\begin{tabular}{ | c | c | c | c | c | }
	$ p $ & $ q $ & $ p \implies q $ & $ q \implies p $ & $ ( p \implies q ) \land ( q \implies p ) $ \\
	\hline
	T & T & T & T & T \\
	T & F & F & T & F \\
	F & T & T & F & F \\
	F & F & T & T & T \\
\end{tabular}

\subsection{28}

Show that $ p \iff q $ and $ \neg p \iff \neg q $ are logically equivalent.

\begin{tabular}{ | c | c | c | c | c | c | }
	$ p $ & $ q $ & $ p \iff q $ & $ \neg p $ & $ \neg q $ & $ \neg p \iff \neg q $ \\
	\hline
	T & T & T & F & F & T \\
	T & F & F & F & T & F \\
	F & T & F & T & F & F \\
	F & F & T & T & T & T \\
\end{tabular}

\end{document}

\cmt{
\begin{tabular}{ | c | c | c | }
	$ p $ & $ q $ & $ r $ \\
	\hline
	T & T & T \\
	T & T & F \\
	T & F & T \\
	T & F & F \\
	F & T & T \\
	F & T & F \\
	F & F & T \\
	F & F & F \\
\end{tabular}
}
