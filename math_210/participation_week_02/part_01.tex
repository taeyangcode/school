\documentclass{article}

% Document extensibility %
%
% Disables native paragraph indentation
\usepackage{parskip} 
%
% Provides further bullet options for lists
\usepackage{enumitem}

% Mathematical symbol and statement packages %
%
% Necessary
\usepackage{amsmath}
\usepackage{amssymb}
%
% Extensive fraction notation
\usepackage{xfrac}
%
% Generic mathematical commands
% Notable: \degree, \celcius
\usepackage{gensymb}
%
% Variable vector notation (arrow above variable)
\usepackage{esvect}
%
% Multiline boxed equations
\usepackage{empheq}
%
% SI Unit
\usepackage{siunitx}
\usepackage{physunits}
%
% More intuitive arrays/matrices
\usepackage{array}
%
% Linear Equations
\usepackage{systeme}
%
% Boxes!
\usepackage{mdframed}
%
% Matrix Notation
\usepackage{bm}

% Graphic packages %
%
% Diagrams and illustrations
\usepackage{tikz}
\usetikzlibrary{positioning}
%
% Image insertion
\usepackage{graphicx}

% LaTeX Commands
%
% Argument Parser
\usepackage{xparse}

% Document content %
%
% Change title of table of contents
% \renewcommand{\contentsname}{Title}

\title{Week 02 Participation Assignment Part 01}
\author{Corey Mostero - 2566652}
\date{8 September 2023}

\begin{document}

% Command `\hr` to insert horizontal rules
\newcommand{\hr}{\par\noindent\rule{\textwidth}{0.4pt}}

% Command to box and center math equations
\newcommand{\bc}[1]{
	\begin{equation*}
		\begin{boxed}
			{#1}
		\end{boxed}
	\end{equation*}
}

% Command for single line equations with a condition
\newcommand{\cond}[2]{
	\ifmmode
		{#1} \quad {#2}
	\else
		$$ {#1} \quad {#2} $$
	\fi
}

% Matrix and Vector notation
\newcommand{\matr}[1]{
	\ifmmode \bm{#1}
	\else \textit{\textbf{#1}}
	\fi
}
\newcommand{\vect}[1]{
	\ifmmode \mathbf{#1}
	\else \textbf{#1}
	\fi
}

% Laplace
\NewDocumentCommand{\lap}{o}{
	\IfNoValueTF{#1}
		{ \mathcal{L} }
		{ \mathcal{L} \left\{ {#1} \right\} }
}
\NewDocumentCommand{\ilap}{o}{
	\IfNoValueTF{#1}
		{ \mathcal{L}^{-1} }
		{ \mathcal{L}^{-1} \left\{ {#1} \right\} }
}

\maketitle
\newpage

\tableofcontents

\section{Part 01}

Suppose the domain of the propositional function $ P(x, y) $ consists of pairs $ x $ and $ y $, where $ x $ is 1, 2, or 3 and $ y $ is 1, 2, or 3. Write out these propositions using disjunctions and conjunctions.

\subsection{a)}
\begin{equation*}
	\forall x \forall y P(x, y)
\end{equation*}

\begin{align*}
	\forall x \forall y P(x, y) & \\
								& \equiv P(1, 1) \land P(1, 2) \land P(1, 3) \\
								& \land P(2, 1) \land P(2, 2) \land P(2, 3) \\
								& \land P(3, 1) \land P(3, 2) \land P(3, 3)
\end{align*}

\subsection{b)}
$$ \exists x \exists y P(x, y) $$

\begin{align*}
	\exists x \exists y P(x, y) & \\
	& \equiv P(1, 1) \lor P(1, 2) \lor P(1, 3) \\
	& \lor P(2, 1) \lor P(2, 2) \lor P(2, 3) \\
	& \lor P(3, 1) \lor P(3, 2) \lor P(3, 3)
\end{align*}

\subsection{c)}
$$ \exists x \forall y P(x, y) $$

\begin{align*}
	\exists x \forall y P(x, y) & \\
								& \equiv \left( P(1, 1) \land P(1, 2) \land P(1, 3) \right) \\
								& \lor \left( P(2, 1) \land P(2, 2) \land P(2, 3) \right) \\
								& \lor \left( P(3, 1) \land P(3, 2) \land P(3, 3) \right)
\end{align*}

\subsection{d)}
$$ \forall y \exists x P(x, y) $$

\begin{align*}
	\forall y \exists x P(x, y) & \\
								& \equiv \left( P(1, 1) \lor P(2, 1) \lor P(3, 1) \right) \\
								& \land \left( P(1, 2) \lor P(2, 2) \lor P(3, 2) \right) \\
								& \land \left( P(1, 3) \lor P(2, 3) \lor P(3, 3) \right)
\end{align*}

\end{document}
