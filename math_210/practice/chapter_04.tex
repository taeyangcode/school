\documentclass{article}

% Document extensibility %
%
% Disables native paragraph indentation
\usepackage{parskip}
%
% Provides further bullet options for lists
\usepackage{enumitem}

% Mathematical symbol and statement packages %
%
% Necessary
\usepackage{amsmath}
\usepackage{amssymb}
%
% Extensive fraction notation
\usepackage{xfrac}
%
% Generic mathematical commands
% Notable: \degree, \celcius
\usepackage{gensymb}
%
% Variable vector notation (arrow above variable)
\usepackage{esvect}
%
% Multiline boxed equations
\usepackage{empheq}
%
% SI Unit
\usepackage{siunitx}
\usepackage{physunits}
%
% More intuitive arrays/matrices
\usepackage{array}
%
% Linear Equations
\usepackage{systeme}
%
% Boxes!
\usepackage{mdframed}
%
% Matrix Notation
\usepackage{bm}

% Graphic packages %
%
% Diagrams and illustrations
\usepackage{tikz}
\usetikzlibrary{positioning}
%
% Image insertion
\usepackage{graphicx}

% LaTeX Commands
%
% Argument Parser
\usepackage{xparse}

% Document content %
%
% Change title of table of contents
% \renewcommand{\contentsname}{Title}

\begin{document}

% Command `\hr` to insert horizontal rules
\newcommand{\hr}{\par\noindent\rule{\textwidth}{0.4pt}}

% Command to box and center math equations
\newcommand{\bc}[1]{
	\begin{equation*}
		\begin{boxed}
			{#1}
		\end{boxed}
	\end{equation*}
}

% Command for single line equations with a condition
\newcommand{\cond}[2]{
	\ifmmode
	{#1} \quad {#2}
	\else
	$$ {#1} \quad {#2} $$
	\fi
}

% Matrix and Vector notation
\newcommand{\matr}[1]{
	\ifmmode \bm{#1}
	\else \textit{\textbf{#1}}
	\fi
}
\newcommand{\vect}[1]{
	\ifmmode \mathbf{#1}
	\else \textbf{#1}
	\fi
}

% Laplace
\NewDocumentCommand{\lap}{o}{
	\IfNoValueTF{#1}
	{ \mathcal{L} }
	{ \mathcal{L} \left\{ {#1} \right\} }
}
\NewDocumentCommand{\ilap}{o}{
	\IfNoValueTF{#1}
	{ \mathcal{L}^{-1} }
	{ \mathcal{L}^{-1} \left\{ {#1} \right\} }
}

\newcommand{\modulus}[1]{\mathrm{mod}\ #1}

\tableofcontents

\section{1}

Use the Euclidean Algorithm to find $ \gcd(7544, 115) $. Then express the greatest common divisor as a linear combination of 7544 and 115.
\begin{align*}
	7544                       & = 65 \cdot 115 + 69 \\
	115                        & = 1 \cdot 69 + 46   \\
	69                         & = 1 \cdot 46 + 23   \\
	46                         & = 2 \cdot 23 + 0    \\
	\therefore \gcd(7544, 115) & = 23
\end{align*}
\begin{align*}
	23 & = 69 - 1 \cdot 46                      \\
	46 & = 115 - 1 \cdot 69                     \\
	23 & = 69 - 1 \cdot (115 - 1 \cdot 69)      \\
	23 & = 2 \cdot 69 - 1 \cdot 115             \\
	69 & = 7544 - 65 \cdot 115                  \\
	23 & = 2(7544 - 65 \cdot 115) - 1 \cdot 115 \\
	23 & = 2 \cdot 7544 - 131 \cdot 115
\end{align*}

\section{2}

Find an inverse of $ a $ modulo $ m $ by Euclidean Algorithm, where $ a = 74, m = 389 $.
\begin{align*}
	389                      & = 5 \cdot 74 + 19 \\
	74                       & = 3 \cdot 19 + 17 \\
	19                       & = 1 \cdot 17 + 2  \\
	17                       & = 8 \cdot 2 + 1   \\
	2                        & = 2 \cdot 1       \\
	\therefore \gcd(74, 389) & = 1
\end{align*}
\begin{align*}
	1  & = 17 - 8 \cdot 2                           \\
	2  & = 19 - 1 \cdot 17                          \\
	1  & = 17 - 8 \cdot (19 - 1 \cdot 17)           \\
	1  & = 9 \cdot 17 - 8 \cdot 19                  \\
	17 & = 74 - 3 \cdot 19                          \\
	1  & = 9 \cdot (74 - 3 \cdot 19) - 8 \cdot 19   \\
	1  & = 9 \cdot 74 - 35 \cdot 19                 \\
	19 & = 389 - 5 \cdot 74                         \\
	1  & = 9 \cdot 74 - 35 \cdot (389 - 5 \cdot 74) \\
	1  & = 184 \cdot 74 - 35 \cdot 389
\end{align*}
\begin{align*}
	sa + tm                     & = 1(\bmod(m))         \\
	184 \cdot 74 - 35 \cdot 389 & \equiv 1 (\bmod(389)) \\
	184 \cdot 74                & \equiv 1 (\bmod(389))
\end{align*}
184 is an inverse of $ a \bmod(m) $.

\section{3}

Solve the congruence $ 74x \equiv 5(\bmod(389)) $ using the modular inverse from the previous problem.
\begin{align*}
	184 \cdot \left[ 74x \right] & \equiv \left[ 5(\bmod(389)) \right] \cdot 184 \\
	x                            & \equiv 142(\bmod(389))
\end{align*}

\section{4}

Show that if $ ac \equiv bc(\bmod(m)) $, where $ a, b, c $, and $ m $ are integers with $ m > 2 $, and $ d = \gcd(m, c) $, then $ a \equiv b \left( \bmod \left( \frac{ m }{ d } \right) \right) $.

\begin{align*}
	ac                                                                                  & \equiv bc(\bmod(m)) \iff m \mid ac - bc                                         \\
	ac - bc                                                                             & = k \cdot m                                                                     \\
	a \left( d \cdot \frac{ c }{ d } \right) - b \left( d \cdot \frac{ c }{ d } \right) & = k \left( d \cdot \frac { m }{ d } \right)                                     \\
	a \left(  \frac{ c }{ d } \right) - b \left(  \frac{ c }{ d } \right)               & = k \left(  \frac { m }{ d } \right)                                            \\
	a \left(  \frac{ c }{ d } \right)                                                   & \equiv b \left(  \frac{ c }{ d } \right) \bmod \left(  \frac { m }{ d } \right) \\
	a                                                                                   & \equiv b  \bmod \left(  \frac { m }{ d } \right)
\end{align*}

\end{document}
