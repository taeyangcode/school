\documentclass{article}

% Document extensibility %
%
% Disables native paragraph indentation
\usepackage{parskip} 
%
% Provides further bullet options for lists
\usepackage{enumitem}

% Mathematical symbol and statement packages %
%
% Necessary
\usepackage{amsmath}
\usepackage{amssymb}
%
% Extensive fraction notation
\usepackage{xfrac}
%
% Generic mathematical commands
% Notable: \degree, \celcius
\usepackage{gensymb}
%
% Variable vector notation (arrow above variable)
\usepackage{esvect}
%
% Multiline boxed equations
\usepackage{empheq}
%
% SI Unit
\usepackage{siunitx}

% Graphic packages %
%
% Diagrams and illustrations
\usepackage{tikz}
%
% Image insertion
\usepackage{graphicx}

% Document content %
%
% Change title of table of contents
% \renewcommand{\contentsname}{Title}

\title{Week 03 Participation Assignment (1 of 3)}
\date{03 March 2023}
\author{Corey Mostero}

\begin{document}

% Command `\hr` to insert horizontal rules
\newcommand{\hr}{\par\noindent\rule{\textwidth}{0.4pt}}

\maketitle
\newpage

\tableofcontents

\section{Part 1}
To test for a first order differential equation to be exact or not, we must write it in the form of $ M(x,y)dx + N(x,y)dy = 0 $. Determine whether the given equation is exact or not. (For your own practice, you may also identify the equation as separable or linear as well as exact equation).

\subsection{a)}
$$ (2x + yx^{-1})dx + (xy - 1)dy = 0 $$
\begin{align*}
	\frac{\partial M}{\partial y} & = \frac{\partial}{\partial y} \left( 2x + yx^{-1} \right) \\
	& = (0) + x^{-1} \\
	& = x^{-1} \\
	\frac{\partial N}{\partial x} & = \frac{\partial}{\partial x} \left( xy - 1 \right) \\
	& = y - (0) \\
	& = y
\end{align*}
\begin{equation*}
	\boxed{
		\frac{\partial M}{\partial y} \neq \frac{\partial N}{\partial x} \therefore \text{not exact}
	}
\end{equation*}

\subsection{b)}
$$ (2y^3 + 2y^2)dx + (3xy^2 + 2xy)dy = 0 $$
\begin{align*}
	\frac{\partial M}{\partial y} & = \frac{\partial}{\partial y} \left( 2y^3 + 2y^2 \right) \\
								  & = 6y^2 + 4y \\
	\frac{\partial N}{\partial x} & = \frac{\partial}{\partial x} \left( 3xy^2 + 2xy \right) \\
								  & = 3y^2 + 2y
\end{align*}
\begin{equation*}
	\boxed{
		\frac{\partial M}{\partial y} \neq \frac{\partial N}{\partial x} \therefore \text{not exact}
	}
\end{equation*}

\subsection{c)}
$$ (2x + y)dx + (x - 2y)dy = 0 $$
\begin{align*}
	\frac{\partial M}{\partial y} & = \frac{\partial}{\partial y} \left( 2x + y \right) \\
								  & = (0) + 1 \\
	\frac{\partial N}{\partial x} & = \frac{\partial}{\partial x} \left( x - 2y \right) \\
								  & = 1 - (0)
\end{align*}
\begin{equation*}
	\boxed{
		\frac{\partial M}{\partial y} = \frac{\partial N}{\partial x} \therefore \text{exact}
	}
\end{equation*}

\subsection{d)}
$$ (y^2 + 2xy)dx - x^2dy = 0 $$
\begin{align*}
	\frac{\partial M}{\partial y} & = \frac{\partial}{\partial y} \left( y^2 + 2xy \right) \\
								  & = 2y + 2x \\
	\frac{\partial N}{\partial x} & = \frac{\partial}{\partial x} \left( -x^2 \right) \\
								  & = -2x
\end{align*}
\begin{equation*}
	\boxed{
		\frac{\partial M}{\partial y} \neq \frac{\partial N}{\partial x} \therefore \text{not exact}
	}
\end{equation*}

\subsection{e)}
$$ (x^2\sin(x) + 4y)dx + xdy = 0 $$
\begin{align*}
	\frac{\partial M}{\partial y} & = \frac{\partial}{\partial y} \left( x^2\sin(x) + 4y \right) \\
								  & = (0) + 4 \\
								  & = 4 \\
	\frac{\partial N}{\partial x} & = \frac{\partial}{\partial x} \left( x \right) \\
								  & = 1
\end{align*}
\begin{equation*}
	\boxed{
		\frac{\partial M}{\partial y} \neq \frac{\partial N}{\partial x} \therefore \text{not exact}
	}
\end{equation*}

\subsection{f)}
$$ [\sin(xy) + xy\cos(xy)]dx + [1 + x^2\cos(xy)]dy = 0 $$
\begin{align*}
	\frac{\partial M}{\partial x} & = \frac{\partial}{\partial x} \left( \sin(xy) + xy\cos(xy) \right) \\
								  & = x\cos(xy) + (x \cdot \cos(xy)) + (-x\sin(xy) \cdot xy) \\
								  & = 2x\cos(xy) -x^2y\sin(xy) \\
	\frac{\partial N}{\partial x} & = \frac{\partial}{\partial x} \left( 1 + x^2\cos(xy) \right) \\
								  & = (0) + (2x \cdot \cos(xy)) + (-y\sin(xy) \cdot x^2) \\
								  & = 2x\cos(xy) -x^2y\sin(xy)
\end{align*}
\begin{equation*}
	\boxed{
		\frac{\partial M}{\partial y} = \frac{\partial N}{\partial x} \therefore \text{exact}
	}
\end{equation*}

\end{document}
