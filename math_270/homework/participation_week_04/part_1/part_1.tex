\documentclass{article}

% Document extensibility %
%
% Disables native paragraph indentation
\usepackage{parskip} 
%
% Provides further bullet options for lists
\usepackage{enumitem}

% Mathematical symbol and statement packages %
%
% Necessary
\usepackage{amsmath}
\usepackage{amssymb}
%
% Extensive fraction notation
\usepackage{xfrac}
%
% Generic mathematical commands
% Notable: \degree, \celcius
\usepackage{gensymb}
%
% Variable vector notation (arrow above variable)
\usepackage{esvect}
%
% Multiline boxed equations
\usepackage{empheq}
%
% SI Unit
\usepackage{siunitx}

% Graphic packages %
%
% Diagrams and illustrations
\usepackage{tikz}
%
% Image insertion
\usepackage{graphicx}

% Document content %
%
% Change title of table of contents
% \renewcommand{\contentsname}{Title}

\title{Week 04 Participation Assignment (1 of 2)}
\date{10 March 2023}
\author{Corey Mostero}

\begin{document}

% Command `\hr` to insert horizontal rules
\newcommand{\hr}{\par\noindent\rule{\textwidth}{0.4pt}}

% Command to box and center math equations
\newcommand{\bc}[1]{
	\begin{equation*}
		\begin{boxed}
			{#1}
		\end{boxed}
	\end{equation*}
}

\maketitle
\newpage

\tableofcontents

\section{Part 1}
One of the most important steps on solving higher order linear differential equation with constant coefficients is to identify the characteristic/auxiliary polynomial. Then from the roots (single roots, repeated roots or complex roots), we can determine the solution to the homogeneous equation.

For this exercise, we would like to practice on the relation between the roots and the solutions (not only from the roots to the solution, but also from the solution/function to the roots):

\subsection{1)}
\begin{align*}
	r_1 & = -1 \\
	r_2 & = 5 \\
	p(r) & = (r - r_1)(r - r_2) \\
		 & = (r - (-1))(r - 5) \\
	p(r) & = r^2 -4x - 5
\end{align*}
\bc{p(r) = r^2 -4x - 5}

\subsection{2)}
\begin{align*}
	r_1 & = -1 \\
	r_2 & = -1 \\
	p(r) & = (r - r_1)(r - r_2) \\
		 & = (r + 1)(r + 1) \\
	p(r) & = r^2 + 2r + 1
\end{align*}
\bc{p(r) = r^2 + 2r + 1}

\subsection{3)}
\begin{align*}
	r_1 & = 4 \\
	r_2 & = -4 \\
	p(r) & = (r - r_1)(r - r_2) \\
		 & = (r - 4)(r - (-4)) \\
	p(r) & = r^2 + 16
\end{align*}
\bc{p(r) = r^2 + 16}

\subsection{4)}
\begin{align*}
	r_1 & = -3 - 2i \\
	r_2 & = -3 + 2i \\
	p(r) & = (r - r_1)(r - r_2) \\
		 & = (r - (-3 - 2i))(r - (-3 + 2i)) \\
	p(r) & = r^2 + 6r + 13
\end{align*}
\bc{p(r) = r^2 + 6r + 13}

\subsection{5)}
\begin{align*}
	r_1 & = -3 \\
	r_2 & = -3 \\
	r_3 & = -3 \\
	p(r) & = (r - r_1)(r - r_2)(r - r_3) \\
		 & = (r - (-3))(r - (-3))(r - (-3)) \\
	p(r) & = r^3 + 9r^2 + 27r + 27
\end{align*}
\bc{p(r) = r^3 + 9r^2 + 27r + 27}

\subsection{6)}
\begin{align*}
	r_1 & = \pm1 \\
	r_2 & = \pm1 \\
	p(r) & = (r - r_1)(r - r_2) \\
		 & = (r - (-1))^2(r - 1)^2 \\
	p(r) & = r^4 + 2r^2 + 1
\end{align*}
\bc{p(r) = r^4 + 2r^2 + 1}

\subsection{7)}
\begin{align*}
	r_1 & = -5 \pm 4i \\
	r_2 & = -5 \pm 4i \\
	p(r) & = (r - r_1)(r - r_2) \\
		 & = (r - (-5 + 4i))^2(r - (-5 - 4i))^2 \\
	p(r) & = r^4 + 20r^3 + 182r^2 + 820r + 168
\end{align*}
\bc{p(r) = r^4 + 20r^3 + 182r^2 + 820r + 168}

\end{document}
