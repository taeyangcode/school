\documentclass{article}

% Document extensibility %
%
% Disables native paragraph indentation
\usepackage{parskip} 
%
% Provides further bullet options for lists
\usepackage{enumitem}

% Mathematical symbol and statement packages %
%
% Necessary
\usepackage{amsmath}
\usepackage{amssymb}
%
% Extensive fraction notation
\usepackage{xfrac}
%
% Generic mathematical commands
% Notable: \degree, \celcius
\usepackage{gensymb}
%
% Variable vector notation (arrow above variable)
\usepackage{esvect}
%
% Multiline boxed equations
\usepackage{empheq}
%
% SI Unit
\usepackage{siunitx}

% Graphic packages %
%
% Diagrams and illustrations
\usepackage{tikz}
%
% Image insertion
\usepackage{graphicx}

% Document content %
%
% Change title of table of contents
% \renewcommand{\contentsname}{Title}

\title{Week 05 Participation Assignment (1 of 2)}
\date{17 March 2023}
\author{Corey Mostero}

\begin{document}

% Command `\hr` to insert horizontal rules
\newcommand{\hr}{\par\noindent\rule{\textwidth}{0.4pt}}

% Command to box and center math equations
\newcommand{\bc}[1]{
	\begin{equation*}
		\begin{boxed}
			{#1}
		\end{boxed}
	\end{equation*}
}

% Command for single line equations with a condition
\newcommand{\cond}[2]{
	\ifmmode
		{#1} \quad {#2}
	\else
		$$ {#1} \quad {#2} $$
	\fi
}

\maketitle
\newpage

\tableofcontents

\section{Part 1}
When we use the method of Undetermined Coefficients to find a participation solution for a given differential equation (non-homogeneous), it is limited to the case that $ f(x) $  is in the form of sine function, cosine function, exponential function, polynomial and product/sum of the functions mentioned. This method won't work if $ f(x) = \sec(3x) $, for example, $ y'' + 9y = 2\sec(3x) $. Then we will use another method called Variation of Parameter.

Let's use this method to solve the differential equation $ y'' + 9y = 2\sec(3x) $

\subsection{1}
\begin{align*}
	y'' + 9y & = 2\sec(3x) \\
	r^2 + 9 & = 0 \\
	r & = 0 \pm 3i \\
	y(x) & = c_1\cos(3x) + c_2\sin(3x) \\
	y_1(x) & = \cos(3x) \\
	y_1'(x) & = -3\sin(3x) \\
	y_2(x) & = \sin(3x) \\
	y_2'(x) & = 3\cos(3x)
\end{align*}
\begin{align*}
	W & = 
		\begin{vmatrix}
			\cos(3x) & \sin(3x) \\
			-3\sin(3x) & 3\cos(3x)
		\end{vmatrix} \\
	  & = \cos(3x) \cdot 3\cos(3x) - \sin(3x) \cdot -3\sin(3x) \\
	W & = 3\cos^2(3x) + 3\sin^2(3x) \\
	W & = 3 \left( \cos^2(3x) + \sin^2(3x) \right) \\
	W & = 3
\end{align*}
\begin{align*}
	W_1 & =
		\begin{vmatrix}
			0 & \sin(3x) \\
			2\sec(3x) & 3\cos(3x)
		\end{vmatrix} \\
	W_1 & = -2\tan(3x)
\end{align*}
\begin{align*}
	W_2 & =
		\begin{vmatrix}
			\cos(3x) & 0 \\
			-3\sin(3x) & 2\sec(3x)
		\end{vmatrix} \\
	W_2 & = 2
\end{align*}
\begin{align*}
	\int \frac{W_1}{W} dx & = \int \frac{-2\tan(3x)}{3} dx \\
						  & = -\frac{2}{3} \int \frac{\sin(3x)}{\cos(3x)} dx \\
						  & = \frac{2}{3 \cdot 3} \int \frac{1}{u} du \\
	\int \frac{W_1}{W} dx & = \frac{2}{9} \ln(\cos(3x)) + C_3
\end{align*}
\begin{align*}
	\int \frac{W_2}{W} dx & = \int \frac{2}{3} dx \\
	\int \frac{W_2}{W} dx & = \frac{2x}{3} + C_4
\end{align*}
\begin{align*}
	y(x) & = \left( \frac{2}{9} \ln(\cos(3x)) + C_3 \right) \cos(3x) + \left( \frac{2x}{3} + C_4 \right) \sin(3x) \\
	y(x) & = \frac{2}{9}\ln(\cos(3x))\cos(3x) + C_3\cos(3x) + \frac{2}{3}x\sin(3x) + C_4\sin(3x)
\end{align*}
\bc{y(x) = \frac{2}{9}\ln(\cos(3x))\cos(3x) + C_3\cos(3x) + \frac{2}{3}x\sin(3x) + C_4\sin(3x)}

\end{document}
