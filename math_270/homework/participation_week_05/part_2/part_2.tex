\documentclass{article}

% Document extensibility %
%
% Disables native paragraph indentation
\usepackage{parskip} 
%
% Provides further bullet options for lists
\usepackage{enumitem}

% Mathematical symbol and statement packages %
%
% Necessary
\usepackage{amsmath}
\usepackage{amssymb}
%
% Extensive fraction notation
\usepackage{xfrac}
%
% Generic mathematical commands
% Notable: \degree, \celcius
\usepackage{gensymb}
%
% Variable vector notation (arrow above variable)
\usepackage{esvect}
%
% Multiline boxed equations
\usepackage[overload]{empheq}
%
% SI Unit
\usepackage{siunitx}

% Graphic packages %
%
% Diagrams and illustrations
\usepackage{tikz}
%
% Image insertion
\usepackage{graphicx}

% Document content %
%
% Change title of table of contents
% \renewcommand{\contentsname}{Title}

\title{Week 05 Participation Assignment (2 of 2)}
\date{17 March 2023}
\author{Corey Mostero}

\begin{document}

% Command `\hr` to insert horizontal rules
\newcommand{\hr}{\par\noindent\rule{\textwidth}{0.4pt}}

% Command to box and center math equations
\newcommand{\bc}[1]{
	\begin{equation*}
		\begin{boxed}
			{#1}
		\end{boxed}
	\end{equation*}
}

% Command for single line equations with a condition
\newcommand{\cond}[2]{
	\ifmmode
		{#1} \quad {#2}
	\else
		$$ {#1} \quad {#2} $$
	\fi
}

\maketitle
\newpage

\tableofcontents

\section{Part 2}

\subsection{1}
\begin{align*}
	A & =
		\begin{bmatrix}
			3 & 1 & 1 \\
			5 & 1 & 2 \\
			-2 & 4 & -3
		\end{bmatrix} \\
	[A | \vec{b}] & =
		\begin{bmatrix}
			3 & 1 & 1 & | & 3 \\
			5 & 1 & 2 & | & 4 \\
			-2 & 4 & -3 & | & 5
		\end{bmatrix}
\end{align*}

\subsection{2}
\begin{align*}
	A & =
		\begin{bmatrix}
			4 & 2 & -3 & -14 \\
			3 & 0 & 4 & 22 \\
			-3 & 1 & -2 & -19 \\
			1 & -1 & -4 & -9
		\end{bmatrix} \\
	[A | \vec{b}] & =
		\begin{bmatrix}
			4 & 2 & -3 & -14 & | & -7 \\
			3 & 0 & 4 & 22 & | & 29 \\
			-3 & 1 & -2 & -19 & | & -21 \\
			1 & -1 & -4 & -9 & | & -15
		\end{bmatrix}
\end{align*}

\subsection{3}
\begin{align*}
	A & =
		\begin{bmatrix}
			5 & 6 & -22 & -27 & 0 \\
			-1 & -3 & 17 & 0 & 9 \\
			7 & 4 & 0 & -51 & 22
		\end{bmatrix} \\
	[A | \vec{b}] & =
		\begin{bmatrix}
			5 & 6 & -22 & -27 & 0 & | & -3 \\
			-1 & -3 & 17 & 0 & 9 & | & -3 \\
			7 & 4 & 0 & -51 & 22 & | & -13
		\end{bmatrix}
\end{align*}

\subsection{4}
\begin{align*}
	A & =
		\begin{bmatrix}
			9 & 3 & 36 & 7 & -2 & 31 \\
			7 & 4 & 23 & 2 & -5 & 31 \\
			-2 & 4 & -22 & 10 & -5 & 72 \\
			5 & 2 & 10 & -7 & -2 & -21 \\
			3 & 6 & -3 & -2 & -5 & 19
		\end{bmatrix} \\
	[A | \vec{b}] & =
		\begin{bmatrix}
			9 & 3 & 36 & 7 & -2 & 31 & | & 23 \\
			7 & 4 & 23 & 2 & -5 & 31 & | & 14 \\
			-2 & 4 & -22 & 10 & -5 & 72 & | & -29 \\
			5 & 2 & 10 & -7 & -2 & -21 & | & 31 \\
			3 & 6 & -3 & -2 & -5 & 19 & | & 26
		\end{bmatrix}
\end{align*}

\end{document}
