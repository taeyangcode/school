\documentclass{article}

% Document extensibility %
%
% Disables native paragraph indentation
\usepackage{parskip} 
%
% Provides further bullet options for lists
\usepackage{enumitem}

% Mathematical symbol and statement packages %
%
% Necessary
\usepackage{amsmath}
\usepackage{amssymb}
%
% Extensive fraction notation
\usepackage{xfrac}
%
% Generic mathematical commands
% Notable: \degree, \celcius
\usepackage{gensymb}
%
% Variable vector notation (arrow above variable)
\usepackage{esvect}
%
% Multiline boxed equations
\usepackage{empheq}
%
% SI Unit
\usepackage{siunitx}

% Graphic packages %
%
% Diagrams and illustrations
\usepackage{tikz}
%
% Image insertion
\usepackage{graphicx}

% Document content %
%
% Change title of table of contents
% \renewcommand{\contentsname}{Title}

\begin{document}

% Command `\hr` to insert horizontal rules
\newcommand{\hr}{\par\noindent\rule{\textwidth}{0.4pt}}

% Command to box and center math equations
\newcommand{\bc}[1]{
	\begin{equation*}
		\begin{boxed}
			{#1}
		\end{boxed}
	\end{equation*}
}

% Command for single line equations with a condition
\newcommand{\cond}[2]{
	\ifmmode
		{#1} \quad {#2}
	\else
		$$ {#1} \quad {#2} $$
	\fi
}

\tableofcontents

\section{Part 1}
In section 3.2. I will focus on the method of Gaussian Elimination and later on the Gauss-Jordan Elimination (for section 3.3).

Consider the augmented matrix
$$ [A|\vec{b}] =
	\begin{bmatrix}
		3 & -15 & -5 & 2 & | & 27 \\
		-2 & 10 & 3 & -4 & | & -28 \\
		5 & -25 & -2 & -1 & | & 15
	\end{bmatrix}
$$

\begin{enumerate}[label=\textbf{Set \arabic*}]
	\item
		\begin{align*}
			R_1 & = \frac{1}{3}r_1 \\
			[A|\vec{b}] & = \begin{bmatrix}
				1 & -5 & -\frac{5}{3} & \frac{2}{3} & | & 9 \\
				-2 & 10 & 3 & -4 & | & -28 \\
				5 & -25 & -2 & -1 & | & 15
			\end{bmatrix}
		\end{align*}
		\begin{align*}
			R_2 & = r_2 + 2r_1 \\
			[A|\vec{b}] & = \begin{bmatrix}
				1 & -5 & -\frac{5}{3} & \frac{2}{3} & | & 9 \\
				0 & 0 & -\frac{1}{3} & -\frac{8}{3} & | & -10 \\
				5 & -25 & -2 & -1 & | & 15
			\end{bmatrix}
		\end{align*}
		\begin{align*}
			R_3 & = r_3 - 5r_1 \\
			[A|\vec{b}] & = \begin{bmatrix}
				1 & -5 & -\frac{5}{3} & \frac{2}{3} & | & 9 \\
				0 & 0 & -\frac{1}{3} & -\frac{8}{3} & | & -10 \\
				0 & 0 & \frac{19}{3} & -\frac{13}{3} & | & -30
			\end{bmatrix}
		\end{align*}
		\begin{align*}
			R_2 & = -3r_2 \\
			[A|\vec{b}] & = \begin{bmatrix}
				1 & -5 & -\frac{5}{3} & \frac{2}{3} & | & 9 \\
				0 & 0 & 1 & 8 & | & 30 \\
				0 & 0 & \frac{19}{3} & -\frac{13}{3} & | & -30
			\end{bmatrix}
		\end{align*}
		\begin{align*}
			R_3 & = r_3 - \frac{19}{3}r_2 \\
			[A|\vec{b}] & = \begin{bmatrix}
				1 & -5 & -\frac{5}{3} & \frac{2}{3} & | & 9 \\
				0 & 0 & 1 & 8 & | & 30 \\
				0 & 0 & 0 & -55 & | & -220
			\end{bmatrix}
		\end{align*}
		\begin{align*}
			R_3 & = -\frac{1}{55}r_3 \\
			[A|\vec{b}] & = \begin{bmatrix}
				1 & -5 & -\frac{5}{3} & \frac{2}{3} & | & 9 \\
				0 & 0 & 1 & 8 & | & 30 \\
				0 & 0 & 0 & 1 & | & 4
			\end{bmatrix}
		\end{align*}
	\item
		\begin{align*}
			R_1 & = r_1 + r_2 \\
			[A|\vec{b}] & = \begin{bmatrix}
				1 & -5 & -2 & -2 & | & -1 \\
				-2 & 10 & 3 & -4 & | & -28 \\
				5 & -25 & -2 & -1 & | & 15
			\end{bmatrix}
		\end{align*}
		\begin{align*}
			R_2 & = r_2 + 2r_1 \\
			[A|\vec{b}] & = \begin{bmatrix}
				1 & -5 & -2 & -2 & | & -1 \\
				0 & 0 & -1 & -8 & | & -30 \\
				5 & -25 & -2 & -1 & | & 15
			\end{bmatrix}
		\end{align*}
		\begin{align*}
			R_3 & = r_3 - 5r_1 \\
			[A|\vec{b}] & = \begin{bmatrix}
				1 & -5 & -2 & -2 & | & -1 \\
				0 & 0 & -1 & -8 & | & -30 \\
				0 & 0 & 8 & 9 & | & 20
			\end{bmatrix}
		\end{align*}
		\begin{align*}
			R_2 & = -r_2 \\
			[A|\vec{b}] & = \begin{bmatrix}
				1 & -5 & -2 & -2 & | & -1 \\
				0 & 0 & 1 & 8 & | & 30 \\
				0 & 0 & 8 & 9 & | & 20
			\end{bmatrix}
		\end{align*}
		\begin{align*}
			R_3 & = r_3 - 8r_2 \\
			[A|\vec{b}] & = \begin{bmatrix}
				1 & -5 & -2 & -2 & | & -1 \\
				0 & 0 & 1 & 8 & | & 30 \\
				0 & 0 & 0 & -55 & | & -220
			\end{bmatrix}
		\end{align*}
		\begin{align*}
			R_3 & = -\frac{1}{55}r_3 \\
			[A|\vec{b}] & = \begin{bmatrix}
				1 & -5 & -2 & -2 & | & -1 \\
				0 & 0 & 1 & 8 & | & 30 \\
				0 & 0 & 0 & 1 & | & 4
			\end{bmatrix}
		\end{align*}
\end{enumerate}

\end{document}
