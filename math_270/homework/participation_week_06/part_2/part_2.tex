\documentclass{article}

% Document extensibility %
%
% Disables native paragraph indentation
\usepackage{parskip} 
%
% Provides further bullet options for lists
\usepackage{enumitem}

% Mathematical symbol and statement packages %
%
% Necessary
\usepackage{amsmath}
\usepackage{amssymb}
%
% Extensive fraction notation
\usepackage{xfrac}
%
% Generic mathematical commands
% Notable: \degree, \celcius
\usepackage{gensymb}
%
% Variable vector notation (arrow above variable)
\usepackage{esvect}
%
% Multiline boxed equations
\usepackage{empheq}
%
% SI Unit
\usepackage{siunitx}

% Graphic packages %
%
% Diagrams and illustrations
\usepackage{tikz}
%
% Image insertion
\usepackage{graphicx}

% Document content %
%
% Change title of table of contents
% \renewcommand{\contentsname}{Title}

\begin{document}

% Command `\hr` to insert horizontal rules
\newcommand{\hr}{\par\noindent\rule{\textwidth}{0.4pt}}

% Command to box and center math equations
\newcommand{\bc}[1]{
	\begin{equation*}
		\begin{boxed}
			{#1}
		\end{boxed}
	\end{equation*}
}

% Command for single line equations with a condition
\newcommand{\cond}[2]{
	\ifmmode
		{#1} \quad {#2}
	\else
		$$ {#1} \quad {#2} $$
	\fi
}

\tableofcontents

\section{Part 2}
Now, let's continue with the same augmented matrix
$$
	[A|\vec{b}] = \begin{bmatrix}
		3 & -15 & -5 & 2 & | & 27 \\
		-2 & 10 & 3 & -4 & | & -28 \\
		5 & -25 & -2 & -1 & | & 15
	\end{bmatrix}
$$
\hr
\begin{align*}
	R_1 & = r_1 + r_2 \\
	[A|\vec{b}] & = \begin{bmatrix}
		1 & -5 & -2 & -2 & | & -1 \\
		-2 & 10 & 3 & -4 & | & -28 \\
		5 & -25 & -2 & -1 & | & 15
	\end{bmatrix}
\end{align*}
\begin{align*}
	R_2 & = r_2 + 2r_1 \\
	[A|\vec{b}] & = \begin{bmatrix}
		1 & -5 & -2 & -2 & | & -1 \\
		0 & 0 & -1 & -8 & | & -30 \\
		5 & -25 & -2 & -1 & | & 15
	\end{bmatrix}
\end{align*}
\begin{align*}
	R_3 & = r_3 - 5r_1 \\
	[A|\vec{b}] & = \begin{bmatrix}
		1 & -5 & -2 & -2 & | & -1 \\
		0 & 0 & -1 & -8 & | & -30 \\
		0 & 0 & 8 & 9 & | & 20
	\end{bmatrix}
\end{align*}
\begin{align*}
	R_2 & = -r_2 \\
	[A|\vec{b}] & = \begin{bmatrix}
		1 & -5 & -2 & -2 & | & -1 \\
		0 & 0 & 1 & 8 & | & 30 \\
		0 & 0 & 8 & 9 & | & 20
	\end{bmatrix}
\end{align*}
\begin{align*}
	R_3 & = r_3 - 8r_2 \\
	[A|\vec{b}] & = \begin{bmatrix}
		1 & -5 & -2 & -2 & | & -1 \\
		0 & 0 & 1 & 8 & | & 30 \\
		0 & 0 & 0 & -55 & | & -220
	\end{bmatrix}
\end{align*}
\begin{align*}
	R_3 & = -\frac{1}{55}r_3 \\
	[A|\vec{b}] & = \begin{bmatrix}
		1 & -5 & -2 & -2 & | & -1 \\
		0 & 0 & 1 & 8 & | & 30 \\
		0 & 0 & 0 & 1 & | & 4
	\end{bmatrix}
\end{align*}
\begin{align*}
	R_1 & = r_1 + 2r_2 \\
	[A|\vec{b}] & = \begin{bmatrix}
		1 & -5 & 0 & 14 & | & 59 \\
		0 & 0 & 1 & 8 & | & 30 \\
		0 & 0 & 0 & 1 & | & 4
	\end{bmatrix}
\end{align*}
\begin{align*}
	R_2 & = r_2 - 8r_3 \\
	[A|\vec{b}] & = \begin{bmatrix}
		1 & -5 & 0 & 14 & | & 59 \\
		0 & 0 & 1 & 0 & | & -2 \\
		0 & 0 & 0 & 1 & | & 4
	\end{bmatrix}
\end{align*}
\begin{align*}
	R_1 & = r_3 - 14r_3 \\
	[A|\vec{b}] & = \begin{bmatrix}
		1 & -5 & 0 & 0 & | & 3 \\
		0 & 0 & 1 & 0 & | & -2 \\
		0 & 0 & 0 & 1 & | & 4
	\end{bmatrix}
\end{align*}
\bc{\text{Reduced Row Echelon Form:} \quad
	[A|\vec{b}] = \begin{bmatrix}
		1 & -5 & 0 & 0 & | & 3 \\
		0 & 0 & 1 & 0 & | & -2 \\
		0 & 0 & 0 & 1 & | & 4
	\end{bmatrix}
}

\end{document}
