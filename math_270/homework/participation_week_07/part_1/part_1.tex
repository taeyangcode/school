\documentclass{article}

% Document extensibility %
%
% Disables native paragraph indentation
\usepackage{parskip} 
%
% Provides further bullet options for lists
\usepackage{enumitem}

% Mathematical symbol and statement packages %
%
% Necessary
\usepackage{amsmath}
\usepackage{amssymb}
%
% Extensive fraction notation
\usepackage{xfrac}
%
% Generic mathematical commands
% Notable: \degree, \celcius
\usepackage{gensymb}
%
% Variable vector notation (arrow above variable)
\usepackage{esvect}
%
% Multiline boxed equations
\usepackage{empheq}
%
% SI Unit
\usepackage{siunitx}
%
% More intuitive arrays/matrices
\usepackage{array}

% Graphic packages %
%
% Diagrams and illustrations
\usepackage{tikz}
%
% Image insertion
\usepackage{graphicx}

% Document content %
%
% Change title of table of contents
% \renewcommand{\contentsname}{Title}

\title{Week 07 Participation Assignment (1 of 3)}
\author{Corey Mostero}
\date{31 March 2023}

\begin{document}

% Command `\hr` to insert horizontal rules
\newcommand{\hr}{\par\noindent\rule{\textwidth}{0.4pt}}

% Command to box and center math equations
\newcommand{\bc}[1]{
	\begin{equation*}
		\begin{boxed}
			{#1}
		\end{boxed}
	\end{equation*}
}

% Command for single line equations with a condition
\newcommand{\cond}[2]{
	\ifmmode
		{#1} \quad {#2}
	\else
		$$ {#1} \quad {#2} $$
	\fi
}

\maketitle
\newpage

\tableofcontents

\section{Part 1}
Consider the following matrices:
\begin{align*}
	A & = \begin{bmatrix}
		1 & -3 & 2 \\
		5 & 4 & -1 \\
		-3 & 2 & -4
	\end{bmatrix} \\
	B & = \begin{bmatrix}
		7 & 1 & -4 \\
		2 & -3 & 2 \\
		-1 & -3 & 2
	\end{bmatrix} \\
	C & = \begin{bmatrix}
		6 & -6 \\
		6 & 7
	\end{bmatrix} \\
	D & = \begin{bmatrix}
		\lambda - 5 & -1 & -1 \\
		-1 & \lambda - 5 & -1 \\
		-1 & -1 & \lambda - 5
	\end{bmatrix}
\end{align*}
Perform the following calculations:
\begin{enumerate}[label=\textbf{\arabic*).}]
	\item $ 2A + B $
	\item $ A \cdot B $
	\item $ B \cdot A $
	\item $ A^2 $
\end{enumerate}

\subsection{$ 2A + B $}
\begin{align*}
	2A & = \begin{bmatrix}
		2 & -6 & 4 \\
		10 & 8 & -2 \\
		-6 & 4 & -8
	\end{bmatrix} \\
	2A + B & = \begin{bmatrix}
		9 & -5 & 0 \\
		12 & 5 & 0 \\
		-7 & 1 & -6
	\end{bmatrix}
\end{align*}
\bc{
	2A + B = \begin{bmatrix}
		9 & -5 & 0 \\
		12 & 5 & 0 \\
		-7 & 1 & -6
	\end{bmatrix}
}

\subsection{$ A \cdot B $}
\begin{align*}
	A \cdot B & = \begin{bmatrix}
		(1 \cdot 7) + (-3 \cdot 2) + (2 \cdot -1) & (1 \cdot 1) + (-3 \cdot -3) + (2 \cdot -3) & (1 \cdot -4) + (-3 \cdot 2) + (2 \cdot 2) \\
		(5 \cdot 7) + (4 \cdot 2) + (-1 \cdot -1) & (5 \cdot 1) + (4 \cdot -3) + (-1 \cdot -3) & (5 \cdot -4) + (4 \cdot 2) + (-1 \cdot 2) \\
		(-3 \cdot 7) + (2 \cdot 2) + (-4 \cdot -1) & (-3 \cdot 1) + (2 \cdot -3) + (-4 \cdot -3) & (-3 \cdot -4) + (2 \cdot 2) + (-4 \cdot 2)
	\end{bmatrix} \\
	A \cdot B & = \begin{bmatrix}
		-1 & 4 & -6 \\
		44 & -4 & -14 \\
		-13 & 3 & 8
	\end{bmatrix}
\end{align*}
\bc{
	A \cdot B = \begin{bmatrix}
		-1 & 4 & -6 \\
		44 & -4 & -14 \\
		-13 & 3 & 8
	\end{bmatrix}
}

\subsection{$ B \cdot A $}
\begin{align*}
	B \cdot A & = \begin{bmatrix}
		(7 \cdot 1) + (1 \cdot 5) + (-4 \cdot -3) & (7 \cdot -3) + (1 \cdot 4) + (-4 \cdot 2) & (7 \cdot 2) + (1 \cdot -1) + (-4 \cdot -4) \\
		(2 \cdot 1) + (-3 \cdot 5) + (2 \cdot -3) & (2 \cdot -3) + (-3 \cdot 4) + (2 \cdot 2) & (2 \cdot 2) + (-3 \cdot -1) + (2 \cdot -4) \\
		(-1 \cdot 1) + (-3 \cdot 5) + (2 \cdot -3) & (-1 \cdot -3) + (-3 \cdot 4) + (2 \cdot 2) & (-1 \cdot 2) + (-3 \cdot -1) + (2 \cdot -4)
	\end{bmatrix} \\
	B \cdot A & = \begin{bmatrix}
		24 & -25 & 29 \\
		-19 & -14 & -1 \\
		-22 & -5 & -7
	\end{bmatrix}
\end{align*}
\bc{
	B \cdot A = \begin{bmatrix}
		24 & -25 & 29 \\
		-19 & -14 & -1 \\
		-22 & -5 & -7
	\end{bmatrix}
}

\subsection{$ A^2 $}
\begin{align*}
	A^2 & = \begin{bmatrix}
		(1 \cdot 1) + (-3 \cdot 5) + (2 \cdot -3) & (1 \cdot -3) + (-3 \cdot 4) + (2 \cdot 2) & (1 \cdot 2) + (-3 \cdot -1) + (2 \cdot -4) \\
		(5 \cdot 1) + (4 \cdot 5) + (-1 \cdot -3) & (5 \cdot -3) + (4 \cdot 4) + (-1 \cdot 2) & (5 \cdot 2) + (4 \cdot -1) + (-1 \cdot -4) \\
		(-3 \cdot 1) + (2 \cdot 5) + (-4 \cdot -3) & (-3 \cdot -3) + (2 \cdot 4) + (-4 \cdot 2) & (-3 \cdot 2) + (2 \cdot -1) + (-4 \cdot -4)
	\end{bmatrix} \\
	A^2 & = \begin{bmatrix}
		-20 & -11 & -3 \\
		28 & -1 & 10 \\
		19 & 9 & 8
	\end{bmatrix}
\end{align*}
\bc{
	A^2 = \begin{bmatrix}
		-20 & -11 & -3 \\
		28 & -1 & 10 \\
		19 & 9 & 8
	\end{bmatrix}
}

\end{document}
