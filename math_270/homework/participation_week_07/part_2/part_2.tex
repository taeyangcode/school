\documentclass{article}

% Document extensibility %
%
% Disables native paragraph indentation
\usepackage{parskip} 
%
% Provides further bullet options for lists
\usepackage{enumitem}

% Mathematical symbol and statement packages %
%
% Necessary
\usepackage{amsmath}
\usepackage{amssymb}
%
% Extensive fraction notation
\usepackage{xfrac}
%
% Generic mathematical commands
% Notable: \degree, \celcius
\usepackage{gensymb}
%
% Variable vector notation (arrow above variable)
\usepackage{esvect}
%
% Multiline boxed equations
\usepackage{empheq}
%
% SI Unit
\usepackage{siunitx}
%
% More intuitive arrays/matrices
\usepackage{array}

% Graphic packages %
%
% Diagrams and illustrations
\usepackage{tikz}
%
% Image insertion
\usepackage{graphicx}

% Document content %
%
% Change title of table of contents
% \renewcommand{\contentsname}{Title}

\title{Week 07 Participation Assignment (2 of 3)}
\author{Corey Mostero}
\date{31 March 2023}

\begin{document}

% Command `\hr` to insert horizontal rules
\newcommand{\hr}{\par\noindent\rule{\textwidth}{0.4pt}}

% Command to box and center math equations
\newcommand{\bc}[1]{
	\begin{equation*}
		\begin{boxed}
			{#1}
		\end{boxed}
	\end{equation*}
}

% Command for single line equations with a condition
\newcommand{\cond}[2]{
	\ifmmode
		{#1} \quad {#2}
	\else
		$$ {#1} \quad {#2} $$
	\fi
}

\maketitle
\newpage

\tableofcontents

\section{Part 2}
Consider the following matrices:
\begin{align*}
	A & = \begin{bmatrix}
		1 & -3 & 2 \\
		5 & 4 & -1 \\
		-3 & 2 & -4
	\end{bmatrix} \\
	B & = \begin{bmatrix}
		7 & 1 & -4 \\
		2 & -3 & 2 \\
		-1 & -3 & 2
	\end{bmatrix} \\
	C & = \begin{bmatrix}
		6 & -6 \\
		6 & 7
	\end{bmatrix} \\
	D & = \begin{bmatrix}
		\lambda - 5 & -1 & -1 \\
		-1 & \lambda - 5 & -1 \\
		-1 & -1 & \lambda - 5
	\end{bmatrix} \\
	E & = \begin{bmatrix}
		3 & 2 & -4 & 1 \\
		-2 & 1 & 0 & 3 \\
		5 & 3 & 2 & -1 \\
		2 & -5 & 3 & 6
	\end{bmatrix}
\end{align*}
Perform the following calculations:
\begin{enumerate}[label=\textbf{\arabic*).}]
	\item $ \det(C) $ using the formula for two by two matrix.
	\item $ \det(D) $ using the co-factor expansion.
	\item $ \det(E) $ using row/column operations.
\end{enumerate}

Find $ \det(E) $ using elementary row/column operations

\subsection{$ \det(C) $}
\begin{align*}
	\det(C) & = (6)(7) - (-6)(6) \\
	\det(C) & = 78
\end{align*}
\bc{
	\det(C) = 78
}

\subsection{$ \det(D) $}
\begin{align*}
	\det(D) & = 
		(\lambda - 5) \det \begin{bmatrix} \lambda - 5 & -1 \\ -1 & \lambda - 5 \end{bmatrix} \\
		& - (-1) \det \begin{bmatrix} -1 & -1 \\ -1 & \lambda - 5 \end{bmatrix} \\
		& + (-1) \det \begin{bmatrix} -1 & \lambda - 5 \\ -1 & -1 \end{bmatrix} \\
	\det(D) & = (\lambda - 5)(\lambda^2 - 10\lambda + 24) + (1)(-\lambda + 4) + (-1)(\lambda - 4) \\
	\det(D) & = \lambda^3 - 15\lambda^2 + 72\lambda - 112
\end{align*}
\bc{
	\det(D) = \lambda^3 - 15\lambda^2 + 72\lambda - 112
}

\subsection{$ \det(E) $}
\begin{align*}
	E_2 & = E_2 + \frac{2}{3}E_1 \\
	E_3 & = E_3 - \frac{5}{3}E_1 \\
	E_4 & = E_4 - \frac{2}{3}E_1 \\
	E & = \begin{bmatrix}
		3 & 2 & -4 & 1 \\
		0 & \frac{7}{3} & -\frac{8}{3} & \frac{11}{3} \\
		0 & -\frac{1}{3} & \frac{26}{3} & -\frac{8}{3} \\
		0 & -\frac{19}{3} & \frac{17}{3} & \frac{16}{3}
	\end{bmatrix}
\end{align*}
\begin{align*}
	E_3 & = E_3 + \frac{1}{7}E_2 \\
	E_4 & = E_4 + \frac{19}{7}E_2 \\
	E & = \begin{bmatrix}
		3 & 2 & -4 & 1 \\
		0 & \frac{7}{3} & -\frac{8}{3} & \frac{11}{3} \\
		0 & 0 & \frac{58}{7} & -\frac{15}{7} \\
		0 & 0 & -\frac{11}{7} & \frac{107}{7}
	\end{bmatrix}
\end{align*}
\begin{align*}
	E_4 & = E_4 + \frac{11}{58}E_3 \\
	E & = \begin{bmatrix}
		3 & 2 & -4 & 1 \\
		0 & \frac{7}{3} & -\frac{8}{3} & \frac{11}{3} \\
		0 & 0 & \frac{58}{7} & -\frac{15}{7} \\
		0 & 0 & 0 & \frac{863}{58}
	\end{bmatrix}
\end{align*}
\begin{align*}
	\det(E) & = (3) \left( \frac{7}{3} \right) \left( \frac{58}{7} \right) \left( \frac{863}{58} \right) \\
	\det(E) & = 863
\end{align*}
\bc{
	\det(E) = 863
}

\end{document}
