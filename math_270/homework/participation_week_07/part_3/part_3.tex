\documentclass{article}

% Document extensibility %
%
% Disables native paragraph indentation
\usepackage{parskip} 
%
% Provides further bullet options for lists
\usepackage{enumitem}

% Mathematical symbol and statement packages %
%
% Necessary
\usepackage{amsmath}
\usepackage{amssymb}
%
% Extensive fraction notation
\usepackage{xfrac}
%
% Generic mathematical commands
% Notable: \degree, \celcius
\usepackage{gensymb}
%
% Variable vector notation (arrow above variable)
\usepackage{esvect}
%
% Multiline boxed equations
\usepackage{empheq}
%
% SI Unit
\usepackage{siunitx}
%
% More intuitive arrays/matrices
\usepackage{array}

% Graphic packages %
%
% Diagrams and illustrations
\usepackage{tikz}
%
% Image insertion
\usepackage{graphicx}

% Document content %
%
% Change title of table of contents
% \renewcommand{\contentsname}{Title}

\title{Week 07 Participation Assignment (3 of 3)}
\author{Corey Mostero}
\date{31 March 2023}

\begin{document}

% Command `\hr` to insert horizontal rules
\newcommand{\hr}{\par\noindent\rule{\textwidth}{0.4pt}}

% Command to box and center math equations
\newcommand{\bc}[1]{
	\begin{equation*}
		\begin{boxed}
			{#1}
		\end{boxed}
	\end{equation*}
}

% Command for single line equations with a condition
\newcommand{\cond}[2]{
	\ifmmode
		{#1} \quad {#2}
	\else
		$$ {#1} \quad {#2} $$
	\fi
}

\maketitle
\newpage

\tableofcontents

\section{Part 3}
Consider the matrix
$$
	\begin{bmatrix}
		1 & -3 & 2 \\
		5 & 4 & -1 \\
		-3 & 2 & -4
	\end{bmatrix}
$$
. Find the inverse of $ A $ using the method indicated:
\begin{enumerate}[label=\textbf{\arabic*)}]
	\item Use elementary row operations on $ [A|I] $ to convert it into $ [I|A^{-1}] $.
	\item Use the adjoint matrix.
\end{enumerate}

\subsection{1)}
\begin{align*}
	[I|A^{-1}] & = \left[ \begin{array}{ccc|ccc}
		1 & -3 & 2 & 1 & 0 & 0 \\
		5 & 4 & -1 & 0 & 1 & 0 \\
		-3 & 2 & -4 & 0 & 0 & 1
	\end{array} \right]
\end{align*}
\begin{align*}
	A_2 & = A_2 - 5A_1 \\
	A_3 & = A_3 + 3A_1 \\
	[I|A^{-1}] & = \left[ \begin{array}{ccc|ccc}
		1 & -3 & 2 & 1 & 0 & 0 \\
		0 & 19 & -11 & -5 & 1 & 0 \\
		0 & -7 & 2 & 3 & 0 & 1
	\end{array} \right]
\end{align*}
\begin{align*}
	A_3 & = A_3 + \frac{7}{19}A_2 \\
	[I|A^{-1}] & = \left[ \begin{array}{ccc|ccc}
		1 & -3 & 2 & 1 & 0 & 0 \\
		0 & 19 & -11 & -5 & 1 & 0 \\
		0 & 0 & -\frac{39}{19} & \frac{22}{19} & \frac{7}{19} & 1
	\end{array} \right]
\end{align*}
\begin{align*}
	A_1 & = A_1 + \frac{3}{19}A_2 \\
	[I|A^{-1}] & = \left[ \begin{array}{ccc|ccc}
		1 & 0 & \frac{5}{19} & \frac{4}{19} & \frac{3}{19} & 0 \\
		0 & 19 & -11 & -5 & 1 & 0 \\
		0 & 0 & -\frac{39}{19} & \frac{22}{19} & \frac{7}{19} & 1
	\end{array} \right]
\end{align*}
\begin{align*}
	A_1 & = A_1 + \frac{5}{39}A_3 \\
	[I|A^{-1}] & = \left[ \begin{array}{ccc|ccc}
		1 & 0 & 0 & \frac{14}{39} & \frac{8}{39} & \frac{5}{39} \\
		0 & 19 & -11 & -5 & 1 & 0 \\
		0 & 0 & -\frac{39}{19} & \frac{22}{19} & \frac{7}{19} & 1
	\end{array} \right]
\end{align*}
\begin{align*}
	A_2 & = \frac{1}{19}A_2 \\
	[I|A^{-1}] & = \left[ \begin{array}{ccc|ccc}
		1 & 0 & 0 & \frac{14}{39} & \frac{8}{39} & \frac{5}{39} \\
		0 & 1 & -\frac{11}{19} & -\frac{5}{19} & \frac{1}{19} & 0 \\
		0 & 0 & -\frac{39}{19} & \frac{22}{19} & \frac{7}{19} & 1
	\end{array} \right]
\end{align*}
\begin{align*}
	A_2 & = A_2 - \frac{11}{39}A_3 \\
	[I|A^{-1}] & = \left[ \begin{array}{ccc|ccc}
		1 & 0 & 0 & \frac{14}{39} & \frac{8}{39} & \frac{5}{39} \\
		0 & 1 & 0 & -\frac{23}{39} & -\frac{2}{39} & -\frac{11}{39} \\
		0 & 0 & -\frac{39}{19} & \frac{22}{19} & \frac{7}{19} & 1
	\end{array} \right]
\end{align*}
\begin{align*}
	A_3 & = -\frac{19}{39}A_3 \\
	[I|A^{-1}] & = \left[ \begin{array}{ccc|ccc}
		1 & 0 & 0 & \frac{14}{39} & \frac{8}{39} & \frac{5}{39} \\
		0 & 1 & 0 & -\frac{23}{39} & -\frac{2}{39} & -\frac{11}{39} \\
		0 & 0 & 1 & -\frac{22}{39} & -\frac{7}{39} & -\frac{19}{39}
	\end{array} \right]
\end{align*}
\bc{
	[I|A^{-1}] = \left[ \begin{array}{ccc|ccc}
		1 & 0 & 0 & \frac{14}{39} & \frac{8}{39} & \frac{5}{39} \\
		0 & 1 & 0 & -\frac{23}{39} & -\frac{2}{39} & -\frac{11}{39} \\
		0 & 0 & 1 & -\frac{22}{39} & -\frac{7}{39} & -\frac{19}{39}
	\end{array} \right]
}

\subsection{2)}
\begin{align*}
	\det(A) & = (1)(4 \cdot -4 - -1 \cdot 2) - (-3)(5 \cdot -4 - -1 \cdot -3) + (2)(5 \cdot 2 - 4 \cdot -3) \\
	\det(A) & = -39
\end{align*}
\begin{align*}
	C & = \begin{bmatrix}
		\det \begin{bmatrix} 4 & -1 \\ 2 & -4 \end{bmatrix} & \det \begin{bmatrix} 5 & -1 \\ -3 & -4 \end{bmatrix} & \det \begin{bmatrix} 5 & 4 \\ -3 & 2 \end{bmatrix} \\
		\det \begin{bmatrix} -3 & 2 \\ 2 & -4 \end{bmatrix} & \det \begin{bmatrix} 1 & 2 \\ -3 & -4 \end{bmatrix} & \det \begin{bmatrix} 1 & -3 \\ -3 & 2 \end{bmatrix} \\
		\det \begin{bmatrix} -3 & 2 \\ 4 & -1 \end{bmatrix} & \det \begin{bmatrix} 1 & 2 \\ 5 & -1 \end{bmatrix} & \det \begin{bmatrix} 1 & -3 \\ 5 & 4 \end{bmatrix}
	\end{bmatrix} \\
	C & = \begin{bmatrix}
		-14 & -23 & 22 \\
		8 & 2 & 7 \\
		-5 & -11 & 19
	\end{bmatrix} \\
	\text{Adj} & = \begin{bmatrix}
		-14 & -8 & -5 \\
		23 & 2 & 11 \\
		22 & 7 & 19
	\end{bmatrix}
\end{align*}
\begin{align*}
	A^{-1} & = \frac{1}{-39}\begin{bmatrix}
		-14 & -8 & -5 \\
		23 & 2 & 11 \\
		22 & 7 & 19
	\end{bmatrix} \\
	A^{-1} & = \begin{bmatrix}
		\frac{14}{39} & \frac{8}{39} & \frac{5}{39} \\
		-\frac{23}{39} & -\frac{2}{39} & -\frac{11}{39} \\
		-\frac{22}{39} & -\frac{7}{39} & -\frac{19}{39}
	\end{bmatrix}
\end{align*}
\bc{
	A^{-1} = \begin{bmatrix}
		\frac{14}{39} & \frac{8}{39} & \frac{5}{39} \\
		-\frac{23}{39} & -\frac{2}{39} & -\frac{11}{39} \\
		-\frac{22}{39} & -\frac{7}{39} & -\frac{19}{39}
	\end{bmatrix}
}

\end{document}
