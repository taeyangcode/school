\documentclass{article}

% Document extensibility %
%
% Disables native paragraph indentation
\usepackage{parskip} 
%
% Provides further bullet options for lists
\usepackage{enumitem}

% Mathematical symbol and statement packages %
%
% Necessary
\usepackage{amsmath}
\usepackage{amssymb}
%
% Extensive fraction notation
\usepackage{xfrac}
%
% Generic mathematical commands
% Notable: \degree, \celcius
\usepackage{gensymb}
%
% Variable vector notation (arrow above variable)
\usepackage{esvect}
%
% Multiline boxed equations
\usepackage{empheq}
%
% SI Unit
\usepackage{siunitx}
%
% More intuitive arrays/matrices
\usepackage{array}

% Graphic packages %
%
% Diagrams and illustrations
\usepackage{tikz}
%
% Image insertion
\usepackage{graphicx}

% Document content %
%
% Change title of table of contents
% \renewcommand{\contentsname}{Title}

\begin{document}
\title{Week 08 Participation Part 1}
\date{07 April 2023}
\author{Corey Mostero}

% Command `\hr` to insert horizontal rules
\newcommand{\hr}{\par\noindent\rule{\textwidth}{0.4pt}}

% Command to box and center math equations
\newcommand{\bc}[1]{
	\begin{equation*}
		\begin{boxed}
			{#1}
		\end{boxed}
	\end{equation*}
}

% Command for single line equations with a condition
\newcommand{\cond}[2]{
	\ifmmode
		{#1} \quad {#2}
	\else
		$$ {#1} \quad {#2} $$
	\fi
}

\maketitle
\newpage

\tableofcontents

\section{Part 1}
Consider the set
$$ S = { \vec{v}_1, \vec{v}_2, \vec{v}_3 }, $$
where $ \vec{v}_1 = <1,1,0> $, $ \vec{v}_2 = <0,1,1> $, $ \vec{v}_3 = <2,-1,-3> $.

Verify the following:
\begin{enumerate}[label=\textbf{\alph*)}]
	\item None of the pairs are parallel. That is $ v_1 $ is not a multiple of $ v_2 $ nor $ v_3 $, $ v_2 $ is not a multiple of $ v_1 $ nor $ v_3 $, $ v_3 $ is not a multiple of $ v_1 $ nor $ v_2 $.
	\item Show that the set is linear dependent by finding a none zero solution to the dependence test equation.
\end{enumerate}

\subsection{a)}
Find if $ \vec{v}_1 \parallel \vec{v}_2 $:
\begin{align*}
	\vec{v}_1 \times \vec{v}_2 & = <1 \cdot 1 - 0 \cdot 1, -(1 \cdot 1 - 0 \cdot 0), 1 \cdot 1 - 1 \cdot 0> \\
	\vec{v}_1 \times \vec{v}_2 & = <1, -1, 1>
\end{align*}
\bc{<1, -1, 1> \neq <0, 0, 0> \therefore \vec{v}_1 \nparallel \vec{v}_2}
Find if $ \vec{v}_1 \parallel \vec{v}_3 $:
\begin{align*}
	\vec{v}_1 \times \vec{v}_3 & = <1 \cdot -3 - 0 \cdot -1, -(1 \cdot -3 - 0 \cdot 2), 1 \cdot -1 - 1 \cdot 2> \\
	\vec{v}_1 \times \vec{v}_3 & = <-3, 3, -3>
\end{align*}
\bc{<-3, 3, -3> \neq <0, 0, 0> \therefore \vec{v}_1 \nparallel \vec{v}_3}
Find if $ \vec{v}_2 \parallel \vec{v}_3 $:
\begin{align*}
	\vec{v}_2 \times \vec{v}_3 & = <1 \cdot -3 - 1 \cdot -1, -(0 \cdot -3 - 1 \cdot 2), 0 \cdot -1 - 1 \cdot 2> \\
	\vec{v}_2 \times \vec{v}_3 & = <-2, 2, -2>
\end{align*}
\bc{<-2, 2, -2> \neq <0, 0, 0> \therefore \vec{v}_2 \nparallel \vec{v}_3}

\subsection{b)}
\begin{align*}
	A & = \left[ \begin{array}{ccc|c}
		1 & 0 & 2 & 0 \\
		1 & 1 & -1 & 0 \\
		0 & 1 & -3 & 0
	\end{array} \right]
\end{align*}
\begin{align*}
	A_2 & = A_2 - A_1 \\
	A & = \left[ \begin{array}{ccc|c}
		1 & 0 & 2 & 0 \\
		0 & 1 & -3 & 0 \\
		0 & 1 & -3 & 0
	\end{array} \right]
\end{align*}
\begin{align*}
	A_3 & = A_3 - A_2 \\
	A & = \left[ \begin{array}{ccc|c}
		1 & 0 & 2 & 0 \\
		0 & 1 & -3 & 0 \\
		0 & 0 & 0 & 0
	\end{array} \right]
\end{align*}
\begin{align*}
	a + 2c & = 0 \\
	b - 3c & = 0 \\
	a & = -2c \\
	b & = 3c
\end{align*}
\bc{-2c\vec{v}_1 + 3c\vec{v}_2 + \vec{v}_3 = 0}

\end{document}
