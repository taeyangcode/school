\documentclass{article}

% Document extensibility %
%
% Disables native paragraph indentation
\usepackage{parskip} 
%
% Provides further bullet options for lists
\usepackage{enumitem}

% Mathematical symbol and statement packages %
%
% Necessary
\usepackage{amsmath}
\usepackage{amssymb}
%
% Extensive fraction notation
\usepackage{xfrac}
%
% Generic mathematical commands
% Notable: \degree, \celcius
\usepackage{gensymb}
%
% Variable vector notation (arrow above variable)
\usepackage{esvect}
%
% Multiline boxed equations
\usepackage{empheq}
%
% SI Unit
\usepackage{siunitx}
%
% More intuitive arrays/matrices
\usepackage{array}

% Graphic packages %
%
% Diagrams and illustrations
\usepackage{tikz}
%
% Image insertion
\usepackage{graphicx}

% Document content %
%
% Change title of table of contents
% \renewcommand{\contentsname}{Title}

\begin{document}
\title{Week 08 Participation Part 2}
\date{07 April 2023}
\author{Corey Mostero}

% Command `\hr` to insert horizontal rules
\newcommand{\hr}{\par\noindent\rule{\textwidth}{0.4pt}}

% Command to box and center math equations
\newcommand{\bc}[1]{
	\begin{equation*}
		\begin{boxed}
			{#1}
		\end{boxed}
	\end{equation*}
}

% Command for single line equations with a condition
\newcommand{\cond}[2]{
	\ifmmode
		{#1} \quad {#2}
	\else
		$$ {#1} \quad {#2} $$
	\fi
}

\newcommand{\R}[1]{\mathbb{R}^{#1}}

\maketitle
\newpage

\tableofcontents

\section{Part 2}
\begin{enumerate}[label=\textbf{\arabic*.}]
	\item $ W $ is the set of all vectors in $ \R{3} $ such that $ x_3 = 0 $.
	\item $ W $ is the set of all vectors in $ \R{3} $ such that $ x_1 = 5x_2 $.
	\item $ W $ is the set of all vectors in $ \R{3} $ such that $ x_2 = 1 $.
	\item $ W $ is the set of all vectors in $ \R{3} $ such that $ x_1 + x_2 + x_3 = 1 $.
	\item $ W $ is the set of all vectors in $ \R{4} $ such that $ x_1 + 2x_2 + 3x_3 + 4x_4 = 0 $.
	\item $ W $ is the set of all vectors in $ \R{4} $ such that $ x_1 = 3x_3 $ and $ x_2 = 4x_4 $.
	\item $ W $ is the set of all vectors in $ \R{2} $ such that $ \|x_1\| = \|x_2 \|$.
	\item $ W $ is the set of all vectors in $ \R{2} $ such that $ (x_1)^2 + (x_2)^2 = 0 $.
	\item $ W $ is the set of all vectors in $ \R{2} $ such that $ (x_1)^2 + (x_2)^2 = 1 $.
	\item $ W $ is the set of all vectors in $ \R{2} $ such that $ \|x_1\| + \|x_2\| = 1 $.
	\item $ W $ is the set of all vectors in $ \R{4} $ such that $ x_1 + x_2 = x_3 + x_4 $.
	\item $ W $ is the set of all vectors in $ \R{4} $ such that $ x_1x_2 = x_3x_4 $.
	\item $ W $ is the set of all vectors in $ \R{4} $ such that $ x_1x_2x_3x_4 = 0 $.
	\item $ W $ is the set of all vectors in $ \R{4} $ whose components are all nonzero.
\end{enumerate}
For the above definitions, give an example of an element that belongs to the set $ W $ and an example of elements that does not belong to the set $ W $.

\subsection{1.}
\begin{enumerate}[label=\textbf{\alph*.}]
	\item $ \vec{v}_1 = <1, 1, 0> $
	\item
		\begin{enumerate}[label=\textbf{\arabic*.}]
			\item $ \vec{v}_2 = <1, 1, 1> $
			\item $ \vec{v}_3 = <2, 2, 2> $
		\end{enumerate}
\end{enumerate}

\subsection{2.}
\begin{enumerate}[label=\textbf{\alph*.}]
	\item $ \vec{v}_1 = <5, 1, 1> $
	\item
		\begin{enumerate}[label=\textbf{\arabic*.}]
			\item $ \vec{v}_2 = <6, 1, 1> $
			\item $ \vec{v}_3 = <7, 1, 1> $
		\end{enumerate}
\end{enumerate}

\subsection{3.}
\begin{enumerate}[label=\textbf{\alph*.}]
	\item $ \vec{v}_1 = <1, 1, 1> $
	\item
		\begin{enumerate}[label=\textbf{\arabic*.}]
			\item $ \vec{v}_2 = <2, 2, 2> $
			\item $ \vec{v}_3 = <3, 3, 3> $
		\end{enumerate}
\end{enumerate}

\subsection{4.}
\begin{enumerate}[label=\textbf{\alph*.}]
	\item $ \vec{v}_1 = <1, 0, 0> $
	\item
		\begin{enumerate}[label=\textbf{\arabic*.}]
			\item $ \vec{v}_2 = <1, 1, 1> $
			\item $ \vec{v}_3 = <2, 2, 2> $
		\end{enumerate}
\end{enumerate}

\subsection{5.}
\begin{enumerate}[label=\textbf{\alph*.}]
	\item $ \vec{v}_1 = <0, 0, 0, 0> $
	\item
		\begin{enumerate}[label=\textbf{\arabic*.}]
			\item $ \vec{v}_2 = <1, 1, 1, 1> $
			\item $ \vec{v}_3 = <2, 2, 2, 2> $
		\end{enumerate}
\end{enumerate}

\subsection{6.}
\begin{enumerate}[label=\textbf{\alph*.}]
	\item $ \vec{v}_1 = <3, 4, 1, 1> $
	\item
		\begin{enumerate}[label=\textbf{\arabic*.}]
			\item $ \vec{v}_2 = <1, 1, 1, 1> $
			\item $ \vec{v}_3 = <2, 2, 2, 2> $
		\end{enumerate}
\end{enumerate}

\subsection{7.}
\begin{enumerate}[label=\textbf{\alph*.}]
	\item $ \vec{v}_1 = <1, 1> $
	\item
		\begin{enumerate}[label=\textbf{\arabic*.}]
			\item $ \vec{v}_2 = <1, 2> $
			\item $ \vec{v}_3 = <2, 1> $
		\end{enumerate}
\end{enumerate}

\subsection{8.}
\begin{enumerate}[label=\textbf{\alph*.}]
	\item $ \vec{v}_1 = <0, 0> $
	\item
		\begin{enumerate}[label=\textbf{\arabic*.}]
			\item $ \vec{v}_2 = <1, 1> $
			\item $ \vec{v}_3 = <2, 2> $
		\end{enumerate}
\end{enumerate}

\subsection{9.}
\begin{enumerate}[label=\textbf{\alph*.}]
	\item $ \vec{v}_1 = <1, 0> $
	\item
		\begin{enumerate}[label=\textbf{\arabic*.}]
			\item $ \vec{v}_2 = <1, 1> $
			\item $ \vec{v}_3 = <2, 2> $
		\end{enumerate}
\end{enumerate}

\subsection{10.}
\begin{enumerate}[label=\textbf{\alph*.}]
	\item $ \vec{v}_1 = <1, 0> $
	\item
		\begin{enumerate}[label=\textbf{\arabic*.}]
			\item $ \vec{v}_2 = <1, 1> $
			\item $ \vec{v}_3 = <2, 2> $
		\end{enumerate}
\end{enumerate}

\subsection{11.}
\begin{enumerate}[label=\textbf{\alph*.}]
	\item $ \vec{v}_1 = <1, 1, 1, 1> $
	\item
		\begin{enumerate}[label=\textbf{\arabic*.}]
			\item $ \vec{v}_2 = <1, 1, 2, 2> $
			\item $ \vec{v}_3 = <2, 2, 1, 1> $
		\end{enumerate}
\end{enumerate}

\subsection{12.}
\begin{enumerate}[label=\textbf{\alph*.}]
	\item $ \vec{v}_1 = <1, 1, 1, 1> $
	\item
		\begin{enumerate}[label=\textbf{\arabic*.}]
			\item $ \vec{v}_1 = <1, 1, 2, 2> $
			\item $ \vec{v}_2 = <2, 2, 1, 1> $
		\end{enumerate}
\end{enumerate}

\subsection{13.}
\begin{enumerate}[label=\textbf{\alph*.}]
	\item $ \vec{v}_1 = <0, 0, 0, 1> $
	\item
		\begin{enumerate}[label=\textbf{\arabic*.}]
			\item $ \vec{v}_2 = <1, 1, 1, 1> $
			\item $ \vec{v}_3 = <2, 2, 2, 2> $
		\end{enumerate}
\end{enumerate}

\subsection{14.}
\begin{enumerate}[label=\textbf{\alph*.}]
	\item $ \vec{v}_1 = <1, 1, 1, 1> $
	\item
		\begin{enumerate}[label=\textbf{\arabic*.}]
			\item $ \vec{v}_2 = <1, 0, 0, 0> $
			\item $ \vec{v}_3 = <0, 0, 0, 1> $
		\end{enumerate}
\end{enumerate}

\end{document}
