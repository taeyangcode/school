\documentclass{article}

% Document extensibility %
%
% Disables native paragraph indentation
\usepackage{parskip} 
%
% Provides further bullet options for lists
\usepackage{enumitem}

% Mathematical symbol and statement packages %
%
% Necessary
\usepackage{amsmath}
\usepackage{amssymb}
%
% Extensive fraction notation
\usepackage{xfrac}
%
% Generic mathematical commands
% Notable: \degree, \celcius
\usepackage{gensymb}
%
% Variable vector notation (arrow above variable)
\usepackage{esvect}
%
% Multiline boxed equations
\usepackage{empheq}
%
% SI Unit
\usepackage{siunitx}
%
% More intuitive arrays/matrices
\usepackage{array}
%
%
\usepackage{systeme}

% Graphic packages %
%
% Diagrams and illustrations
\usepackage{tikz}
%
% Image insertion
\usepackage{graphicx}

% Document content %
%
% Change title of table of contents
% \renewcommand{\contentsname}{Title}

\title{Week 09 Participation Assignment (1 of 1)}
\date{21 April 2023}
\author{Corey Mostero - 2566652}

\begin{document}

% Command `\hr` to insert horizontal rules
\newcommand{\hr}{\par\noindent\rule{\textwidth}{0.4pt}}

% Command to box and center math equations
\newcommand{\bc}[1]{
	\begin{equation*}
		\begin{boxed}
			{#1}
		\end{boxed}
	\end{equation*}
}

% Command for single line equations with a condition
\newcommand{\cond}[2]{
	\ifmmode
		{#1} \quad {#2}
	\else
		$$ {#1} \quad {#2} $$
	\fi
}

\maketitle
\newpage

\tableofcontents

\section{Part 1}
In this participation assignment, we would like to apply the term "linear combination" to two different types of question and observe the difference.
\begin{enumerate}[label = \textbf{\arabic*)}]
\item
	Let $ S = {\vec{v}_1, \vec{v}_2, \vec{v}_3, \vec{v}_4} \subseteq \mathbb{R}^6 $, where $ \vec{v}_1 = \langle 3, 4, 1, -6, -1, 9 \rangle, \vec{v}_2 = \langle 1, -3, -2, 1, 1, 2 \rangle, \vec{v}_3 = \langle -4, -5, -1, 9, -3, 18 \rangle, \vec{v}_4 = \langle 0, 2, 1, -2, 2, -18 \rangle $.

	To show that the set $ S $ is linearly independent, we must begin with the dependence test equation (set the linear combination of the vectors equal to the zero vector). Here are what you need to work on
	\begin{enumerate}[label = \alph*)]
		\item Write the dependence test equation as a \textbf{\underline{vector equation}}.
		\item Rewrite the dependence test equation as a \textbf{\underline{matrix equation}}.
		\item Begin with your vector equation and combine the linear combination as one vector (with variables), then write a system of linear equations by equating the vector on the left hand side and the right hand side.
		\item With your matrix equation, write the coefficient matrix and augmented matrix separately.
	\end{enumerate}
\item
	Let $ S = {\vec{v}_1, \vec{v}_2, \vec{v}_3, \vec{v}_4} \subseteq \mathbb{R}^6 $, where $ \vec{v}_1 = \langle 3, 4, 1, -6, -1, 9 \rangle, \vec{v}_2 = \langle 1, -3, -2, 1, 1, 2 \rangle, \vec{v}_3 = \langle -4, -5, -1, 9, -3, 18 \rangle, \vec{v}_4 = \langle 0, 2, 1, -2, 2, -18 \rangle $.

	Now given another vector $ \vec{b} = \langle 1, 8, 4, -6, -1, -5 \rangle $, In order to determine if vector $ \vec{b} $ is in the $ \text{span}(S) $, we must begin with the vector equation $ \vec{b} = x_1\vec{v}_1 + x_2\vec{v}_2 + x_3\vec{v}_3 + x_4\vec{v}_4 $. Here is what you need to work on
	\begin{enumerate}[label = \alph*)]
		\item Write the \textbf{\underline{vector equation}} explicitly. (This means substitute the numbers into the vector equation).
		\item Rewrite the dependence test equation as a \textbf{\underline{matrix equation}}.
		\item Begin with your vector equation and combine the linear combination as one vector (with variables), then write a system of linear equation by equating the vector on the right hand side and the right hand side.
		\item With your matrix equation, write the coefficient matrix and augmented matrix separately.
	\end{enumerate}
\end{enumerate}

\subsection{1)}
\begin{enumerate}[label = \textbf{\alph*)}]
	\item
		\begin{align*}
			& a_1 \langle 3, 4, 1, -6, -1, 9 \rangle \\
			& + a_2 \langle 1, -3, -2, 1, 1, 2 \rangle \\
			& + a_3 \langle -4, -5, -1, 9, -3, 18 \rangle \\
			& + a_4 \langle 0, 2, 1, -2, 2, -18 \rangle = \langle 0, 0, 0, 0, 0, 0 \rangle
		\end{align*}
	\item
		\begin{align*}
			a_1 \begin{bmatrix} 3 \\ 4 \\ 1 \\ -6 \\ -1 \\ 9 \end{bmatrix}
			+ a_2 \begin{bmatrix} 1 \\ -3 \\ -2 \\ 1 \\ 1 \\ 2 \end{bmatrix}
			+ a_3 \begin{bmatrix} -4 \\ -5 \\ -1 \\ 9 \\ -3 \\ 18 \end{bmatrix}
			+ a_4 \begin{bmatrix} 0 \\ 2 \\ 1 \\ -2 \\ 2 \\ -18 \end{bmatrix}
			& = \begin{bmatrix} 0 \\ 0 \\ 0 \\ 0 \\ 0 \\ 0 \end{bmatrix}
		\end{align*}
	\item
		\[
		\systeme{
			3a_1 + a_2 - 4a_3 = 0,
			4a_1 - 3a_2 - 5a_3 + 2a_4 = 0,
			a_1 - 2a_2 - a_3 + a_4 = 0,
			-6a_1 + a_2 + 9a_3 - 2a_4 = 0,
			-a_1 + a_2 - 3a_3 + 2a_4 = 0,
			9a_1 + 2a_2 + 18a_3 - 18a_4 = 0
		}
		\]
	\item
		Coefficient Matrix:
		\begin{align*}
			\left[
			\begin{array}{ c c c c }
				3 & 1 & -4 & 0 \\
				4 & -3 & -5 & 2 \\
				1 & -2 & -1 & 1 \\
				-6 & 1 & 9 & -2 \\
				-1 & 1 & -3 & 2 \\
				9 & 2 & 18 & -18
			\end{array}
			\right]
		\end{align*}

		Augmented Matrix:
		\begin{align*}
			\left[
			\begin{array}{ c c c c | c }
				3 & 1 & -4 & 0 & 0 \\
				4 & -3 & -5 & 2 & 0 \\
				1 & -2 & -1 & 1 & 0 \\
				-6 & 1 & 9 & -2 & 0 \\
				-1 & 1 & -3 & 2 & 0 \\
				9 & 2 & 18 & -18 & 0
			\end{array}
			\right]
		\end{align*}
\end{enumerate}

\subsection{2)}
\begin{enumerate}[label = \textbf{\alph*)}]
	\item
		\begin{align*}
			& a_1 \langle 3, 4, 1, -6, -1, 9 \rangle \\
			& + a_2 \langle 1, -3, -2, 1, 1, 2 \rangle \\
			& + a_3 \langle -4, -5, -1, 9, -3, 18 \rangle \\
			& + a_4 \langle 0, 2, 1, -2, 2, -18 \rangle = \langle 1, 8, 4, -6, -1, -5 \rangle
		\end{align*}
	\item
		\begin{align*}
			a_1 \begin{bmatrix} 3 \\ 4 \\ 1 \\ -6 \\ -1 \\ 9 \end{bmatrix}
			+ a_2 \begin{bmatrix} 1 \\ -3 \\ -2 \\ 1 \\ 1 \\ 2 \end{bmatrix}
			+ a_3 \begin{bmatrix} -4 \\ -5 \\ -1 \\ 9 \\ -3 \\ 18 \end{bmatrix}
			+ a_4 \begin{bmatrix} 0 \\ 2 \\ 1 \\ -2 \\ 2 \\ -18 \end{bmatrix}
			& = \begin{bmatrix} 1 \\ 8 \\ 4 \\ -6 \\ -1 \\ -5 \end{bmatrix}
		\end{align*}
	\item
		\[
		\systeme{
			3a_1 + a_2 - 4a_3 = 1,
			4a_1 - 3a_2 - 5a_3 + 2a_4 = 8,
			a_1 - 2a_2 - a_3 + a_4 = 4,
			-6a_1 + a_2 + 9a_3 - 2a_4 = -6,
			-a_1 + a_2 - 3a_3 + 2a_4 = -1,
			9a_1 + 2a_2 + 18a_3 - 18a_4 = -5
		}
		\]
	\item
		Coefficient Matrix:
		\begin{align*}
			\left[
			\begin{array}{ c c c c }
				3 & 1 & -4 & 0 \\
				4 & -3 & -5 & 2 \\
				1 & -2 & -1 & 1 \\
				-6 & 1 & 9 & -2 \\
				-1 & 1 & -3 & 2 \\
				9 & 2 & 18 & -18
			\end{array}
			\right]
		\end{align*}

		Augmented Matrix:
		\begin{align*}
			\left[
			\begin{array}{ c c c c | c }
				3 & 1 & -4 & 0 & 1 \\
				4 & -3 & -5 & 2 & 8 \\
				1 & -2 & -1 & 1 & 4 \\
				-6 & 1 & 9 & -2 & -6 \\
				-1 & 1 & -3 & 2 & -1 \\
				9 & 2 & 18 & -18 & -5
			\end{array}
			\right]
		\end{align*}
\end{enumerate}

\end{document}
