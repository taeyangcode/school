\documentclass{article}

% Document extensibility %
%
% Disables native paragraph indentation
\usepackage{parskip} 
%
% Provides further bullet options for lists
\usepackage{enumitem}

% Mathematical symbol and statement packages %
%
% Necessary
\usepackage{amsmath}
\usepackage{amssymb}
%
% Extensive fraction notation
\usepackage{xfrac}
%
% Generic mathematical commands
% Notable: \degree, \celcius
\usepackage{gensymb}
%
% Variable vector notation (arrow above variable)
\usepackage{esvect}
%
% Multiline boxed equations
\usepackage{empheq}
%
% SI Unit
\usepackage{siunitx}
\usepackage{physunits}
%
% More intuitive arrays/matrices
\usepackage{array}
%
% Linear Equations
\usepackage{systeme}

% Graphic packages %
%
% Diagrams and illustrations
\usepackage{tikz}
%
% Image insertion
\usepackage{graphicx}

% Document content %
%
% Change title of table of contents
% \renewcommand{\contentsname}{Title}

\DeclareMathOperator{\colspace}{colspace}
\DeclareMathOperator{\rowspace}{rowspace}
\DeclareMathOperator{\nullspace}{nullspace}
\DeclareMathOperator{\rref}{rref}

\title{Week 10 Participation Assignment (1 of 2)}
\author{Corey Mostero - 2566652}
\date{28 April 2023}

\begin{document}

% Command `\hr` to insert horizontal rules
\newcommand{\hr}{\par\noindent\rule{\textwidth}{0.4pt}}

% Command to box and center math equations
\newcommand{\bc}[1]{
	\begin{equation*}
		\begin{boxed}
			{#1}
		\end{boxed}
	\end{equation*}
}

% Command for single line equations with a condition
\newcommand{\cond}[2]{
	\ifmmode
		{#1} \quad {#2}
	\else
		$$ {#1} \quad {#2} $$
	\fi
}

\maketitle
\newpage

\tableofcontents

\section{Part 1}

Let's consider the following matrix
\begin{equation*}
	A = \begin{bmatrix}
		3 & -1 & 3 & 7 & 2 & 2 & 15 \\
		-4 & 3 & 11 & 4 & 2 & 3 & -17 \\
		-3 & 2 & 6 & 1 & 1 & 1 & -16 \\
		1 & 4 & 40 & 37 & 12 & 17 & 24 \\
		-5 & 3 & 7 & -1 & 0 & 1 & -22
	\end{bmatrix}
\end{equation*}

We can define the following subspaces:
\begin{equation*}
	W_1 = \colspace(A), W_2 = \rowspace(A), W_3 = \nullspace(A)
\end{equation*}
Questions:
\begin{enumerate}[label = \textbf{\arabic*)}]
	\item
		Write $ W_1 $, $ W_2 $ as spaces of vectors. Make sure you write all the vectors as $ \langle x_1, x_2, x_3, \cdots, x_k \rangle $.
	\item
		Identify the ambient spaces of $ \colspace(A) $, $ \rowspace(A) $, and $ \nullspace(A) $.
	\item
		Next, we can define $ W_1^\perp $, $ W_2^\perp $, $ W_3^\perp $. Then what are the ambient spaces of the orthogonal complements?
\end{enumerate}

\subsection{1)}

\subsubsection{Find $ \rref(A) $}
\begin{align*}
	A_2 & = 3A_2 + 4A_1 \\
	A & = \begin{bmatrix}
		3 & -1 & 3 & 7 & 2 & 2 & 15 \\
		0 & 5 & 45 & 40 & 14 & 17 & 9 \\
		-3 & 2 & 6 & 1 & 1 & 1 & -16 \\
		1 & 4 & 40 & 37 & 12 & 17 & 24 \\
		-5 & 3 & 7 & -1 & 0 & 1 & -22
	\end{bmatrix}
\end{align*}
\begin{align*}
	A_3 & = A_3 + A_1 \\
	A & = \begin{bmatrix}
		3 & -1 & 3 & 7 & 2 & 2 & 15 \\
		0 & 5 & 45 & 40 & 14 & 17 & 9 \\
		0 & 1 & 9 & 8 & 3 & 3 & -1 \\
		1 & 4 & 40 & 37 & 12 & 17 & 24 \\
		-5 & 3 & 7 & -1 & 0 & 1 & -22
	\end{bmatrix}
\end{align*}
\begin{align*}
	A_3 & = 5A_3 - A_2 \\
	A & = \begin{bmatrix}
		3 & -1 & 3 & 7 & 2 & 2 & 15 \\
		0 & 5 & 45 & 40 & 14 & 17 & 9 \\
		0 & 0 & 0 & 0 & 1 & -2 & -14 \\
		1 & 4 & 40 & 37 & 12 & 17 & 24 \\
		-5 & 3 & 7 & -1 & 0 & 1 & -22
	\end{bmatrix}
\end{align*}
\begin{align*}
	A_4 & = 3A_4 - A_1 \\
	A & = \begin{bmatrix}
		3 & -1 & 3 & 7 & 2 & 2 & 15 \\
		0 & 5 & 45 & 40 & 14 & 17 & 9 \\
		0 & 0 & 0 & 0 & 1 & -2 & -14 \\
		0 & 13 & 117 & 104 & 34 & 49 & 57 \\
		-5 & 3 & 7 & -1 & 0 & 1 & -22
	\end{bmatrix}
\end{align*}
\begin{align*}
	A_4 & = 5A_4 - 13A_2 \\
	A & = \begin{bmatrix}
		3 & -1 & 3 & 7 & 2 & 2 & 15 \\
		0 & 5 & 45 & 40 & 14 & 17 & 9 \\
		0 & 0 & 0 & 0 & 1 & -2 & -14 \\
		0 & 0 & 0 & 0 & -12 & 24 & 168 \\
		-5 & 3 & 7 & -1 & 0 & 1 & -22
	\end{bmatrix}
\end{align*}
\begin{align*}
	A_4 & = A_4 + 12A_3 \\
	A & = \begin{bmatrix}
		3 & -1 & 3 & 7 & 2 & 2 & 15 \\
		0 & 5 & 45 & 40 & 14 & 17 & 9 \\
		0 & 0 & 0 & 0 & 1 & -2 & -14 \\
		0 & 0 & 0 & 0 & 0 & 0 & 0 \\
		-5 & 3 & 7 & -1 & 0 & 1 & -22
	\end{bmatrix}
\end{align*}
\begin{align*}
	& \text{Swap } A_4, A_5 \\
	A_4 & = 3A_4 + 5A_1 \\
	A & = \begin{bmatrix}
		3 & -1 & 3 & 7 & 2 & 2 & 15 \\
		0 & 5 & 45 & 40 & 14 & 17 & 9 \\
		0 & 0 & 0 & 0 & 1 & -2 & -14 \\
		0 & 4 & 36 & 32 & 10 & 13 & 9 \\
		0 & 0 & 0 & 0 & 0 & 0 & 0
	\end{bmatrix}
\end{align*}
\begin{align*}
	A_4 & = 5A_4 - 4A_2 \\
	A & = \begin{bmatrix}
		3 & -1 & 3 & 7 & 2 & 2 & 15 \\
		0 & 5 & 45 & 40 & 14 & 17 & 9 \\
		0 & 0 & 0 & 0 & 1 & -2 & -14 \\
		0 & 0 & 0 & 0 & -6 & -3 & 9 \\
		0 & 0 & 0 & 0 & 0 & 0 & 0
	\end{bmatrix}
\end{align*}
\begin{align*}
	A_4 & = A_4 + 6A_3 \\
	A & = \begin{bmatrix}
		3 & -1 & 3 & 7 & 2 & 2 & 15 \\
		0 & 5 & 45 & 40 & 14 & 17 & 9 \\
		0 & 0 & 0 & 0 & 1 & -2 & -14 \\
		0 & 0 & 0 & 0 & 0 & -15 & -75 \\
		0 & 0 & 0 & 0 & 0 & 0 & 0
	\end{bmatrix}
\end{align*}
\begin{align*}
	A_1 & = \frac{1}{3}A_1 \\
	A_2 & = \frac{1}{5}A_2 \\
	A_4 & = -\frac{1}{15}A_4 \\
	A & = \begin{bmatrix}
		1 & -\frac{1}{3} & 1 & \frac{7}{3} & \frac{2}{3} & \frac{2}{3} & 5 \\
		0 & 1 & 9 & 8 & \frac{14}{5} & \frac{17}{5} & \frac{9}{5} \\
		0 & 0 & 0 & 0 & 1 & -2 & -14 \\
		0 & 0 & 0 & 0 & 0 & 1 & 5 \\
		0 & 0 & 0 & 0 & 0 & 0 & 0
	\end{bmatrix}
\end{align*}
Using calculator to find reduced form:
\bc{
	\rref(A) = \begin{bmatrix}
		1 & 0 & 4 & 5 & 0 & 0 & 3 \\
		0 & 1 & 9 & 8 & 0 & 0 & -4 \\
		0 & 0 & 0 & 0 & 1 & 0 & -4 \\
		0 & 0 & 0 & 0 & 0 & 1 & 5 \\
		0 & 0 & 0 & 0 & 0 & 0 & 0
	\end{bmatrix}
}

\begin{align*}
	& W_1 = \colspace(A) = \\
	& \text{span} \left(
		c_1 \begin{bmatrix} 3 \\ -4 \\ -3 \\ 1 \\ -5 \end{bmatrix}
		+ c_2 \begin{bmatrix} -1 \\ 3 \\ 2 \\ 4 \\ 3 \end{bmatrix}
		+ c_3 \begin{bmatrix} 3 \\ 11 \\ 6 \\ 40 \\ 7 \end{bmatrix}
		+ c_4 \begin{bmatrix} 7 \\ 4 \\ 1 \\ 37 \\ -1 \end{bmatrix}
		+ c_5 \begin{bmatrix} 2 \\ 2 \\ 1 \\ 12 \\ 0 \end{bmatrix}
		+ c_6 \begin{bmatrix} 2 \\ 3 \\ 1 \\ 17 \\ 1 \end{bmatrix}
		+ c_7 \begin{bmatrix} 15 \\ -17 \\ -16 \\ 24 \\ -22 \end{bmatrix}
	\right)
\end{align*}

\begin{align*}
	& W_2 = \rowspace(A) = \\
	& \text{span} \left(
		c_1 \begin{bmatrix} 3 \\ -1 \\ 3 \\ 7 \\ 2 \\ 2 \\ 15 \end{bmatrix}
		+ c_2 \begin{bmatrix} -4 \\ 3 \\ 11 \\ 4 \\ 2 \\ 3 \\ -17 \end{bmatrix}
		+ c_3 \begin{bmatrix} -3 \\ 2 \\ 6 \\ 1 \\ 1 \\ 1 \\ -16 \end{bmatrix}
		+ c_4 \begin{bmatrix} 1 \\ 4 \\ 40 \\ 37 \\ 12 \\ 17 \\ 24 \end{bmatrix}
		+ c_5 \begin{bmatrix} -5 \\ 3 \\ 7 \\ -1 \\ 0 \\ 1 \\ -22 \end{bmatrix}
	\right)
\end{align*}

\subsection{2)}

\begin{enumerate}[label = \textbf{ $ W_\arabic* $ ambient space: }]
	\item $ \mathbb{R}^7 $
	\item $ \mathbb{R}^5 $
	\item $ \mathbb{R}^7 $
\end{enumerate}

\subsection{3)}

\begin{enumerate}[label = \textbf{ $ W_\arabic*^\perp $ ambient space: }]
	\item $ \mathbb{R}^7 $
	\item $ \mathbb{R}^5 $
	\item $ \mathbb{R}^7 $
\end{enumerate}

\end{document}
