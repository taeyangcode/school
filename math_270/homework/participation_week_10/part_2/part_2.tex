\documentclass{article}

% Document extensibility %
%
% Disables native paragraph indentation
\usepackage{parskip} 
%
% Provides further bullet options for lists
\usepackage{enumitem}

% Mathematical symbol and statement packages %
%
% Necessary
\usepackage{amsmath}
\usepackage{amssymb}
%
% Extensive fraction notation
\usepackage{xfrac}
%
% Generic mathematical commands
% Notable: \degree, \celcius
\usepackage{gensymb}
%
% Variable vector notation (arrow above variable)
\usepackage{esvect}
%
% Multiline boxed equations
\usepackage{empheq}
%
% SI Unit
\usepackage{siunitx}
\usepackage{physunits}
%
% More intuitive arrays/matrices
\usepackage{array}
%
% Linear Equations
\usepackage{systeme}

% Graphic packages %
%
% Diagrams and illustrations
\usepackage{tikz}
%
% Image insertion
\usepackage{graphicx}

% Document content %
%
% Change title of table of contents
% \renewcommand{\contentsname}{Title}

\title{Week 10 Participation Assignment (2 of 2)}
\author{Corey Mostero - 2566652}
\date{28 April 2023}

\DeclareMathOperator{\colspace}{colspace}
\DeclareMathOperator{\rowspace}{rowspace}
\DeclareMathOperator{\nullspace}{nullspace}
\DeclareMathOperator{\rref}{rref}

\begin{document}

% Command `\hr` to insert horizontal rules
\newcommand{\hr}{\par\noindent\rule{\textwidth}{0.4pt}}

% Command to box and center math equations
\newcommand{\bc}[1]{
	\begin{equation*}
		\begin{boxed}
			{#1}
		\end{boxed}
	\end{equation*}
}

% Command for single line equations with a condition
\newcommand{\cond}[2]{
	\ifmmode
		{#1} \quad {#2}
	\else
		$$ {#1} \quad {#2} $$
	\fi
}

\maketitle
\newpage

\tableofcontents

\section{Part 2}

\begin{equation*}
	A = \begin{bmatrix}
		3 & -1 & 3 & 7 & 2 & 2 & 15 \\
		-4 & 3 & 11 & 4 & 2 & 3 & -17 \\
		-3 & 2 & 6 & 1 & 1 & 1 & -16 \\
		1 & 4 & 40 & 37 & 12 & 17 & 24 \\
		-5 & 3 & 7 & -1 & 0 & 1 & -22
	\end{bmatrix} \xrightarrow{rref}
	R = \begin{bmatrix}
		1 & 0 & 4 & 5 & 0 & 0 & 3 \\
		0 & 1 & 9 & 8 & 0 & 0 & -4 \\
		0 & 0 & 0 & 0 & 1 & 0 & -4 \\
		0 & 0 & 0 & 0 & 0 & 1 & 5 \\
		0 & 0 & 0 & 0 & 0 & 0 & 0
	\end{bmatrix}
\end{equation*}
Find a basis for
\begin{enumerate}[label = \textbf{\arabic*).}]
	\item $ W_1 = \colspace(A) $
	\item $ W_2 = \rowspace(A) $
	\item $ W_3 = \nullspace(A) $
	\item Next, we can define $ W_1^\perp $, $ W_2^\perp $, $ W_3^\perp $. Last time, we knew the ambient space of the orthogonal complements. Then let's use the definition of the orthogonal complements as well as the basis you found above to set up system of linear equations so that we can find a basis for the complement.
\end{enumerate}

\subsection{1).}

\begin{align*}
	\left\{
		\begin{bmatrix} 3 \\ -4 \\ -3 \\ 1 \\ -5 \end{bmatrix},
		\begin{bmatrix} -1 \\ 3 \\ 2 \\ 4 \\ 3 \end{bmatrix},
		\begin{bmatrix} 2 \\ 2 \\ 1 \\ 12 \\ 0 \end{bmatrix},
		\begin{bmatrix} 2 \\ 3 \\ 1 \\ 17 \\ 1 \end{bmatrix}
	\right\}
\end{align*}

\subsection{2).}

\begin{align*}
	\left\{
		\begin{bmatrix} 3 \\ -1 \\ 3 \\ 7 \\ 2 \\ 2 \\ 15 \end{bmatrix},
		\begin{bmatrix} -4 \\ 3 \\ 11 \\ 4 \\ 2 \\ 3 \\ -17 \end{bmatrix},
		\begin{bmatrix} -3 \\ 2 \\ 6 \\ 1 \\ 1 \\ 1 \\ -16 \end{bmatrix},
		\begin{bmatrix} 1 \\ 4 \\ 40 \\ 37 \\ 12 \\ 17 \\ 24 \end{bmatrix}
	\right\}
\end{align*}

\subsection{3).}

\begin{align*}
	& \begin{bmatrix} x_1 \\ x_2 \\ x_3 \\ x_4 \\ x_5 \\ x_6 \\ x_7 \end{bmatrix} \\
	& = \begin{bmatrix}
		-4a - 3c - 3d \\
		-4a - 8c + 4d \\
		a \\
		c \\
		4d \\
		-5d \\
		d
	\end{bmatrix} \\
	& = a \begin{bmatrix} -4 \\ -9 \\ 1 \\ 0 \\ 0 \\ 0 \\ 0 \end{bmatrix}
	+ c \begin{bmatrix} -5 \\ -8 \\ 0 \\ 1 \\ 0 \\ 0 \\ 0 \end{bmatrix}
	+ d \begin{bmatrix} -3 \\ 4 \\ 0 \\ 0 \\ 4 \\ -5 \\ 1 \end{bmatrix} \\
	& = \left\{
		\begin{bmatrix} -4 \\ -9 \\ 1 \\ 0 \\ 0 \\ 0 \\ 0 \end{bmatrix},
		\begin{bmatrix} -5 \\ -8 \\ 0 \\ 1 \\ 0 \\ 0 \\ 0 \end{bmatrix},
		\begin{bmatrix} -3 \\ 4 \\ 0 \\ 0 \\ 4 \\ -5 \\ 1 \end{bmatrix}
	\right\}
\end{align*}

\subsection{4).}

\begin{align*}
	A^T & = \begin{bmatrix}
		3 & -4 & -3 & 1 & -5 \\
		-1 & 3 & 2 & 4 & 3 \\
		3 & 11 & 6 & 40 & 7 \\
		7 & 4 & 1 & 37 & -1 \\
		2 & 2 & 1 & 12 & 0 \\
		2 & 3 & 1 & 17 & 1 \\
		15 & -17 & -16 & 24 & -22
	\end{bmatrix} \\
	\rref(A^T) & = \begin{bmatrix}
		1 & 0 & 0 & 3 & 0 \\
		0 & 1 & 0 & 5 & 0 \\
		0 & 0 & 1 & -4 & 0 \\
		0 & 0 & 0 & 0 & 1 \\
		0 & 0 & 0 & 0 & 0 \\
		0 & 0 & 0 & 0 & 0 \\
		0 & 0 & 0 & 0 & 0
	\end{bmatrix}
\end{align*}
\begin{enumerate}[label = \textbf{ basis $ W_\arabic*^\perp $: }]
	\item
		\begin{align*}
			\left\{
				\begin{bmatrix} 3 \\ -4 \\ -3 \\ 1 \\ -5 \end{bmatrix},
				\begin{bmatrix} -1 \\ 3 \\ 2 \\ 4 \\ 3 \end{bmatrix},
				\begin{bmatrix} 3 \\ 11 \\ 6 \\ 40 \\ 7 \end{bmatrix},
				\begin{bmatrix} 7 \\ 4 \\ 1 \\ 37 \\ -1 \end{bmatrix}
			\right\}
		\end{align*}
	\item
		\begin{align*}
			\left\{
				\begin{bmatrix} 1 \\ 0 \\ 4 \\ 5 \\ 0 \\ 0 \\ 3 \end{bmatrix},
				\begin{bmatrix} 0 \\ 1 \\ 9 \\ 8 \\ 0 \\ 0 \\ -4 \end{bmatrix},
				\begin{bmatrix} 0 \\ 0 \\ 0 \\ 0 \\ 1 \\ 0 \\ -4 \end{bmatrix},
				\begin{bmatrix} 0 \\ 0 \\ 0 \\ 0 \\ 0 \\ 1 \\ 5 \end{bmatrix}
			\right\}
		\end{align*}
	\item
		\begin{align*}
			\left\{
				\begin{bmatrix} 1 \\ 0 \\ 4 \\ 5 \\ 0 \\ 0 \\ 3 \end{bmatrix},
				\begin{bmatrix} 0 \\ 1 \\ 9 \\ 8 \\ 0 \\ 0 \\ -4 \end{bmatrix},
				\begin{bmatrix} 0 \\ 0 \\ 0 \\ 0 \\ 1 \\ 0 \\ -4 \end{bmatrix},
				\begin{bmatrix} 0 \\ 0 \\ 0 \\ 0 \\ 0 \\ 1 \\ 5 \end{bmatrix}
			\right\}
		\end{align*}
\end{enumerate}

\end{document}
