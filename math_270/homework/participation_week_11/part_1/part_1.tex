\documentclass{article}

% Document extensibility %
%
% Disables native paragraph indentation
\usepackage{parskip} 
%
% Provides further bullet options for lists
\usepackage{enumitem}

% Mathematical symbol and statement packages %
%
% Necessary
\usepackage{amsmath}
\usepackage{amssymb}
%
% Extensive fraction notation
\usepackage{xfrac}
%
% Generic mathematical commands
% Notable: \degree, \celcius
\usepackage{gensymb}
%
% Variable vector notation (arrow above variable)
\usepackage{esvect}
%
% Multiline boxed equations
\usepackage{empheq}
%
% SI Unit
\usepackage{siunitx}
\usepackage{physunits}
%
% More intuitive arrays/matrices
\usepackage{array}
%
% Linear Equations
\usepackage{systeme}
%
% Boxes!
\usepackage{mdframed}
%
% Matrix Notation
\usepackage{bm}

% Graphic packages %
%
% Diagrams and illustrations
\usepackage{tikz}
\usetikzlibrary{positioning}
%
% Image insertion
\usepackage{graphicx}

% Document content %
%
% Change title of table of contents
% \renewcommand{\contentsname}{Title}

\title{Week 11 Participation Assignment (1 of 2)}
\author{Corey Mostero - 2566652}
\date{12 May 2023}

\begin{document}

% Command `\hr` to insert horizontal rules
\newcommand{\hr}{\par\noindent\rule{\textwidth}{0.4pt}}

% Command to box and center math equations
\newcommand{\bc}[1]{
	\begin{equation*}
		\begin{boxed}
			{#1}
		\end{boxed}
	\end{equation*}
}

% Command for single line equations with a condition
\newcommand{\cond}[2]{
	\ifmmode
		{#1} \quad {#2}
	\else
		$$ {#1} \quad {#2} $$
	\fi
}

% Matrix and Vector notation
\newcommand{\matr}[1]{
	\ifmmode \bm{#1}
	\else \textit{\textbf{#1}}
	\fi
}
\newcommand{\vect}[1]{
	\ifmmode \mathbf{#1}
	\else \textbf{#1}
	\fi
}

\maketitle
\newpage

\tableofcontents

\section{Part 1}

Given matrix $ \matr{A} = \begin{bmatrix} 29 & 64 & -32 \\ -16 & -35 & 16 \\ -8 & -16 & 5 \end{bmatrix} $; let's try to form the matrix $ \matr{A} - \lambda \vect{I} $ and find the rref so that we can determine whether the given value is the eigenvalue or not.
\begin{enumerate}[label = \textbf{\arabic*)}]
	\item $ \lambda = 1 $
	\item $ \lambda = 3 $
	\item $ \lambda = 5 $
	\item $ \lambda = -1 $
	\item $ \lambda = -3 $
\end{enumerate}
\hr
\begin{align*}
	\det( \matr{A} - \lambda \vect{I} ) & =
		\begin{bmatrix}
			29 - \lambda & 64 & -32 \\
			-16 & -35 - \lambda & 16 \\
			-8 & -16 & 5 - \lambda
		\end{bmatrix} \\
	\det( \matr{A} - \lambda \vect{I} )
		& = (29 - \lambda)( (-35 - \lambda)(5 - \lambda) - (16)(-16) ) \\
		& + (64)( (16)(-8) - (-16)(5 - \lambda) ) \\
		& + (-32)( (-16)(-16) - (-35 - \lambda)(-8) ) \\
	\det( \matr{A} - \lambda \vect{I} ) & =
		-\lambda^3 - \lambda^2 + 21\lambda + 45 = -(\lambda - 5)(\lambda + 3)^2 \\
	\lambda_{1,2} & = 5, -3
\end{align*}
\begin{align*}
	\left[ \matr{A} - \lambda_1 \right] \vect{x} & = 0 \\
	\begin{bmatrix}
		24 & 64 & -32 \\
		-16 & -40 & 16 \\
		-8 & -16 & 0
	\end{bmatrix}
	\begin{bmatrix} \vect{x}_1 \\ \vect{x}_2 \\ \vect{x}_3 \end{bmatrix} & = 0 \\
	\begin{bmatrix}
		1 & 0 & 4 \\
		0 & 1 & -2 \\
		0 & 0 & 0
	\end{bmatrix}
	\begin{bmatrix} \vect{x}_1 \\ \vect{x}_2 \\ \vect{x}_3 \end{bmatrix} & = 0
\end{align*}
\begin{align*}
	(1)\vect{x}_1 + (0)\vect{x}_2 + (4)\vect{x}_3 & = 0 \\
	\vect{x}_1 & = (-4)\vect{x}_3
\end{align*}
\begin{align*}
	(0)\vect{x}_1 + (1)\vect{x}_2 + (-2)\vect{x}_3 & = 0 \\
	\vect{x}_2 & = (2)\vect{x}_3
\end{align*}
\begin{align*}
	(0)\vect{x}_1 + (0)\vect{x}_2 + (0)\vect{x}_3 & = 0 \\
	0 & = 0
\end{align*}
\begin{align*}
	\vect{x} & = \begin{bmatrix} (-4)\vect{x}_3 \\ (2)\vect{x}_3 \\ \vect{x}_3 \end{bmatrix} \\
	\vect{x} & = \vect{x}_3 \begin{bmatrix} -4 \\ 2 \\ 1 \end{bmatrix}
\end{align*}
\begin{align*}
	\left[ \matr{A} - \lambda_2 \right] \vect{x} & =  0 \\
	\begin{bmatrix}
		32 & 64 & -32 \\
		-16 & -32 & 16 \\
		-8 & -16 & 8
	\end{bmatrix}
	\begin{bmatrix} \vect{x}_1 \\ \vect{x}_2 \\ \vect{x}_3 \end{bmatrix} & = 0 \\
	\begin{bmatrix}
		1 & 2 & -1 \\
		0 & 0 &  0 \\
		0 & 0 & 0
	\end{bmatrix}
	\begin{bmatrix} \vect{x}_1 \\ \vect{x}_2 \\ \vect{x}_3 \end{bmatrix} & = 0 \\
\end{align*}
For both $ \left[ \matr{A} - \lambda_2 \right]_{1,2} $:
\begin{align*}
	(0)\vect{x}_1 + (0)\vect{x}_2 + (0)\vect{x}_3 & = 0 \\
	0 & = 0
\end{align*}
\begin{align*}
	(1)\vect{x}_1 + (2)\vect{x}_2 + (-1)\vect{x}_3 & = 0 \\
	\vect{x}_1 & = (-2)\vect{x}_2 + \vect{x}_3
\end{align*}
\begin{align*}
	\vect{x} & =
		\begin{bmatrix} (-2)\vect{x}_2 \\ \vect{x}_2 \\ 0 \end{bmatrix}
		+ \begin{bmatrix} \vect{x}_3 \\ 0 \\ \vect{x}_3 \end{bmatrix} \\
	\vect{x} & =
		\vect{x}_2 \begin{bmatrix} -2 \\ 1 \\ 0 \end{bmatrix}
		+ \vect{x}_3 \begin{bmatrix} 1 \\ 0 \\ 1 \end{bmatrix}
\end{align*}
Where $ \matr{P} $ is the modal matrix and $ \matr{D} $ is the diagonal matrix:
\begin{align*}
	\matr{P} & =
		\begin{bmatrix}
			-4 & -2 & 1 \\
			2 & 1 & 0 \\
			1 & 0 & 1
		\end{bmatrix} \\
	\matr{D} & =
		\begin{bmatrix}
			5 & 0 & 0 \\
			0 & -3 & 0 \\
			0 & 0 & -3 \\
		\end{bmatrix}
\end{align*}

\end{document}
