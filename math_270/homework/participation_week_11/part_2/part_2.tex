\documentclass{article}

% Document extensibility %
%
% Disables native paragraph indentation
\usepackage{parskip} 
%
% Provides further bullet options for lists
\usepackage{enumitem}

% Mathematical symbol and statement packages %
%
% Necessary
\usepackage{amsmath}
\usepackage{amssymb}
%
% Extensive fraction notation
\usepackage{xfrac}
%
% Generic mathematical commands
% Notable: \degree, \celcius
\usepackage{gensymb}
%
% Variable vector notation (arrow above variable)
\usepackage{esvect}
%
% Multiline boxed equations
\usepackage{empheq}
%
% SI Unit
\usepackage{siunitx}
\usepackage{physunits}
%
% More intuitive arrays/matrices
\usepackage{array}
%
% Linear Equations
\usepackage{systeme}
%
% Boxes!
\usepackage{mdframed}
%
% Matrix Notation
\usepackage{bm}

% Graphic packages %
%
% Diagrams and illustrations
\usepackage{tikz}
\usetikzlibrary{positioning}
%
% Image insertion
\usepackage{graphicx}

% Document content %
%
% Change title of table of contents
% \renewcommand{\contentsname}{Title}

\title{Week 11 Participation Assignment (2 of 2)}
\author{Corey Mostero - 2566652}
\date{12 May 2023}

\begin{document}

% Command `\hr` to insert horizontal rules
\newcommand{\hr}{\par\noindent\rule{\textwidth}{0.4pt}}

% Command to box and center math equations
\newcommand{\bc}[1]{
	\begin{equation*}
		\begin{boxed}
			{#1}
		\end{boxed}
	\end{equation*}
}

% Command for single line equations with a condition
\newcommand{\cond}[2]{
	\ifmmode
		{#1} \quad {#2}
	\else
		$$ {#1} \quad {#2} $$
	\fi
}

% Matrix and Vector notation
\newcommand{\matr}[1]{
	\ifmmode \bm{#1}
	\else \textit{\textbf{#1}}
	\fi
}
\newcommand{\vect}[1]{
	\ifmmode \mathbf{#1}
	\else \textbf{#1}
	\fi
}

\maketitle
\newpage

\tableofcontents

\section{Part 2}

Given a square matrix $ \matr{A} $, we can find the eigenvalues and the corresponding eigenvectors. Then with the eigenvalues and the eigenvectors, we can construct the matrices $ \matr{P} $ and $ \matr{D} $ such that $ \matr{A} = \matr{P}\matr{D}\matr{P}^{-1} $. \\
For the following given information, construct at least three ways for the matrix $ \matr{D} $.
\begin{enumerate}[label = \textbf{\arabic*)}]
	\item For $ \lambda = 1 : \left\{ \langle -1, 0, 1 \rangle \right\} $, for $ \lambda = 3 : \left\{ \langle 1, 0, 1 \rangle \right\} $, for $ \lambda = 5 : \left\{ \langle 0, 1, 0 \rangle \right\} $
		\begin{align*}
			\matr{P} & =
				\begin{bmatrix}
					-1 & 0 & 1 \\
					1 & 0 & 1 \\
					0 & 1 & 0
				\end{bmatrix} \\
			\matr{P}^{-1} & =
				\begin{bmatrix}
					-\frac{1}{2} & \frac{1}{2} & 0 \\
					0 & 0 & 1 \\
					\frac{1}{2} & \frac{1}{2} & 0
				\end{bmatrix} \\
			\matr{D} & =
				\begin{bmatrix}
					1 & 0 & 0 \\
					0 & 3 & 0 \\
					0 & 0 & 5
				\end{bmatrix} \\
			\matr{A} & = \matr{P}\matr{D}\matr{P}^{-1} \\
			\matr{A} & =
				\begin{bmatrix}
					3 & 2 & 0 \\
					2 & 3 & 0 \\
					0 & 0 & 3
				\end{bmatrix}
		\end{align*}
	\item For $ \lambda = 7 : \left\{ \langle 1, 1, 1 \rangle \right\} $, for $ \lambda = 4 : \left\{ \langle -1, 1, 0 \rangle, \langle -1, 0, 1 \rangle \right\} $
		\begin{align*}
			\matr{P} & =
				\begin{bmatrix}
					1 & 1 & 1 \\
					-1 & 1 & 0 \\
					-1 & 0 & 1
				\end{bmatrix} \\
			\matr{P}^{-1} & =
				\begin{bmatrix}
					\frac{1}{3} & -\frac{1}{3} & -\frac{1}{3} \\
					\frac{1}{3} & \frac{2}{3} & -\frac{1}{3} \\
					\frac{1}{3} & -\frac{1}{3} & \frac{2}{3}
				\end{bmatrix} \\
			\matr{D} & =
				\begin{bmatrix}
					7 & 0 & 0 \\
					0 & 4 & 0 \\
					0 & 0 & 4
				\end{bmatrix} \\
			\matr{A} & =
				\begin{bmatrix}
					5 & -1 & -1 \\
					-1 & 5 & 1 \\
					-1 & 1 & 5
				\end{bmatrix}
		\end{align*}
	\item For $ \lambda = 4 : \left\{ \langle 4, 0, 5 \rangle, \langle 2, -5, 0 \rangle \right\} $, for $ \lambda = 7 : \left\{ \langle 1, -1, 1 \rangle \right\} $
		\begin{align*}
			\matr{P} & =
				\begin{bmatrix}
					4 & 0 & 5 \\
					2 & -5 & 0 \\
					1 & -1 & 1
				\end{bmatrix} \\
			\matr{P}^{-1} & =
				\begin{bmatrix}
					1 & 1 & -5 \\
					\frac{2}{5} & \frac{1}{5} & -2 \\
					-\frac{3}{5} & -\frac{4}{5} & 4
				\end{bmatrix} \\
			\matr{D} & =
				\begin{bmatrix}
					4 & 0 & 0 \\
					0 & 4 & 0 \\
					0 & 0 & 7
				\end{bmatrix} \\
			\matr{A} & =
				\begin{bmatrix}
					-5 & -12 & 60 \\
					0 & 4 & 0 \\
					-\frac{9}{5} & -\frac{12}{5} & 16
				\end{bmatrix}
		\end{align*}
\end{enumerate}

\end{document}
