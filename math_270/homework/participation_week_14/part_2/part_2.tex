\documentclass{article}

% Document extensibility %
%
% Disables native paragraph indentation
\usepackage{parskip} 
%
% Provides further bullet options for lists
\usepackage{enumitem}

% Mathematical symbol and statement packages %
%
% Necessary
\usepackage{amsmath}
\usepackage{amssymb}
%
% Extensive fraction notation
\usepackage{xfrac}
%
% Generic mathematical commands
% Notable: \degree, \celcius
\usepackage{gensymb}
%
% Variable vector notation (arrow above variable)
\usepackage{esvect}
%
% Multiline boxed equations
\usepackage{empheq}
%
% SI Unit
\usepackage{siunitx}
\usepackage{physunits}
%
% More intuitive arrays/matrices
\usepackage{array}
%
% Linear Equations
\usepackage{systeme}
%
% Boxes!
\usepackage{mdframed}
%
% Matrix Notation
\usepackage{bm}

% Graphic packages %
%
% Diagrams and illustrations
\usepackage{tikz}
\usetikzlibrary{positioning}
%
% Image insertion
\usepackage{graphicx}

% Document content %
%
% Change title of table of contents
% \renewcommand{\contentsname}{Title}

\begin{document}

% Command `\hr` to insert horizontal rules
\newcommand{\hr}{\par\noindent\rule{\textwidth}{0.4pt}}

% Command to box and center math equations
\newcommand{\bc}[1]{
	\begin{equation*}
		\begin{boxed}
			{#1}
		\end{boxed}
	\end{equation*}
}

% Command for single line equations with a condition
\newcommand{\cond}[2]{
	\ifmmode
		{#1} \quad {#2}
	\else
		$$ {#1} \quad {#2} $$
	\fi
}

% Matrix and Vector notation
\newcommand{\matr}[1]{
	\ifmmode \bm{#1}
	\else \textit{\textbf{#1}}
	\fi
}
\newcommand{\vect}[1]{
	\ifmmode \mathbf{#1}
	\else \textbf{#1}
	\fi
}

% Laplace
\newcommand{\lap}{\mathcal{L}}
\newcommand{\ilap}{\mathcal{L}^{-1}}

\section{Part 2}

Find the inverse Laplace of the following $ F(s) $:

\begin{enumerate}[label = \textbf{\arabic*)}]
	\item $ F(s) = \frac{-s + 5}{s^2 + 2s - 3} $
	\item $ F(s) = \frac{5s^3 - 39s^2 + 14s - 32}{(s^2 - 7s + 10)(s^2 + 1)} $
	\item $ F(s) = \frac{5s^3 + 2s^2 + 4s + 8}{s^2(s + 2)^2(s - 2)} $
\end{enumerate}

\subsection{1)}

\begin{align*}
	F(s) & = \frac{-s + 5}{s^2 + 2s - 3} \\
	F(s) & = \frac{1}{s - 1} - 2\frac{1}{s + 3} \\
	y(t) & = \ilap \left( \frac{1}{s - 1} \right) - 2\ilap \left( \frac{1}{s + 3} \right) \\
	y(t) & = e^t - 2e^{-3t}
\end{align*}
\bc{ y(t) = e^t - 2e^{-3t} }

\subsection{2)}

\begin{align*}
	F(s) & = \frac{5s^3 - 39s^2 + 14s - 32}{(s^2 - 7s + 10)(s^2 + 1)} \\
	F(s) & = 8\frac{1}{s - 2} - 4\frac{1}{s - 5} + \frac{s}{s^2 + 1} \\
	y(t) & = 8\ilap \left( \frac{1}{s - 2} \right) - 4\ilap \left( \frac{1}{s - 5} \right) + \ilap \left( \frac{s}{s^2 + 1} \right) \\
	y(t) & = 8e^{2t} - 4e^{5t} + \cos(t)
\end{align*}
\bc{ y(t) = 8e^{2t} - 4e^{5t} + \cos(t) }

\subsection{3)}

\begin{align*}
	F(s) & = \frac{5s^3 + 2s^2 + 4s + 8}{s^2(s + 2)^2(s - 2)} \\
	F(s) & = -\frac{1}{s^2} - \frac{1}{s + 2} + 2\frac{1}{(s + 2)^2} + \frac{1}{s - 2} \\
	y(t) & = -\ilap \left( \frac{1}{s^2} \right) - \ilap \left( \frac{1}{s + 2} \right) + 2\ilap \left( \frac{1}{(s + 2)^2} \right) + \ilap \left( \frac{1}{s - 2} \right) \\
	y(t) & = -t - e^{-2t} + 2te^{-2t} + e^{2t}
\end{align*}
\bc{ y(t) = -t - e^{-2t} + 2te^{-2t} + e^{2t} }

\end{document}
