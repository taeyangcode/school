\documentclass{article}

% Document extensibility %
%
% Disables native paragraph indentation
\usepackage{parskip} 
%
% Provides further bullet options for lists
\usepackage{enumitem}

% Mathematical symbol and statement packages %
%
% Necessary
\usepackage{amsmath}
\usepackage{amssymb}
%
% Extensive fraction notation
\usepackage{xfrac}
%
% Generic mathematical commands
% Notable: \degree, \celcius
\usepackage{gensymb}
%
% Variable vector notation (arrow above variable)
\usepackage{esvect}
%
% Multiline boxed equations
\usepackage{empheq}
%
% SI Unit
\usepackage{siunitx}
\usepackage{physunits}
%
% More intuitive arrays/matrices
\usepackage{array}
%
% Linear Equations
\usepackage{systeme}
%
% Boxes!
\usepackage{mdframed}

% Graphic packages %
%
% Diagrams and illustrations
\usepackage{tikz}
\usetikzlibrary{positioning}
%
% Image insertion
\usepackage{graphicx}

% Document content %
%
% Change title of table of contents
% \renewcommand{\contentsname}{Title}

\title{Week 12 and Week 13 Participation Assignment (1 of 3)}
\date{12 May 2023}
\author{Corey Mostero - 2566652}

\begin{document}

% Command `\hr` to insert horizontal rules
\newcommand{\hr}{\par\noindent\rule{\textwidth}{0.4pt}}

% Command to box and center math equations
\newcommand{\bc}[1]{
	\begin{equation*}
		\begin{boxed}
			{#1}
		\end{boxed}
	\end{equation*}
}

% Command for single line equations with a condition
\newcommand{\cond}[2]{
	\ifmmode
		{#1} \quad {#2}
	\else
		$$ {#1} \quad {#2} $$
	\fi
}

\newcommand{\vect}[1]{ \textbf{#1} }
\newcommand{\matr}[1]{ \textbf{\textit{#1}} }

\maketitle
\newpage

\tableofcontents

\section{Part 1}

\begin{enumerate}[label = \textbf{\alph*)}]
	\item
		Write the given system in the matrix form $ \mathbf{x' = Ax + f} $. Then identify whether it is a homogeneous or non-homogeneous equation.
		\begin{enumerate}[label = \textbf{\arabic*)}]
			\item
				\systeme*{
					x'(t) = 3x(t) - y(t) + t^2,
					y'(t) = -x(t) + 2y(t) + e^t
				}
			\item
				\systeme*{
					\frac{dx}{dt} = x + y + z,
					\frac{dy}{dt} = 2x - y + 3z,
					\frac{dz}{dt} = x + 5z
				}
		\end{enumerate}
	\item
		Rewrite the given scalar equation as a first order system and then express the system in the matrix form $ \mathbf{x' = Ax + f} $.
		$$ y''(t) - 3y'(t) - 10y(t) = \sin(t) $$
\end{enumerate}

\subsection{a)}

\begin{enumerate}[label = \textbf{\arabic*)}]
	\item
		\begin{align*}
			\frac{d}{dt} \begin{bmatrix} x(t) \\ y(t) \end{bmatrix} & = \begin{bmatrix} 3 & -1 \\ -1 & 2 \end{bmatrix} \begin{bmatrix} x(t) \\ y(t) \end{bmatrix} + \begin{bmatrix} t^2 \\ e^t \end{bmatrix}
		\end{align*}
		The system can also be expressed similarly if we let $ \vect{x} = \begin{bmatrix} x(t) \\ y(t) \end{bmatrix} $ and $ \vect{x}' = \begin{bmatrix} x'(t) \\ y'(t) \end{bmatrix} $.
		\begin{align*}
			\vect{x}' & = \matr{A}\vect{x} + \vect{f} \\
			\vect{x}' & = \begin{bmatrix} 3 & -1 \\ -1 & 2 \end{bmatrix} \vect{x} + \begin{bmatrix} t^2 \\ e^t \end{bmatrix}
		\end{align*}
	\item
		\begin{align*}
			\frac{d}{dt} \begin{bmatrix} x(t) \\ y(t) \\ z(t) \end{bmatrix} & =
				\begin{bmatrix}
					1 & 1 & 1 \\
					2 & -1 & 3 \\
					1 & 0 & 5
				\end{bmatrix}
				\begin{bmatrix} x(t) \\ y(t) \\ z(t) \end{bmatrix} + 0
		\end{align*}
		If we let $ \vect{x} = \begin{bmatrix} x(t) \\ y(t) \\ z(t) \end{bmatrix} $ and $ \vect{x}' = \begin{bmatrix} x'(t) \\ y'(t) \\ z'(t) \end{bmatrix} $:
		\begin{align*}
			\vect{x} & = \matr{A}\vect{x} + \vect{f} \\
			\vect{x}' & =
				\begin{bmatrix}
					1 & 1 & 1 \\
					2 & -1 & 3 \\
					1 & 0 & 5
				\end{bmatrix} \vect{x}
		\end{align*}
\end{enumerate}

\subsection{b)}

$$ y''(t) - 3y'(t) - 10y(t) = \sin(t) $$
Let $ x_0 = y'(t), x_1 = y''(t) $; $ \therefore x_1' = y''(t) $. A linear system of ODE can now be constructed.
\begin{align*}
	y''(t) & = 3y'(t) + 10y(t) + \sin(t) \\
	x_1 & = 3x_0 + 10y + \sin(t) \\
	x_0 & = y'(t)
\end{align*}
\begin{align*}
	\frac{d}{dt} \begin{bmatrix} y \\ x_0 \end{bmatrix} & =
		\begin{bmatrix}
			0 & 1 \\
			10 & 3
		\end{bmatrix}
		\begin{bmatrix} y \\ x_0 \end{bmatrix}
		+ \begin{bmatrix} \sin(t) \\ \sin(t) \end{bmatrix}
\end{align*}

\end{document}
