\documentclass{article}

% Document extensibility %
%
% Disables native paragraph indentation
\usepackage{parskip} 
%
% Provides further bullet options for lists
\usepackage{enumitem}

% Mathematical symbol and statement packages %
%
% Necessary
\usepackage{amsmath}
\usepackage{amssymb}
%
% Extensive fraction notation
\usepackage{xfrac}
%
% Generic mathematical commands
% Notable: \degree, \celcius
\usepackage{gensymb}
%
% Variable vector notation (arrow above variable)
\usepackage{esvect}
%
% Multiline boxed equations
\usepackage{empheq}
%
% SI Unit
\usepackage{siunitx}
\usepackage{physunits}
%
% More intuitive arrays/matrices
\usepackage{array}
%
% Linear Equations
\usepackage{systeme}
%
% Boxes!
\usepackage{mdframed}
%
\usepackage{bm}

% Graphic packages %
%
% Diagrams and illustrations
\usepackage{tikz}
\usetikzlibrary{positioning}
%
% Image insertion
\usepackage{graphicx}

% Document content %
%
% Change title of table of contents
% \renewcommand{\contentsname}{Title}

\title{Week 12 and Week 13 Participation Assignment (2 of 3)}
\date{12 May 2023}
\author{Corey Mostero - 2566652}

\begin{document}

% Command `\hr` to insert horizontal rules
\newcommand{\hr}{\par\noindent\rule{\textwidth}{0.4pt}}

% Command to box and center math equations
\newcommand{\bc}[1]{
	\begin{equation*}
		\begin{boxed}
			{#1}
		\end{boxed}
	\end{equation*}
}

% Command for single line equations with a condition
\newcommand{\cond}[2]{
	\ifmmode
		{#1} \quad {#2}
	\else
		$$ {#1} \quad {#2} $$
	\fi
}

\newcommand{\matr}[1]{\bm{#1}}

\newcommand{\vect}[1]{\mathbf{#1}}

\maketitle
\newpage

\tableofcontents

\section{Part 2}

\begin{enumerate}[label = \textbf{\arabic*)}]
	\item
		$ \begin{bmatrix}
			-1 & -1 & 0 \\
			2 & -1 & 1 \\
			0 & 1 & -1
		\end{bmatrix} $, the complex eigenvalues are $ {-1 + i, -1 - i} $
	\item
		$ \begin{bmatrix}
			5 & -5 & -5 \\
			-1 & 4 & 2 \\
			3 & -5 & -3
		\end{bmatrix} $, the complex eigenvalues are $ {2 + i, 2 - i} $
\end{enumerate}
Find the complex eigenvectors for the above matrices and write it as
\begin{equation*}
	\vec{a} + i\vec{b}, \vec{a} - i\vec{b}
\end{equation*}

\subsection{1)}
\begin{align*}
	\lambda_0 & = -1 + i \\
	\lambda_1 & = -1 - i
\end{align*}
\begin{align*}
	\matr{A} & = \begin{bmatrix}
		-1 - \lambda_0 & -1 & 0 \\
		2 & -1 - \lambda_0 & 1 \\
		0 & 1 & -1 - \lambda_0
	\end{bmatrix} \\
	\matr{A} & = \begin{bmatrix}
		-1 - (-1 + i) & -1 & 0 \\
		2 & -1 - (-1 + i) & 1 \\
		0 & 1 & -1 - (-1 + i)
	\end{bmatrix} \\
	\matr{A} & = \begin{bmatrix}
		-i & -1 & 0 \\
		2 & -i & 1 \\
		0 & 1 & -i
	\end{bmatrix}
\end{align*}
\begin{align*}
	\matr{A}_1 & = \frac{1}{-i}\matr{A}_1 \\
	\matr{A}_2 & \leftrightarrows \matr{A}_3 \\
	\matr{A} & = \begin{bmatrix}
		1 & -i & 0 \\
		0 & 1 & -i \\
		2 & -i & 1
	\end{bmatrix}
\end{align*}
\begin{align*}
	\matr{A}_3 & = \matr{A}_3 - 2\matr{A}_1 \\
	\matr{A}_3 & = \matr{A}_3 - i\matr{A}_2 \\
	\matr{A} & = \begin{bmatrix}
		1 & -i & 0 \\
		0 & 1 & -i \\
		0 & 0 & 0
	\end{bmatrix}
\end{align*}
\begin{align*}
	\matr{A}_1 & = \matr{A}_1 + (i)\matr{A}_2 \\
	\matr{A} & = \begin{bmatrix}
		1 & 0 & 1 \\
		0 & 1 & -i \\
		0 & 0 & 0
	\end{bmatrix}
\end{align*}
\begin{align*}
	\begin{bmatrix}
		1 & 0 & 1 \\
		0 & 1 & -i \\
		0 & 0 & 0
	\end{bmatrix}
	\begin{bmatrix} x_0 \\ x_1 \\ x_2 \end{bmatrix} & =
	\begin{bmatrix} 0 \\ 0 \\ 0 \end{bmatrix}
\end{align*}
\begin{align*}
	(1)x_0 + (0)x_1 + (1)x_2 & = 0 \\
	x_0 & = -x_2
\end{align*}
\begin{align*}
	(0)x_0 + (1)x_1 + (-i)x_2 & = 0 \\
	x_1 & = (i)x_2
\end{align*}
\begin{align*}
	\vect{x} & = \begin{bmatrix} -x_2 \\ (i)x_2 \\ x_2 \end{bmatrix} \\
	\vect{x} & = x_2 \begin{bmatrix} -1 \\ i \\ 1 \end{bmatrix} \\
	\vect{x} & = \begin{bmatrix} -1 \\ 0 \\ 1 \end{bmatrix} + \begin{bmatrix} 0 \\ i \\ 0 \end{bmatrix} \\
	\vect{x} & = \vect{a} + i\vect{b} \\
	\vect{x} & = \begin{bmatrix} -1 \\ 0 \\ 1 \end{bmatrix} + i\begin{bmatrix} 0 \\ 1 \\ 0 \end{bmatrix}
\end{align*}
\bc{ \vect{x} = \begin{bmatrix} -1 \\ 0 \\ 1 \end{bmatrix} + i\begin{bmatrix} 0 \\ 1 \\ 0 \end{bmatrix} }

\begin{align*}
	\matr{A} & = \begin{bmatrix}
		-1 - \lambda_1 & -1 & 0 \\
		2 & -1 - \lambda_1 & 1 \\
		0 & 1 & -1 - \lambda_1
	\end{bmatrix} \\
	\matr{A} & = \begin{bmatrix}
		-1 - (-1 - i) & -1 & 0 \\
		2 & -1 - (-1 - i) & 1 \\
		0 & 1 & -1 - (-1 - i)
	\end{bmatrix} \\
	\matr{A} & = \begin{bmatrix}
		i & -1 & 0 \\
		2 & i & 1 \\
		0 & 1 & i
	\end{bmatrix} \\
\end{align*}
\begin{align*}
	\matr{A}_2 & \leftrightarrows \matr{A}_3 \\
	\matr{A}_1 & = \frac{1}{i}\matr{A}_1 \\
	\matr{A} & =\begin{bmatrix}
		1 & i & 0 \\
		0 & 1 & i \\
		2 & i & 1
	\end{bmatrix}
\end{align*}
\begin{align*}
	\matr{A}_3 & = \matr{A}_3 - 2\matr{A}_1 \\
	\matr{A}_3 & = \matr{A}_3 + i\matr{A}_2 \\
	\matr{A}_1 & = \matr{A}_1 - i\matr{A}_2 \\
	\matr{A} & =\begin{bmatrix}
		1 & 0 & 1 \\
		0 & 1 & i \\
		0 & 0 & 0
	\end{bmatrix}
\end{align*}
\begin{align*}
	(0)x_0 + (1)x_1 + (i)x_2 & = 0 \\
	x_1 & = (-i)x_2
\end{align*}
\begin{align*}
	(1)x_0 + (0)x_1 + (1)x_2 & = 0 \\
	x_0 & = -x_2
\end{align*}
\begin{align*}
	\vect{x} & = \begin{bmatrix} -x_2 \\ (-i)x_2 \\ x_2 \end{bmatrix} \\
	\vect{x} & = x_2 \begin{bmatrix} -1 \\ -i \\ 1 \end{bmatrix} \\
	\vect{x} & = \begin{bmatrix} -1 \\ 0 \\ 1 \end{bmatrix} + i\begin{bmatrix} 0 \\ -1 \\ 0 \end{bmatrix}
\end{align*}
\bc{ \vect{x} = \begin{bmatrix} -1 \\ 0 \\ 1 \end{bmatrix} - i\begin{bmatrix} 0 \\ 1 \\ 0 \end{bmatrix} }

\subsection{2)}

\begin{align*}
	\lambda_0 & = 2 + i \\
	\lambda_1 & = 2 - i
\end{align*}
\begin{align*}
	\matr{A} & = \begin{bmatrix}
		5 - \lambda_0 & -5 & -5 \\
		-1 & 4 - \lambda_0 & 2 \\
		3 & -5 & -3 - \lambda_0
	\end{bmatrix} \\
	\matr{A} & = \begin{bmatrix}
		3 - i & -5 & -5 \\
		-1 & 2 - i & 2 \\
		3 & -5 & -5 - i
	\end{bmatrix} \\
	\matr{A} & = \begin{bmatrix}
		1 & 0 & -1 \\
		0 & 1 & \frac{2}{5} + \frac{i}{5} \\
		0 & 0 & 0
	\end{bmatrix}
\end{align*}
\begin{align*}
	(0)x_0 + (1)x_1 + \left( \frac{2}{5} + \frac{i}{5} \right)x_2 & = 0 \\
	x_1 & = \left( - \frac{2}{5} - \frac{i}{5} \right)x_2
\end{align*}
\begin{align*}
	(1)x_0 + (0)x_1 + (-1)x_2 & = 0 \\
	x_0 & = x_2
\end{align*}
\begin{align*}
	\vect{x} & = \begin{bmatrix} x_2 \\ \left( - \frac{2}{5} - \frac{i}{5} \right)x_2 \\ x_2 \end{bmatrix} \\
	\vect{x} & = \begin{bmatrix} 1 \\ -\frac{2}{5} \\ 1 \end{bmatrix} + i\begin{bmatrix} 0 \\ -\frac{1}{5} \\ 0 \end{bmatrix}
\end{align*}
\bc{ \vect{x} = \begin{bmatrix} 1 \\ -\frac{2}{5} \\ 1 \end{bmatrix} - i\begin{bmatrix} 0 \\ \frac{1}{5} \\ 0 \end{bmatrix} }
\begin{align*}
	\matr{A} & = \begin{bmatrix}
		5 - \lambda_1 & -5 & -5 \\
		-1 & 4 - \lambda_1 & 2 \\
		3 & -5 & -3 - \lambda_1
	\end{bmatrix} \\
	\matr{A} & = \begin{bmatrix}
		i + 3 & -5 & -5 \\
		-1 & i + 2 & 2 \\
		3 & -5 & i - 5
	\end{bmatrix} \\
	\matr{A} & = \begin{bmatrix}
		1 & 0 & -1 \\
		0 & 1 & \frac{2}{5} - \frac{i}{5} \\
		0 & 0 & 0
	\end{bmatrix}
\end{align*}
\begin{align*}
	(0)x_0 + (1)x_1 + \left( \frac{2}{5} - \frac{i}{5} \right)x_2 & = 0 \\
	x_1 & = \left( -\frac{2}{5} + \frac{i}{5} \right)x_2
\end{align*}
\begin{align*}
	(1)x_0 + (0)x_1 + (-1)x_2 & = 0 \\
	x_0 & = x_2
\end{align*}
\begin{align*}
	\vect{x} & = \begin{bmatrix} x_2 \\ \left( -\frac{2}{5} + \frac{i}{5} \right)x_2 \\ x_2 \end{bmatrix} \\
	\vect{x} & = \begin{bmatrix} 1 \\ -\frac{2}{5} \\ 1 \end{bmatrix} + i\begin{bmatrix} 0 \\ \frac{1}{5} \\ 0 \end{bmatrix}
\end{align*}
\bc{ \vect{x} = \begin{bmatrix} 1 \\ -\frac{2}{5} \\ 1 \end{bmatrix} + i\begin{bmatrix} 0 \\ \frac{1}{5} \\ 0 \end{bmatrix} }

\end{document}
