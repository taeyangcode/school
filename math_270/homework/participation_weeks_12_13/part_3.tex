\documentclass{article}

% Document extensibility %
%
% Disables native paragraph indentation
\usepackage{parskip} 
%
% Provides further bullet options for lists
\usepackage{enumitem}

% Mathematical symbol and statement packages %
%
% Necessary
\usepackage{amsmath}
\usepackage{amssymb}
%
% Extensive fraction notation
\usepackage{xfrac}
%
% Generic mathematical commands
% Notable: \degree, \celcius
\usepackage{gensymb}
%
% Variable vector notation (arrow above variable)
\usepackage{esvect}
%
% Multiline boxed equations
\usepackage{empheq}
%
% SI Unit
\usepackage{siunitx}
\usepackage{physunits}
%
% More intuitive arrays/matrices
\usepackage{array}
%
% Linear Equations
\usepackage{systeme}
%
% Boxes!
\usepackage{mdframed}

% Graphic packages %
%
% Diagrams and illustrations
\usepackage{tikz}
\usetikzlibrary{positioning}
%
% Image insertion
\usepackage{graphicx}

% Document content %
%
% Change title of table of contents
% \renewcommand{\contentsname}{Title}

\title{Week 12 and Week 13 Participation Assignment (1 of 3)}
\date{12 May 2023}
\author{Corey Mostero - 2566652}

\begin{document}

% Command `\hr` to insert horizontal rules
\newcommand{\hr}{\par\noindent\rule{\textwidth}{0.4pt}}

% Command to box and center math equations
\newcommand{\bc}[1]{
	\begin{equation*}
		\begin{boxed}
			{#1}
		\end{boxed}
	\end{equation*}
}

% Command for single line equations with a condition
\newcommand{\cond}[2]{
	\ifmmode
		{#1} \quad {#2}
	\else
		$$ {#1} \quad {#2} $$
	\fi
}

\newcommand{\matr}[1]{\bm{#1}}
\newcommand{\vect}[1]{\mathbf{#1}}

\section{Part 3}

After finding the complex eigenvectors from the previous assignment, it's time to solve the system of first order linear differential equations $ \vec{x}' = A\vec{x} $ completely, where $ A $ is given as followed:
\begin{enumerate}[label = \textbf{\arabic*)}]
	\item
		$ \begin{bmatrix}
			-1 & -1 & 0 \\
			2 & -1 & 1 \\
			0 & 1 & -1
		\end{bmatrix} $
	\item
		$ \begin{bmatrix}
			5 & -5 & -5 \\
			-1 & 4 & 2 \\
			3 & -5 & -3
		\end{bmatrix} $
\end{enumerate}

\subsection{1)}

\begin{equation*}
	\vect{x} = \begin{bmatrix} -1 \\ 0 \\ 1 \end{bmatrix} + i\begin{bmatrix} 0 \\ 1 \\ 0 \end{bmatrix}
\end{equation*}
\begin{align*}
	\vect{x}(t) & = e^{(-1 + i)t} \begin{bmatrix} -1 \\ i \\ 1 \end{bmatrix} \\
	\vect{x}(t) & = e^{-t}e^{it} \begin{bmatrix} -1 \\ i \\ 1 \end{bmatrix} \\
	\vect{x}(t) & = e^{-t}(\cos(t) + i\sin(t)) \begin{bmatrix} -1 \\ i \\ 1 \end{bmatrix} \\
	\vect{x}(t) & = e^{-t}
		\begin{bmatrix}
			-\cos(t) - i\sin(t) \\
			i\cos(t) - \sin(t) \\
			\cos(t) + i\sin(t)
		\end{bmatrix} \\
	\vect{x}(t) & = e^{-t}
		\begin{bmatrix}
			-\cos(t) \\
			-\sin(t) \\
			\cos(t)
		\end{bmatrix} +
		ie^{-t} \begin{bmatrix}
			-\sin(t) \\
			\cos(t) \\
			\sin(t)
		\end{bmatrix}
\end{align*}
General Solution:
\bc{
	\vect{x}(t) = C_0 e^{-t}
		\begin{bmatrix}
			-\cos(t) \\
			-\sin(t) \\
			\cos(t)
		\end{bmatrix} +
		C_1 ie^{-t} \begin{bmatrix}
			-\sin(t) \\
			\cos(t) \\
			\sin(t)
		\end{bmatrix}
}

\subsection{2)}

\begin{equation*}
	\vect{x} = \begin{bmatrix} 1 \\ -\frac{2}{5} \\ 1 \end{bmatrix} - i \begin{bmatrix} 0 \\ \frac{1}{5} \\ 0 \end{bmatrix}
\end{equation*}
\begin{align*}
	\vect{x}(t) & = e^{(2 + i)t} \begin{bmatrix} 1 \\ -\frac{2}{5} - \frac{i}{5} \\ 1 \end{bmatrix} \\
	\vect{x}(t) & = e^{2t}e^{it} \begin{bmatrix} 1 \\ -\frac{2}{5} - \frac{i}{5} \\ 1 \end{bmatrix} \\
	\vect{x}(t) & = e^{2t}(\cos(t) + i\sin(t)) \begin{bmatrix} 1 \\ -\frac{2}{5} - \frac{i}{5} \\ 1 \end{bmatrix} \\
	\vect{x}(t) & = e^{2t}
		\begin{bmatrix}
			\cos(t) + i\sin(t) \\
			-\frac{2\cos(t)}{5} - \frac{i\cos(t)}{5} - \frac{2i\sin(t)}{5} + \frac{\sin(t)}{5} \\
			\cos(t) + i\sin(t)
		\end{bmatrix} \\
	\vect{x}(t) & = e^{2t}
		\begin{bmatrix}
			\cos(t) \\
			\frac{\sin(t) - 2\cos(t)}{5} \\
			\cos(t)
		\end{bmatrix}
		+ ie^{2t}
		\begin{bmatrix}
			\sin(t) \\
			-\frac{\cos(t) + 2\sin(t)}{5} \\
			\sin(t)
		\end{bmatrix}
\end{align*}
General Solution:
\bc{
	\vect{x}(t) = C_0 e^{2t}
		\begin{bmatrix}
			\cos(t) \\
			\frac{\sin(t) - 2\cos(t)}{5} \\
			\cos(t)
		\end{bmatrix}
		+ C_1 ie^{2t}
		\begin{bmatrix}
			\sin(t) \\
			-\frac{\cos(t) + 2\sin(t)}{5} \\
			\sin(t)
		\end{bmatrix}
}

\end{document}
