\documentclass{article}

% Document extensibility %
%
% Disables native paragraph indentation
\usepackage{parskip} 
%
% Provides further bullet options for lists
\usepackage{enumitem}

% Mathematical symbol and statement packages %
%
% Necessary
\usepackage{amsmath}
\usepackage{amssymb}
%
% Extensive fraction notation
\usepackage{xfrac}
%
% Generic mathematical commands
% Notable: \degree, \celcius
\usepackage{gensymb}
%
% Variable vector notation (arrow above variable)
\usepackage{esvect}
%
% Multiline boxed equations
\usepackage{empheq}
%
% SI Unit
\usepackage{siunitx}
\usepackage{physunits}
%
% More intuitive arrays/matrices
\usepackage{array}
%
% Linear Equations
\usepackage{systeme}
%
% Boxes!
\usepackage{mdframed}
%
% Matrix Notation
\usepackage{bm}

% Graphic packages %
%
% Diagrams and illustrations
\usepackage{tikz}
\usetikzlibrary{positioning}
%
% Image insertion
\usepackage{graphicx}

% LaTeX Commands
%
% Argument Parser
\usepackage{xparse}

% Document content %
%
% Change title of table of contents
% \renewcommand{\contentsname}{Title}

\begin{document}

% Command `\hr` to insert horizontal rules
\newcommand{\hr}{\par\noindent\rule{\textwidth}{0.4pt}}

% Command to box and center math equations
\newcommand{\bc}[1]{
	\begin{equation*}
		\begin{boxed}
			{#1}
		\end{boxed}
	\end{equation*}
}

% Command for single line equations with a condition
\newcommand{\cond}[2]{
	\ifmmode
		{#1} \quad {#2}
	\else
		$$ {#1} \quad {#2} $$
	\fi
}

% Matrix and Vector notation
\newcommand{\matr}[1]{
	\ifmmode \bm{#1}
	\else \textit{\textbf{#1}}
	\fi
}
\newcommand{\vect}[1]{
	\ifmmode \mathbf{#1}
	\else \textbf{#1}
	\fi
}

% Laplace
\NewDocumentCommand{\lap}{o}{
	\IfNoValueTF{#1}
		{ \mathcal{L} }
		{ \mathcal{L} \left\{ {#1} \right\} }
}
\NewDocumentCommand{\ilap}{o}{
	\IfNoValueTF{#1}
		{ \mathcal{L}^{-1} }
		{ \mathcal{L}^{-1} \left\{ {#1} \right\} }
}

\section{Week 15 and Week 16 Participation Assignment (1 of 2)}
For the following equations in the form of $ A(x)y'' + B(x)y' + C(x)y = 0 $, classify each singular point (real or complex) of the given equation as regular or irregular by finding the functions $ P(x), Q(x), p(x) = xP(x), q(x) = x^2Q(x) $ and then determine whether $ p(x) $ and $ q(x) $ are \text{ analytic} or not.
\begin{enumerate}[label = \textbf{\arabic*)}]
	\item
		\begin{equation*}
			(x^2 - 1)y'' + xy' + 3y = 0
		\end{equation*}
		\begin{align*}
			P(x) & = \frac{ x }{ x^2 - 1 } \\
			p(x) & = \frac{ x }{ x + 1 }, \text{ analytic} \\
			Q(x) & = \frac{ 3 }{ x^2 - 1 } \\
			q(x) & = \frac{ 3(x - 1) }{ x + 1 }, \text{ analytic}
		\end{align*}
	\item
		\begin{equation*}
			x^2y'' + 8xy' - 3xy = 0
		\end{equation*}
		\begin{align*}
			P(x) & = \frac{ 8 }{ x } \\
			p(x) & = 8, \text{ analytic} \\
			Q(x) & = -\frac{ 3 }{ x } \\
			q(x) & = -3x, \text{ analytic}
		\end{align*}
	\item
		\begin{equation*}
			(x^2 + 1)y'' + 7x^2y' - 3xy = 0
		\end{equation*}
		\begin{align*}
			x & = \pm i, \quad \therefore \text{ordinary}
		\end{align*}
	\item
		\begin{equation*}
			(x^2 - x)y'' + xy' + 7y = 0
		\end{equation*}
		\begin{align*}
			P(x) & = \frac{ 1 }{ x - 1 } \\
			p(x) & = 1, \text{ analytic} \\
			Q(x) & = \frac{ 7 }{ x(x - 1) } \\
			q(x) & = \frac{ 7(x - 1) }{ x }, \text{ not analytic}
		\end{align*}
\end{enumerate}

\end{document}
