\documentclass{article}

% Document extensibility %
%
% Disables native paragraph indentation
\usepackage{parskip} 
%
% Provides further bullet options for lists
\usepackage{enumitem}

% Mathematical symbol and statement packages %
%
% Necessary
\usepackage{amsmath}
\usepackage{amssymb}
%
% Extensive fraction notation
\usepackage{xfrac}
%
% Generic mathematical commands
% Notable: \degree, \celcius
\usepackage{gensymb}
%
% Variable vector notation (arrow above variable)
\usepackage{esvect}
%
% Multiline boxed equations
\usepackage{empheq}
%
% SI Unit
\usepackage{siunitx}
\usepackage{physunits}
%
% More intuitive arrays/matrices
\usepackage{array}
%
% Linear Equations
\usepackage{systeme}
%
% Boxes!
\usepackage{mdframed}
%
% Matrix Notation
\usepackage{bm}

% Graphic packages %
%
% Diagrams and illustrations
\usepackage{tikz}
\usetikzlibrary{positioning}
%
% Image insertion
\usepackage{graphicx}

% Document content %
%
% Change title of table of contents
% \renewcommand{\contentsname}{Title}

\begin{document}

% Command `\hr` to insert horizontal rules
\newcommand{\hr}{\par\noindent\rule{\textwidth}{0.4pt}}

% Command to box and center math equations
\newcommand{\bc}[1]{
	\begin{equation*}
		\begin{boxed}
			{#1}
		\end{boxed}
	\end{equation*}
}

% Command for single line equations with a condition
\newcommand{\cond}[2]{
	\ifmmode
		{#1} \quad {#2}
	\else
		$$ {#1} \quad {#2} $$
	\fi
}

% Matrix and Vector notation
\newcommand{\matr}[1]{
	\ifmmode \bm{#1}
	\else \textit{\textbf{#1}}
	\fi
}
\newcommand{\vect}[1]{
	\ifmmode \mathbf{#1}
	\else \textbf{#1}
	\fi
}

% Laplace
\NewDocumentCommand{\lap}{o}{
	\IfNoValueTF{#1}
		{ \mathcal{L} }
		{ \mathcal{L} \left\{ {#1} \right\} }
}
\NewDocumentCommand{\ilap}{o}{
	\IfNoValueTF{#1}
		{ \mathcal{L}^{-1} }
		{ \mathcal{L}^{-1} \left\{ {#1} \right\} }
}

\tableofcontents

\section{Section 10.3}

\subsection{10.3.1}

Apply the translation theorem to find the Laplace transform of the following function.
\begin{equation*}
	f(t) = t^5e^{\pi t}
\end{equation*}
\begin{align*}
	\lap[t^5] & = \frac{5!}{s^6} \\
	\lap[t^5e^{\pi t}] & = \frac{5!}{(s - \pi)^6}
\end{align*}

\subsection{10.3.3}

Apply the translation theorem to find the Laplace transform of the following function.
\begin{equation*}
	f(t) = e^{-9t}\sin(7\pi t)
\end{equation*}
\begin{align*}
	\lap[7\pi t] & = \frac{7\pi}{s^2 + 49\pi^2} \\
	\lap[e^{-9t}\sin(7\pi t)] & = \frac{7\pi}{(s + 9)^2 + 49\pi^2}
\end{align*}

\subsection{10.3.5}

Apply the translation theorem to find the inverse Laplace transform of the following function.
\begin{equation*}
	F(s) = \frac{5}{2s - 18}
\end{equation*}
\begin{align*}
	F(s) & = \frac{5}{2} \left( \frac{1}{s - 9} \right) \\
	\ilap[ \frac{1}{s - 9} ] & = e^{9t} \\
	\ilap[ \frac{5}{2s - 18} ] & = \frac{5e^{9t}}{2}
\end{align*}

\subsection{10.3.9}

Apply the translation theorem to find the inverse Laplace transform of the following function.
\begin{equation*}
	F(s) = \frac{7s + 13}{s^2 - 6s + 73}
\end{equation*}
\begin{align*}
	F(s) & = \frac{7s + 13}{s^2 - 6s + 73} \\
	F(s) & = \frac{7(s + 3 - 3) + 13}{(s - 3)^2 + 64} \\
	F(s) & = \frac{7(s - 3) + 34}{(s - 3)^2 + 64} \\
	F(s) & = \frac{7(s - 3)}{(s - 3)^2 + 64} + \frac{34}{(s - 3)^2 + 64} \\
	\ilap[\frac{7(s - 3)}{(s - 3)^2 + 64}] & = e^{3t}\cos(8t) \\
	\ilap[\frac{34}{(s - 3)^2 + 64}] & = e^{3t}\frac{34}{8}\sin(8t) \\
	\ilap[F(s)] & = e^{3t} \left( \cos(8t) + \frac{17\sin(t)}{4} \right)
\end{align*}

\subsection{10.3.13}

Use partial fractions to find the inverse Laplace transform of the following function.
\begin{equation*}
	F(s) = \frac{-50 - 15s}{s^2 + 13s + 36}
\end{equation*}
\begin{align*}
	F(s) & = \frac{2}{s + 4} - \frac{17}{x + 9} \\
	\ilap[F(s)] & = 2e^{-4t} - 17e^{-9t}
\end{align*}

\subsection{10.3.15}

Use partial fractions to find the inverse Laplace transform of the following function.
\begin{equation*}
	F(s) = \frac{5}{s^3 - 4s^2}
\end{equation*}
\begin{align*}
	F(s) & = \frac{5}{16(s - 4)} - \frac{5}{16s} - \frac{5}{4s^2} \\
	F(s) & = \frac{5e^{4t}}{16} - \frac{5}{16} - \frac{5t}{4}
\end{align*}

\subsection{10.3.37}

Use Laplace transforms to solve the following initial value problem.
\begin{equation*}
	x'' + 10x' + 29x = te^{-t}; x(0) = 0, x'(0) = 3
\end{equation*}
\begin{align*}
	\lap[x''] + 10\lap[x'] + 29\lap[x] & = \lap[te^{-t}] \\
	\left[ s^2\lap[x] - sx(0) - x'(0) \right] + 10 \left[ s\lap[x] - x(0) \right] + 29\lap[x] & = \frac{1}{(s + 1)^{2}} \\
	\lap[x](s^2 + 10s + 29) - 0 - 3 - 0 & = \frac{1}{(s + 1)^2} \\
	\lap[x] & = \frac{3s^2 + 6s + 4}{(s + 1)^2(s^2 + 10s + 29)}
\end{align*}
\begin{align*}
	x & = \ilap[\frac{3s^2 + 6s + 4}{(s + 1)^2(s^2 + 10s + 29)}] \\
	x & = \frac{1}{50}\ilap[\frac{s}{s^2 + 10s + 29}] + \frac{313}{100}\ilap[\frac{1}{s^2 + 10s + 29}] - \frac{1}{50}\ilap[\frac{1}{s + 1}] + \frac{1}{20}\ilap[\frac{1}{(s + 1)^2}]
\end{align*}
\begin{align*}
	\ilap[\frac{s}{s^2 + 10s + 29}] & = \ilap[\frac{s + 5 - 5}{(s + 5)^2 + 4}] \\
									& = \ilap[\frac{s + 5}{(s + 5)^2 + 4}] - 5\ilap[\frac{1}{(s + 5)^2 + 4}] \\
									& = e^{-5t}\cos(2t) - 5e^{-5t}\frac{1}{2}\sin(2t) \\
									& = e^{-5t} \left( \cos(2t) - \frac{5}{2}\sin(2t) \right)
\end{align*}
\begin{align*}
	\ilap[\frac{1}{s^2 + 10s + 29}] & = \ilap[\frac{1}{(s + 5)^2 + 4}] \\
									& = \frac{e^{-5t}\sin(2t)}{2}
\end{align*}
\begin{align*}
	\ilap[\frac{1}{s + 1}] & = e^{-t}
\end{align*}
\begin{align*}
	\ilap[\frac{1}{(s + 1)^2}] & = te^{-t}
\end{align*}
\begin{align*}
	x(t) & = \frac{1}{50}e^{-5t} \left( \cos(2t) - \frac{5}{2}\sin(2t) \right) + \frac{313}{100}\frac{e^{-5t}\sin(2t)}{2} - \frac{1}{50}e^{-t} + \frac{1}{20}te^{-t}
\end{align*}

\end{document}

