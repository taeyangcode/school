\documentclass{article}

% Document extensibility %
%
% Disables native paragraph indentation
\usepackage{parskip} 
%
% Provides further bullet options for lists
\usepackage{enumitem}

% Mathematical symbol and statement packages %
%
% Necessary
\usepackage{amsmath}
\usepackage{amssymb}
%
% Extensive fraction notation
\usepackage{xfrac}
%
% Generic mathematical commands
% Notable: \degree, \celcius
\usepackage{gensymb}
%
% Variable vector notation (arrow above variable)
\usepackage{esvect}
%
% Multiline boxed equations
\usepackage{empheq}
%
% SI Unit
\usepackage{siunitx}
\usepackage{physunits}
%
% More intuitive arrays/matrices
\usepackage{array}
%
% Linear Equations
\usepackage{systeme}
%
% Boxes!
\usepackage{mdframed}
%
% Matrix Notation
\usepackage{bm}

% Graphic packages %
%
% Diagrams and illustrations
\usepackage{tikz}
\usetikzlibrary{positioning}
%
% Image insertion
\usepackage{graphicx}

% LaTeX Commands
%
% Argument Parser
\usepackage{xparse}

% Document content %
%
% Change title of table of contents
% \renewcommand{\contentsname}{Title}

\begin{document}

% Command `\hr` to insert horizontal rules
\newcommand{\hr}{\par\noindent\rule{\textwidth}{0.4pt}}

% Command to box and center math equations
\newcommand{\bc}[1]{
	\begin{equation*}
		\begin{boxed}
			{#1}
		\end{boxed}
	\end{equation*}
}

% Command for single line equations with a condition
\newcommand{\cond}[2]{
	\ifmmode
		{#1} \quad {#2}
	\else
		$$ {#1} \quad {#2} $$
	\fi
}

% Matrix and Vector notation
\newcommand{\matr}[1]{
	\ifmmode \bm{#1}
	\else \textit{\textbf{#1}}
	\fi
}
\newcommand{\vect}[1]{
	\ifmmode \mathbf{#1}
	\else \textbf{#1}
	\fi
}

% Laplace
\NewDocumentCommand{\lap}{o}{
	\IfNoValueTF{#1}
		{ \mathcal{L} }
		{ \mathcal{L} \left\{ {#1} \right\} }
}
\NewDocumentCommand{\ilap}{o}{
	\IfNoValueTF{#1}
		{ \mathcal{L}^{-1} }
		{ \mathcal{L}^{-1} \left\{ {#1} \right\} }
}

\tableofcontents

\section{Section 11.1}

\subsection{11.1.4}

Find a power series solution of the differential equation below. Determine the radius of convergence of the resulting series, and use the series given below to identify the series in terms of familiar elementary functions.
\begin{equation*}
	y' - 4xy = 0
\end{equation*}
\begin{align*}
	y & = \sum_{n = 0}^{\infty} a_nx^n \\
	y' & = \sum_{n = 1}^{\infty} na_nx^{n - 1}
\end{align*}
\begin{align*}
	\sum_{n = 1}^{\infty} na_nx^{n - 1} - 4x \left( \sum_{n = 0}^{\infty} a_nx^n \right) & = 0 \\
	\sum_{n = 1}^{\infty} na_nx^{n - 1} + \sum_{n = 0}^{\infty} -4a_nx^{n + 1} & = 0 \\
	\sum_{n = 0}^{\infty} (n + 1)a_{n + 1}x^{n} + \sum_{n = 1}^{\infty} -4a_{n - 1}x^{n} & = 0 \\
	a_n + \sum_{n = 1}^{\infty} \left[ (n + 1)a_{n + 1} - 4a_{n - 1} \right] x^n & = 0 \\
	a_n & = 0 \\
	a_{n + 1} & = \frac{ 4a_{n - 1} }{ n + 1 }
\end{align*}
\begin{align*}
\end{align*}

\end{document}
