\documentclass{article}

% Document extensibility %
%
% Disables native paragraph indentation
\usepackage{parskip} 
%
% Provides further bullet options for lists
\usepackage{enumitem}

% Mathematical symbol and statement packages %
%
% Necessary
\usepackage{amsmath}
\usepackage{amssymb}
%
% Extensive fraction notation
\usepackage{xfrac}
%
% Generic mathematical commands
% Notable: \degree, \celcius
\usepackage{gensymb}
%
% Variable vector notation (arrow above variable)
\usepackage{esvect}
%
% Multiline boxed equations
\usepackage{empheq}
%
% SI Unit
\usepackage{siunitx}

% Graphic packages %
%
% Diagrams and illustrations
\usepackage{tikz}
%
% Image insertion
\usepackage{graphicx}

% Document content %
%
% Change title of table of contents
% \renewcommand{\contentsname}{Title}

\begin{document}

% Command `\hr` to insert horizontal rules
\newcommand{\hr}{\par\noindent\rule{\textwidth}{0.4pt}}

% Command to box and center math equations
\newcommand{\bc}[1]{
	\begin{equation*}
		\begin{boxed}
			{#1}
		\end{boxed}
	\end{equation*}
}

% Command for single line equations with a condition
\newcommand{\cond}[2]{
	\ifmmode
		{#1} \quad {#2}
	\else
		$$ {#1} \quad {#2} $$
	\fi
}

\tableofcontents

\section{Section 2.4 - Homework Problems}

\subsection{2.4.1}
Apply Euler's method twice to approximate the solution to the initial value problem on the interval $ \left[ 0,\frac{1}{2} \right] $, first with step size $ h = 0.25 $, then with step size $ h = 0.1 $. Compare the three-decimal place values of the two approximations at $ x = \frac{1}{2} $ with the value of $ y \left( \frac{1}{2} \right) $ of the actual solution. (Round to three decimal places)
$$ y' = -y, y(0) = 7, y(x) = 7e^{-x} $$
\begin{enumerate}[label=\textbf{(\alph*)}]
	\item
		\begin{align*}
			x_0 & = 0 \\
			y_0 & = 7 \\
			f(x,y) & = -y \\
			h & = 0.25
		\end{align*}
		\begin{align*}
			y_1 & = (7) + (0.25) \left[ -(7) \right] = 5.25 \\
			y_2 & = (5.25) + (0.25) \left[ -(5.25) \right] = 3.9375
		\end{align*}
		The Euler approximation when $ h = 0.25 $ of $ y \left( \frac{1}{2} \right) $ is \bc{3.938}
	\item
		\begin{align*}
			x_0 & = 0 \\
			y+0 & = 7 \\
			f(x,y) & = -y \\
			h & = 0.1
		\end{align*}
		\begin{align*}
			y_1 & = (7) + (0.1) \left[ -(7) \right] = 6.3 \\
			y_2 & = (6.3) + (0.1) \left[ -(6.3) \right] = 5.67 \\
			y_3 & = (5.67) + (0.1) \left[ -(5.67) \right] = 5.103 \\
			y_4 & = (5.103) + (0.1) \left[ -(5.103) \right] = 4.5927 \\
			y_5 & = (4.5927) + (0.1) \left[ -(4.5927) \right] = 4.13343
		\end{align*}
		The Euler approximation when $ h = 0.1 $ of $ y \left( \frac{1}{2} \right) $ is \bc{4.133}
	\item
		\begin{align*}
			y & = \left( 7e^{-x} \right) \\
			  & = \left( 7e^{-\frac{1}{2}} \right) \\
			y & = 4.24571
		\end{align*}
		The value of $ y \left( \frac{1}{2} \right) $ using the actual solution is \bc{4.246}
	\item
		The approximation $ 4.133 $, using the \underline{lesser} value of $ h $, is closer to the value of $ y \left( \frac{1}{2} \right) $ found using the actual solution.
\end{enumerate}

\subsection{2.4.7}
Apply Euler's method twice to approximate the solution to the initial value problem on the interval $ \left[ 0,\frac{1}{2} \right] $, first with step size $ h = 0.25 $, then with step size $ h = 0.1 $. Compare the three-decimal-place values of the two approximations at $ x = \frac{1}{2} $ with the value of $ y \left( \frac{1}{2} \right) $ of the actual solution. (Round to three decimal places)
$$ y' = -3x^2y, y(0) = 8, y(x) = 8e^{-x^3} $$
\begin{enumerate}[label=\textbf{(\alph*)}]
	\item
		\begin{align*}
			x_0 & = 0 \\
			y_0 & = 8 \\
			f(x,y) & = -3x^2y \\
			h & = 0.25
		\end{align*}
		\begin{align*}
			y_1 & = (8) + (0.25) \left[ -3(0)^2(8) \right] = 8 \\
			y_2 & = (8) + (0.25) \left[ -3(0.25)^2(8) \right] = 7.625
		\end{align*}
		The Euler approximation when $ h = 0.25 $ of $ y \left( \frac{1}{2} \right) $ is \bc{7.625}
	\item
		\begin{align*}
			x_0 & = 0 \\
			y_0 & = 8 \\
			f(x,y) & = -3x^2y \\
			h & = 0.1
		\end{align*}
		\begin{align*}
			y_1 & = (8) + (0.1) \left[ -3(0)^2(8) \right] = 8 \\
			y_2 & = (8) + (0.1) \left[ -3(0.1)^2(8) \right] = 7.976 \\
			y_3 & = (7.976) + (0.1) \left[ -3(0.2)^2(7.976) \right] \approx 7.88029 \\
			y_4 & \approx (7.88029) + (0.1) \left[ -3(0.3)^2(7.88029) \right] \approx 7.66752 \\
			y_5 & \approx (7.66752) + (0.1) \left[ -3(0.4)^2(7.66752) \right] \approx 7.29948
		\end{align*}
		The Euler approximation when h $ h = 0.1 $ of $ y \left( \frac{1}{2} \right) $ is \bc{7.299}
	\item
		\begin{align*}
			y & = 8e^{-x^3} \\
			  & = 8e^{-(\frac{1}{2})^3} \\
			y & = 7.05998
		\end{align*}
		The value of $ y \left( \frac{1}{2} \right) $ using the actual solution is \bc{7.060}
	\item
		The approximation $ 7.299 $, using the \underline{lesser} value of $ h $, is closer to the value of $ y \left( \frac{1}{2} \right) $ found using the actual solution.
\end{enumerate}

\end{document}
