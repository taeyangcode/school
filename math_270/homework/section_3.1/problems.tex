\documentclass{article}

% Document extensibility %
%
% Disables native paragraph indentation
\usepackage{parskip} 
%
% Provides further bullet options for lists
\usepackage{enumitem}

% Mathematical symbol and statement packages %
%
% Necessary
\usepackage{amsmath}
\usepackage{amssymb}
%
% Extensive fraction notation
\usepackage{xfrac}
%
% Generic mathematical commands
% Notable: \degree, \celcius
\usepackage{gensymb}
%
% Variable vector notation (arrow above variable)
\usepackage{esvect}
%
% Multiline boxed equations
\usepackage{empheq}
%
% SI Unit
\usepackage{siunitx}

% Graphic packages %
%
% Diagrams and illustrations
\usepackage{tikz}
%
% Image insertion
\usepackage{graphicx}

% Document content %
%
% Change title of table of contents
% \renewcommand{\contentsname}{Title}

\begin{document}

% Command `\hr` to insert horizontal rules
\newcommand{\hr}{\par\noindent\rule{\textwidth}{0.4pt}}

% Command to box and center math equations
\newcommand{\bc}[1]{
	\begin{equation*}
		\begin{boxed}
			{#1}
		\end{boxed}
	\end{equation*}
}

% Command for single line equations with a condition
\newcommand{\cond}[2]{
	\ifmmode
		{#1} \quad {#2}
	\else
		$$ {#1} \quad {#2} $$
	\fi
}

\tableofcontents

\section{Section 3.1}

\subsection{3.1.3}
Use the method of elimination to determine whether the given linear system is consistent or inconsistent. If the linear system is consistent, find the solution if it is unique; otherwise, describe the infinite solution set in terms of an arbitrary parameter $ t $.
\begin{equation*}
	\left\{
		\begin{aligned}
			7x + 5y & = -22 \\
			2x + 9y & = 24
		\end{aligned}
	\right.
\end{equation*}
\begin{align*}
	& \left\{
		\begin{aligned}
			x + \frac{5}{7}y & = -\frac{22}{7} \\
			-x - \frac{9}{2}y & = -12
		\end{aligned}
	\right. \\
	-\frac{53}{14}y & = -\frac{106}{7} \\
	y & = 4 \\
	7x + 5(4) & = -22 \\
	x & = -6
\end{align*}
\bc{\text{Unique solution:} \quad x = -6, y = 4}

\end{document}
