\documentclass{article}

% Document extensibility %
%
% Disables native paragraph indentation
\usepackage{parskip} 
%
% Provides further bullet options for lists
\usepackage{enumitem}

% Mathematical symbol and statement packages %
%
% Necessary
\usepackage{amsmath}
\usepackage{amssymb}
%
% Extensive fraction notation
\usepackage{xfrac}
%
% Generic mathematical commands
% Notable: \degree, \celcius
\usepackage{gensymb}
%
% Variable vector notation (arrow above variable)
\usepackage{esvect}
%
% Multiline boxed equations
\usepackage{empheq}
%
% SI Unit
\usepackage{siunitx}

% Graphic packages %
%
% Diagrams and illustrations
\usepackage{tikz}
%
% Image insertion
\usepackage{graphicx}

% Document content %
%
% Change title of table of contents
% \renewcommand{\contentsname}{Title}

\begin{document}

% Command `\hr` to insert horizontal rules
\newcommand{\hr}{\par\noindent\rule{\textwidth}{0.4pt}}

% Command to box and center math equations
\newcommand{\bc}[1]{
	\begin{equation*}
		\begin{boxed}
			{#1}
		\end{boxed}
	\end{equation*}
}

% Command for single line equations with a condition
\newcommand{\cond}[2]{
	\ifmmode
		{#1} \quad {#2}
	\else
		$$ {#1} \quad {#2} $$
	\fi
}

\tableofcontents

\section{Section 3.1}

\subsection{3.1.3}
Use the method of elimination to determine whether the given linear system is consistent or inconsistent. If the linear system is consistent, find the solution if it is unique; otherwise, describe the infinite solution set in terms of an arbitrary parameter $ t $.
\begin{equation*}
	\left\{
		\begin{aligned}
			7x + 5y & = -22 \\
			2x + 9y & = 24
		\end{aligned}
	\right.
\end{equation*}
\begin{align*}
	& \left\{
		\begin{aligned}
			x + \frac{5}{7}y & = -\frac{22}{7} \\
			-x - \frac{9}{2}y & = -12
		\end{aligned}
	\right. \\
	-\frac{53}{14}y & = -\frac{106}{7} \\
	y & = 4 \\
	7x + 5(4) & = -22 \\
	x & = -6
\end{align*}
\bc{\text{Unique solution:} \quad x = -6, y = 4}

\subsection{3.1.5}
Use the method of elimination to determine whether the given linear system is consistent or inconsistent. If the linear system is consistent, find the solution if it is unique; otherwise, describe the infinite solution set in terms of an arbitrary parameter $ t $.
\begin{equation*}
	\left\{
		\begin{aligned}
			x + 5y & = 18 \\
			3x + 15y & = 55
		\end{aligned}
	\right.
\end{equation*}
\begin{align*}
	& \left\{
		\begin{aligned}
			-3x - 15y & = -54 \\
			3x + 15y & = 55
		\end{aligned}
	\right. \\
	0 + 0 & = 1
\end{align*}
\bc{\text{Linear system is inconsistent and no solution exists}}

\subsection{3.1.7}
Use the method of elimination to determine whether the given linear system is consistent or inconsistent. If the linear system is consistent, find the solution if it is unique; otherwise, describe the infinite solution set in terms of an arbitrary parameter
\begin{equation*}
	\left\{
		\begin{aligned}
			x - 4y & = -8 \\
			-2x + 8y & = 16
		\end{aligned}
	\right.
\end{equation*}
\begin{align*}
	\left\{
		\begin{aligned}
			2x - 8y & = -16 \\
			-2x + 8y & = 16
		\end{aligned}
	\right. \\
	0 & = 0 \\
	x - 2t & = -15 \\
	x & = 2t - 15
\end{align*}
\bc{\text{Infinite many solutions}, x = 2t - 15}

\subsection{3.1.9}
Use the method of elimination to determine whether the given linear system is consistent or inconsistent. If the linear system is consistent, find the solution if it is unique; otherwise, describe the infinite solution set in terms of an arbitrary parameter $ t $.
\begin{equation*}
	\left\{
		\begin{aligned}
			x + 4y + z & = -3 \\
			2x + y - 4z & = -16 \\
			x + 6y + 2z & = -3
		\end{aligned}
	\right.
\end{equation*}
\begin{align*}
	\left\{
		\begin{aligned}
			x + 4y + z & = -3 \\
			x + 6y + 2z & = -3
		\end{aligned}
	\right. \\
	-2y - z & = 0 \\
	z & = -2y \\
	\left\{
		\begin{aligned}
			2x + 8y + 2z & = -6 \\
			2x + y - 4z & = -16 \\
		\end{aligned}
	\right. \\
	7y + 6z & = 10 \\
	7y + 6(-2y) & = 10 \\
	y & = -2 \\
	z & = -2(-2) \\
	z & = 4 \\
	x + 4(-2) + 4 & = -3 \\
	x & = 1
\end{align*}
\bc{x = 1, y = -2, z = 4}

\subsection{3.1.19}
Use the method of elimination to determine whether the given linear system is consistent or inconsistent. If the linear system is consistent, find the solution if it is unique; otherwise, describe the infinite solution set in terms of an arbitrary parameter $ t $.
\begin{equation*}
	\left\{
		\begin{aligned}
			x - 2y + 2 & = -1 \\
			2x - y - 8z & = 31 \\
			x - y - 2z & = 10
		\end{aligned}
	\right.
\end{equation*}
\begin{align*}
	x & = 2y - 2z - 1 \\
	2(2y - 2z - 1) - y - 8z & = 31 \\
	y & = 4z + 11 \\
	x & = 2(4z + 11) - 2z - 1 \\
	x & = 6z + 21 \\
	(6z + 21) - (4z + 11) - 2z & = 10 \\
	(6z - 4z - 2z) & = (-21 + 11 + 10) \\
	0 & = 0, \quad \text{Infinite many solutions} \\
	  & \left\{
		\begin{aligned}
			x & = 6t + 21 \\
			y & = 4t + 11
		\end{aligned}
	\right.
\end{align*}
\bc{
	\left\{
		\begin{aligned}
			x & = 6t + 21 \\
			y & = 4t + 11
		\end{aligned}
	\right.
}

\subsection{3.1.23}
A second-order differential equation and its general solution $ y(x) $ are given. Determine the constants $ A $ and $ B $ so as to find a solution of the differential equation that satisfies the given initial conditions involving $ y(0) $ and $ y'(0) $.
$$ y'' + 9y = 0, y(x) = A\cos(3x) + B\sin(3x), y(0) = 3, y'(0) = 27 $$
\begin{align*}
	y(x) & = A\cos(3x) + B\sin(3x) \\
	y'(x) & = -3A\sin(3x) + 3B\cos(3x) \\
	y''(x) & = -9A\cos(3x) - 9B\sin(3x)
\end{align*}
\begin{align*}
	y(0) & = A\cos(3(0)) + B\sin(3(0)) = 3 \\
	A & = 3 \\
	y'(0) & = -3(3)\sin(3(0)) + 3B\cos(3(0)) = 27 \\
	B & = 9
\end{align*}
\bc{A = 3, B = 9}

\subsection{3.1.25}
A second-order differential equation and its general solution $ y(x) $ are given. Determine the constants $ A $ and $ B $ so as to find a solution of the differential equation that satisfies the given initial conditions involving $ y(0) $ and $ y'(0) $.
$$ y'' - 9y = 0, y(x) = Ae^{3x} + Be^{-3x}, y(0) = 12, y'(0) = 6 $$
\begin{align*}
	y(x) & = Ae^{3x} + Be^{-3x} \\
	y'(x) & = 3Ae^{3x} - 3Be^{-3x} \\
	y''(x) & = 9Ae^{3x} + 9Be^{-3x}
\end{align*}
\begin{align*}
	y(0) & = Ae^{3(0)} + Be^{-3(0)} = 12 \\
	A + B & = 12 \\
	A & = 12 - B \\
	y'(0) & = 3Ae^{3(0)} - 3Be^{-3(0)} = 6 \\
	3A - 3B & = 6 \\
	3(12 - B) - 3B & = 6 \\
	B & = 5 \\
	A + (5) & = 12 \\
	A & = 7
\end{align*}
\bc{A = 7, B = 5}

\end{document}
