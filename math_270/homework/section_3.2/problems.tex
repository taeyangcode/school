\documentclass{article}

% Document extensibility %
%
% Disables native paragraph indentation
\usepackage{parskip} 
%
% Provides further bullet options for lists
\usepackage{enumitem}

% Mathematical symbol and statement packages %
%
% Necessary
\usepackage{amsmath}
\usepackage{amssymb}
%
% Extensive fraction notation
\usepackage{xfrac}
%
% Generic mathematical commands
% Notable: \degree, \celcius
\usepackage{gensymb}
%
% Variable vector notation (arrow above variable)
\usepackage{esvect}
%
% Multiline boxed equations
\usepackage{empheq}
%
% SI Unit
\usepackage{siunitx}

% Graphic packages %
%
% Diagrams and illustrations
\usepackage{tikz}
%
% Image insertion
\usepackage{graphicx}

% Document content %
%
% Change title of table of contents
% \renewcommand{\contentsname}{Title}

\begin{document}

% Command `\hr` to insert horizontal rules
\newcommand{\hr}{\par\noindent\rule{\textwidth}{0.4pt}}

% Command to box and center math equations
\newcommand{\bc}[1]{
	\begin{equation*}
		\begin{boxed}
			{#1}
		\end{boxed}
	\end{equation*}
}

% Command for single line equations with a condition
\newcommand{\cond}[2]{
	\ifmmode
		{#1} \quad {#2}
	\else
		$$ {#1} \quad {#2} $$
	\fi
}

\tableofcontents

\section{Section 3.2}

\subsection{3.2.1}
The following linear system is in echelon form. Solve the linear system by back substitution.
\begin{equation*}
	\left\{
		\begin{aligned}
			x_1 + x_2 + 2x_3 & = 2 \\
			x_2 + 3x_3 & = 4 \\
			x_3 & = 3
		\end{aligned}
	\right.
\end{equation*}
\begin{align*}
	x_2 + 3(3) & = 4 \\
	x_2 & = -5 \\
	x_1 + (-5) + 2(3) & = 2 \\
	x_1 & = 1
\end{align*}
\bc{x_1 = 1, x_2 = -5, x_3 = 3}

The following linear system is in echelon form. Solve the linear system by back substitution.
\begin{equation*}
	\left\{
		\begin{aligned}
			x_1 - 9x_2 + x_3 & = 18 \\
			x_2 + x_3 & = 2
		\end{aligned}
	\right.
\end{equation*}
\begin{align*}
	x_1 - 10x_2 & = 16 \\
	x_2 & = 2 - x_3
\end{align*}

\end{document}
