\documentclass{article}

% Document extensibility %
%
% Disables native paragraph indentation
\usepackage{parskip} 
%
% Provides further bullet options for lists
\usepackage{enumitem}

% Mathematical symbol and statement packages %
%
% Necessary
\usepackage{amsmath}
\usepackage{amssymb}
%
% Extensive fraction notation
\usepackage{xfrac}
%
% Generic mathematical commands
% Notable: \degree, \celcius
\usepackage{gensymb}
%
% Variable vector notation (arrow above variable)
\usepackage{esvect}
%
% Multiline boxed equations
\usepackage{empheq}
%
% SI Unit
\usepackage{siunitx}
%
% More intuitive arrays/matrices
\usepackage{array}

% Graphic packages %
%
% Diagrams and illustrations
\usepackage{tikz}
%
% Image insertion
\usepackage{graphicx}

% Document content %
%
% Change title of table of contents
% \renewcommand{\contentsname}{Title}

\begin{document}

% Command `\hr` to insert horizontal rules
\newcommand{\hr}{\par\noindent\rule{\textwidth}{0.4pt}}

% Command to box and center math equations
\newcommand{\bc}[1]{
	\begin{equation*}
		\begin{boxed}
			{#1}
		\end{boxed}
	\end{equation*}
}

% Command for single line equations with a condition
\newcommand{\cond}[2]{
	\ifmmode
		{#1} \quad {#2}
	\else
		$$ {#1} \quad {#2} $$
	\fi
}

\tableofcontents

\section{Section 3.3}

\subsection{3.3.1}
Find the reduced echelon form of the given matrix.
\begin{equation*}
	A = \begin{bmatrix}
		1 & 5 \\
		7 & 36
	\end{bmatrix}
\end{equation*}
\begin{align*}
	A_2 & = A_2 - 7A_1 \\
	A & = \begin{bmatrix}
		1 & 5 \\
		0 & 1
	\end{bmatrix}
\end{align*}
\begin{align*}
	A_1 & = A_1 - 5A_2 \\
	A & = \begin{bmatrix}
		1 & 0 \\
		0 & 1
	\end{bmatrix}
\end{align*}
\bc{
	A = \begin{bmatrix}
		1 & 0 \\
		0 & 1
	\end{bmatrix}
}

\subsection{3.3.3}
Find the reduced echelon form of the given matrix.
\begin{equation*}
	A = \begin{bmatrix}
		3 & 10 & 12 \\
		2 & 7 & 9
	\end{bmatrix}
\end{equation*}
\begin{align*}
	A_2 & = 3A_2 - 2A_1 \\
	A & = \begin{bmatrix}
		3 & 10 & 12 \\
		0 & 1 & 3
	\end{bmatrix}
\end{align*}
\begin{align*}
	A_1 & = A_1 - 10A_2 \\
	A & = \begin{bmatrix}
		3 & 0 & -18 \\
		0 & 1 & 3
	\end{bmatrix}
\end{align*}
\begin{align*}
	A_1 & = \frac{1}{3}A_1 \\
	A & = \begin{bmatrix}
		1 & 0 & -6 \\
		0 & 1 & 3
	\end{bmatrix}
\end{align*}
\bc{
	A = \begin{bmatrix}
		1 & 0 & -6 \\
		0 & 1 & 3
	\end{bmatrix}
}

\subsection{3.3.21}
Solve using Gauss-Jordan elimination.
\begin{equation*}
	\left[
		\begin{array}{ccc|c}
			2 & 5 & -12 & 5 \\
			4 & 37 & -85 & 84 \\
			1 & 7 & -16 & 15
		\end{array}
	\right]
\end{equation*}
\begin{align*}
	E_2 & = E_2 - 2E_1 \\
	E & = \left[
		\begin{array}{ccc|c}
			2 & 5 & -12 & 5 \\
			0 & 27 & -61 & 74 \\
			1 & 7 & -16 & 15
		\end{array}
	\right]
\end{align*}
\begin{align*}
	E_3 & = 2E_3 - E_1 \\
	E & = \left[
		\begin{array}{ccc|c}
			2 & 5 & -12 & 5 \\
			0 & 27 & -61 & 74 \\
			0 & 9 & -20 & 25
		\end{array}
	\right]
\end{align*}
\begin{align*}
	E_1 & = \frac{1}{2}E_1 \\
	E & = \left[
		\begin{array}{ccc|c}
			1 & \frac{5}{2} & -6 & \frac{5}{2} \\
			0 & 27 & -61 & 74 \\
			0 & 9 & -20 & 25
		\end{array}
	\right]
\end{align*}
\begin{align*}
	E_3 & = E_3 - \frac{1}{3}E_2 \\
	E & = \left[
		\begin{array}{ccc|c}
			1 & \frac{5}{2} & -6 & \frac{5}{2} \\
			0 & 27 & -61 & 74 \\
			0 & 0 & \frac{1}{3} & \frac{1}{3}
		\end{array}
	\right]
\end{align*}
\begin{align*}
	E_2 & = \frac{1}{27}E_2 \\
	E & = \left[
		\begin{array}{ccc|c}
			1 & \frac{5}{2} & -6 & \frac{5}{2} \\
			0 & 1 & -\frac{61}{27} & \frac{74}{27} \\
			0 & 0 & \frac{1}{3} & \frac{1}{3}
		\end{array}
	\right]
\end{align*}
\begin{align*}
	E_3 & = 3E_3 \\
	E & = \left[
		\begin{array}{ccc|c}
			1 & \frac{5}{2} & -6 & \frac{5}{2} \\
			0 & 1 & -\frac{61}{27} & \frac{74}{27} \\
			0 & 0 & 1 & -\frac{3}{23}
		\end{array}
	\right]
\end{align*}

\end{document}
