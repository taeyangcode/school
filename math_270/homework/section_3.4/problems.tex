\documentclass{article}

% Document extensibility %
%
% Disables native paragraph indentation
\usepackage{parskip} 
%
% Provides further bullet options for lists
\usepackage{enumitem}

% Mathematical symbol and statement packages %
%
% Necessary
\usepackage{amsmath}
\usepackage{amssymb}
%
% Extensive fraction notation
\usepackage{xfrac}
%
% Generic mathematical commands
% Notable: \degree, \celcius
\usepackage{gensymb}
%
% Variable vector notation (arrow above variable)
\usepackage{esvect}
%
% Multiline boxed equations
\usepackage{empheq}
%
% SI Unit
\usepackage{siunitx}
%
% More intuitive arrays/matrices
\usepackage{array}

% Graphic packages %
%
% Diagrams and illustrations
\usepackage{tikz}
%
% Image insertion
\usepackage{graphicx}

% Document content %
%
% Change title of table of contents
% \renewcommand{\contentsname}{Title}

\begin{document}

% Command `\hr` to insert horizontal rules
\newcommand{\hr}{\par\noindent\rule{\textwidth}{0.4pt}}

% Command to box and center math equations
\newcommand{\bc}[1]{
	\begin{equation*}
		\begin{boxed}
			{#1}
		\end{boxed}
	\end{equation*}
}

% Command for single line equations with a condition
\newcommand{\cond}[2]{
	\ifmmode
		{#1} \quad {#2}
	\else
		$$ {#1} \quad {#2} $$
	\fi
}

\section{Section 3.4}

\subsection{3.4.1}
Find $ 2A + 3B $
$$
	A = \begin{bmatrix}
		5 & 5 \\
		4 & 3
	\end{bmatrix} \quad
	B = \begin{bmatrix}
		-6 & 8 \\
		1 & -1
	\end{bmatrix}
$$
\begin{align*}
	2A + 3B & = \begin{bmatrix}
		-8 & 34 \\
		11 & 3
	\end{bmatrix}
\end{align*}
\bc{
	2A + 3B = \begin{bmatrix}
		-8 & 34 \\
		11 & 3
	\end{bmatrix}
}

\subsection{3.4.2}
Two matrices $ A $ and $ B $ and two numbers $ c $ and $ d $ are given. Compute the matrix $ cA + dB $.
$$
	A = \begin{bmatrix}
		2 & 0 & -2 \\
		-1 & 6 & 6
	\end{bmatrix},
	B = \begin{bmatrix}
		-2 & 2 & 3 \\
		5 & 2 & 5
	\end{bmatrix},
	c = 6, d = -4
$$
\begin{align*}
	cA + dB & = \begin{bmatrix}
		12 & 0 & -12 \\
		-6 & 36 & 36
	\end{bmatrix}
	+ \begin{bmatrix}
		8 & -8 & -12 \\
		-20 & -8 & -20
	\end{bmatrix} \\
	cA + dB & = \begin{bmatrix}
		20 & -8 & -24 \\
		-26 & 28 & 16
	\end{bmatrix}
\end{align*}
\bc{
	cA + dB = \begin{bmatrix}
		20 & -8 & -24 \\
		-26 & 28 & 16
	\end{bmatrix}
}

\end{document}
