\documentclass{article}

% Document extensibility %
%
% Disables native paragraph indentation
\usepackage{parskip} 
%
% Provides further bullet options for lists
\usepackage{enumitem}

% Mathematical symbol and statement packages %
%
% Necessary
\usepackage{amsmath}
\usepackage{amssymb}
%
% Extensive fraction notation
\usepackage{xfrac}
%
% Generic mathematical commands
% Notable: \degree, \celcius
\usepackage{gensymb}
%
% Variable vector notation (arrow above variable)
\usepackage{esvect}
%
% Multiline boxed equations
\usepackage{empheq}
%
% SI Unit
\usepackage{siunitx}
%
% More intuitive arrays/matrices
\usepackage{array}

% Graphic packages %
%
% Diagrams and illustrations
\usepackage{tikz}
%
% Image insertion
\usepackage{graphicx}

% Document content %
%
% Change title of table of contents
% \renewcommand{\contentsname}{Title}

\DeclareMathOperator{\cof}{cof}
\DeclareMathOperator{\adj}{adj}

\begin{document}

% Command `\hr` to insert horizontal rules
\newcommand{\hr}{\par\noindent\rule{\textwidth}{0.4pt}}

% Command to box and center math equations
\newcommand{\bc}[1]{
	\begin{equation*}
		\begin{boxed}
			{#1}
		\end{boxed}
	\end{equation*}
}

% Command for single line equations with a condition
\newcommand{\cond}[2]{
	\ifmmode
		{#1} \quad {#2}
	\else
		$$ {#1} \quad {#2} $$
	\fi
}

\tableofcontents

\section{Section 3.5}

\subsection{3.5.1}
First use the formula $ A^{-1} = \frac{1}{ad - bc} \begin{bmatrix} d & -b \\ -c & a \end{bmatrix} $ for the general $ 2 \times 2 $ matrix $ A = \begin{bmatrix} a & b \\ c & d \end{bmatrix} $ to find $ A^{-1} $. Then use $ A^{-1} $ to solve the system $ Ax = b $.
\begin{equation*}
	A =
		\begin{bmatrix}
			7 & 6 \\
			8 & 7
		\end{bmatrix};
	b =
		\begin{bmatrix}
			17 \\
			19
		\end{bmatrix}
\end{equation*}
\begin{align*}
	\det(A) & = (7 \cdot 7) - (6 \cdot 8) = 1 \\
	\cof(A) & =
		\begin{bmatrix}
			7 & 6 \\
			8 & 7
		\end{bmatrix} \\
	\adj{A} & =
		\begin{bmatrix}
			7 & -6 \\
			-8 & 7
		\end{bmatrix}
\end{align*}
\begin{align*}
	A^{-1} & = \frac{1}{\det(A)}\adj(A) \\
	A^{-1} & = \frac{1}{1}
		\begin{bmatrix}
			7 & -6 \\
			-8 & 7
		\end{bmatrix} \\
	A^{-1} & = 
		\begin{bmatrix}
			7 & -6 \\
			-8 & 7
		\end{bmatrix}
\end{align*}
\begin{align*}
	Ax & = b \\
	x & = bA^{-1} \\
	x & =
		\begin{bmatrix}
			7 & -6 \\
			-8 & 7
		\end{bmatrix}
		\begin{bmatrix}
			17 \\
			19
		\end{bmatrix} \\
	x & =
		\begin{bmatrix}
			5 \\
			-3
		\end{bmatrix}
\end{align*}
\bc{
	x =
		\begin{bmatrix}
			5 \\
			-3
		\end{bmatrix}
}

\subsection{3.5.5}
Solve $ Ax = b $
\begin{equation*}
	A =
		\begin{bmatrix}
			4 & 3 \\
			6 & 5
		\end{bmatrix};
	b =
		\begin{bmatrix}
			6 \\
			4
		\end{bmatrix}
\end{equation*}
\begin{align*}
	A^{-1} & = \frac{1}{\det(A)}\adj(A)
\end{align*}
\begin{align*}
	\det(A) & = 4 \cdot 5 - 3 \cdot 6 = 2 \\
	\cof(A) & =
		\begin{bmatrix}
			5 & 6 \\
			3 & 4
		\end{bmatrix} \\
	\adj(A) & =
		\begin{bmatrix}
			5 & -3 \\
			-6 & 4
		\end{bmatrix} \\
	A^{-1} & = \frac{1}{2}
		\begin{bmatrix}
			5 & -3 \\
			-6 & 4
		\end{bmatrix} \\
	A^{-1} & =
		\begin{bmatrix}
			\frac{5}{2} & -\frac{3}{2} \\
			-3 & 2
		\end{bmatrix}
\end{align*}
\begin{align*}
	x & =
		\begin{bmatrix}
			\frac{5}{2} & -\frac{3}{2} \\
			-3 & 2
		\end{bmatrix}
		\begin{bmatrix}
			6 \\
			4
		\end{bmatrix} \\
	x & =
		\begin{bmatrix}
			(\frac{5}{2} \cdot 6) - (-\frac{3}{2} \cdot 4) \\
			(-3 \cdot 6) - (2 \cdot 4)
		\end{bmatrix} \\
	x & =
		\begin{bmatrix}
			9 \\
			-10
		\end{bmatrix}
\end{align*}
\bc{
	x =
		\begin{bmatrix}
			9 \\
			-10
		\end{bmatrix}
}

\subsection{3.5.8}
Solve $ Ax = b $
\begin{equation*}
	A =
		\begin{bmatrix}
			7 & 13 \\
			5 & 10
		\end{bmatrix};
	b =
		\begin{bmatrix}
			6 \\
			2
		\end{bmatrix}
\end{equation*}
\begin{align*}
	\det(A) & = (7 \cdot 10) - (13 \cdot 5) = 5 \\
	\cof(A) & =
		\begin{bmatrix}
			10 & 5 \\
			13 & 7
		\end{bmatrix} \\
	\adj(A) & =
		\begin{bmatrix}
			10 & -13 \\
			-5 & 7
		\end{bmatrix} \\
	A^{-1} & = \frac{1}{5}
		\begin{bmatrix}
			10 & -13 \\
			-5 & 7
		\end{bmatrix} \\
	A^{-1} & =
		\begin{bmatrix}
			2 & -\frac{13}{5} \\
			-1 & \frac{7}{5}
		\end{bmatrix}
\end{align*}
\begin{align*}
	x & =
		\begin{bmatrix}
			2 & -\frac{13}{5} \\
			-1 & \frac{7}{5}
		\end{bmatrix}
		\begin{bmatrix}
			6 \\
			2
		\end{bmatrix} \\
	x & =
		\begin{bmatrix}
			(2 \cdot 6) + (-\frac{13}{5} \cdot 2) \\
			(-1 \cdot 6) + (\frac{7}{5} \cdot 2)
		\end{bmatrix} \\
	x & =
		\begin{bmatrix}
			\frac{34}{5} \\
			-\frac{16}{5}
		\end{bmatrix}
\end{align*}
\bc{
	x =
		\begin{bmatrix}
			\frac{34}{5} \\
			-\frac{16}{5}
		\end{bmatrix}
}

\subsection{3.5.11}
Adjoin $ I $ on the right of $ A $, then use row operations to find the inverse $ A^{-1} $ of the given matrix $ A $.
\begin{equation*}
	\begin{bmatrix}
		1 & 15 & 1 \\
		2 & 15 & 0 \\
		2 & 22 & 1
	\end{bmatrix}
\end{equation*}
\begin{align*}
	[I|A^{-1}] & =
		\left[ \begin{array}{ccc|ccc}
			1 & 15 & 1 & 1 & 0 & 0 \\
			2 & 15 & 0 & 0 & 1 & 0 \\
			2 & 22 & 1 & 0 & 0 & 1
		\end{array} \right]
\end{align*}
\begin{align*}
	I_2 & = I_2 - 2I_1 \\
	I_3 & = I_3 - 2I_1 \\
	[I|A^{-1}] & =
		\left[ \begin{array}{ccc|ccc}
			1 & 15 & 1 & 1 & 0 & 0 \\
			0 & -15 & -2 & -2 & 1 & 0 \\
			0 & -8 & -1 & -2 & 0 & 1
		\end{array} \right]
\end{align*}
\begin{align*}
	I_3 & = I_3 - \frac{8}{15}I_2 \\
	[I|A^{-1}] & =
		\left[ \begin{array}{ccc|ccc}
			1 & 15 & 1 & 1 & 0 & 0 \\
			0 & -15 & -2 & -2 & 1 & 0 \\
			0 & 0 & \frac{1}{15} & -\frac{14}{15} & -\frac{8}{15} & 1
		\end{array} \right]
\end{align*}
\begin{align*}
	I_1 & = I_1 + I_2 \\
	[I|A^{-1}] & =
		\left[ \begin{array}{ccc|ccc}
			1 & 0 & -1 & -1 & 1 & 0 \\
			0 & -15 & -2 & -2 & 1 & 0 \\
			0 & 0 & \frac{1}{15} & -\frac{14}{15} & -\frac{8}{15} & 1
		\end{array} \right]
\end{align*}
\begin{align*}
	I_1 & = I_1 + 15I_3 \\
	[I|A^{-1}] & =
		\left[ \begin{array}{ccc|ccc}
			1 & 0 & 0 & -15 & -7 & 15 \\
			0 & -15 & -2 & -2 & 1 & 0 \\
			0 & 0 & \frac{1}{15} & -\frac{14}{15} & -\frac{8}{15} & 1
		\end{array} \right]
\end{align*}
\begin{align*}
	I_2 & = I_2 + 30I_3 \\
	[I|A^{-1}] & =
		\left[ \begin{array}{ccc|ccc}
			1 & 0 & 0 & -15 & -7 & 15 \\
			0 & -15 & 0 & -30 & -15 & 30 \\
			0 & 0 & \frac{1}{15} & -\frac{14}{15} & -\frac{8}{15} & 1
		\end{array} \right]
\end{align*}
\begin{align*}
	I_2 & = -\frac{1}{15}I_2 \\
	I_3 & = 15I_3 \\
	[I|A^{-1}] & =
		\left[ \begin{array}{ccc|ccc}
			1 & 0 & 0 & -15 & -7 & 15 \\
			0 & 1 & 0 & 2 & 1 & -2 \\
			0 & 0 & 1 & -14 & -8 & 15
		\end{array} \right]
\end{align*}
\bc{
	[I|A^{-1}] =
		\left[ \begin{array}{ccc|ccc}
			1 & 0 & 0 & -15 & -7 & 15 \\
			0 & 1 & 0 & 2 & 1 & -2 \\
			0 & 0 & 1 & -14 & -8 & 15
		\end{array} \right]
}

\subsection{3.5.23}
Find a matrix $ X $ such that $ AX = B $
\begin{equation*}
	A =
		\begin{bmatrix}
			4 & 3 \\
			5 & 4
		\end{bmatrix},
	B =
		\begin{bmatrix}
			1 & 3 & -4 \\
			-1 & -2 & 4
		\end{bmatrix}
\end{equation*}
Find $ A^{-1} $
\begin{align*}
	\det(A) & = 4 \cdot 4 - 3 \cdot 5 = 1 \\
	\adj(A) & =
		\begin{bmatrix}
			4 & -3 \\
			-5 & 4
		\end{bmatrix} \\
	A^{-1} & =
		\begin{bmatrix}
			4 & -3 \\
			-5 & 4
		\end{bmatrix}
\end{align*}
Find $ A^{-1} \cdot B $
\begin{align*}
	X & =
		\begin{bmatrix}
			4 & -3 \\
			-5 & 4
		\end{bmatrix}
		\begin{bmatrix}
			1 & 3 & -4 \\
			-1 & -2 & 4
		\end{bmatrix} \\
	X & =
		\begin{bmatrix}
			4 \cdot 1 + -3 \cdot -1 & 4 \cdot 3 + -3 \cdot -2 & 4 \cdot -4 + -3 \cdot 4 \\
			-5 \cdot 1 + 4 \cdot -1 & -5 \cdot 3 + 4 \cdot -2 & -5 \cdot -4 + 4 \cdot 4
		\end{bmatrix} \\
	X & =
		\begin{bmatrix}
			7 & 18 & -28 \\
			-9 & -23 & 36
		\end{bmatrix}
\end{align*}
\bc{
	X =
		\begin{bmatrix}
			7 & 18 & -28 \\
			-9 & -23 & 36
		\end{bmatrix}
}

\end{document}
