\documentclass{article}

% Document extensibility %
%
% Disables native paragraph indentation
\usepackage{parskip} 
%
% Provides further bullet options for lists
\usepackage{enumitem}

% Mathematical symbol and statement packages %
%
% Necessary
\usepackage{amsmath}
\usepackage{amssymb}
%
% Extensive fraction notation
\usepackage{xfrac}
%
% Generic mathematical commands
% Notable: \degree, \celcius
\usepackage{gensymb}
%
% Variable vector notation (arrow above variable)
\usepackage{esvect}
%
% Multiline boxed equations
\usepackage{empheq}
%
% SI Unit
\usepackage{siunitx}
%
% More intuitive arrays/matrices
\usepackage{array}

% Graphic packages %
%
% Diagrams and illustrations
\usepackage{tikz}
%
% Image insertion
\usepackage{graphicx}

% Document content %
%
% Change title of table of contents
% \renewcommand{\contentsname}{Title}

\begin{document}

% Command `\hr` to insert horizontal rules
\newcommand{\hr}{\par\noindent\rule{\textwidth}{0.4pt}}

% Command to box and center math equations
\newcommand{\bc}[1]{
	\begin{equation*}
		\begin{boxed}
			{#1}
		\end{boxed}
	\end{equation*}
}

% Command for single line equations with a condition
\newcommand{\cond}[2]{
	\ifmmode
		{#1} \quad {#2}
	\else
		$$ {#1} \quad {#2} $$
	\fi
}

\tableofcontents

\section{Section 4.1}

\subsection{4.1.1}
Find $ |a - b|, 2a + b, 3a - 4b $
\begin{equation*}
	a =
		\begin{bmatrix}
			5 \\
			5 \\
			-6
		\end{bmatrix},
	b =
		\begin{bmatrix}
			2 \\
			-2 \\
			-5
		\end{bmatrix}
\end{equation*}
\begin{align*}
	\|a - b\| & =
		\begin{bmatrix}
			3 \\
			7 \\
			-1
		\end{bmatrix} \\
			  & = \sqrt{(3)^2 + (7)^1 + (-1)^2} \\
	\|a - b\| & = \sqrt{59}
\end{align*}
\bc{\|a - b\| = \sqrt{59}}
\begin{align*}
	2a + b & =
		\begin{bmatrix}
			12 \\
			8 \\
			-17
		\end{bmatrix} \\
	2a + b & = <12,8,-17>
\end{align*}
\bc{2a + b = <12,8,-17>}
\begin{align*}
	3a - 4b & =
		\begin{bmatrix}
			7 \\
			23 \\
			2
		\end{bmatrix} \\
	3a - 4b & = <7,23,2>
\end{align*}
\bc{3a - 4b = <7,23,2>}

\subsection{4.1.3}
Find $ \|a - b\| $, $ 2a + b $, and $ 4a - 5b $.
$$ a = 5\hat{i} + 2\hat{j} + 7\hat{k}, b = 7\hat{i} + 8\hat{j} - 5\hat{k} $$
\begin{align*}
	\|a - b\| & = \sqrt{ (5 - 7)^2 + (2 - 8)^2 + (7- (-5))^2 } \\
	\|a - b\| & = 2\sqrt{46}
\end{align*}
\bc{2\sqrt{46}}
\begin{align*}
	2a + b & =
		\begin{bmatrix}
			2(5) + 7 \\
			2(2) + 8 \\
			2(7) + (-5)
		\end{bmatrix} \\
	2a + b & =
		\begin{bmatrix}
			17 \\
			12 \\
			9
		\end{bmatrix}
\end{align*}
\bc{17\hat{i} + 12\hat{j} + 9\hat{k}}
\begin{align*}
	4a - 5b & =
		\begin{bmatrix}
			4(5) - 5(7) \\
			4(2) - 5(8) \\
			4(7) - 5(-5)
		\end{bmatrix} \\
	4a - 5b & =
		\begin{bmatrix}
			-15 \\
			-32 \\
			53
		\end{bmatrix} \\
\end{align*}
\bc{-15\hat{i} - 32\hat{j} + 53\hat{k}}

\subsection{4.1.5}
Determine whether the given vectors $ u $ and $ v $ are linearly dependent or linearly independent.
$$
	u =
		\begin{bmatrix}
			4 \\
			0
		\end{bmatrix},
	v =
		\begin{bmatrix}
			3 \\
			0
		\end{bmatrix}
$$
\bc{\text{Linearly Dependent, } u = \frac{3}{4}v}

\subsection{4.1.7}
Determine whether the given vectors $ u $ and $ v $ are linearly dependent or linearly independent.
$$
	u =
		\begin{bmatrix}
			-9 \\
			9
		\end{bmatrix},
	v =
		\begin{bmatrix}
			9 \\
			9
		\end{bmatrix}
$$
\begin{align*}
	a \begin{bmatrix} -9 \\ 9 \end{bmatrix} + b \begin{bmatrix} 9 \\ 9 \end{bmatrix} & = \begin{bmatrix} 0 \\ 0 \end{bmatrix}
\end{align*}
\begin{align*}
	-9a + 9b & = 0 \\
	9b & = 9a \\
	b & = a
\end{align*}
\begin{align*}
	9a + 9b & = 0 \\
	9a + 9a & = 0 \\
	18a & = 0 \\
	a & = 0
\end{align*}
\begin{align*}
	-9a + 9b & = 0 \\
	-9(0) + 9b & = 0 \\
	b & = 0
\end{align*}
\bc{\text{Linearly Independent}}

\subsection{4.1.9}
Express $ w $ as a linear combination of $ u $ and $ v $.
$$
	u = \begin{bmatrix} 1 \\ -3 \end{bmatrix},
	v = \begin{bmatrix} -1 \\ 8 \end{bmatrix},
	w = \begin{bmatrix} 5 \\ 0 \end{bmatrix}
$$
\begin{align*}
	A & = \begin{bmatrix}
		1 & -1 & 5 \\
		-3 & 8 & 0
	\end{bmatrix}
\end{align*}
\begin{align*}
	A_2 & = A_2 + 3A_1 \\
	A & = \begin{bmatrix}
		1 & -1 & 5 \\
		0 & 5 & 15
	\end{bmatrix}
\end{align*}
\begin{align*}
	5b & = 15 \\
	b & = 3
\end{align*}
\begin{align*}
	a - b & = 5 \\
	a - (3) & = 5 \\
	a & = 8
\end{align*}
\bc{w = 8u + 3v}

\subsection{4.1.13}
Express $ w $ as a linear combination of $ u $ and $ v $.
$$
	u = \begin{bmatrix} 7 \\ 8 \end{bmatrix},
	v = \begin{bmatrix} 5 \\ 7 \end{bmatrix},
	w = \begin{bmatrix} -4 \\ -2 \end{bmatrix}
$$
\begin{align*}
	A & = \begin{bmatrix}
		7 & 5 & -4 \\
		8 & 7 & -2
	\end{bmatrix}
\end{align*}
\begin{align*}
	A_1 & = \frac{1}{7}A_1 \\
	A & = \begin{bmatrix}
		1 & \frac{5}{7} & -\frac{4}{7} \\
		8 & 7 & -2
	\end{bmatrix}
\end{align*}
\begin{align*}
	A_2 & = A_2 - 8A_1 \\
	A & = \begin{bmatrix}
		1 & \frac{5}{7} & -\frac{4}{7} \\
		0 & \frac{9}{7} & \frac{18}{7}
	\end{bmatrix}
\end{align*}
\begin{align*}
	\frac{9}{7}b & = \frac{18}{7} \\
	b & = 2
\end{align*}
\begin{align*}
	a + \frac{5}{7}b & = -\frac{4}{7} \\
	a + \frac{5}{7}(2) & = -\frac{4}{7} \\
	a & = -2
\end{align*}
\bc{w = -2u + 2v}

\subsection{4.1.15}
Use the theorem for three linearly independent vectors (that is, calculate a determinant) to determine whether the given vectors $ u $, $ v $, and $ w $ are linearly dependent or independent.
$$
	u = \begin{bmatrix} 3 \\ -1 \\ 4 \end{bmatrix},
	v = \begin{bmatrix} 4 \\ 6 \\ -7 \end{bmatrix},
	w = \begin{bmatrix} 7 \\ 5 \\ -3 \end{bmatrix}
$$
\begin{align*}
	\det(A) & = \begin{bmatrix}
		3 & 4 & 7 \\
		-1 & 6 & 5 \\
		4 & -7 & -3
	\end{bmatrix} \\
	\det(A) & =
		3 \begin{bmatrix} 6 & 5 \\ -7 & -3 \end{bmatrix}
		- 4 \begin{bmatrix} -1 & 5 \\ 4 & -3 \end{bmatrix}
		+ 7 \begin{bmatrix} -1 & 6 \\ 4 & -7 \end{bmatrix} \\
	\det(A) & = 3(6 \cdot -3 - 5 \cdot -7) - 4(-1 \cdot -3 - 5 \cdot 4) + 7(-1 \cdot -7 - 6 \cdot 4) \\
	\det(A) & = 0
\end{align*}
\bc{\det(A) = 0}

\subsection{4.1.19}
Solve a linear system to determine whether the given vectors $ u $, $ v $, and $ w $ are linearly independent or dependent. If they are linearly dependent, find scalars $ a $, $ b $, and $ c $ not all zero such that $ au + bv + cw = 0 $.
\begin{equation*}
	u = \begin{bmatrix} 2 \\ 0 \\ 1 \end{bmatrix},
	v = \begin{bmatrix} -3 \\ 1 \\ -1 \end{bmatrix},
	w = \begin{bmatrix} -6 \\ -2 \\ -4 \end{bmatrix}
\end{equation*}
\begin{align*}
	A & = \begin{bmatrix}
		2 & -3 & -6 \\
		0 & 1 & -2 \\
		1 & -1 & -4
	\end{bmatrix}
\end{align*}
\begin{align*}
	A_1 & = \frac{1}{2}A_1 \\
	A_3 & = A_3 - A_1 \\
	A & = \begin{bmatrix}
		1 & -\frac{3}{2} & -3 \\
		0 & 1 & -2 \\
		0 & \frac{1}{2} & -1
	\end{bmatrix}
\end{align*}
\begin{align*}
	A_3 & = A_3 - \frac{1}{2}A_2 \\
	A & = \begin{bmatrix}
		1 & -\frac{3}{2} & -3 \\
		0 & 1 & -2 \\
		0 & 0 & 0
	\end{bmatrix}
\end{align*}
\begin{align*}
	b - 2c & = 0 \\
	b & = 2c
\end{align*}
\begin{align*}
	a - \frac{3}{2}b - 3c & = 0 \\
	a - \frac{3}{2}(2c) - 3c & = 0 \\
	a & = 6c
\end{align*}
If $ c = 1 $, then
\begin{align*}
	a & = 6(1) = 6 \\
	b & = 2(1) = 2
\end{align*}
\bc{6u + 2v + w = 0}

\subsection{4.1.27}
Express the vector $ t $ as a linear combination of the vectors $ u $, $ v $, and $ w $.
\begin{equation*}
	t = \begin{bmatrix} 0 \\ 0 \\ 14 \end{bmatrix},
	u = \begin{bmatrix} 1 \\ 5 \\ 2 \end{bmatrix},
	v = \begin{bmatrix} -1 \\ -2 \\ 3 \end{bmatrix},
	w = \begin{bmatrix} 4 \\ -1 \\ -13 \end{bmatrix}
\end{equation*}
\begin{align*}
	A & = \left[ \begin{array}{c c c | c}
		1 & -1 & 3 & 0 \\
		3 & -1 & -3 & 0 \\
		4 & 2 & -10 & 14
	\end{array} \right]
\end{align*}
\begin{align*}
	A_3 & = A_3 - 4A_1 \\
	A & = \left[ \begin{array}{c c c | c}
		1 & -1 & 3 & 0 \\
		0 & 2 & -12 & 0 \\
		0 & 6 & -22 & 14
	\end{array} \right]
\end{align*}
\begin{align*}
	A_3 & = A_3 - 3A_2 \\
	A & = \left[ \begin{array}{c c c | c}
		1 & -1 & 3 & 0 \\
		0 & 2 & -12 & 0 \\
		0 & 0 & 14 & 14
	\end{array} \right]
\end{align*}
\begin{align*}
	14c & = 14 \\
	c & = 1
\end{align*}
\begin{align*}
	2b - 12c & = 0 \\
	2b - 12(1) & = 0 \\
	b & = 6
\end{align*}
\begin{align*}
	a - b + 3c & = 0 \\
	a - (6) + 3(1) & = 0 \\
	a & = 3
\end{align*}
\bc{t = 3u + 6v + w}

\subsection{4.1.29}
Show that the given set $ V $ is closed under addition and multiplication by scalars and is therefore a subspace of $ \mathbb{R}^3 $.
$ V $ is the set of all $ \begin{bmatrix} x \\ y \\ z \end{bmatrix} $ such that $ x = 0 $.
Let $ a = \begin{bmatrix} 0 \\ y \\ z \end{bmatrix} $ and $ b = \begin{bmatrix} 0 \\ v \\ w \end{bmatrix} $ be two vectors in $ V $. Find their sum $ a + b $.
\begin{align*}
	a + b & =
		\begin{bmatrix}
			0 + 0 \\
			y + v \\
			z + w
		\end{bmatrix} \\
	a + b & =
		\begin{bmatrix}
			0 \\
			y + v \\
			z + w
		\end{bmatrix}
\end{align*}
\bc{
	a + b =
		\begin{bmatrix}
			0 \\
			y + v \\
			z + w
		\end{bmatrix}
}

\end{document}
