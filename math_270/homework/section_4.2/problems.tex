\documentclass{article}

% Document extensibility %
%
% Disables native paragraph indentation
\usepackage{parskip} 
%
% Provides further bullet options for lists
\usepackage{enumitem}

% Mathematical symbol and statement packages %
%
% Necessary
\usepackage{amsmath}
\usepackage{amssymb}
%
% Extensive fraction notation
\usepackage{xfrac}
%
% Generic mathematical commands
% Notable: \degree, \celcius
\usepackage{gensymb}
%
% Variable vector notation (arrow above variable)
\usepackage{esvect}
%
% Multiline boxed equations
\usepackage{empheq}
%
% SI Unit
\usepackage{siunitx}
\usepackage{physunits}
%
% More intuitive arrays/matrices
\usepackage{array}
%
% Linear Equations
\usepackage{systeme}

% Graphic packages %
%
% Diagrams and illustrations
\usepackage{tikz}
%
% Image insertion
\usepackage{graphicx}

% Document content %
%
% Change title of table of contents
% \renewcommand{\contentsname}{Title}

\DeclareMathOperator{\rref}{rref}

\begin{document}

% Command `\hr` to insert horizontal rules
\newcommand{\hr}{\par\noindent\rule{\textwidth}{0.4pt}}

% Command to box and center math equations
\newcommand{\bc}[1]{
	\begin{equation*}
		\begin{boxed}
			{#1}
		\end{boxed}
	\end{equation*}
}

% Command for single line equations with a condition
\newcommand{\cond}[2]{
	\ifmmode
		{#1} \quad {#2}
	\else
		$$ {#1} \quad {#2} $$
	\fi
}

\newcommand{\matr}[1]{\mathbf{#1}}

\tableofcontents

\section{Section 4.2}

\subsection{4.2.15}

Find the solution vectors $ \vec{u} $ and $ \vec{v} $ such that the solution space is the set of all linear combinations of the form $ s\vec{u} + t\vec{v} $.
\begin{equation*}
	\systeme{
		x_1 - 4x_2 + x_3 - 8x_4 = 0,
		x_1 + 2x_2 + x_3 + 16x_4 = 0,
		x_1 + x_2 + x_3 + 12x_4 = 0
	}
\end{equation*}

Find $ \rref(\matr{A}) $
\begin{align*}
	\matr{A} & = \begin{bmatrix}
		1 & -4 & 1 & -8 \\
		1 & 2 & 1 & 16 \\
		1 & 1 & 1 & 12
	\end{bmatrix}
\end{align*}
\begin{align*}
	\matr{A}_2 & = \matr{A}_2 - \matr{A}_1 \\
	\matr{A}_2 & = \frac{1}{6}\matr{A}_2 \\
	\matr{A}_3 & = \matr{A}_3 - \matr{A}_1 \\
	\matr{A} & = \begin{bmatrix}
		1 & -4 & 1 & -8 \\
		0 & 1 & 0 & 4 \\
		0 & 5 & 0 & 20
	\end{bmatrix}
\end{align*}
\begin{align*}
	\matr{A}_3 & = \matr{A}_3 - 5\matr{A}_2 \\
	\matr{A}_1 & = \matr{A}_1 + 4\matr{A}_2 \\
	\matr{A} & = \begin{bmatrix}
		1 & 0 & 1 & 8 \\
		0 & 1 & 0 & 4 \\
		0 & 0 & 0 & 0
	\end{bmatrix}
\end{align*}
\begin{equation*}
	\rref(\matr{A}) = \begin{bmatrix}
		1 & 0 & 1 & 8 \\
		0 & 1 & 0 & 4 \\
		0 & 0 & 0 & 0
	\end{bmatrix}
\end{equation*}

Find the system solution
\begin{equation*}
	\systeme{
		x_1 + x_3 + 8x_4 = 0,
		x_2 + 4x_4 = 0,
		0 = 0
	}
\end{equation*}
From the $ \rref(\matr{A}) $ and system of equations, the leading variables  $ x_1, x_2 $ and free variables $ x_3, x_4 $ can be determined. Solving for the leading variables:
\begin{align*}
	x_1 + x_3 + 8x_4 & = 0 \\
	x_1 & = -x_3 - 8x_4
\end{align*}
\begin{align*}
	x_2 + 4x_4 & = 0 \\
	x_2 & = -4x_4
\end{align*}
Thus the solution can be found as:
\begin{align*}
	\vec{x} & =
		\begin{bmatrix}
			x_1 \\
			x_2 \\
			x_3 \\
			x_4
		\end{bmatrix} \\
	\vec{x} & =
		\begin{bmatrix}
			-x_3 - 8x_4 \\
			-4x_4 \\
			x_3 \\
			x_4
		\end{bmatrix} \\
	\vec{x} & =
		\begin{bmatrix}
			-x_3 \\
			0 \\
			x_3 \\
			0
		\end{bmatrix}
		+ \begin{bmatrix}
			-8x_4 \\
			-4x_4 \\
			0 \\
			x_4
		\end{bmatrix} \\
	\vec{x} & =
		x_3 \begin{bmatrix}
			-1 \\
			0 \\
			1 \\
			0
		\end{bmatrix}
		+ x_4 \begin{bmatrix}
			-8 \\
			-4 \\
			0 \\
			1
		\end{bmatrix}
\end{align*}
Therefore the system can be described in the form $ \vec{x} = s\vec{x_3} + t\vec{x_4} $:
\bc{
	\vec{x} = 
		s \begin{bmatrix}
			-1 \\
			0 \\
			1 \\
			0
		\end{bmatrix}
		+ t \begin{bmatrix}
			-8 \\
			-4 \\
			0 \\
			1
		\end{bmatrix}
}

\subsection{4.2.21}

Reduce the given system to echelon form to find a single solution vector $ \vec{u} $ such that the solution space is the set of all scalar multiples of $ \vec{u} $.
\begin{equation*}
	\systeme{
		x_1 + 7x_2 + 3x_3 - 4x_4 = 0,
		2x_1 + 7x_2 + 3x_3 - x_4 = 0,
		3x_1 + 5x_2 + 2x_3 + 3x_4 = 0
	}
\end{equation*}

Begin with finding $ \rref(\matr{A}) $:
\begin{align*}
	\matr{A} & =
		\begin{bmatrix}
			1 & 7 & 3 & -4 \\
			2 & 7 & 3 & -1 \\
			3 & 5 & 2 & 3
		\end{bmatrix}
\end{align*}
\begin{align*}
	\matr{A}_2 & = \matr{A}_2 - 2\matr{A}_1 \\
	\matr{A}_3 & = \matr{A}_3 - 3\matr{A}_1 \\
	\matr{A} & =
		\begin{bmatrix}
			1 & 7 & 3 & -4 \\
			0 & -7 & -3 & 7 \\
			0 & -16 & -7 & 15
		\end{bmatrix}
\end{align*}
\begin{align*}
	\matr{A}_1 & = \matr{A}_1 + \matr{A}_2 \\
	\matr{A}_2 & = -\frac{1}{7}\matr{A}_2 \\
	\matr{A}_3 & = \matr{A}_3 + 16\matr{A}_2 \\
	\matr{A} & =
		\begin{bmatrix}
			1 & 0 & 0 & 3 \\
			0 & 1 & \frac{3}{7} & -1 \\
			0 & 0 & -\frac{1}{7} & -1
		\end{bmatrix}
\end{align*}
\begin{align*}
	\matr{A}_2 & = \matr{A}_2 - \matr{A}_3 \\
	\matr{A}_3 & = -7\matr{A}_3 \\
	\matr{A}_2 & = \matr{A}_2 - \frac{4}{7}\matr{A}_3 \\
	\matr{A} & =
		\begin{bmatrix}
			1 & 0 & 0 & 3 \\
			0 & 1 & 0 & -4 \\
			0 & 0 & 1 & 7
		\end{bmatrix}
\end{align*}
The system of equations can be found as:
\begin{align*}
	& \systeme{
		x_1 + 3x_4 = 0,
		x_2 - 4x_4 = 0,
		x_3 + 7x_4 = 0
	} \\
	& \systeme{
		x_1 = -3x_4,
		x_2 = 4x_4,
		x_3 = -7x_4
	}
\end{align*}
Aliasing $ x_4 = s $, and finding $ \vec{x} $ in the form $ \vec{x} = s\vec{u} $:
\begin{align*}
	\vec{x} & =
		\begin{bmatrix}
			x_1 \\
			x_2 \\
			x_3 \\
			x_4
		\end{bmatrix} \\
	\vec{x} & =
		\begin{bmatrix}
			-3x_4 \\
			4x_4 \\
			-7x_4 \\
			x_4
		\end{bmatrix} \\
	\vec{x} & =
		\begin{bmatrix}
			-3s \\
			4s \\
			-7s \\
			s
		\end{bmatrix} \\
	\vec{x} & =
		s \begin{bmatrix}
			-3 \\
			4 \\
			-7 \\
			1
		\end{bmatrix}
\end{align*}
\bc{
	\vec{x} =
		s \begin{bmatrix}
			-3 \\
			4 \\
			-7 \\
			1
		\end{bmatrix}
}

\end{document}
