\documentclass{article}

% Document extensibility %
%
% Disables native paragraph indentation
\usepackage{parskip} 
%
% Provides further bullet options for lists
\usepackage{enumitem}

% Mathematical symbol and statement packages %
%
% Necessary
\usepackage{amsmath}
\usepackage{amssymb}
%
% Extensive fraction notation
\usepackage{xfrac}
%
% Generic mathematical commands
% Notable: \degree, \celcius
\usepackage{gensymb}
%
% Variable vector notation (arrow above variable)
\usepackage{esvect}
%
% Multiline boxed equations
\usepackage{empheq}
%
% SI Unit
\usepackage{siunitx}
\usepackage{physunits}
%
% More intuitive arrays/matrices
\usepackage{array}
%
% Linear Equations
\usepackage{systeme}

% Graphic packages %
%
% Diagrams and illustrations
\usepackage{tikz}
%
% Image insertion
\usepackage{graphicx}

% Document content %
%
% Change title of table of contents
% \renewcommand{\contentsname}{Title}

\begin{document}

% Command `\hr` to insert horizontal rules
\newcommand{\hr}{\par\noindent\rule{\textwidth}{0.4pt}}

% Command to box and center math equations
\newcommand{\bc}[1]{
	\begin{equation*}
		\begin{boxed}
			{#1}
		\end{boxed}
	\end{equation*}
}

% Command for single line equations with a condition
\newcommand{\cond}[2]{
	\ifmmode
		{#1} \quad {#2}
	\else
		$$ {#1} \quad {#2} $$
	\fi
}

\newcommand{\matr}[1]{\mathbf{#1}}

\tableofcontents

\section{Section 4.3}

\subsection{4.3.11}

Express the indicated vector $ \vec{w} $ as a linear combination of the given vectors $ \vec{v}_1 $ and $ \vec{v}_2 $ if this is possible. If not, show that it is impossible.
\begin{equation*}
	\vec{w} =
		\begin{bmatrix}
			-8 \\ 0 \\ 0 \\ 6
		\end{bmatrix},
	\vec{v}_1 =
		\begin{bmatrix}
			7 \\ 9 \\ -6 \\ 3
		\end{bmatrix},
	\vec{v}_2 =
		\begin{bmatrix}
			2 \\ 6 \\ -4 \\ 4
		\end{bmatrix}
\end{equation*}

\begin{align*}
	(\matr{A}|\vec{v}) & =
		\left[ \begin{array}{ c c | c }
			7 & 2 & -8 \\
			9 & 6 & 0 \\
			-6 & -4 & 0 \\
			3 & 4 & 6
		\end{array} \right]
\end{align*}
\begin{align*}
	\matr{A}_2 & = 7\matr{A}_2 - 9\matr{A}_1 \\
	\matr{A}_2 & = \frac{1}{24}\matr{A}_2 \\
	\matr{A}_3 & = 7\matr{A}_3 + 6\matr{A}_1 \\
	\matr{A}_3 & = -\frac{1}{16}\matr{A}_3 \\
	\matr{A}_4 & = 7\matr{A}_4 - 3\matr{A}_1 \\
	\matr{A}_4 & = \frac{1}{22}\matr{A}_4 \\
	(\matr{A}|\vec{v}) & =
		\left[ \begin{array}{ c c | c }
			7 & 2 & -8 \\
			0 & 1 & 3 \\
			0 & 1 & 3 \\
			0 & 1 & 3
		\end{array} \right]
\end{align*}
\begin{align*}
	\matr{A}_3 & = \matr{A}_3 - \matr{A}_2 \\
	\matr{A}_4 & = \matr{A}_4 - \matr{A}_2 \\
	(\matr{A}|\vec{v}) & =
		\left[ \begin{array}{ c c | c }
			7 & 2 & -8 \\
			0 & 1 & 3 \\
			0 & 0 & 0 \\
			0 & 0 & 0
		\end{array} \right]
\end{align*}
\begin{align*}
	& \systeme{
		7x_1 + 2x_2 = -8,
		x_2 = 3
	} \\
	7x_1 + 2(3) & = -8 \\
	x_1 & = -2
\end{align*}
Therefore $ \vec{w} $ can be expressed as:
\bc{
	\vec{w} = (-2)\vec{v}_1 + (3)\vec{v}_2
}

\end{document}
