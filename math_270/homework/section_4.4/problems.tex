\documentclass{article}

% Document extensibility %
%
% Disables native paragraph indentation
\usepackage{parskip} 
%
% Provides further bullet options for lists
\usepackage{enumitem}

% Mathematical symbol and statement packages %
%
% Necessary
\usepackage{amsmath}
\usepackage{amssymb}
%
% Extensive fraction notation
\usepackage{xfrac}
%
% Generic mathematical commands
% Notable: \degree, \celcius
\usepackage{gensymb}
%
% Variable vector notation (arrow above variable)
\usepackage{esvect}
%
% Multiline boxed equations
\usepackage{empheq}
%
% SI Unit
\usepackage{siunitx}
\usepackage{physunits}
%
% More intuitive arrays/matrices
\usepackage{array}
%
% Linear Equations
\usepackage{systeme}
%
% Boxes!
\usepackage{mdframed}

% Graphic packages %
%
% Diagrams and illustrations
\usepackage{tikz}
%
% Image insertion
\usepackage{graphicx}

% Document content %
%
% Change title of table of contents
% \renewcommand{\contentsname}{Title}

\begin{document}

% Command `\hr` to insert horizontal rules
\newcommand{\hr}{\par\noindent\rule{\textwidth}{0.4pt}}

% Command to box and center math equations
\newcommand{\bc}[1]{
	\begin{equation*}
		\begin{boxed}
			{#1}
		\end{boxed}
	\end{equation*}
}

% Command for single line equations with a condition
\newcommand{\cond}[2]{
	\ifmmode
		{#1} \quad {#2}
	\else
		$$ {#1} \quad {#2} $$
	\fi
}

\newcommand{\matr}[1]{\mathbf{#1}}

\tableofcontents

\section{Section 4.4}

\subsection{4.4.1}

Determine whether or not the given vectors in $ \mathbb{R}^2 $ form a basis for $ \mathbb{R}^2 $.
\begin{equation*}
	\vec{v}_1 = \begin{bmatrix} 8 \\ 9 \end{bmatrix},
	\vec{v}_2 = \begin{bmatrix} 9 \\ 8 \end{bmatrix}
\end{equation*}

Do the given vectors form a basis for $ \mathbb{R}^2 $?
\bc{ \text{Yes, because } \vec{v}_1 \text{ and } \vec{v}_2 \text{ are linearly independent.} }

\subsection{4.4.5}

Determine whether or not the given vectors in $ \mathbb{R}^3 $ form a basis for $ \mathbb{R}^3 $.
\begin{equation*}
	\vec{v}_1 = \begin{bmatrix} 0 \\ -7 \\ 9 \end{bmatrix},
	\vec{v}_2 = \begin{bmatrix} 0 \\ 7 \\ -4 \end{bmatrix},
	\vec{v}_3 = \begin{bmatrix} 0 \\ -5 \\ -3 \end{bmatrix}
\end{equation*}

Do the given vectors form a basis for $ \mathbb{R}^3 $?

Begin by finding the determinant:
\begin{align*}
	\matr{A} & =
		\begin{bmatrix}
			0 & 0 & 0 \\
			-7 & 7 & -5 \\
			9 & -4 & -3
		\end{bmatrix}
\end{align*}
\begin{align*}
	\matr{A}_2 & = 7\matr{A}_2 + 9\matr{A}_1 \\
	\matr{A} & =
		\begin{bmatrix}
			-7 & 7 & -5 \\
			0 & 35 & -66 \\
			0 & 0 & 0
		\end{bmatrix}
\end{align*}
$ \det(\matr{A}) = -7 \cdot 35 \cdot 0 = 0 $

\subsection{4.4.7}

Determine whether or not the given vectors in $ \mathbb{R}^3 $ form a basis for $ \mathbb{R}^3 $.
\begin{equation*}
	v_1 = \begin{bmatrix} 0 \\ 0 \\ 1 \end{bmatrix},
	v_2 = \begin{bmatrix} 8 \\ 3 \\ 13 \end{bmatrix},
	v_3 = \begin{bmatrix} 7 \\ 2 \\ 9 \end{bmatrix}
\end{equation*}

Do the given vectors form a basis for $ \mathbb{R}^3 $?

\begin{align*}
	\matr{A} & =
		\begin{bmatrix}
			0 & 8 & 7 \\
			0 & 3 & 2 \\
			1 & 13 & 9
		\end{bmatrix}
\end{align*}
\begin{align*}
	\matr{A}_1 & \leftrightarrows \matr{A}_3 \\
	\matr{A}_3 & = 3\matr{A}_3 - 8\matr{A}_2 \\
	\matr{A} & =
		\begin{bmatrix}
			1 & 13 & 9 \\
			0 & 3 & 2 \\
			0 & 0 & 5
		\end{bmatrix}
\end{align*}
\bc{
	\det(\matr{A}) = 1 \cdot 3 \cdot 5 = 15 \therefore \text{ Linearly Independent}
}

\subsection{4.4.9}

Find a basis for the indicated subspace of $ \mathbb{R}^3 $.

The plane with equation $ x - 8y + 9z = 0 $.

\begin{align*}
	\matr{A} & = \begin{bmatrix*} 1 \\ -8 \\ 9 \end{bmatrix*} \\
	x & = 8y - 9z
\end{align*}
\begin{align*}
	\matr{A} & = \begin{bmatrix} 8y - 9z \\ y \\ z \end{bmatrix} \\
	\matr{A} & =
		\begin{bmatrix} 8y \\ y \\ 0 \end{bmatrix}
		+ \begin{bmatrix} -9z \\ 0 \\ z \end{bmatrix} \\
	\matr{A} & =
		y \begin{bmatrix} 8 \\ 1 \\ 0 \end{bmatrix}
		+ z \begin{bmatrix} -9 \\ 0 \\ 1 \end{bmatrix}
\end{align*}
\begin{mdframed}
	A basis for the indicated subspace of $ \mathbb{R}^3 $ is $ \left\{ \begin{bmatrix} 8 \\ 1 \\ 0 \end{bmatrix}, \begin{bmatrix} -9 \\ 0 \\ 1 \end{bmatrix} \right\} $.
\end{mdframed}

\subsection{4.4.13}

Find a basis for the indicated subspace of $ \mathbb{R}^4 $.

The set of all vectors of the form $ (a, b, c, d) $ such that $ a = 6c $ and $ b = 2d $.

\begin{align*}
	\matr{A} & = \begin{bmatrix} a \\ b \\ c \\ d \end{bmatrix} \\
	\matr{A} & = \begin{bmatrix} 6c \\ 2d \\ c \\ d \end{bmatrix} \\
	\matr{A} & =
	\begin{bmatrix} 6c \\ 0 \\ c \\ 0 \end{bmatrix}
	+ \begin{bmatrix} 0 \\ 2d \\ 0 \\ d \end{bmatrix} \\
	\matr{A} & =
		c \begin{bmatrix} 6 \\ 0 \\ 1 \\ 0 \end{bmatrix}
		+ d \begin{bmatrix} 0 \\ 2 \\ 0 \\ 1 \end{bmatrix}
\end{align*}
\begin{mdframed}
	A basis for the indicated subspace of $ \mathbb{R}^4 $ is $ \left\{ \begin{bmatrix} 6 \\ 0 \\ 1 \\ 0 \end{bmatrix}, \begin{bmatrix} 0 \\ 2 \\ 0 \\ 1 \end{bmatrix} \right\} $.
\end{mdframed}

\subsection{4.4.15}

Find a basis for the solution space of the given homogeneous linear system.
\begin{equation*}
	\systeme{
		x_1 - 2x_2 + 11x_3 = 0,
		2x_1 - 3x_2 + 14x_3 = 0
	}
\end{equation*}

\begin{align*}
	\matr{A} & =
		\begin{bmatrix}
			1 & -2 & 11 \\
			2 & -3 & 14
		\end{bmatrix}
\end{align*}
\begin{align*}
	\matr{A}_2 & = \matr{A}_2 - 2\matr{A}_1 \\
	\matr{A} & =
		\begin{bmatrix}
			1 & -2 & 11 \\
			0 & 1 & -8
		\end{bmatrix}
\end{align*}
\begin{align*}
	x_1 & = 2x_2 - 11x_3 \\
	x_2 & = 8x_3 \\
	x_1 & = 2(8x_3) - 11x_3 \\
	x_1 & = 5x_3
\end{align*}
\begin{align*}
	x & = \begin{bmatrix} x_1 \\ x_2 \\ x_3 \end{bmatrix} \\
	x & = \begin{bmatrix} 5x_3 \\ 8x_3 \\ x_3 \end{bmatrix} \\
	x & = x_3 \begin{bmatrix} 5 \\ 8 \\ 1 \end{bmatrix}
\end{align*}
\begin{mdframed}
	A basis for the solution space of the given homogeneous linear system is $ \left\{ \begin{bmatrix} 5 \\ 8 \\ 1 \end{bmatrix} \right\} $.
\end{mdframed}

\subsection{4.4.17}

Find a basis for the solution space of the given homogeneous linear system.
\begin{align*}
	\systeme{
		x_1 - 3x_2 + 3x_3 - 3x_4 = 0,
		2x_1 - 5x_2 + 11x_3 - 4x_4 = 0
	}
\end{align*}

\begin{align*}
	\matr{A} & =
		\begin{bmatrix}
			1 & -3 & 3 & -3 \\
			2 & -5 & 11 & -4
		\end{bmatrix}
\end{align*}
\begin{align*}
	\matr{A}_2 & = \matr{A}_2 - 2\matr{A}_1 \\
	\matr{A} & =
		\begin{bmatrix}
			1 & -3 & 3 & -3 \\
			0 & 1 & 5 & 2
		\end{bmatrix}
\end{align*}
\begin{align*}
	x_1 - 3x_2 + 3x_3 - 3x_4 & = 0 \\
	x_1 & = 3x_2 - 3x_3 + 3x_4 \\
	x_2 + 5x_3 + 2x_4 & = 0 \\
	x_2 & = -5x_3 - 2x_4 \\
	x_1 & = 3(-5x_3 - 2x_4) - 3x_3 + 3x_4 \\
	x_1 & = -18x_3 - 3x_4
\end{align*}
\begin{align*}
	x & =
		\begin{bmatrix}
			-18x_3 - 3x_4 \\
			-5x_3 - 2x_4 \\
			x_3 \\
			x_4
		\end{bmatrix} \\
	x & =
		\begin{bmatrix}
			-18x_3 \\ -5x_3 \\ x_3 \\ 0
		\end{bmatrix}
		+ \begin{bmatrix}
			-3x_4 \\ -2x_4 \\ 0 \\ x_4
		\end{bmatrix} \\
	x & =
		x_3 \begin{bmatrix}
			-18 \\ -5 \\ 1 \\ 0
		\end{bmatrix}
		+ x_4 \begin{bmatrix}
			-3 \\ -2 \\ 0 \\ 1
		\end{bmatrix}
\end{align*}
\begin{mdframed}
	A basis for the solution space of the given homogeneous linear system is $ \left\{ \begin{bmatrix} -18 \\ -5 \\ 1 \\ 0 \end{bmatrix}, \begin{bmatrix} -3 \\ -2 \\ 0 \\ 1 \end{bmatrix} \right\} $.
\end{mdframed}

\subsection{4.4.21}

Find a basis for the solution space of the given homogeneous linear system.
\begin{align*}
	\systeme{
		x_1 - 3x_2 - 9x_3 - 5x_4 = 0,
		2x_1 + x_2 - 4x_3 + 11x_4 = 0,
		x_1 - x_2 - 5x_3 + x_4 = 0
	}
\end{align*}

\begin{align*}
	\matr{A} & =
		\begin{bmatrix}
			1 & -3 & -9 & -5 \\
			2 & 1 & -4 & 11 \\
			1 & -1 & -5 & 1
		\end{bmatrix}
\end{align*}
\begin{align*}
	\matr{A}_2 & = \matr{A}_2 - 2\matr{A}_1 \\
	\matr{A}_3 & = \matr{A}_3 - \matr{A}_1 \\
	\matr{A} & =
		\begin{bmatrix}
			1 & -3 & -9 & -5 \\
			0 & 7 & 14 & 21 \\
			0 & 2 & 4 & 6
		\end{bmatrix}
\end{align*}
\begin{align*}
	\matr{A}_3 & = 7\matr{A}_3 - 2\matr{A}_2 \\
	\matr{A}_2 & = \frac{1}{7}\matr{A}_2 \\
	\matr{A} & =
		\begin{bmatrix}
			1 & -3 & -9 & -5 \\
			0 & 1 & 2 & 3 \\
			0 & 0 & 0 & 0
		\end{bmatrix}
\end{align*}
\begin{align*}
	x_2 + 2x_3 + 3x_4 & = 0 \\
	x_2 & = -2x_3 - 3x_4 \\
	x_1 - 3x_2 - 9x_3 - 5x_4 & = 0 \\
	x_1 & = 3x_2 + 9x_3 + 5x_4 \\
	x_1 & = 3(-2x_3 - 3x_4) + 9x_3 + 5x_4 \\
	x_1 & = 15x_3 - 4x_4
\end{align*}
\begin{align*}
	x & =
		\begin{bmatrix}
			15x_3 - 4x_4 \\
			-2x_3 - 3x_4 \\
			x_3 \\
			x_4
		\end{bmatrix} \\
	x & =
		\begin{bmatrix}
			15x_3 \\
			2x_3 \\
			x_3 \\
			0
		\end{bmatrix}
		+ \begin{bmatrix}
			-4x_4 \\
			-3x_4 \\
			0 \\
			x_4
		\end{bmatrix} \\
	x & =
		x_3 \begin{bmatrix} 15 \\ 2 \\ 1 \\ 0 \end{bmatrix}
		+ x_4 \begin{bmatrix} -4 \\ -3 \\ 0 \\ 1 \end{bmatrix}
\end{align*}
\begin{mdframed}
	A basis for the solution space of the given homogeneous linear system is $ \left\{ \begin{bmatrix} 15 \\ 2 \\ 1 \\ 0 \end{bmatrix}, \begin{bmatrix} -4 \\ -3 \\ 0 \\ 1 \end{bmatrix} \right\} $.
\end{mdframed}

\end{document}
