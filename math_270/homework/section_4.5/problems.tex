\documentclass{article}

% Document extensibility %
%
% Disables native paragraph indentation
\usepackage{parskip} 
%
% Provides further bullet options for lists
\usepackage{enumitem}

% Mathematical symbol and statement packages %
%
% Necessary
\usepackage{amsmath}
\usepackage{amssymb}
%
% Extensive fraction notation
\usepackage{xfrac}
%
% Generic mathematical commands
% Notable: \degree, \celcius
\usepackage{gensymb}
%
% Variable vector notation (arrow above variable)
\usepackage{esvect}
%
% Multiline boxed equations
\usepackage{empheq}
%
% SI Unit
\usepackage{siunitx}
\usepackage{physunits}
%
% More intuitive arrays/matrices
\usepackage{array}
%
% Linear Equations
\usepackage{systeme}
%
% Boxes!
\usepackage{mdframed}

% Graphic packages %
%
% Diagrams and illustrations
\usepackage{tikz}
%
% Image insertion
\usepackage{graphicx}

% Document content %
%
% Change title of table of contents
% \renewcommand{\contentsname}{Title}

\begin{document}

% Command `\hr` to insert horizontal rules
\newcommand{\hr}{\par\noindent\rule{\textwidth}{0.4pt}}

% Command to box and center math equations
\newcommand{\bc}[1]{
	\begin{equation*}
		\begin{boxed}
			{#1}
		\end{boxed}
	\end{equation*}
}

% Command for single line equations with a condition
\newcommand{\cond}[2]{
	\ifmmode
		{#1} \quad {#2}
	\else
		$$ {#1} \quad {#2} $$
	\fi
}

\newcommand{\matr}[1]{\mathbf{#1}}
\newcommand{\rows}{\text{rowspace}}
\newcommand{\cols}{\text{colspace}}
\newcommand{\rref}{\text{rref}}
\newcommand{\set}[1]{\mathit{#1}}

\tableofcontents

\section{Section 4.5}

\subsection{4.5.1}

Find both a basis for the row space and a basis for the column space of the given matrix $ \matr{A} $.
\begin{equation*}
	\begin{bmatrix}
		1 & 3 & 2 \\
		1 & 2 & 6 \\
		2 & 5 & 8
	\end{bmatrix}
\end{equation*}

Find $ \rref(\matr{A}) $:
\begin{align*}
	\matr{A} & =
		\begin{bmatrix}
			1 & 3 & 2 \\
			1 & 2 & 6 \\
			2 & 5 & 8
		\end{bmatrix}
\end{align*}
\begin{align*}
	\matr{A}_2 & = \matr{A}_2 - \matr{A}_1 \\
	\matr{A}_3 & = \matr{A}_3 - 2\matr{A}_1 \\
	\matr{A} & =
		\begin{bmatrix}
			1 & 3 & 2 \\
			0 & -1 & 4 \\
			0 & -1 & 4
		\end{bmatrix}
\end{align*}
\begin{align*}
	\matr{A}_3 & = \matr{A}_3 + \matr{A}_2 \\
	\matr{A}_2 & = -1\matr{A}_2 \\
	\matr{A} & =
		\begin{bmatrix}
			1 & 3 & 2 \\
			0 & 1 & -4 \\
			0 & 0 & 0
		\end{bmatrix}
\end{align*}
\begin{mdframed}
	A basis for the row space is
	$ \left\{ \begin{bmatrix} 1 \\ 3 \\ 2 \end{bmatrix}, \begin{bmatrix} 0 \\ 1 \\ -4 \end{bmatrix} \right\} $.
	
	A basis for the column space is
	$ \left\{ \begin{bmatrix} 1 \\ 1 \\ 2 \end{bmatrix}, \begin{bmatrix} 3 \\ 2 \\ 5 \end{bmatrix} \right\} $.
\end{mdframed}

\subsection{4.5.3}

Find both a basis for the row space and a basis for the column space of the given matrix $ \matr{A} $.
\begin{align*}
	\begin{bmatrix}
		1 & -3 & -2 & -5 \\
		4 & -7 & -3 & -5 \\
		1 & 3 & 4 & 13
	\end{bmatrix}
\end{align*}

\begin{align*}
	\matr{A} & =
		\begin{bmatrix}
			1 & -3 & -2 & -5 \\
			4 & -7 & -3 & -5 \\
			1 & 3 & 4 & 13
		\end{bmatrix}
\end{align*}
\begin{align*}
	\matr{A}_2 & = \matr{A}_2 - 4\matr{A}_1 \\
	\matr{A}_2 & = \frac{1}{5}\matr{A}_2 \\
	\matr{A}_3 & = \matr{A}_3 - \matr{A}_1 \\
	\matr{A}_3 & = \frac{1}{6}\matr{A}_3 \\
	\matr{A} & =
		\begin{bmatrix}
			1 & -3 & -2 & -5 \\
			0 & 1 & 1 & 3 \\
			0 & 1 & 1 & 3
		\end{bmatrix}
\end{align*}
\begin{align*}
	\matr{A}_3 & = \matr{A}_3 - \matr{A}_2 \\
	\matr{A}_1 & = \matr{A}_1 + 3\matr{A}_2 \\
	\matr{A} & =
		\begin{bmatrix}
			1 & 0 & 1 & 4 \\
			0 & 1 & 1 & 3 \\
			0 & 0 & 0 & 0
		\end{bmatrix}
\end{align*}
\begin{mdframed}
	A basis for the row space is
	$ \left\{
		\begin{bmatrix} 1 \\ 0 \\ 1 \\ 4 \end{bmatrix},
		\begin{bmatrix} 0 \\ 1 \\ 1 \\ 3 \end{bmatrix}
	\right\} $
	
	A basis for the column space is
	$ \left\{
		\begin{bmatrix} 1 \\ 4 \\ 1 \end{bmatrix},
		\begin{bmatrix} -3 \\ -7 \\ 3 \end{bmatrix}
	\right\} $
\end{mdframed}

\subsection{4.5.9}

Find both a basis for the row space and a basis for the column space of the given matrix $ \matr{A} $.
\begin{align*}
	\begin{bmatrix}
		5 & 1 & 2 & 9 \\
		10 & 7 & 5 & 7 \\
		5 & 16 & 3 & 13 \\
		15 & 28 & 9 & 9
	\end{bmatrix}
\end{align*}

\begin{align*}
	\matr{A} & =
		\begin{bmatrix}
			5 & 1 & 2 & 9 \\
			10 & 7 & 5 & 7 \\
			5 & 16 & 3 & 13 \\
			15 & 28 & 9 & 9
		\end{bmatrix}
\end{align*}
\begin{align*}
	\matr{A}_2 & = \matr{A}_2 - 2\matr{A}_1 \\
	\matr{A}_3 & = \matr{A}_3 - \matr{A}_1 \\
	\matr{A}_4 & = \matr{A}_4 - 3\matr{A}_1 \\
	\matr{A} & =
		\begin{bmatrix}
			5 & 1 & 2 & 9 \\
			0 & 5 & 1 & -11 \\
			0 & 15 & 1 & 4 \\
			0 & 25 & 3 & -18
		\end{bmatrix}
\end{align*}
\begin{align*}
	\matr{A}_3 & = \matr{A}_3 - 3\matr{A}_2 \\
	\matr{A}_4 & = \matr{A}_4 - 5\matr{A}_2 \\
	\matr{A} & =
		\begin{bmatrix}
			5 & 1 & 2 & 9 \\
			0 & 5 & 1 & -11 \\
			0 & 0 & -2 & 37 \\
			0 & 0 & -2 & 37
		\end{bmatrix}
\end{align*}
\begin{align*}
	\matr{A}_4 & = \matr{A}_4 - \matr{A}_3 \\
	\matr{A} & =
		\begin{bmatrix}
			5 & 1 & 2 & 9 \\
			0 & 5 & 1 & -11 \\
			0 & 0 & -2 & 37 \\
			0 & 0 & 0 & 0
		\end{bmatrix}
\end{align*}
\begin{align*}
	\matr{A}_2 & = 2\matr{A}_2 + \matr{A}_3 \\
	\matr{A}_2 & = \frac{1}{5}\matr{A}_2 \\
	\matr{A} & =
		\begin{bmatrix}
			5 & 1 & 2 & 9 \\
			0 & 2 & 0 & 3 \\
			0 & 0 & -2 & 37 \\
			0 & 0 & 0 & 0
		\end{bmatrix}
\end{align*}
\begin{align*}
	\matr{A}_1 & = 2\matr{A}_1 - \matr{A}_2 \\
	\matr{A}_1 & = \matr{A}_1 + 2\matr{A}_3 \\
	\matr{A}_1 & = \frac{1}{10}A_1 \\
	\matr{A}_2 & = \frac{1}{2}A_2 \\
	\matr{A}_3 & = -\frac{1}{2}A_3 \\
	\matr{A} & =
		\begin{bmatrix}
			1 & 0 & 0 & \frac{89}{10} \\
			0 & 1 & 0 & \frac{3}{2}\\
			0 & 0 & 1 & -\frac{37}{2} \\
			0 & 0 & 0 & 0
		\end{bmatrix}
\end{align*}
\begin{mdframed}
	A basis for the row space is
	$ \left\{
		\begin{bmatrix} 1 \\ 0 \\ 0 \\ \frac{89}{10} \end{bmatrix},
		\begin{bmatrix} 0 \\ 1 \\ 0 \\ \frac{3}{2} \end{bmatrix},
		\begin{bmatrix} 0 \\ 0 \\ 1 \\ -\frac{37}{2} \end{bmatrix}
	\right\} $
	
	A basis for the column space is
	$ \left\{
		\begin{bmatrix} 5 \\ 10 \\ 5 \\ 15 \end{bmatrix},
		\begin{bmatrix} 1 \\ 7 \\ 16 \\ 28 \end{bmatrix},
		\begin{bmatrix} 2 \\ 5 \\ 3 \\ 9 \end{bmatrix}
	\right\} $
\end{mdframed}

\subsection{4.5.13}

A set $ \set{S} $ of vectors in $ \mathbb{R}^4 $ is given. Find a subset of $ \set{S} $ that forms a basis for the subspace of $ \mathbb{R}^4 $ spanned by $ \set{S} $.
\begin{equation*}
	\vec{v}_1 = \begin{bmatrix} 2 \\ 2 \\ -2 \\ 6 \end{bmatrix},
	\vec{v}_2 = \begin{bmatrix} 2 \\ 23 \\ -44 \\ 30 \end{bmatrix},
	\vec{v}_3 = \begin{bmatrix} 5 \\ 26 \\ -47 \\ 39 \end{bmatrix}
\end{equation*}

\begin{align*}
	\matr{A} & =
		\begin{bmatrix}
			2 & 2 & 5 \\
			2 & 23 & 26 \\
			-2 & -44 & -47 \\
			6 & 30 & 39
		\end{bmatrix}
\end{align*}
\begin{align*}
	\matr{A}_2 & = \matr{A}_2 - \matr{A}_1 \\
	\matr{A}_3 & = \matr{A}_3 + \matr{A}_1 \\
	\matr{A}_4 & = \matr{A}_4 - 3\matr{A}_1 \\
	\matr{A} & =
		\begin{bmatrix}
			2 & 2 & 5 \\
			0 & 21 & 21 \\
			0 & -42 & -42 \\
			0 & 24 & 24
		\end{bmatrix}
\end{align*}
\begin{align*}
	\matr{A}_2 & = \frac{1}{21}\matr{A}_2 \\
	\matr{A}_3 & = -\frac{1}{42}\matr{A}_3 \\
	\matr{A}_4 & = \frac{1}{24}\matr{A}_4 \\
	\matr{A}_4 & = \matr{A}_4 - \matr{A}_3 \\
	\matr{A}_3 & = \matr{A}_3 - \matr{A}_2 \\
	\matr{A} & =
		\begin{bmatrix}
			2 & 2 & 5 \\
			0 & 1 & 1 \\
			0 & 0 & 0 \\
			0 & 0 & 0
		\end{bmatrix}
\end{align*}
\begin{align*}
	\matr{A}_1 & = \matr{A}_1 - 2\matr{A}_2 \\
	\matr{A}_1 & = \frac{1}{2}\matr{A}_1 \\
	\matr{A} & =
		\begin{bmatrix}
			1 & 0 & \frac{3}{2} \\
			0 & 1 & 1 \\
			0 & 0 & 0 \\
			0 & 0 & 0
		\end{bmatrix}
\end{align*}
\begin{mdframed}
	A basis for the row space is
	$ \left\{
		\begin{bmatrix} 1 \\ 0 \\ \frac{3}{2} \end{bmatrix},
		\begin{bmatrix} 0 \\ 1 \\ 1 \end{bmatrix}
	\right\} $
	
	A basis for the column space is
	$ \left\{
		\begin{bmatrix} 2 \\ 2 \\ -2 \\ 6 \end{bmatrix},
		\begin{bmatrix} 2 \\ 23 \\ -44 \\ 30 \end{bmatrix}
	\right\} $

	A basis for the subspace is given by
	$ \left\{
		\begin{bmatrix} 2 \\ 2 \\ -2 \\ 6 \end{bmatrix},
		\begin{bmatrix} 2 \\ 23 \\ -44 \\ 30 \end{bmatrix}
	\right\} $
\end{mdframed}

\end{document}
