\documentclass{article}

% Document extensibility %
%
% Disables native paragraph indentation
\usepackage{parskip} 
%
% Provides further bullet options for lists
\usepackage{enumitem}

% Mathematical symbol and statement packages %
%
% Necessary
\usepackage{amsmath}
\usepackage{amssymb}
%
% Extensive fraction notation
\usepackage{xfrac}
%
% Generic mathematical commands
% Notable: \degree, \celcius
\usepackage{gensymb}
%
% Variable vector notation (arrow above variable)
\usepackage{esvect}
%
% Multiline boxed equations
\usepackage{empheq}
%
% SI Unit
\usepackage{siunitx}
\usepackage{physunits}
%
% More intuitive arrays/matrices
\usepackage{array}
%
% Linear Equations
\usepackage{systeme}
%
% Boxes!
\usepackage{mdframed}

% Graphic packages %
%
% Diagrams and illustrations
\usepackage{tikz}
%
% Image insertion
\usepackage{graphicx}

% Document content %
%
% Change title of table of contents
% \renewcommand{\contentsname}{Title}

\begin{document}

% Command `\hr` to insert horizontal rules
\newcommand{\hr}{\par\noindent\rule{\textwidth}{0.4pt}}

% Command to box and center math equations
\newcommand{\bc}[1]{
	\begin{equation*}
		\begin{boxed}
			{#1}
		\end{boxed}
	\end{equation*}
}

% Command for single line equations with a condition
\newcommand{\cond}[2]{
	\ifmmode
		{#1} \quad {#2}
	\else
		$$ {#1} \quad {#2} $$
	\fi
}

\newcommand{\matr}[1]{\mathbf{#1}}

\tableofcontents

\section{Section 4.6}

\subsection{4.6.1}

Determine whether the following vectors are mutually orthogonal.
\begin{equation*}
	\vec{u}_1 = (1, -2, 1),
	\vec{u}_2 = (0, 1, 2),
	\vec{u}_3 = (-5, -2, 1)
\end{equation*}
\begin{align*}
	\vec{u}_1 \cdot \vec{u}_2 & = 1 \cdot 0 + -2 \cdot 1 + 1 \cdot 2 = 0 \therefore \vec{u}_1 \perp \vec{u}_2 \\
	\vec{u}_1 \cdot \vec{u}_3 & = 1 \cdot -5 + -2 \cdot -2 + 1 \cdot 1 = 0 \therefore \vec{u}_1 \perp \vec{u}_3 \\
	\vec{u}_2 \cdot \vec{u}_3 & = 0 \cdot -5 + 1 \cdot -2 + 2 \cdot 1 = 0 \therefore \vec{u}_2 \perp \vec{u}_3
\end{align*}
\begin{mdframed}
	The vectors are mutually orthogonal because each pair of distinct vectors is orthogonal.
\end{mdframed}

\subsection{4.6.5}

The three vertices $ A $, $ B $, and $ C $, of a triangle are given. Prove that the triangle is a right triangle by showing that its sides $ a $, $ b $, and $ c $ satisfy the Pythagorean relation $ a^2 + b^2 = c^2 $.

Find the length of each side $ a $, $ b $, and $ c $.
\begin{align*}
	d(B, C) & = \sqrt{ (3 - 4)^2 + (8 - 9)^2 + (3 - 6)^2 + (5 - 3)^2 } \\
	d(B, C) & = \sqrt{15}
\end{align*}
\begin{align*}
	d(A, C) & = \sqrt{ (3 - 4)^2 + (8 - 7)^2 + (3 - 5)^2 + (5 - 8)^2 } \\
	d(A, C) & = \sqrt{15}
\end{align*}
\begin{align*}
	d(A, B) & = \sqrt{ ((4 - 4)^2 + (9 - 7)^2 + (6 - 5)^2 + (3 - 8)^2 } \\
	d(A, B) & = \sqrt{30}
\end{align*}
As the distance $ d(A, B) $ is the longest of the three sides, it is therefore the hypotenuse.
\begin{align*}
	a^2 + b^2 & = c^2 \\
	(\sqrt{15})^2 + (\sqrt{15})^2 & = (\sqrt{30})^2 \\
	0 & = 0
\end{align*}

\subsection{4.6.13}

The vector $ \vec{v}_1 = \begin{bmatrix} 1 \\ -7 \\ 8 \end{bmatrix} $ spans a subspace $ \matr{V} $ of the indicated Euclidean space. Find a basis for the orthogonal complement $ \matr{V}^\perp $ of $ \matr{V} $.
\begin{align*}
	\matr{V} & = \begin{bmatrix} 1 & -7 & 8 \end{bmatrix}
\end{align*}
\begin{align*}
	x_1 - 7x_2 + 8x_3 & = 0 \\
	x_1 & = 7x_2 - 8x_3 \\
	\matr{V}^\perp & =
		\begin{bmatrix}
			7x_2 - 8x_3 \\
			x_2 \\
			x_3
		\end{bmatrix} \\
	\matr{V}^\perp & =
		x_2 \begin{bmatrix} 7 \\ 1 \\ 0 \end{bmatrix}
		+ x_3 \begin{bmatrix} -8 \\ 0 \\ 1 \end{bmatrix}
\end{align*}
\begin{mdframed}
	A basis for the orthogonal complement $ \matr{V}^\perp $ is
	$ \left\{
		\begin{bmatrix} 7 \\ 1 \\ 0 \end{bmatrix},
		\begin{bmatrix} -8 \\ 0 \\ 1 \end{bmatrix}
	\right\} $
\end{mdframed}

\end{document}
