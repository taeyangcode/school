\documentclass{article}

% Document extensibility %
%
% Disables native paragraph indentation
\usepackage{parskip} 
%
% Provides further bullet options for lists
\usepackage{enumitem}

% Mathematical symbol and statement packages %
%
% Necessary
\usepackage{amsmath}
\usepackage{amssymb}
%
% Extensive fraction notation
\usepackage{xfrac}
%
% Generic mathematical commands
% Notable: \degree, \celcius
\usepackage{gensymb}
%
% Variable vector notation (arrow above variable)
\usepackage{esvect}
%
% Multiline boxed equations
\usepackage{empheq}
%
% SI Unit
\usepackage{siunitx}

% Graphic packages %
%
% Diagrams and illustrations
\usepackage{tikz}
%
% Image insertion
\usepackage{graphicx}

% Document content %
%
% Change title of table of contents
% \renewcommand{\contentsname}{Title}

\begin{document}

% Command `\hr` to insert horizontal rules
\newcommand{\hr}{\par\noindent\rule{\textwidth}{0.4pt}}

% Command to box and center math equations
\newcommand{\bc}[1]{
	\begin{equation*}
		\begin{boxed}
			{#1}
		\end{boxed}
	\end{equation*}
}

% Command for single line equations with a condition
\newcommand{\cond}[2]{
	\ifmmode
		{#1} \quad {#2}
	\else
		$$ {#1} \quad {#2} $$
	\fi
}

\tableofcontents

\section{Section 5.2}

\subsection{5.2.1}
Show directly that the given functions are linearly dependent on the real line. That is, find a nontrivial linear combination of the following functions that vanishes identically.
$$ f(x) = 6x, g(x) = 4x^2, h(x) = 6x - 21x^2 $$
Enter the non-trivial linear combination.
$$ (24)6x + (-126)4x^2 + (-24)(6x - 21x^2) = 0 $$

\subsection{5.2.4}
Show directly that the given functions are linearly dependent on the real line. That is, find a nontrivial linear combination of the following functions that vanishes identically.
$$ f(x) = 23, g(x) = 3\sin^2(x), h(x) = 5\cos^2(x) $$
Enter the non-trivial linear combination.
$$ (15)(23) + (-115)(3\sin^2(x)) + (-69)(5\cos^2(x)) = 0 $$

\subsection{5.2.7}
Use the Wronskian to determine if the given functions are linearly independent on the indicated interval.
$$ f(x) = 19, g(x) = 5x, h(x) = 5x^2 \text{; the real line} $$
The Wronskian $ W(f,g,h) = 950 $. As $ W $ is never $ 0 $ on the real line $ f(x), g(x), $ and $ h(x) $ are linearly independent.

\subsection{5.2.13}

A third-order homogeneous linear equation and three linearly independent solutions are given below. Find a particular solution satisfying the given initial conditions.
$$ y^{(3)} + 2y'' - y' - 2y = 0; y(0) = 8, y'(0) = 12, y''(0) = 0; $$
$$ y_1 = e^x, y_2 = e^{-x}, y_3 = e^{-2x} $$
\begin{align*}
	y(x) & = c_1e^x + c_2e^{-x} + c_3e^{-2x} \\
	y(0) & = c_1e^0 + c_2e^0 + c_3e^0 = 8 \\
		 & = c_1 + c_2 + c_3 = 8 \\
	c_1 & = 8 - c_2 - c_3 \\
	y'(x) & = c_1e^x - c_2e^{-x} - 2c_3e^{-2x} \\
	y'(0) & = c_1e^0 - c_2e^0 - 2c_3e^0 = 12 \\
		  & = c_1 - c_2 - 2c_3 = 12 \\
	y''(x) & = c_1e^x + c_2e^{-x} + 4c_3e^{-2x} \\
	y''(0) & = c_1e^0 + c_2e^0 + 4c_3e^0 = 0 \\
		   & = c_1 + c_2 + 4c_3 = 0 \\
	(8 - c_2 - c_3) + c_2 + 4c_3 & = 0 \\
	8 + 3c_3 & = 0 \\
	3c_3 & = -8 \\
	c_3 & = -\frac{8}{3} \\
	c_2 & = -c_1 - c_3 + 8 \\
	c_1 - (-c_1 - c_3 + 8) - 2c_3 & = 12 \\
	2c_1 - c_3 & = 20 \\
	2c_1 - \left( -\frac{8}{3} \right) & = 20 \\
	c_1 & = \frac{20 - \frac{8}{3}}{2} \\
	c_1 & = \frac{26}{3} \\
	\frac{26}{3} + c_2 - \frac{8}{3} & = 8 \\
	c_2 & = 2
\end{align*}
\bc{y(x) = \frac{26}{3}e^x + 2e^{-x} - \frac{8}{3}e^{-2x}}

\subsection{5.2.19}
A third-order homogeneous linear equation and three linearly independent solutions are given below. Find a particular solution satisfying the given initial conditions.
$$ x^3y^{(3)} - 3x^2y'' + 6xy' - 6y = 0; y(1) = 9, y'(1) = 15, y''(1) = 28; $$
$$ y_1 = x, y_2 = x^2, y_3 = x^3 $$
\begin{align*}
	y(x) & = c_1x + c_2x^2 + c_3x^3 \\
	y'(x) & = c_1 + 2c_2x + 3c_3x^2 \\
	y''(x) & = 2c_2 + 6c_3x \\
	y''(1) & = 2c_2 + 6c_3(1) = 28 \\
	c_2 & = 14 - 3c_3 \\
	y'(1) & = c_1 + 2(14 - 3c_3)(1) + 3c_3(1) = 15 \\
	c_1 + 28 - 6c_3 + 3c_3 & = 15 \\
	c_1 & = -13 + 3c_3 \\
	y(1) & = (-13 + 3c_3)(1) + (14 - 3c_3)(1)^2 + c_3(1)^3 = 9 \\
	c_3 & = 8 \\
	2c_2 + 6(8)(1) & = 28 \\
	c_2 & = -10 \\
	c_1 + (-10) + (8) & = 9 \\
	c_1 & = 11
\end{align*}
\bc{y(x) = 11x - 10x^2 + 8x^3}

\end{document}
