\documentclass{article}

% Document extensibility %
%
% Disables native paragraph indentation
\usepackage{parskip} 
%
% Provides further bullet options for lists
\usepackage{enumitem}

% Mathematical symbol and statement packages %
%
% Necessary
\usepackage{amsmath}
\usepackage{amssymb}
%
% Extensive fraction notation
\usepackage{xfrac}
%
% Generic mathematical commands
% Notable: \degree, \celcius
\usepackage{gensymb}
%
% Variable vector notation (arrow above variable)
\usepackage{esvect}
%
% Multiline boxed equations
\usepackage{empheq}
%
% SI Unit
\usepackage{siunitx}
\usepackage{physunits}
%
% More intuitive arrays/matrices
\usepackage{array}
%
% Linear Equations
\usepackage{systeme}
%
% Boxes!
\usepackage{mdframed}

% Graphic packages %
%
% Diagrams and illustrations
\usepackage{tikz}
\usetikzlibrary{positioning}
%
% Image insertion
\usepackage{graphicx}

% Document content %
%
% Change title of table of contents
% \renewcommand{\contentsname}{Title}

\begin{document}

% Command `\hr` to insert horizontal rules
\newcommand{\hr}{\par\noindent\rule{\textwidth}{0.4pt}}

% Command to box and center math equations
\newcommand{\bc}[1]{
	\begin{equation*}
		\begin{boxed}
			{#1}
		\end{boxed}
	\end{equation*}
}

% Command for single line equations with a condition
\newcommand{\cond}[2]{
	\ifmmode
		{#1} \quad {#2}
	\else
		$$ {#1} \quad {#2} $$
	\fi
}

\newcommand{\matr}[1]{
	\ifmmode \boldsymbol{#1}
	\else \textbf{\textit{#1}}
	\fi
}

\newcommand{\vect}[1]{
	\ifmmode \mathbf{#1}
	\else \textbf{#1}
	\fi
}

\tableofcontents

\section{Section 6.1}

\subsection{6.1.1}

Find the (real) eigenvalues and associated eigenvectors of the larger matrix \matr{A}. Find a basis of each eigenspace of dimension 2 or larger.
\begin{align*}
	\begin{bmatrix} 4 & -1 \\ 2 & 1 \end{bmatrix}
\end{align*}
\begin{align*}
	|\matr{A} - \lambda \vect{I}| & =
		\begin{bmatrix}
			4 - \lambda & -1 \\
			2 & 1 - \lambda
		\end{bmatrix} \\
	\det(\matr{A}) & = (4 - \lambda)(1 - \lambda) - (-1)(2) \\
	\det(\matr{A}) & = \lambda^2 - 5\lambda + 2 = (\lambda - 2)(\lambda - 3) \\
	\lambda & = 2, 3
\end{align*}
\begin{align*}
	\lambda & = 2 \\
	\matr{A} - \lambda \vect{I} & =
		\begin{bmatrix}
			4 - \lambda & -1 \\
			2 & 1 - \lambda
		\end{bmatrix} \\
	\matr{A} - \lambda \vect{I} & =
		\begin{bmatrix}
			2 & -1 \\
			2 & -1
		\end{bmatrix}
\end{align*}
\begin{align*}
	(\matr{A} - \lambda \vect{I})\vect{v} & = 0 \\
	\begin{bmatrix} 2 & -1 \\ 2 & -1 \end{bmatrix}
	\begin{bmatrix} x \\ y \end{bmatrix} & =
	\begin{bmatrix} 0 \\ 0 \end{bmatrix} \\
	2x - y & = 0 \\
	x & = \frac{1}{2}y \\
	2 \left( \frac{1}{2}y \right) - y & = 0 \\
	0 & = 0
\end{align*}
\begin{align*}
	\lambda = 2 & \rightarrow \begin{bmatrix} x \\ y \end{bmatrix} = \begin{bmatrix} \frac{1}{2}y \\ y \end{bmatrix} = y \begin{bmatrix} \frac{1}{2} \\ 1 \end{bmatrix}
\end{align*}
\begin{align*}
	\lambda & = 3 \\
	\matr{A} - \lambda \vect{I} & =
		\begin{bmatrix}
			1 & -1 \\
			2 & -2
		\end{bmatrix} \\
	(\matr{A} - \lambda \vect{I})\vect{v} & = 0 \\
	\begin{bmatrix} 1 & -1 \\ 2 & -2 \end{bmatrix}
	\begin{bmatrix} x \\ y \end{bmatrix} & =
	\begin{bmatrix} 0 \\ 0 \end{bmatrix} \\
	x & = y \\
	2x & = 2y \\
	2(y) & = 2y \\
	0 & = 0
\end{align*}
\begin{align*}
	\lambda = 3 & \rightarrow \begin{bmatrix} x \\ y \end{bmatrix} = \begin{bmatrix} y \\ y \end{bmatrix} = y \begin{bmatrix} 1 \\ 1 \end{bmatrix}
\end{align*}

\end{document}
