\documentclass{article}

% Document extensibility %
%
% Disables native paragraph indentation
\usepackage{parskip} 
%
% Provides further bullet options for lists
\usepackage{enumitem}

% Mathematical symbol and statement packages %
%
% Necessary
\usepackage{amsmath}
\usepackage{amssymb}
%
% Extensive fraction notation
\usepackage{xfrac}
%
% Generic mathematical commands
% Notable: \degree, \celcius
\usepackage{gensymb}
%
% Variable vector notation (arrow above variable)
\usepackage{esvect}
%
% Multiline boxed equations
\usepackage{empheq}
%
% SI Unit
\usepackage{siunitx}
\usepackage{physunits}
%
% More intuitive arrays/matrices
\usepackage{array}
%
% Linear Equations
\usepackage{systeme}
%
% Boxes!
\usepackage{mdframed}

% Graphic packages %
%
% Diagrams and illustrations
\usepackage{tikz}
\usetikzlibrary{positioning}
%
% Image insertion
\usepackage{graphicx}

% Document content %
%
% Change title of table of contents
% \renewcommand{\contentsname}{Title}

\begin{document}

% Command `\hr` to insert horizontal rules
\newcommand{\hr}{\par\noindent\rule{\textwidth}{0.4pt}}

% Command to box and center math equations
\newcommand{\bc}[1]{
	\begin{equation*}
		\begin{boxed}
			{#1}
		\end{boxed}
	\end{equation*}
}

% Command for single line equations with a condition
\newcommand{\cond}[2]{
	\ifmmode
		{#1} \quad {#2}
	\else
		$$ {#1} \quad {#2} $$
	\fi
}

\newcommand{\matr}[1]{
	\ifmmode \boldsymbol{#1}
	\else \textbf{\textit{#1}}
	\fi
}

\newcommand{\vect}[1]{
	\ifmmode \mathbf{#1}
	\else \textbf{#1}
	\fi
}

\tableofcontents

\section{Section 6.1}

\subsection{6.1.1}

Find the (real) eigenvalues and associated eigenvectors of the larger matrix \matr{A}. Find a basis of each eigenspace of dimension 2 or larger.
\begin{align*}
	\begin{bmatrix} 4 & -1 \\ 2 & 1 \end{bmatrix}
\end{align*}
\begin{align*}
	|\matr{A} - \lambda \vect{I}| & =
		\begin{bmatrix}
			4 - \lambda & -1 \\
			2 & 1 - \lambda
		\end{bmatrix} \\
	\det(\matr{A}) & = (4 - \lambda)(1 - \lambda) - (-1)(2) \\
	\det(\matr{A}) & = \lambda^2 - 5\lambda + 2 = (\lambda - 2)(\lambda - 3) \\
	\lambda & = 2, 3
\end{align*}
\begin{align*}
	\lambda & = 2 \\
	\matr{A} - \lambda \vect{I} & =
		\begin{bmatrix}
			4 - \lambda & -1 \\
			2 & 1 - \lambda
		\end{bmatrix} \\
	\matr{A} - \lambda \vect{I} & =
		\begin{bmatrix}
			2 & -1 \\
			2 & -1
		\end{bmatrix}
\end{align*}
\begin{align*}
	(\matr{A} - \lambda \vect{I})\vect{v} & = 0 \\
	\begin{bmatrix} 2 & -1 \\ 2 & -1 \end{bmatrix}
	\begin{bmatrix} x \\ y \end{bmatrix} & =
	\begin{bmatrix} 0 \\ 0 \end{bmatrix} \\
	2x - y & = 0 \\
	x & = \frac{1}{2}y \\
	2 \left( \frac{1}{2}y \right) - y & = 0 \\
	0 & = 0
\end{align*}
\begin{align*}
	\lambda = 2 & \rightarrow \begin{bmatrix} x \\ y \end{bmatrix} = \begin{bmatrix} \frac{1}{2}y \\ y \end{bmatrix} = y \begin{bmatrix} \frac{1}{2} \\ 1 \end{bmatrix}
\end{align*}
\begin{align*}
	\lambda & = 3 \\
	\matr{A} - \lambda \vect{I} & =
		\begin{bmatrix}
			1 & -1 \\
			2 & -2
		\end{bmatrix} \\
	(\matr{A} - \lambda \vect{I})\vect{v} & = 0 \\
	\begin{bmatrix} 1 & -1 \\ 2 & -2 \end{bmatrix}
	\begin{bmatrix} x \\ y \end{bmatrix} & =
	\begin{bmatrix} 0 \\ 0 \end{bmatrix} \\
	x & = y \\
	2x & = 2y \\
	2(y) & = 2y \\
	0 & = 0
\end{align*}
\begin{align*}
	\lambda = 3 & \rightarrow \begin{bmatrix} x \\ y \end{bmatrix} = \begin{bmatrix} y \\ y \end{bmatrix} = y \begin{bmatrix} 1 \\ 1 \end{bmatrix}
\end{align*}

\subsection{6.1.2}

Find the (real) eigenvalues and associated eigenvectors of the given matrix \matr{A}. Find a basis of each eigenspace of dimension 2 or larger.
\begin{equation*}
	\begin{bmatrix}
		6 & -7 \\
		4 & -5
	\end{bmatrix}
\end{equation*}
\begin{align*}
	\left| \matr{A} - \lambda \vect{I} \right| & =
		\begin{bmatrix}
			6 - \lambda & -7 \\
			4 & -5 - \lambda
		\end{bmatrix} \\
	\det( \matr{A} - \lambda \vect{I} ) & = (6 - \lambda)(-5 - \lambda) - (-7)(4) \\
	\det( \matr{A} - \lambda \vect{I} ) & = \lambda^2 - \lambda - 2 = (\lambda - 2)(\lambda + 1) \\
	\lambda_{1,2} & = 2, -1
\end{align*}
\begin{align*}
	\lambda_1 & = 2 \\
	\left[ \matr{A} - \lambda_1 \right] \vect{x} & = 0 \\
	\begin{bmatrix}
		6 - 2 & -7 \\
		4 & -5 - 2
	\end{bmatrix} \vect{x} & = 0 \\
	\begin{bmatrix}
		4 & -7 \\
		4 & -7
	\end{bmatrix}
	\begin{bmatrix}
		x_1 \\
		x_2
	\end{bmatrix} & = 0 \\
	(4)x_1 + (-7)x_2 & = 0 \\
	(4)x_1 + (-7)x_2 & = 0
\end{align*}
\begin{align*}
	(4)x_1 & = (7)x_2 \\
	x_1 & = \left( \frac{7}{4} \right) x_2
\end{align*}
\begin{align*}
	(4)x_1 + (-7)x_2 & = 0 \\
	(4)\left( \frac{7}{4} \right) x_2 + (-7)x_2 & = 0 \\
	0 & = 0
\end{align*}
\begin{align*}
	\vect{x} & = \begin{bmatrix} \left( \frac{7}{4} \right) x_2 \\ x_2 \end{bmatrix} \\
	\vect{x} & = x_2 \begin{bmatrix} \frac{7}{4} \\ 1 \end{bmatrix}
\end{align*}
\begin{align*}
	\lambda_2 & = -1 \\
	\left[ \matr{A} - \lambda_2 \right] \vect{x} & = 0 \\
	\begin{bmatrix}
		6 - (-1) & -7 \\
		4 & -5 - (-1)
	\end{bmatrix} \vect{x} & = 0 \\
	\begin{bmatrix}
		7 & -7 \\
		4 & -4
	\end{bmatrix} \vect{x} & = 0 \\
	\begin{bmatrix} 7 & -7 \\ 4 & -4 \end{bmatrix}
	\begin{bmatrix} x_1 \\ x_2 \end{bmatrix} & = 0 \\
	(7)x_1 + (-7)x_2 & = 0 \\
	(4)x_1 + (-4)x_2 & = 0
\end{align*}
\begin{align*}
	(7)x_1 + (-7)x_2 & = 0 \\
	x_1 & = x_2
\end{align*}
\begin{align*}
	(4)x_1 + (-4)x_2 & = 0 \\
	x_1 & = x_2
\end{align*}
\begin{align*}
	\vect{x} & = \begin{bmatrix} x_2 \\ x_2 \end{bmatrix} \\
	\vect{x} & = x_2 \begin{bmatrix} 1 \\ 1 \end{bmatrix}
\end{align*}
\begin{mdframed}
	Eigenvalues: $ \lambda_{1,2} = 2, -1 $ \\
	Eigenvectors: $ \vect{x}_{1,2} = \begin{bmatrix} \frac{7}{4} \\ 1 \end{bmatrix}, \begin{bmatrix} 1 \\ 1 \end{bmatrix} $
\end{mdframed}

\subsection{6.1.13}

Find the (real) eigenvalues and associated eigenvectors of the larger matrix \matr{A}. Find a basis of each eigenspace of dimension 2 or larger.
\begin{align*}
	\begin{bmatrix}
		3 & 0 & 0 \\
		7 & -2 & -2 \\
		-3 & 4 & 4
	\end{bmatrix}
\end{align*}
\begin{align*}
	\left| \matr{A} - \lambda \vect{I} \right| & =
		\begin{bmatrix}
			3 - \lambda & 0 & 0 \\
			7 & -2 - \lambda & -2 \\
			-3 & 4 & 4 - \lambda
		\end{bmatrix} \\
	\det( \matr{A} - \lambda \vect{I} ) & =
		(3 - \lambda)( (-2 - \lambda)(4 - \lambda) - (-2)(4) ) - 0 - 0 \\
	\det( \matr{A} - \lambda \vect{I} ) & = -\lambda(\lambda - 3)(\lambda - 2) \\
	\lambda_{1,2,3} & = 0, 3, 2
\end{align*}
\begin{align*}
	\left[ \matr{A} - \lambda_1 \right] \vect{x}_1 & = 0 \\
	\begin{bmatrix}
		3 - (0) & 0 & 0 \\
		7 & -2 - (0) & -2 \\
		-3 & 4 & 4 - (0)
	\end{bmatrix}
	\begin{bmatrix} x_1 \\ x_2 \\ x_3 \end{bmatrix} & = 0 \\
	\begin{bmatrix}
		3 & 0 & 0 \\
		7 & -2 & -2 \\
		-3 & 4 & 4
	\end{bmatrix}
	\begin{bmatrix} x_1 \\ x_2 \\ x_3 \end{bmatrix} & = 0
\end{align*}
\begin{align*}
	(3)x_1 + (0)x_2 + (0)x_3 & = 0 \\
	x_1 & = 0
\end{align*}
\begin{align*}
	(7)x_1 + (-2)x_2 + (-2)x_3 & = 0 \\
	x_2 & = -x_3
\end{align*}
\begin{align*}
	(-3)x_1 + (4)x_2 + (4)x_3 & = 0 \\
	(4)(-x_3) + (4)x_3 & = 0 \\
	0 & = 0
\end{align*}
\begin{align*}
	\vect{x} & = \begin{bmatrix} 0 \\ -x_3 \\ x_3 \end{bmatrix} \\
	\vect{x} & = x_3 \begin{bmatrix} 0 \\ -1 \\ 1 \end{bmatrix}
\end{align*}
\begin{align*}
	\left[ \matr{A} - \lambda_2 \right] \vect{x}_2 & = 0 \\
	\begin{bmatrix}
		3 - (3) & 0 & 0 \\
		7 & -2 - (3) & -2 \\
		-3 & 4 & 4- (3)
	\end{bmatrix}
	\begin{bmatrix} x_1 \\ x_2 \\ x_3 \end{bmatrix} & = 0 \\
	\begin{bmatrix}
		0 & 0 & 0 \\
		7 & -5 & -2 \\
		-3 & 4 & 1
	\end{bmatrix}
	\begin{bmatrix} x_1 \\ x_2 \\ x_3 \end{bmatrix} & = 0
\end{align*}
\begin{align*}
	(0)x_1 + (0)x_2 + (0)x_3 & = 0 \\
	0 & = 0
\end{align*}
\begin{align*}
	(7)x_1 + (-5)x_2 + (-2)x_3 & = 0 \\
	x_1 & = \left( \frac{5}{7} \right) x_2 + \left( \frac{2}{7} \right) x_3
\end{align*}
\begin{align*}
	(-3)x_1 + (4)x_2 + (1)x_3 & = 0 \\
	(-3) \left( \left( \frac{5}{7} \right) x_2 + \left( \frac{2}{7} \right) x_3 \right) + (4)x_2 + (1)x_3 & = 0 \\
	\left( \frac{13}{7} \right) x_2 & = \left( -\frac{1}{7} \right) x_3 \\
	x_2 & = \left( -\frac{1}{13} \right) x_3
\end{align*}
\begin{align*}
	x_1 & = \left( \frac{5}{7} \right) \left( -\frac{1}{13} \right) x_3 + \left( \frac{2}{7} \right) x_3 \\
	x_1 & = \left( \frac{3}{13} \right) x_3
\end{align*}
\begin{align*}
	\vect{x} & = \begin{bmatrix} \left( \frac{3}{13} \right) x_3 \\ \left( -\frac{1}{13} \right) \\ x_3 \end{bmatrix} \\
	\vect{x} & = x_3 \begin{bmatrix} \frac{3}{13} \\ -\frac{1}{13} \\ 1 \end{bmatrix}
\end{align*}
\begin{align*}
	\left[ \matr{A} - \lambda_3 \right] \vect{x}_3 & = 0 \\
	\begin{bmatrix}
		3 - (2) & 0 & 0 \\
		7 & -2 - (2) & -2 \\
		-3 & 4 & 4 - (2)
	\end{bmatrix}
	\begin{bmatrix} x_1 \\ x_2 \\ x_3 \end{bmatrix} & = 0 \\
	\begin{bmatrix}
		1 & 0 & 0 \\
		7 & -4 & -2 \\
		-3 & 4 & 2
	\end{bmatrix}
	\begin{bmatrix} x_1 \\ x_2 \\ x_3 \end{bmatrix} & = 0
\end{align*}
\begin{align*}
	(1)x_1 + (0)x_2 + (0)x_3 & = 0 \\
	x_1 & = 0
\end{align*}
\begin{align*}
	(7)x_1 + (-4)x_2 + (-2)x_3 & = 0 \\
	x_2 & = \left( -\frac{1}{2} \right) x_3
\end{align*}
\begin{align*}
	(-3)x_1 + (4)x_2 + (2)x_3 & = 0 \\
	(4) \left( -\frac{1}{2} \right) x_3 + (2)x_3 & = 0 \\
	0 & = 0
\end{align*}
\begin{align*}
	\vect{x} & = \begin{bmatrix} 0 \\ \left( -\frac{1}{2} \right) x_3 \\ x_3 \end{bmatrix} \\
	\vect{x} & = x_3 \begin{bmatrix} 0 \\ -\frac{1}{2} \\ 1 \end{bmatrix}
\end{align*}

\subsection{6.1.19}

Find the (real) eigenvalues and associated eigenvectors of the larger matrix \matr{A}. Find a basis of each eigenspace of dimension 2 or larger.
\begin{align*}
	\begin{bmatrix}
		1 & 0 & 0 \\
		0 & 1 & 0 \\
		-2 & -6 & 3
	\end{bmatrix}
\end{align*}
\begin{align*}
	\left| \matr{A} - \lambda \vect{I} \right| & =
		\begin{bmatrix}
			1 - \lambda & 0 & 0 \\
			0 & 1 - \lambda & 0 \\
			-2 & -7 & 3 - \lambda
		\end{bmatrix} \\
	\det( \matr{A} - \lambda \vect{I} ) & = (1 - \lambda)( (1 - \lambda)(3 - \lambda) - 0) ) = -(\lambda - 3)(\lambda - 1)^2 \\
	\lambda_{1,2} & = 3, 1
\end{align*}
\begin{align*}
	\lambda_2 & = 1 \\
	\left[ \matr{A} - \lambda_2 \right] \vect{x}_2 & = 0 \\
	\begin{bmatrix}
		0 & 0 & 0 \\
		0 & 0 & 0 \\
		-2 & -6 & 2
	\end{bmatrix}
	\begin{bmatrix} x_1 \\ x_2 \\ x_3 \end{bmatrix} & = 0
\end{align*}
\begin{align*}
	(-2)x_1 + (-6)x_2 + (2)x_3 & = 0 \\
	x_1 & = (-3)x_2 + x_3
\end{align*}
\begin{align*}
	\vect{x} & = \begin{bmatrix} (-3) x_2 \\ x_2 \\ 0 \end{bmatrix} + \begin{bmatrix} x_3 \\ 0 \\ x_3 \end{bmatrix} \\
	\vect{x} & = x_2 \begin{bmatrix} -3 \\ 1 \\ 0 \end{bmatrix} + x_3 \begin{bmatrix} 1 \\ 0 \\ 1 \end{bmatrix}
\end{align*}
\begin{align*}
	\lambda_1 & = 3 \\
	\left[ \matr{A} - \lambda_1 \right] \vect{x}_1 & = 0 \\
	\begin{bmatrix}
		-2 & 0 & 0 \\
		0 & -2 & 0 \\
		-2 & -6 & 0
	\end{bmatrix}
	\begin{bmatrix} x_1 \\ x_2 \\ x_3 \end{bmatrix} & = 0
\end{align*}
\begin{align*}
	(-2)x_1 + (0)x_2 + (0)x_3 & = 0 \\
	x_1 & = 0
\end{align*}
\begin{align*}
	(0)x_1 + (-2)x_2 + (0)x_3 & = 0 \\
	x_2 & = 0
\end{align*}
\begin{align*}
	(-2)x_1 + (-6)x_2 + (0)x_3 & = 0 \\
	(-2)(0) + (-6)(0) & = 0 \\
	0 & = 0
\end{align*}
\begin{align*}
	\vect{x} & = \begin{bmatrix} 0 \\ 0 \\ x_3 \end{bmatrix} \\
	\vect{x} & = x_3 \begin{bmatrix} 0 \\ 0 \\ 1 \end{bmatrix}
\end{align*}
Exactly one of the eigenspaces has dimension 2 or larger. The eigenspace associated with the eigenvalue $ \lambda_2 $ has basis $ \left\{ \begin{bmatrix} -3 \\ 1 \\ 0 \end{bmatrix}, \begin{bmatrix} 1 \\ 0 \\ 1 \end{bmatrix} \right\} $.

\end{document}
