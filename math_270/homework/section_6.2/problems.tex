\documentclass{article}

% Document extensibility %
%
% Disables native paragraph indentation
\usepackage{parskip} 
%
% Provides further bullet options for lists
\usepackage{enumitem}

% Mathematical symbol and statement packages %
%
% Necessary
\usepackage{amsmath}
\usepackage{amssymb}
%
% Extensive fraction notation
\usepackage{xfrac}
%
% Generic mathematical commands
% Notable: \degree, \celcius
\usepackage{gensymb}
%
% Variable vector notation (arrow above variable)
\usepackage{esvect}
%
% Multiline boxed equations
\usepackage{empheq}
%
% SI Unit
\usepackage{siunitx}
\usepackage{physunits}
%
% More intuitive arrays/matrices
\usepackage{array}
%
% Linear Equations
\usepackage{systeme}
%
% Boxes!
\usepackage{mdframed}
%
% Matrix Notation
\usepackage{bm}

% Graphic packages %
%
% Diagrams and illustrations
\usepackage{tikz}
\usetikzlibrary{positioning}
%
% Image insertion
\usepackage{graphicx}

% Document content %
%
% Change title of table of contents
% \renewcommand{\contentsname}{Title}

\begin{document}

% Command `\hr` to insert horizontal rules
\newcommand{\hr}{\par\noindent\rule{\textwidth}{0.4pt}}

% Command to box and center math equations
\newcommand{\bc}[1]{
	\begin{equation*}
		\begin{boxed}
			{#1}
		\end{boxed}
	\end{equation*}
}

% Command for single line equations with a condition
\newcommand{\cond}[2]{
	\ifmmode
		{#1} \quad {#2}
	\else
		$$ {#1} \quad {#2} $$
	\fi
}

% Matrix and Vector notation
\newcommand{\matr}[1]{\bm{#1}}
\newcommand{\vect}[1]{
	\ifmmode \mathbf{#1}
	\else \textbf{#1}
	\fi
}

\tableofcontents

\section{Section 6.2}

\subsection{6.2.1}

Determine whether or not the given matrix $ \matr{A} $ is diagonalizable. If it is, find a diagonalizing matrix $ \matr{P} $ and diagonal matrix $ \matr{D} $ such that $ \matr{P}^{-1}\matr{A}\matr{P} = \matr{D} $.
\begin{equation*}
	\matr{A} =
		\begin{bmatrix}
			6 & -4 \\
			3 & -1
		\end{bmatrix}
\end{equation*}
\begin{align*}
	\det( \matr{A} - \lambda \vect{I} ) & =
		\begin{bmatrix}
			6 - \lambda & -4 \\
			3 & -1 - \lambda
		\end{bmatrix} \\
	\det( \matr{A} - \lambda \vect{I} ) & = (6 - \lambda)(-1 - \lambda) - (-4)(3) \\
	\det( \matr{A} - \lambda \vect{I} ) & = \lambda^2 - 5\lambda + 6 = (\lambda - 3)(\lambda - 2) \\
	\lambda_{1,2} & = 3, 2
\end{align*}
\begin{align*}
	\left[ \matr{A} - \lambda_1 \right] \vect{v} & = 0 \\
	\begin{bmatrix}
		3 & -4 \\
		3 & -4
	\end{bmatrix}
	\begin{bmatrix} \vect{v}_1 \\ \vect{v}_2 \end{bmatrix} & = 0
\end{align*}
\begin{align*}
	(3)\vect{v}_1 + (-4)\vect{v}_2 & = 0 \\
	\vect{v}_1 & = \left( \frac{4}{3} \right) \vect{v}_2
\end{align*}
\begin{align*}
	(3)\vect{v}_1 + (-4)\vect{v}_2 & = 0 \\
	(3) \left( \frac{4}{3} \right) \vect{v}_2 + (-4)\vect{v}_2 & = 0 \\
	0 & = 0
\end{align*}
\begin{align*}
	\vect{v} & =
		\begin{bmatrix}
			\left( \frac{4}{3} \right) \vect{v}_2 \\
			\vect{v}_2
		\end{bmatrix} \\
	\vect{v} & =
		\vect{v}_2 \begin{bmatrix}
			\frac{4}{3} \\
			1
		\end{bmatrix}
\end{align*}
\begin{align*}
	\left[ \matr{A} - \lambda_2 \right] \vect{v} & = 0 \\
	\begin{bmatrix}
		4 & -4 \\
		3 & -3 \\
	\end{bmatrix}
	\begin{bmatrix} \vect{v}_1 \\ \vect{v}_2 \end{bmatrix} & = 0
\end{align*}
\begin{align*}
	(4)\vect{v}_1 + (-4)\vect{v}_2 & = 0 \\
	\vect{v}_1 & = \vect{v}_2
\end{align*}
\begin{align*}
	(3)\vect{v}_1 + (-3)\vect{v}_2 & = 0 \\
	\vect{v}_1 & = \vect{v}_2
\end{align*}
\begin{align*}
	\vect{v} & = \begin{bmatrix} \vect{v}_1 \\ \vect{v}_1 \end{bmatrix} \\
	\vect{v} & = \vect{v}_1 \begin{bmatrix} 1 \\ 1 \end{bmatrix}
\end{align*}
\begin{align*}
	\matr{P} & =
		\begin{bmatrix}
			\frac{4}{3} & 1 \\
			1 & 1
		\end{bmatrix} \\
	\matr{D} & =
		\begin{bmatrix}
			3 & 0 \\
			0 & 2
		\end{bmatrix}
\end{align*}

\subsection{6.2.5}

Determine whether or not the given matrix $ \matr{A} $ is diagonalizable. If it is, find a diagonalizing matrix $ \matr{P} $ and diagonal matrix $ \matr{D} $ such that $ \matr{P}^{-1}\matr{A}\matr{P} = \matr{D} $.
\begin{equation*}
	\matr{A} =
		\begin{bmatrix}
			5 & -3 \\
			1 & 1
		\end{bmatrix}
\end{equation*}
\begin{align*}
	\det( \matr{A} - \lambda \vect{I} ) & =
		\begin{bmatrix}
			5 - \lambda & -3 \\
			1 & 1 - \lambda
		\end{bmatrix} \\
	\det( \matr{A} - \lambda \vect{I} ) & =
		(5 - \lambda)(1 - \lambda) - (-3)(1) \\
	\det( \matr{A} - \lambda \vect{I} ) & =
		\lambda^2 - 6\lambda + 8 = (\lambda - 4)(\lambda - 2) \\
	\lambda_{1,2} & = 4, 2
\end{align*}
\begin{align*}
	\left[ \matr{A} - \lambda_1 \right] \vect{v} & = 0 \\
	\begin{bmatrix}
		1 & -3 \\
		1 & -3
	\end{bmatrix}
	\begin{bmatrix} \vect{v}_1 \\ \vect{v}_2 \end{bmatrix} & = 0
\end{align*}
\begin{align*}
	(1)\vect{v}_1 + (-3)\vect{v}_2 & = 0 \\
	\vect{v}_1 & = (3)\vect{v}_2
\end{align*}
\begin{align*}
	(1)\vect{v}_1 + (-3)\vect{v}_2 & = 0 \\
	(1)(3)\vect{v}_2 + (-3)\vect{v}_2 & = 0 \\
	0 & = 0
\end{align*}
\begin{align*}
	\vect{v} & = \begin{bmatrix} (3)\vect{v}_2 \\ \vect{v}_2 \end{bmatrix} \\
	\vect{v} & = \vect{v}_2 \begin{bmatrix} 3 \\ 1 \end{bmatrix}
\end{align*}
\begin{align*}
	\left[ \matr{A} - \lambda_2 \right] \vect{v} & = 0 \\
	\begin{bmatrix}
		3 & -3 \\
		1 & -1
	\end{bmatrix}
	\begin{bmatrix} \vect{v}_1 \\ \vect{v}_2 \end{bmatrix} & = 0
\end{align*}
\begin{align*}
	(3)\vect{v}_1 + (-3)\vect{v}_2 & = 0 \\
	\vect{v}_1 & = \vect{v}_2
\end{align*}
\begin{align*}
	(1)\vect{v}_1 + (-1)\vect{v}_2 & = 0 \\
	\vect{v}_2 & = \vect{v}_2 \\
	0 & = 0
\end{align*}
\begin{align*}
	\vect{v} & = \begin{bmatrix} \vect{v}_2 \\ \vect{v}_2 \end{bmatrix} \\
	\vect{v} & = \vect{v}_2 \begin{bmatrix} 1 \\ 1 \end{bmatrix}
\end{align*}

\subsection{6.2.10}

Determine whether or not the given matrix $ \matr{A} $ is diagonalizable. If it is, find a diagonalizing matrix $ \matr{P} $ and diagonal matrix $ \matr{D} $ such that $ \matr{P}^{-1}\matr{A}\matr{P} = \matr{D} $.
\begin{equation*}
	\matr{A} =
		\begin{bmatrix}
			3 & -1 \\
			1 & 1
		\end{bmatrix}
\end{equation*}
\begin{align*}
	\det( \matr{A} - \lambda \vect{I} ) & =
		\begin{bmatrix}
			3 - \lambda & -1 \\
			1 & 1 - \lambda
		\end{bmatrix} \\
	\det( \matr{A} - \lambda \vect{I} ) & =
		(3 - \lambda)(1 - \lambda) - (-1)(1) \\
	\det( \matr{A} - \lambda \vect{I} ) & =
		\lambda^2 - 4\lambda + 4 = (\lambda - 2)^2 \\
	\lambda & = 2
\end{align*}
\begin{align*}
	\left[ \matr{A} - \lambda \right] \vect{v} & = 0 \\
	\begin{bmatrix}
		1 & -1 \\
		1 & -1
	\end{bmatrix}
	\begin{bmatrix} \vect{v}_1 \\ \vect{v}_2 \end{bmatrix} & = 0
\end{align*}
\begin{align*}
	(1)\vect{v}_1 + (-1)\vect{v}_2 & = 0 \\
	\vect{v}_1 & = \vect{v}_2
\end{align*}
\begin{align*}
	(1)\vect{v}_1 + (-1)\vect{v}_2 & = 0 \\
	\vect{v}_1 & = \vect{v}_2
\end{align*}
\begin{align*}
	\vect{v} & = \begin{bmatrix} \vect{v}_1 \\ \vect{v}_1 \end{bmatrix} \\
	\vect{v} & = \vect{v}_1 \begin{bmatrix} 1 \\ 1 \end{bmatrix}
\end{align*}
The matrix is not diagonalizable.

\subsection{6.2.12}

Determine whether or not the given matrix $ \matr{A} $ is diagonalizable. If it is, find a diagonalizing matrix $ \matr{P} $ and diagonal matrix $ \matr{D} $ such that $ \matr{P}^{-1}\matr{A}\matr{P} = \matr{D} $.
\begin{equation*}
	\matr{A} =
		\begin{bmatrix}
			10 & 8 \\
			-18 & -14
		\end{bmatrix}
\end{equation*}
\begin{align*}
	\det( \matr{A} - \lambda \vect{I} ) & =
		\begin{bmatrix}
			10 - \lambda & 8 \\
			-18 & -14 - \lambda
		\end{bmatrix} \\
	\det( \matr{A} - \lambda \vect{I} ) & = (10 - \lambda)(-14 - \lambda) - (8)(-18) \\
	\det( \matr{A} - \lambda \vect{I} ) & = \lambda^2 + 4\lambda + 4 = (\lambda + 2)^2 \\
	\lambda & = -2
\end{align*}
\begin{align*}
	\left[ \matr{A} - \lambda \right] \vect{v} & = 0 \\
	\begin{bmatrix}
		12 & 8 \\
		-18 & -12
	\end{bmatrix}
	\begin{bmatrix} \vect{v}_1 \\ \vect{v}_2 \end{bmatrix} & = 0
\end{align*}
\begin{align*}
	(12)\vect{v}_1 + (8)\vect{v}_2 & = 0 \\
	\vect{v}_2 & = \left( -\frac{3}{2} \right) \vect{v}_1
\end{align*}
\begin{align*}
	(-18)\vect{v}_1 + (-12)\vect{v}_2 & = 0 \\
	(-18)\vect{v}_1 + (-12) \left( -\frac{3}{2} \right) \vect{v}_1 & = 0 \\
	0 & = 0
\end{align*}
\begin{align*}
	\vect{v} & = \begin{bmatrix} \vect{v}_1 \\ \left( -\frac{3}{2} \right) \vect{v}_1 \end{bmatrix} \\
	\vect{v} & = \vect{v}_1 \begin{bmatrix} 1 \\ -\frac{3}{2} \end{bmatrix}
\end{align*}
The matrix is not diagonalizable.

\subsection{6.2.13}

Determine whether or not the given matrix $ \matr{A} $ is diagonalizable. If it is, find a diagonalizing matrix $ \matr{P} $ and diagonal matrix $ \matr{D} $ such that $ \matr{P}^{-1}\matr{A}\matr{P} = \matr{D} $.
\begin{equation*}
	\matr{A} =
		\begin{bmatrix}
			2 & 3 & 0 \\
			0 & 3 & 0 \\
			0 & 0 & 3
		\end{bmatrix}
\end{equation*}
\begin{align*}
	\det( \matr{A} - \lambda \vect{I} ) & =
		\begin{bmatrix}
			2 - \lambda & 3 & 0 \\
			0 & 3 - \lambda & 0 \\
			0 & 0 & 3 - \lambda
		\end{bmatrix} \\
	\det( \matr{A} - \lambda \vect{I} ) & =
		(2 - \lambda)( (3 - \lambda)(3 - \lambda) - 0 ) - 0 - 0 \\
	\det( \matr{A} - \lambda \vect{I} ) & =
		-(\lambda - 3)^2(\lambda - 2) \\
	\lambda_{1,2} & = 3, 2
\end{align*}
\begin{align*}
	\left[ \matr{A} - \lambda_1 \right] \vect{v} & = 0 \\
	\begin{bmatrix}
		-1 & 3 & 0 \\
		0 & 0 & 0 \\
		0 & 0 & 0
	\end{bmatrix}
	\begin{bmatrix} \vect{v}_1 \\ \vect{v}_2 \\ \vect{v}_3 \end{bmatrix} & = 0
\end{align*}
\begin{align*}
	(-1)\vect{v}_1 + (3)\vect{v}_2 + (0)\vect{v}_3 & = 0 \\
	\vect{v}_1 & = (3)\vect{v}_2
\end{align*}
\begin{align*}
	(0)\vect{v}_1 + (0)\vect{v}_2 + (0)\vect{v}_3 & = 0 \\
	0 & = 0
\end{align*}
\begin{align*}
	(0)\vect{v}_1 + (0)\vect{v}_2 + (0)\vect{v}_3 & = 0 \\
	0 & = 0
\end{align*}
\begin{align*}
	\vect{v} & =
		\begin{bmatrix} (3)\vect{v}_2 \\ \vect{v}_2 \\ 0 \end{bmatrix}
		+ \begin{bmatrix} 0 \\ 0 \\ \vect{v}_3 \end{bmatrix} \\
	\vect{v} & =
		\vect{v}_2 \begin{bmatrix} 3 \\ 1 \\ 0 \end{bmatrix}
		+ \vect{v}_3 \begin{bmatrix} 0 \\ 0 \\ 1 \end{bmatrix}
\end{align*}
\begin{align*}
	\left[ \matr{A} - \lambda_2 \right] \vect{v} & = 0 \\
	\begin{bmatrix}
		0 & 3 & 0 \\
		0 & 1 & 0 \\
		0 & 0 & 1
	\end{bmatrix}
	\begin{bmatrix} \vect{v}_1 \\ \vect{v}_2 \\ \vect{v}_3 \end{bmatrix} & = 0
\end{align*}
\begin{align*}
	(0)\vect{v}_1 + (3)\vect{v}_2 + (0)\vect{v}_3 & = 0 \\
	\vect{v}_2 & = 0
\end{align*}
\begin{align*}
	(0)\vect{v}_1 + (1)\vect{v}_2 + (0)\vect{v}_3 & = 0 \\
	\vect{v}_2 & = 0
\end{align*}
\begin{align*}
	(0)\vect{v}_1 + (0)\vect{v}_2 + (1)\vect{v}_3 & = 0 \\
	\vect{v}_3 & = 0
\end{align*}
\begin{align*}
	\vect{v} & =
		\begin{bmatrix} \vect{v}_1 \\ 0 \\ 0 \end{bmatrix} \\
	\vect{v} & =
		\vect{v}_1 \begin{bmatrix} 1 \\ 0 \\ 0 \end{bmatrix}
\end{align*}
\begin{align*}
	\vect{v}_{1,2,3} & =
		\begin{bmatrix} 3 \\ 1 \\ 0 \end{bmatrix},
		\begin{bmatrix} 0 \\ 0 \\ 1 \end{bmatrix},
		\begin{bmatrix} 1 \\ 0 \\ 0 \end{bmatrix} \\
	\matr{D} & =
		\begin{bmatrix}
			3 & 1 & 0 \\
			0 & 0 & 1 \\
			1 & 0 & 0
		\end{bmatrix} \\
	\matr{P} & =
		\begin{bmatrix}
			3 & 0 & 0 \\
			0 & 3 & 0 \\
			0 & 0 & 2
		\end{bmatrix}
\end{align*}

\subsection{6.2.16}

Determine whether or not the given matrix $ \matr{A} $ is diagonalizable. If it is, find a diagonalizing matrix $ \matr{P} $ and diagonal matrix $ \matr{D} $ such that $ \matr{P}^{-1}\matr{A}\matr{P} = \matr{D} $.
\begin{equation*}
	\begin{bmatrix}
		1 & -3 & 3 \\
		0 & 2 & -1 \\
		0 & 0 & 1
	\end{bmatrix}
\end{equation*}
\begin{align*}
	\det( \matr{A} - \lambda \vect{I} ) & =
		\begin{bmatrix}
			1 - \lambda & -3 & 3 \\
			0 & 2 - \lambda & -1 \\
			0 & 0 & 1 - \lambda
		\end{bmatrix} \\
	\det( \matr{A} - \lambda \vect{I} ) & =
		(1 - \lambda)( (2 - \lambda)(1 - \lambda) - 0 ) - (-3)(0) - (3)(0) \\
	\det( \matr{A} - \lambda \vect{I} ) & =
		-(\lambda - 2)(\lambda - 1)^2 \\
	\lambda_{1,2} & = 2, 1
\end{align*}
\begin{align*}
	\left[ \matr{A} - \lambda_1 \right] \vect{v} & = 0 \\
	\begin{bmatrix}
		-1 & -3 & 3 \\
		0 & 0 & -1 \\
		0 & 0 & -1
	\end{bmatrix}
	\begin{bmatrix} \vect{v}_1 \\ \vect{v}_2 \\ \vect{v}_3 \end{bmatrix} & = 0
\end{align*}
\begin{align*}
	(0)\vect{v}_1 + (0)\vect{v}_2 + (-1)\vect{v}_3 & = 0 \\
	\vect{v}_3 & = 0
\end{align*}
\begin{align*}
	(-1)\vect{v}_1 + (-3)\vect{v}_2 + (3)\vect{v}_3 & = 0 \\
	(-1)\vect{v}_1 + (-3)\vect{v}_2 + (3)(0) & = 0 \\
	\vect{v}_1 & = (-3)\vect{v}_2
\end{align*}
\begin{align*}
	\vect{v} & = \vect{v}_2 \begin{bmatrix} -3 \\ 1 \\ 0 \end{bmatrix}
\end{align*}
\begin{align*}
	\left[ \matr{A} - \lambda_2 \right] \vect{v} & = 0 \\
	\begin{bmatrix}
		0 & -3 & 3 \\
		0 & 1 & -1 \\
		0 & 0 & 0
	\end{bmatrix}
	\begin{bmatrix} \vect{v}_1 \\ \vect{v}_2 \\ \vect{v}_3 \end{bmatrix} & = 0
\end{align*}
\begin{align*}
	(0)\vect{v}_1 + (0)\vect{v}_2 + (0)\vect{v}_3 & = 0 \\
	0 & = 0
\end{align*}
\begin{align*}
	(0)\vect{v}_1 + (1)\vect{v}_2 + (-1)\vect{v}_3 & = 0 \\
	\vect{v}_2 & = \vect{v}_3
\end{align*}
\begin{align*}
	(0)\vect{v}_1 + (-3)\vect{v}_2 + (3)\vect{v}_3 & = 0 \\
	(0)\vect{v}_1 + (-3)(\vect{v}_3) + (3)\vect{v}_3 & = 0 \\
	0 & = 0
\end{align*}
\begin{align*}
	\vect{v} & =
		\begin{bmatrix} \vect{v}_1 \\ 0 \\ 0 \end{bmatrix}
		+ \begin{bmatrix} 0 \\ \vect{v}_3 \\ \vect{v}_3 \end{bmatrix} \\
	\vect{v} & =
		\vect{v}_1 \begin{bmatrix} 1 \\ 0 \\ 0 \end{bmatrix}
		+ \vect{v}_3 \begin{bmatrix} 0 \\ 1 \\ 1 \end{bmatrix}
\end{align*}

\subsection{6.2.19}

Determine whether or not the given matrix $ \matr{A} $ is diagonalizable. If it is, find a diagonalizing matrix $ \matr{P} $ and diagonal matrix $ \matr{D} $ such that $ \matr{P}^{-1}\matr{A}\matr{P} = \matr{D} $.
\begin{equation*}
	\matr{A} =
		\begin{bmatrix}
			2 & -1 & 1 \\
			-4 & 2 & 4 \\
			-2 & -1 & 5
		\end{bmatrix}
\end{equation*}
\begin{align*}
	\det( \matr{A} - \lambda \vect{I} ) & =
		\begin{bmatrix}
			2 - \lambda & -1 & 1 \\
			-4 & 2 - \lambda & 4 \\
			-2 & -1 & 5 - \lambda
		\end{bmatrix} \\
	\det( \matr{A} - \lambda \vect{I} ) & = (2 - \lambda)( (2 - \lambda)(5 - \lambda)- (4)(-1) \\
										& + (-1)( (-4)(5 - \lambda) - (4)(-2) ) \\
										& + (1)( (-4)(-1) - (2 - \lambda)(-2) ) \\
	\det( \matr{A} - \lambda \vect{I} ) & =
		-\lambda^3 + 9\lambda^2 - 26\lambda + 24 = -(\lambda - 4)(\lambda - 3)(\lambda - 2) \\
	\lambda_{1,2,3} & = 4, 3, 2
\end{align*}

\end{document}
