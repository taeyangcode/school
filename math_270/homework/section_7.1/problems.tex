\documentclass{article}

% Document extensibility %
%
% Disables native paragraph indentation
\usepackage{parskip} 
%
% Provides further bullet options for lists
\usepackage{enumitem}

% Mathematical symbol and statement packages %
%
% Necessary
\usepackage{amsmath}
\usepackage{amssymb}
%
% Extensive fraction notation
\usepackage{xfrac}
%
% Generic mathematical commands
% Notable: \degree, \celcius
\usepackage{gensymb}
%
% Variable vector notation (arrow above variable)
\usepackage{esvect}
%
% Multiline boxed equations
\usepackage{empheq}
%
% SI Unit
\usepackage{siunitx}
\usepackage{physunits}
%
% More intuitive arrays/matrices
\usepackage{array}
%
% Linear Equations
\usepackage{systeme}
%
% Boxes!
\usepackage{mdframed}
%
% Matrix Notation
\usepackage{bm}

% Graphic packages %
%
% Diagrams and illustrations
\usepackage{tikz}
\usetikzlibrary{positioning}
%
% Image insertion
\usepackage{graphicx}

% Document content %
%
% Change title of table of contents
% \renewcommand{\contentsname}{Title}

\begin{document}

% Command `\hr` to insert horizontal rules
\newcommand{\hr}{\par\noindent\rule{\textwidth}{0.4pt}}

% Command to box and center math equations
\newcommand{\bc}[1]{
	\begin{equation*}
		\begin{boxed}
			{#1}
		\end{boxed}
	\end{equation*}
}

% Command for single line equations with a condition
\newcommand{\cond}[2]{
	\ifmmode
		{#1} \quad {#2}
	\else
		$$ {#1} \quad {#2} $$
	\fi
}

% Matrix and Vector notation
\newcommand{\matr}[1]{
	\ifmmode \bm{#1}
	\else \textit{\textbf{#1}}
	\fi
}
\newcommand{\vect}[1]{
	\ifmmode \mathbf{#1}
	\else \textbf{#1}
	\fi
}

\tableofcontents

\section{Section 7.1}

\subsection{7.1.1}

Transform the given differential equation into an equivalent system of first-order differential equations.
\begin{equation*}
	x'' + 4x' - 3x = 6t
\end{equation*}

\hr

Let $ x_1 = x $ and $ x_2 = x' $. Complete the system below.
\begin{align*}
	x_1' & = x_2 \\
	x_2' & = -4x' + 3x + 6t = -4x_2 + 3x_1 + 6t
\end{align*}

\subsection{7.1.2}

Transform the given differential equation into an equivalent system of first-order differential equations.
\begin{equation*}
	x^{(4)} + 5x'' + 2x = 5t^3\sin(2t)
\end{equation*}

\hr

Let $ x_1 = x $, $ x_2 = x' $, $ x_3 = x'' $, and $ x_4 = x^{(3)} $. Complete the system below.
\begin{align*}
	x_1' & = x_2 \\
	x_2' & = x_3 \\
	x_3' & = x_4 \\
	x_4' & = -5x'' - 2x + 5t^3\sin(2t) = -5x_3 - 2x_1 + 5t^3\sin(2t)
\end{align*}

\subsection{7.1.5}

Transform the given differential equation into an equivalent system of first-order differential equations.
\begin{equation*}
	x^{(3)} = (x'')^2 - 2\cos(x')
\end{equation*}

\hr

Let $ x_1 = x $, $ x_2 = x' $, and $ x_3 = x'' $. Complete the system below.
\begin{align*}
	x_1' & = x_2 \\
	x_2' & = x_3 \\
	x_3' & = (x_3)^2 - 2\cos(x_2)
\end{align*}

\subsection{7.1.8}

Transform the given differential equation into an equivalent system of first-order differential equations.
\begin{align*}
	x'' + 6x' - 6x - 5y & = 0 \\
	y'' - 4y' + 3x - 5y & = \sin(t)
\end{align*}

\hr

Let $ x_1 = x $, $ x_2 = x' $, $ y_1 = y $, and $ y_2 = y' $. Complete the system below.
\begin{align*}
	x_1' & = x_2 \\
	x_2' & = -6x' + 6x + 5y = -6x_2 + 6x_1 + 5y_1 \\
	y_1' & = y_2 \\
	y_2' & = 4y' - 3x + 5y + \sin(t) = 4y_2 - 3x_1 + 5y_1 + \sin(t)
\end{align*}

\subsection{7.1.9}

Transform the given differential equation into an equivalent system of first-order differential equations.
\begin{align*}
	\systeme*{
		x'' = 9x - y + 3z,
		y'' = x + y - 3z,
		z'' = 6x - y - z
	}
\end{align*}
\begin{align*}
	x_1' & = x_2 \\
	y_1' & = y_2 \\
	z_1' & = z_2
\end{align*}
\begin{align*}
	x_2' & = 9x_1 - y_1 + 3z_1 \\
	y_2' & = x_1 + y_1 - 3z_1 \\
	z_2' & = 6x_1 - y_1 - z_1
\end{align*}

\subsection{7.1.22}

\begin{enumerate}[label = \textbf{(\alph*)}]
	\item Beginning with the general solution of the system $ x' = -2y $, $ y' = 2x $, calculate $ x^2 + y^2 $ to show that the trajectories are circles.
	\item Show similarly that the trajectories of the system $ x' = \frac{1}{2}y $, $ y' = -8x $ are ellipses with equation of the form $ 16x^2 + y^2 = C^2 $.
\end{enumerate}

\hr

\begin{enumerate}[label = \textbf{(\alph*)}]
	\item
		Find the solution of the system $ x' = -2y $, $ y' = 2x $ below. Start with $ x(t) $.
		\begin{align*}
			x' & = -2y \\
			x'' & = -2y' \\
			x'' & = -2(2x) \\
			x'' + 4x & = 0
		\end{align*}
		\begin{align*}
			r^2 + 4 = 0 \\
			r & = \pm 2i
		\end{align*}
		\begin{align*}
			x(t) & = C_1\cos(2t) + C_2\sin(2t)
		\end{align*}
		Now find $ y(t) $ so that $ y(t) $ and the solution for $ x(t) $ found in the previous step are a general solution to the system of differential equations.
		\begin{align*}
			y' & = 2x \\
			y' & = 2(C_1\cos(2t) + C_2\sin(2t)) \\
			\int y' dt & = 2 \int \left( C_1\cos(2t) + C_2\sin(2t) \right) dt \\
			y(t) & = C_1\sin(2t) - C_2\cos(2t)
		\end{align*}
		Now calculate and simplify $ x^2 + y^2 $.
		\begin{align*}
			x^2 + y^2 & = (C_1\cos(2t) + C_2\sin(2t))^2 + (C_1\sin(2t) - C_2\cos(2t))^2 \\
			x^2 + y^2 & = C_1^2 + C_2^2
		\end{align*}
	\item
		Show that the trajectories of the system $ x' = \frac{1}{2}y $, $ y' = -8x $ are ellipses with equations of the form $ 16x^2 + y^2 = C^2 $. First solve for $ x(t) $.
		\begin{align*}
			x' & = \frac{1}{2}y \\
			x'' & = \frac{1}{2}y' \\
			x'' & = \frac{1}{2}(-8x) \\
			x'' + 4x & = 0
		\end{align*}
		\begin{align*}
			r^2 + 4r & = 0 \\
			r & = \pm 2i
		\end{align*}
		\begin{align*}
			x(t) & = A\cos(2t) + B\sin(2t)
		\end{align*}
		Now find $ y(t) $ so that $ y(t) $ and the solution for $ x(t) $ found in the previous step are a general solution to the system of differential equations.
		\begin{align*}
			y' & = -8x \\
			\int y' dt & = -8 \int \left( A\cos(2t) + B\sin(2t) \right) dt \\
			y & = -4A\sin(2t) + 4B\cos(2t)
		\end{align*}
		Letting $ C = \sqrt{A^2 + B^2} $, $ A = C\cos(\alpha) $, and $ B = C\sin(\alpha) $, rewrite $ x(t) $ and $ y(t) $ in terms of $ C $, $ \alpha $, and $ t $ below.
		\begin{align*}
			x(t) & = A\cos(2t) + B\sin(2t) \\
			x(t) & = (C\cos(\alpha))\cos(2t) + (C\sin(\alpha))\sin(2t)
		\end{align*}
		\begin{align*}
			y(t) & = -4A\sin(2t) + 4B\cos(2t) \\
			y(t) & = -4(C\cos(\alpha))\sin(2t) + 4(C\sin(\alpha))\cos(2t)
		\end{align*}
		Using the equations from the previous step, solve for $ \cos(\alpha - 2t) $ and $ \sin(\alpha - 2t) $ and rewrite $ \cos^2(\alpha - 2t) + \sin^2(\alpha - 2t) = 1 $ in terms of $ x, y $ and $ C $.
		\begin{align*}
			x(t) & = (C\cos(\alpha))\cos(2t) + (C\sin(\alpha))\sin(2t) \\
			x(t) & = C\cos(-2t + \alpha) \\
			\cos(2t - \alpha) & = \frac{x}{C}
		\end{align*}
		\begin{align*}
			y(t) & = -4(C\cos(\alpha))\sin(2t) + 4(C\sin(\alpha))\cos(2t) \\
			y(t) & = 4C\sin(-2t + \alpha) \\
			\sin(-2t + \alpha) & = \frac{y}{4C}
		\end{align*}
		\begin{align*}
			\left( \frac{x}{C} \right)^2 + \left( \frac{y}{4C} \right)^2 & = 1
		\end{align*}
		Finally, multiply both sides of the equation found in the previous step by $ 16C^2 $, then replace $ 4C $ with $ C $, resulting in the equation $ 16x^2 + y^2 = C^2 $.
\end{enumerate}

\subsection{7.1.26}

Three \SI{132}{gal} fermentation vats are connected as indicated in the figure, and the mixtures in each tank are kept uniform by stirring. Denote by $ x_i(t) $ the amount (in pounds) of alcohol in tank $ T_i $ at time $ t (i = 1, 2, 3) $. Suppose that the mixture circulates between the tanks at the rate of \SI{11}{gal \per \min}. Derive the equations.
\begin{align*}
	11x_1' & = -x_1 + x_3 \\
	11x_2' & = x_1 - x_2 \\
	11x_3' & = x_2 - x_3
\end{align*}

\hr

\begin{align*}
	11 \begin{bmatrix} x_1' \\ x_2' \\ x_3' \end{bmatrix} & =
	\begin{bmatrix}
		-1 & 0 & 1 \\
		1 & -1 & 0 \\
		0 & 1 & -1
	\end{bmatrix}
	\begin{bmatrix} x_1 \\ x_2 \\ x_3 \end{bmatrix} = 0 \\
	x_1' & = \left( -\frac{1}{11} \right) x_1 + \left( \frac{1}{11} \right) x_3 \\
	x_2' & = \left( \frac{1}{11} \right) x_1 + \left( -\frac{1}{11} \right) x_2 \\
	x_3' & = \left( \frac{1}{11} \right) x_2 + \left( -\frac{1}{11} \right) x_3 \\
	\det( \matr{A} - \lambda \vect{I} ) & =
		\begin{bmatrix}
			-1 - \lambda & 0 & 1  \\
			1 & -1 - \lambda & 0 \\
			0 & 1 & -1 - \lambda
		\end{bmatrix} \\
	\det( \matr{A} - \lambda \vect{I} ) & =
		(-1 - \lambda)( (-1 - \lambda)(-1 - \lambda) - 0 ) + 0 + (1)( (1)(1) - 0 ) \\
	\det( \matr{A} - \lambda \vect{I} ) & =
		-\lambda^3 - 3\lambda^2 - 3\lambda = -\lambda(\lambda^2 + 3\lambda + 3) \\
	\lambda & = 0, -\frac{3}{2}\pm \frac{\sqrt{3}}{2}i
\end{align*}

\end{document}
