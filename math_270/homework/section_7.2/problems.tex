\documentclass{article}

% Document extensibility %
%
% Disables native paragraph indentation
\usepackage{parskip} 
%
% Provides further bullet options for lists
\usepackage{enumitem}

% Mathematical symbol and statement packages %
%
% Necessary
\usepackage{amsmath}
\usepackage{amssymb}
%
% Extensive fraction notation
\usepackage{xfrac}
%
% Generic mathematical commands
% Notable: \degree, \celcius
\usepackage{gensymb}
%
% Variable vector notation (arrow above variable)
\usepackage{esvect}
%
% Multiline boxed equations
\usepackage{empheq}
%
% SI Unit
\usepackage{siunitx}
\usepackage{physunits}
%
% More intuitive arrays/matrices
\usepackage{array}
%
% Linear Equations
\usepackage{systeme}
%
% Boxes!
\usepackage{mdframed}
%
% Matrix Notation
\usepackage{bm}

% Graphic packages %
%
% Diagrams and illustrations
\usepackage{tikz}
\usetikzlibrary{positioning}
%
% Image insertion
\usepackage{graphicx}

% Document content %
%
% Change title of table of contents
% \renewcommand{\contentsname}{Title}

\begin{document}

% Command `\hr` to insert horizontal rules
\newcommand{\hr}{\par\noindent\rule{\textwidth}{0.4pt}}

% Command to box and center math equations
\newcommand{\bc}[1]{
	\begin{equation*}
		\begin{boxed}
			{#1}
		\end{boxed}
	\end{equation*}
}

% Command for single line equations with a condition
\newcommand{\cond}[2]{
	\ifmmode
		{#1} \quad {#2}
	\else
		$$ {#1} \quad {#2} $$
	\fi
}

% Matrix and Vector notation
\newcommand{\matr}[1]{
	\ifmmode \bm{#1}
	\else \textit{\textbf{#1}}
	\fi
}
\newcommand{\vect}[1]{
	\ifmmode \mathbf{#1}
	\else \textbf{#1}
	\fi
}

\section{Section 7.2}

\subsection{7.2.1}

Verify the product law for differentiation, $ (\matr{A}\matr{B})' = \matr{A}'\matr{B} + \matr{A}\matr{B}' $ where $ \matr{A}(t) = \begin{bmatrix} t & 2t - 1 \\ t^3 & \frac{1}{t} \end{bmatrix} $ and $ \matr{B}(t) = \begin{bmatrix} 1 - t & 1 + t \\ 2t^2 & 2t^3 \end{bmatrix} $.

\hr

To calculate $ (\matr{A}\matr{B})' $, first calculate $ \matr{A}\matr{B} $.
\begin{align*}
	\matr{A}\matr{B} & =
		\begin{bmatrix}
			(t)(1 - t) + (2t - 1)(2t^2) & (t)(1 + t) + (2t - 1)(2t^3) \\
			(t^3)(1 - t) + \left( \frac{1}{t} \right) (2t^2) & (t^3)(1 + t) + \left( \frac{1}{t} \right) (2t^3)
		\end{bmatrix} \\
	\matr{A}\matr{B} & =
		\begin{bmatrix}
			4t^3 - 3t^2 + t & 4t^4 - 2t^3 + t^2 + t \\
			-t^4 + t^3 + 2t & t^4 + t^3 + 2t^2
		\end{bmatrix}
\end{align*}
Now take the derivative of $ \matr{A}\matr{B} $ to find $ (\matr{A}\matr{B})' $.
\begin{align*}
	(\matr{A}\matr{B})' & =
		\begin{bmatrix}
			12t^2 - 6t + 1 & 16t^3 -6t^2 + 2t + 1 \\
			-4t^3 + 3t^2 + 2 & 4t^3 + 3t^2 + 4t
		\end{bmatrix}
\end{align*}
To calculate $ \matr{A}'\matr{B} + \matr{A}\matr{B}' $, first calculate $ \matr{A}' $.
\begin{align*}
	\matr{A}' & =
		\begin{bmatrix}
			1 & 2 \\
			3t^2 & -\frac{1}{t^2}
		\end{bmatrix}
\end{align*}
Now find $ \matr{A}'\matr{B} $.
\begin{align*}
	\matr{A}'\matr{B} & =
		\begin{bmatrix}
			4t^2 - t + 1 & 4t^3 + t + 1 \\
			-3t^3 + 3t^2 - 2 & 3t^3 + 3t^2 - 2t
		\end{bmatrix}
\end{align*}

\subsection{7.2.3}

Write the given system in the form $ \vect{x}' = \matr{P}(t)\vect{x} + \vect{f}(t) $.
\begin{align*}
	x' & = -9y \\
	y' & = 5x
\end{align*}
\begin{align*}
	\matr{P} & =
		\begin{bmatrix}
			0 & -9 \\
			5 & 0
		\end{bmatrix} \\
	\vect{f} & = \begin{bmatrix} 0 \\ 0 \end{bmatrix} \\
	\vect{x}' & =
		\begin{bmatrix}
			0 & -9 \\
			5 & 0
		\end{bmatrix} \vect{x}
		+ \begin{bmatrix} 0 \\ 0 \end{bmatrix}
\end{align*}

\subsection{7.2.5}

Write the given system in the form $ \vect{x}' = \matr{P}(t)\vect{x} + \vect{f}(t) $.
\begin{align*}
	x' & = 9x + 4y + 6e^t \\
	y' & = 6x - y - t^3
\end{align*}
\begin{align*}
	\matr{P} & =
		\begin{bmatrix}
			9 & 4 \\
			6 & -1
		\end{bmatrix} \\
	\vect{f} & =
		\begin{bmatrix}
			6e^t \\
			-t^3
		\end{bmatrix} \\
	\vect{x}' & =
		\begin{bmatrix}
			9 & 4 \\
			6 & -1
		\end{bmatrix} \vect{x}
		+ \begin{bmatrix}
			6e^t \\
			-t^3
		\end{bmatrix}
\end{align*}

\subsection{7.2.7}

Write the given system in the form $ \vect{x}' = \matr{P}(t)\vect{x} + \vect{f}(t) $.
\begin{align*}
	x' & = 5y + 5z \\
	y' & = 4z + 8x \\
	z' & = 8x + 2y
\end{align*}
\begin{align*}
	\matr{P} & =
		\begin{bmatrix}
			0 & 5 & 5 \\
			8 & 0 & 4 \\
			8 & 2 & 0
		\end{bmatrix} \\
	\vect{f} & =
		\begin{bmatrix}
			0 \\ 0 \\ 0
		\end{bmatrix} \\
	\vect{x}' & =
		\begin{bmatrix}
			0 & 5 & 5 \\
			8 & 0 & 4 \\
			8 & 2 & 0
		\end{bmatrix} \vect{x}
		+ \begin{bmatrix} 0 \\ 0 \\ 0 \end{bmatrix}
\end{align*}

\subsection{7.2.9}

Write the given system in the form $ \vect{x}' = \matr{P}(t)\vect{x} + \vect{f}(t) $.
\begin{align*}
	x' & = 8x - 9y + z + t \\
	y' & = x - 3z + t^2 \\
	z' & = 3y - 9z + t^3
\end{align*}
\begin{align*}
	\matr{P} & =
		\begin{bmatrix}
			7 & -9 & 1 \\
			1 & 0 & -3 \\
			0 & 3 & -9
		\end{bmatrix} \\
	\vect{x} & =
		\begin{bmatrix}
			t \\
			t^2 \\
			t^3
		\end{bmatrix} \\
	\vect{x}' & =
		\begin{bmatrix}
			7 & -9 & 1 \\
			1 & 0 & -3 \\
			0 & 3 & -9
		\end{bmatrix} \vect{x}
		+ \begin{bmatrix} t \\ t^2 \\ t^3 \end{bmatrix}
\end{align*}

\subsection{7.2.15}

Find a particular solution of the indicated linear system that satisfies the initial conditions $ x_1(0) = 7 $, $ x_2(0) = -5 $.
\begin{equation*}
	\vect{x}' = \begin{bmatrix}
		4 & -1 \\
		7 & -4
	\end{bmatrix};
	\vect{x}_1 = \begin{bmatrix}
		e^{3t} \\
		e^{3t}
	\end{bmatrix};
	\vect{x}_2 = \begin{bmatrix}
		e^{-3t} \\
		7e^{-3t}
	\end{bmatrix}
\end{equation*}
\begin{align*}
	\begin{bmatrix} x_1(t) \\ x_2(t) \end{bmatrix}
	& = C_1 \begin{bmatrix} e^{3t} \\ e^{3t} \end{bmatrix}
	+ C_2 \begin{bmatrix} e^{-3t} \\ 7e^{-3t} \end{bmatrix}
\end{align*}
\begin{align*}
	x_1(t) & = C_1e^{3t} + C_2e^{-3t} \\
	x_2(t) & = C_1e^{3t} + 7C_2e^{-3t}
\end{align*}
\begin{align*}
	x_1(0) & = C_1e^{3(0)} + C_2e^{-3(0)} = 7 \\
	C_1 + C_2 & = 7 \\
	C_1 & = 7 - C_2
\end{align*}
\begin{align*}
	x_2(0) & = C_1e^{3(0)} + 7C_2e^{-3(0)} = -5 \\
	C_1 + 7C_2 & = -5 \\
	(7 - C_2) + 7C_2 & = -5 \\
	C_2 & = -2 \\
	C_1 & = 7 - (-2) \\
	C_1 & = 9
\end{align*}
\begin{align*}
	x_1(t) & = 9e^{3t} - 2e^{-3t} \\
	x_2(t) & = 9e^{3t} - 14e^{-3t}
\end{align*}

\subsection{7.2.29}

Find a particular solution of the indicated linear system that satisfies the initial conditions $ x_1(0) = 5 $, $ x_2(0) = 4 $, and $ x_3(0) = 2 $.
\begin{equation*}
	\vect{x}' = \begin{bmatrix}
		-14 & -17 & -1 \\
		12 & 15 & 1 \\
		-12 & -12 & 2
	\end{bmatrix};
	\vect{x}_1 = e^{-2t} \begin{bmatrix} 4 \\ -3 \\ 3 \end{bmatrix},
	\vect{x}_2 = e^{2t} \begin{bmatrix} 1 \\ -1 \\ 1 \end{bmatrix},
	\vect{x}_3 = e^{3t} \begin{bmatrix} 1 \\ -1 \\ 0 \end{bmatrix}
\end{equation*}
\begin{align*}
	& \left[ \begin{array}{ c c c | c }
		4 & 1 & 1 & 5 \\
		-3 & -1 & -1 & 4 \\
		3 & 1 & 0 & 2
	\end{array} \right] \\
	\begin{bmatrix} C_1 \\ C_2 \\ C_3 \end{bmatrix}
	& = \begin{bmatrix} 9 \\ -25 \\ -6 \end{bmatrix}
\end{align*}
\begin{align*}
	x_1(t) & = (C_1)4e^{-2t} + (C_2)e^{2t} + (C_3)e^{3t} \\
	x_2(t) & = (C_1)-3e^{-2t} + (C_2)-e^{2t} + (C_3)-e^{3t} \\
	x_3(t) & = (C_1)3e^{-2t} + (C_2)e^{2t}
\end{align*}
\begin{align*}
	x_1(t) & = 36e^{-2t} - 25e^{2t} - 6e^{3t} \\
	x_2(t) & = -27e^{-2t} + 25e^{2t} + 6e^{3t} \\
	x_3(t) & = 27e^{-2t} - 25e^{2t}
\end{align*}

\subsection{7.2.33}

\begin{enumerate}[label = \textbf{(\alph*)}]
	\item Show that the vector functions $ \vect{x}_1(t) = \begin{bmatrix} t \\ t^2 \end{bmatrix} $ and $ \vect{x}_2(t) = \begin{bmatrix} t^2 \\ t^3 \end{bmatrix} $ are linearly independent on the real line.
	\item Why does it follow from the Wronskians of solutions that there is no continuous matrix $ \matr{P}(t) $ such that $ \vect{x}_1 $ and $ \vect{x}_2 $ are both solutions of $ \vect{x}' = \matr{P}(t)\vect{x} $?
\end{enumerate}

\hr

\begin{enumerate}[label = \textbf{(\alph*)}]
	\item
		The vector-valued functions $ \vect{x}_1, \vect{x}_2, \cdots, \vect{x}_n $ are linearly dependent on the interval $ \vect{I} $ provided that there exist constants $ c_1, c_2, \cdots, c_n $ not all zero such that $ c_1\vect{x}_1(t) + c_2\vect{x}_2(t) + \cdots + c_n\vect{x}_n(t) = 0 $ for all $ t $ in $ \vect{I} $. Otherwise they are linearly independent. \\
		The m atrix $ \vect{x}_2 = t \cdot \vect{x}_1 $, so neither is a constant multiple of the other. Therefore $ \vect{x}_1(t) = \begin{bmatrix} t \\ t^2 \end{bmatrix} $ and $ \vect{x}_2(t) = \begin{bmatrix} t^2 \\ t^3 \end{bmatrix} $ are linearly independent.
	\item
		The Wronskian of solutions of $ \vect{x}' = \matr{P}(t)\vect{x} $ with a continuous matrix $ \matr{P}(t) $ have only two possibilities for solutions of homoegeneous systems, either $ W = 0 $ at every point of $ \vect{I} $, or $ W = 0 $ at no point of $ \vect{I} $. Find the Wronskian of $ \vect{x}_1(t) = \begin{bmatrix} t \\ t^2 \end{bmatrix} $ and $ \vect{x}_2(t) = \begin{bmatrix} t^2 \\ t^3 \end{bmatrix} $.
		\begin{equation*}
			W = 0
		\end{equation*}
\end{enumerate}

\end{document}
