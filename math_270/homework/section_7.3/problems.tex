\documentclass{article}

% Document extensibility %
%
% Disables native paragraph indentation
\usepackage{parskip} 
%
% Provides further bullet options for lists
\usepackage{enumitem}

% Mathematical symbol and statement packages %
%
% Necessary
\usepackage{amsmath}
\usepackage{amssymb}
%
% Extensive fraction notation
\usepackage{xfrac}
%
% Generic mathematical commands
% Notable: \degree, \celcius
\usepackage{gensymb}
%
% Variable vector notation (arrow above variable)
\usepackage{esvect}
%
% Multiline boxed equations
\usepackage{empheq}
%
% SI Unit
\usepackage{siunitx}
\usepackage{physunits}
%
% More intuitive arrays/matrices
\usepackage{array}
%
% Linear Equations
\usepackage{systeme}
%
% Boxes!
\usepackage{mdframed}
%
% Matrix Notation
\usepackage{bm}

% Graphic packages %
%
% Diagrams and illustrations
\usepackage{tikz}
\usetikzlibrary{positioning}
%
% Image insertion
\usepackage{graphicx}

% Document content %
%
% Change title of table of contents
% \renewcommand{\contentsname}{Title}

\begin{document}

% Command `\hr` to insert horizontal rules
\newcommand{\hr}{\par\noindent\rule{\textwidth}{0.4pt}}

% Command to box and center math equations
\newcommand{\bc}[1]{
	\begin{equation*}
		\begin{boxed}
			{#1}
		\end{boxed}
	\end{equation*}
}

% Command for single line equations with a condition
\newcommand{\cond}[2]{
	\ifmmode
		{#1} \quad {#2}
	\else
		$$ {#1} \quad {#2} $$
	\fi
}

% Matrix and Vector notation
\newcommand{\matr}[1]{
	\ifmmode \bm{#1}
	\else \textit{\textbf{#1}}
	\fi
}
\newcommand{\vect}[1]{
	\ifmmode \mathbf{#1}
	\else \textbf{#1}
	\fi
}

\tableofcontents

\section{Section 7.3}

\subsection{7.3.1-T}

Apply the eigenvalue method to find a general solution of the given system. Use a computer system or graphing calculator to construct a direction field and typical solution curves for the given system.
\begin{align*}
	x'_1 & = 2x_1 + 6x_2 \\
	x'_2 & = 6x_1 + 2x_2
\end{align*}
\begin{align*}
	\begin{bmatrix} \vect{x}_1' \\ \vect{x}_2' \end{bmatrix} & =
		\begin{bmatrix} 2 & 6 \\ 6 & 2 \end{bmatrix}
		\begin{bmatrix} \vect{x}_1 \\ \vect{x}_2 \end{bmatrix} \\
	\det( \matr{A} - \lambda \vect{I} ) & =
		\begin{bmatrix}
			2 - \lambda & 6 \\
			6 & 2 - \lambda
		\end{bmatrix} \\
	\det( \matr{A} - \lambda \vect{I} ) & =
		(2 - \lambda)(2 - \lambda) - (6)(6) = \lambda^2 - 4\lambda - 32 = (\lambda - 8)(\lambda + 4) \\
	\lambda_{1,2} & = 8, -4
\end{align*}
\begin{align*}
	\left[ \matr{A} - \lambda_1 \right] \vect{x} & = 0 \\
	\begin{bmatrix}
		-6 & 6 \\
		6 & -6
	\end{bmatrix} \vect{x} & = 0 \\
	\begin{bmatrix}
		1 & -1 \\
		0 & 0
	\end{bmatrix} \vect{x} & = 0
\end{align*}
\begin{align*}
	\vect{x} & = \begin{bmatrix} \vect{x}_2 \\ \vect{x}_2 \end{bmatrix} \\
	\vect{x} & = \vect{x}_2 \begin{bmatrix} 1 \\ 1 \end{bmatrix}
\end{align*}
\begin{align*}
	\left[ \matr{A} - \lambda_2 \right] & = 0 \\
	\begin{bmatrix}
		6 & 6 \\
		6 & 6
	\end{bmatrix} \vect{x} & = 0 \\
	\begin{bmatrix}
		1 & 1 \\
		0 & 0
	\end{bmatrix} \vect{x} & = 0
\end{align*}
\begin{align*}
	\vect{x} & = \begin{bmatrix} -\vect{x}_2 \\ \vect{x}_2 \end{bmatrix} \\
	\vect{x} & = \vect{x}_2 \begin{bmatrix} -1 \\ 1 \end{bmatrix}
\end{align*}
\begin{align*}
	\vect{x} & = C_1 \vect{x}_1 e^{8t} + C_2 \vect{x}_2 e^{-4t} \\
	\vect{x} & =
		C_1 \begin{bmatrix} 1 \\ 1 \end{bmatrix} e^{8t}
		+ C_2 \begin{bmatrix} -1 \\ 1 \end{bmatrix} e^{-4t} \\
	\vect{x}_1 & = C_1e^{8t} - C_2e^{-4t} \\
	\vect{x}_2 & = C_1e^{8t} + C_2e^{-4t}
\end{align*}

\subsection{7.3.3-T}

Apply the eigenvalue method to find a general solution of the given system. Use a computer system or graphing calculator to construct a direction field and typical solution curves for the given system.
\begin{align*}
	\vect{x}'_1 = 3\vect{x}_1 + 4\vect{x}_2, \vect{x}'_2 = 3\vect{x}_1 + 2\vect{x}_2, \vect{x}_1(0) = \vect{x}_2(0) = 1
\end{align*}
\begin{align*}
	\vect{x}' & =
		\begin{bmatrix}
			3 & 4 \\
			3 & 2
		\end{bmatrix} \vect{x}
\end{align*}
\begin{align*}
	\det( \matr{A} - \lambda \vect{I} ) & =
		\begin{bmatrix}
			3 - \lambda & 4 \\
			3 & 2 - \lambda
		\end{bmatrix} \\
	\det( \matr{A} - \lambda \vect{I} ) & =
		(3 - \lambda)(2 - \lambda) - (4)(3) = (\lambda - 6)(\lambda + 1) \\
	\lambda_{1,2} & = 6, -1
\end{align*}
\begin{align*}
	\left[ \matr{A} - \lambda_1 \right] \vect{x} & = 0 \\
	\begin{bmatrix}
		-3 & 4 \\
		3 & -4
	\end{bmatrix} \vect{x} & = 0 \\
	\begin{bmatrix}
		1 & \frac{4}{3} \\
		0 & 0
	\end{bmatrix} \vect{x} & = 0 \\
	\vect{x} & = \vect{x}_2 \begin{bmatrix} \frac{4}{3} \\ 1 \end{bmatrix}
\end{align*}
\begin{align*}
	\left[ \matr{A} - \lambda_2 \right] \vect{x} & = 0 \\
	\begin{bmatrix} 4 & 4 \\ 3 & 3 \end{bmatrix} \vect{x} & = 0 \\
	\begin{bmatrix} 1 & 1 \\ 0 & 0 \end{bmatrix} \vect{x} & = 0 \\
	\vect{x} & = \vect{x}_2 \begin{bmatrix} -1 \\ 1 \end{bmatrix}
\end{align*}
\begin{align*}
	\vect{x}(t) & = C_1 \vect{x}_1 e^{6t} + C_2 \vect{x}_2 e^{-1t} \\
	\vect{x}(t) & =
		C_1 \begin{bmatrix} \frac{4}{3} \\ 1 \end{bmatrix} e^{6t}
		+ C_2 \begin{bmatrix} -1 \\ 1 \end{bmatrix} e^{-t}
\end{align*}
The general solution in matrix form is $ \vect{x}(t) = C_1 \begin{bmatrix} \frac{4}{3} \\ 1 \end{bmatrix} e^{6t} + C_2 \begin{bmatrix} -1 \\ 1 \end{bmatrix} e^{-t} $.

\hr

Now finding the particular solution.
\begin{align*}
	\vect{x}_1(t) & = C_1 \left( \frac{4}{3} \right) e^{6t} + C_2 (-1) e^{-t} \\
	\vect{x}_1(0) & = C_1 \left( \frac{4}{3} \right) e^{6(0)} + C_2 (-1) e^{-(0)} = 1 \\
	\vect{x}_1(0) & = C_1 \left( \frac{4}{3} \right) - C_2 = 1 \\
	\vect{x}_2(t) & = C_1 (1) e^{6t} + C_2 (1) e^{-t} \\
	\vect{x}_2(0) & = C_1 (1) e^{6(0)} + C_2 (1) e^{-(0)} = 1 \\
	\vect{x}_2(0) & = C_1 + C_2 = 1
\end{align*}
\begin{align*}
	\left[ \vect{x} | 1 \right] & =
		\left[ \begin{array}{ c c | c }
			\frac{4}{3} & -1 & 1 \\
			1 & 1 & 1
		\end{array} \right] \\
	\left[ \vect{x} | 1 \right] & =
		\left[ \begin{array}{ c c | c }
			1 & 0 & \frac{6}{7} \\
			0 & 1 & \frac{1}{7}
		\end{array} \right]
\end{align*}
\begin{align*}
	\vect{x}(t) & = \left( \frac{6}{7} \right) \begin{bmatrix} \frac{4}{3} \\ 1 \end{bmatrix} e^{6t} + \left( \frac{1}{7} \right) \begin{bmatrix} -1 \\ 1 \end{bmatrix} e^{-t}
\end{align*}

\subsection{7.3.7-T}

Apply the eigenvalue method to find a general solution of the given system. Use a computer system or graphing calculator to construct a direction field and typical solution curves for the given system.
\begin{equation*}
	\vect{x}'_1 = -2\vect{x}_1 + 6\vect{x}_2, \vect{x}'_2 = 9\vect{x}_1 - 5\vect{x}_2
\end{equation*}
\begin{align*}
	\det( \matr{A} - \lambda \vect{I} ) & =
		\begin{bmatrix}
			-2 - \lambda & 6 \\
			9 & -5 - \lambda
		\end{bmatrix} \\
	\det( \matr{A} - \lambda \vect{I} ) & =
		(-2 - \lambda)(-5 - \lambda) - (6)(9) = (\lambda - 4)(\lambda + 11) \\
	\lambda_{1,2} & = 4, -11
\end{align*}
\begin{align*}
	\left[ \matr{A} - \lambda_1 \right] \vect{x} & = 0 \\
	\begin{bmatrix}
		-6 & 6 \\
		9 & -9
	\end{bmatrix} \vect{x} & = 0 \\
	\vect{x} & = \vect{x}_2 \begin{bmatrix} 1 \\ 1 \end{bmatrix}
\end{align*}
\begin{align*}
	\left[ \matr{A} - \lambda_2 \right] \vect{x} & = 0 \\
	\begin{bmatrix}
		9 & 6 \\
		9 & 6
	\end{bmatrix} \vect{x} & = 0 \\
	\vect{x} & = \vect{x}_2 \begin{bmatrix} -\frac{2}{3} \\ 1 \end{bmatrix}
\end{align*}
\begin{align*}
	\vect{x} & = C_1 \begin{bmatrix} 1 \\ 1 \end{bmatrix} e^{4t} + C_2 \begin{bmatrix} -\frac{2}{3} \\ 1 \end{bmatrix} e^{-11t}
\end{align*}

\subsection{7.3.15}

Apply the eigenvalue method to find a general solution of the given system. Use a computer system or graphing calculator to construct a direction field and typical solution curves for the given system.
\begin{equation*}
	\vect{x}'_1 = 5\vect{x}_1 - 10\vect{x}_2, \vect{x}'_2 = 8\vect{x}_1 - 3\vect{x}_2
\end{equation*}
\begin{align*}
	\det( \matr{A} - \lambda \vect{I} ) & =
		\begin{bmatrix}
			5 - \lambda & -10 \\
			8 & -3 - \lambda
		\end{bmatrix} \\
	\det( \matr{A} - \lambda \vect{I} ) & =
		(5 - \lambda)(-3 - \lambda) - (-10)(8) = 1 \pm 8i \\
	\lambda_{1,2} & = 1 \pm 8i
\end{align*}
\begin{align*}
	\left[ \matr{A} - \lambda_1 \right] \vect{x} & = 0 \\
	\begin{bmatrix}
		4 - 8i & -10 \\
		8 & -8i - 4
	\end{bmatrix} \vect{x} & = 0
\end{align*}

\end{document}
