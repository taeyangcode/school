\documentclass{article}

\usepackage{parskip}
\usepackage{amsmath}
\usepackage{amssymb}
\usepackage{xfrac}

\renewcommand{\contentsname}{Chapter 1}

\begin{document}

\newpage
    \tableofcontents
\newpage

\section{Introduction}

\subsection{Introduction to Differential Equations}

Equations contain: ``=" and ``\textit{solution}"

There are two types of variables: \textit{dependent} and \textit{independent}

Differential: contains derivative (or partial derivative)

Derivative refers to:
\begin{align*}
    \text{ODE (ordinary differential equation)} & \text{:}\ \frac{ \text{d} }{ \text{d}x } \\
    \text{PDE (partial differential equation)} & \text{:}\ \frac{ \partial }{ \partial x }
\end{align*}

The solution to a differential equation is a \textbf{function} \\
And can be expressed in four ways:
\begin{enumerate}
    \item Verbally
    \item Table
    \item Graph
    \item Expression (implicitly or explicitly)
\end{enumerate}

Regular equation:
$$ x^{3} + 3 \sin(x) = 2 - 2 \cos(x) $$

$ \text{Claim}\ x = 0\ \text{is a solution. Verify by substitution.} $
$$ \text{When}\ x = 0 $$
$$ \left( 0 \right)^{3} + 3 \sin(0) = 2 - 2 \cos(0) $$
$$ 0 + 3 \times 0 = 2 - 2 \times 1 $$
$$ 0 = 0 $$

Although it may be true for $ x = 0 $, it is not a true method of verification. See for $ x = \pi $:
$$ \pi ^{3} + 3 \sin(\pi) = 2 - 2 \cos(\pi) $$
$$ \pi ^{3} + 0 = 2 - 2 \times \left( -1 \right) $$
$$ \pi ^{3} \neq 4 $$

So how do we verify the equation?

Given $ y'' - 2y' + 5y = 0 $ (with respect to $x$) \\
Claim $ y = 4e^{x} \cos(2x) $ is a solution
\begin{align*}
    y' & = 4e^{x} \cos(2x) + 4e^{x} \left( - \sin(2x) \right) \\
       & = 4e^{x} \left( \cos(2x) - 2 \sin(2x) \right) \\
    y'' & = 4e^{x} \left( \cos(2x) - 2 \sin(2x) \right) + 4e^{x} \left( -2 \sin(2x) - 4 \cos(2x) \right) \\
        & = 4e^{x} \left( -3 \cos(2x) - 4 \sin(2x) \right) \\
    -2y' & = 4e^{x} \left( -2 \cos(2x) + 4 \sin(2x) \right) \\
    5y & = 4e^{x} \left( 5 \cos(2x) \right)
\end{align*}
$ \text{LHS} = y'' - 2y' + 5y = 4e^{x} \left( 0 + 0 \right) = 0 = \text{RHS} $ \\
$ \therefore\ y = 4e^{x} \cos(2x)\ \text{is a solution} $

\subsection{Classification of Differential Equations}

\begin{enumerate}
    \item ODE v.s. PDE
    \begin{itemize}
        \item ODE: ordinary differential equation \\
        $ F \left( x,y,y',y'',\cdots,y^{\left( n \right)} \right) = 0 $

        \item PDE: partial differential equation \\
        $ F \left( x_{1},x_{2},\cdots,u,u_{x_{1}},u_{x_{2}},\cdots,u_{x_{k}},\cdots \right) = 0 $
    \end{itemize}
\end{enumerate}

\subsection{Order of Differential Equations}

The order of differential equations is defined by the highest derivative.

Example:
$$ y'' - 2y' + 5y = 0 \rightarrow \text{2nd order} $$ 

Second order ODE is generally written as:
$$ F \left( x, y, y', y'' \right) = 0 $$
\begin{itemize}
    \item independent variable: $x$
    \item dependent variable: $y$
\end{itemize}

To write the second order PDE, the independent variable must be located first
\begin{itemize}
    \item independent variable: $x, t$
    \item dependent variable: $u$
\end{itemize}
$$ F \left( x, t, u, u_{x}, u_{t}, u_{xt}, u_{xx}, u_{tx}, u_{tt} \right) = 0 $$

\subsection{Linear \& Non-Linear ODE Classification}

Linear equations can be solved explicitly.
Non-linear equations may only sometimes be solvable. Generally analyzed or attempting to find the equation's stable point.

\begin{enumerate}
    \item Linear Form
        $$ a_{n}(x) y^{n} + a_{n - 1}(x) y^{n-1} + \cdots + a_{2}(x) y'' + a_{1}(x) y' + a_{0}(x) y = f(x) $$ \\
        Can also be written as so:
        $$ \sum_{i = 0}^{n} a_{i}(x) y^{(i)} = f(x) $$

    \item Non-Linear \\
        Examples include: $ \sin(y), e^{y}, \ln(y), y^{2}, \sqrt{y} $
\end{enumerate}

Orders covered in this course:
\begin{itemize}
    \item 1st order: linear \& some non-linear

    \item 2nd order: linear only
        \begin{itemize}
            \item[$\rightarrow$] $ a_{i}(x) $ are constants
            \item[$\rightarrow$] $ f(x) $ are ``nice functions"
            \item[$\rightarrow$] $ a_{i}(x) $ are polynomials
        \end{itemize}

    \item higher order: linear with constant coefficients \\
        * possibly system of linear differential (using matrices \& eigen-theory)
\end{itemize}

\end{document}
