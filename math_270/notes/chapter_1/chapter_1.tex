\documentclass{article}

\usepackage{parskip}
\usepackage{amsmath}
\usepackage{amssymb}
\usepackage{xfrac}

\renewcommand{\contentsname}{Chapter 1 - First-Order Differential Equations}

\begin{document}
\newcommand{\hr}{\par\noindent\rule{\textwidth}{0.4pt}}

\newpage
    \tableofcontents
\newpage

\section{Differential Equations and Mathematical Models}

\subsection{What is a differential equation?}

An equation relating an unknown function and one or more of its derivatives is called a \textbf{differential equation}.

The differential equation
$$ \frac{dx}{dt} = x^2 + t^2 $$
involves both the unknown function $ x(t) $ and its first derivative $ x'(t) = \frac{dx}{dt} $.

The differential equation
$$ \frac{d^2y}{dx^2} + 3\frac{dy}{dx} + 7y = 0 $$
involves the unknown function $ y $ of the independent variable $ x $ and the first two derivatives $ y' $ and $ y'' $ of $ y $.

\par\noindent\rule{\textwidth}{0.4pt}

The study of differential equations has three principal goals:
\begin{enumerate}
    \item To discover the differential equation that describes a specified physical situation.
    \item To find – either exactly or approximately – the appropriate solution of that equation.
    \item To interpret the solution that is found.
\end{enumerate}

\par\noindent\rule{\textwidth}{0.4pt}

\textbf{Example 1:}

If $ C $ is a constant and
$$ y(x) = C{e^x}^{2}\text{,} $$
then
$$ \frac{dy}{dx} = C\left(2x{e^x}^{2}\right) = (2x)\left(C{e^x}^{2}\right) = 2xy $$
for all $ x $. This example defines an \textit{infinite} family of different solutions of this differential equation; one for each choice of the arbitrary constant \textit{C}.

\textbf{Example 2:}

Note that any function of the form
$$ P(t) = Ce^{kt} $$
is a solution of the differential equation
$$ \frac{dP}{dt} = kP $$
And can be verified as such
\begin{align*}
    P'(t) & = Cke^{kt} \\
          & = k\left(Ce^{kt}\right) \\
          & = kP(t)
\end{align*}
This example exemplifies that the differential equation $ \frac{dP}{dt} = kP $ has \textit{infinitely many} different solutions of the form $ P(t) = Ce^{kt} $ – one for each choice of the arbitrary constant $ C $.

\textbf{Example 3:}

Suppose that $ P(t) = Ce^{kt} $ is the population of a colony of bacteria at time $ t $, that the population at time $ t = 0 $ (hours, h) was $ 1000 $, and that the population doubled after $ 1 $ h. This additional information about $ P(t) $ yields the following equations:
\begin{align*}
    1000 & = P(0) = Ce^0 = C \\
    2000 & = P(1) = Ce^k
\end{align*}
It follows that $ C = 1000 $ and that $ e^k = 2 $, so $ k = \ln(2) \approx 0.693147 $. With this value of $ k $ the differential equation is
$$ \frac{dP}{dt} = (\ln(2))P \approx (0.693147)P $$
Substitution of $ k = \ln(2) $ and $ C = 1000 $ for the function $ P $ yields the particular solution
\begin{align*}
    P(t) & = 1000e^{\left(\ln(2)\right)t} \\
         & = 1000\left(e^{\ln(2)}\right)^t \\
         & = 1000 \cdot 2^t \rightarrow \text{because } e^{ln(2)} = 2
\end{align*}
that satisfies the given conditions. This solution can be used to predict future populations of the bacteria colony – e.g.
$$ P(1.5) = 1000 \cdot 2^{\sfrac{3}{2}} \approx 2828 $$

The condition $ P(0) = 1000 $ used in \textbf{Example 3} is called an \textbf{initial condition} because differential equations are frequently written for which $ t = 0 $ – the "starting time."

\subsection{Mathematical Models}

\textbf{Mathematical Modeling} is
\begin{enumerate}
    \item The formulation fo a real-world problem in mathematical terms; that is, the construction of a mathematical model.
    \item The analysis or solution of the resulting mathematical problem.
    \item The interpretation of the mathematical results in the context of the original real-world situation – for example, answering the question originally posed.
\end{enumerate}

A satisfactory mathematical model is subject to two contradictory requirements: It must be sufficiently detailed to represent the real-world situation with relative accuracy, yet it must be sufficiently simple to make the mathematical analysis practical. In other words, if the model is so detailed that it fully represents the physical situation, then the mathematical analysis may to be too difficult to carry out. On the contrary, if the model is too simple, the results may be so inaccurate as to be useless. Finding the tradeoff is \textit{key}.

\par\noindent\rule{\textwidth}{0.4pt}

\textbf{Example 1}
If $ C $ is a constant and $ y(x) = \frac{1}{(C-x)^2} $, then
\begin{align*}
    \frac{dy}{dx} & = \frac{d}{dx} \left(C-x\right)^{-1} \\
                  & = \frac{-1}{\left(C-x\right)^2} \cdot \frac{d}{dx}\left(C-x\right) \\
                  & = \frac{1}{\left(C-x\right)^2} \rightarrow x \neq C \\
                  & = y^2
\end{align*}
Thus
$$ y(x) = \frac{1}{C-x} $$
defines a solution of the differential equation
$$ \frac{dy}{dx} = y^2 \rightarrow x \neq C $$

\textbf{Example 2}
Verify that the function $ y(x) = 2x^{\sfrac{1}{2}} - x^{\sfrac{1}{2}}\ln(x) $ satisfies the differential equation
$$ 4x^2y'' + y = 0 $$
for all $ x > 0 $.

Find $ y'(x) $
\begin{align*}
    y'(x) & = x^{\sfrac{-1}{2}} - \left( \frac{1}{2}x^{\sfrac{-1}{2}} \cdot \ln(x) + \frac{1}{x} \cdot x^{\sfrac{1}{2}} \right) \\
          & = x^{\sfrac{-1}{2}} - \frac{x^{\sfrac{-1}{2}}\ln(x)}{2} - x^{\sfrac{-1}{2}} \\
          & = -\frac{x^{\sfrac{-1}{2}}\ln(x)}{2} \\
          & = -\frac{1}{2}x^{\sfrac{-1}{2}}\ln(x)
\end{align*}
Find $ y''(x) $
\begin{align*}
    y''(x) & = -\frac{1}{2} \left( -\frac{1}{2}x^{\sfrac{-3}{2}} \cdot \ln(x) + \frac{1}{x} \cdot x^{\sfrac{-1}{2}} \right) \\
           & = \frac{1}{4}x^{\sfrac{-3}{2}}\ln(x) - \frac{1}{2}x^{\sfrac{-3}{2}}
\end{align*}
Substitute for $ 4x^2y'' + y = 0 $
\begin{align*}
    4x^2 \left( \frac{1}{4}x^{\sfrac{-3}{2}}\ln(x) - \frac{1}{2}x^{\sfrac{-3}{2}} \right) + 2x^{\sfrac{1}{2}} - x^{\sfrac{1}{2}}\ln(x) & = 0 \\
    x^{\sfrac{1}{2}}\ln(x) - 2x^{\sfrac{1}{2}} + 2x^{\sfrac{1}{2}} - x^{\sfrac{1}{2}}\ln(x) & = 0 \\
    x^{\sfrac{1}{2}}\ln(x) + \left( -x^{\sfrac{1}{2}}\ln(x) \right) - 2x^{\sfrac{1}{2}} + 2x^{\sfrac{1}{2}} & = 0 \\
    0 + 0 & = 0 \rightarrow 0 = 0
\end{align*}
$ \therefore $ if $ x $ is positive, the differential equation is satisfied for all $ x > 0 $.

\par\noindent\rule{\textwidth}{0.4pt}

The \textbf{order} of a differential equation is the order of the highest derivative that appears in it. e.g.
$$ y^{(4)} + x^2y^{(3)} + x^5y = \sin(x) $$
is a fourth-order equation.

The most general form of an \textbf{\textit{n}th-order} differential equation with independent variable $ x $ and unknown function or dependent variable $ y= y(x) $ is
$$ F \left( x,y,y',y'',\cdots,y^{(n)} \right) = 0 $$
where $ F $ is a specific real-valued function of $ n + 2 $ variables.

A \textbf{solution} is the continuous function $ u = u(x) $ of the differential equation \textbf{on the interval} $ I $ provided that the derivatives $ u',u'',\cdots,u^{(n)} $ exist on $ I $ and
$$ F \left( x,u,u',u'',\cdots,u^{(n)} \right) = 0 $$
for all $ x $ in  $ I $. For the sake of brevity, we may say that $ u = u(x) $ \textbf{satisfies} the differential equation on $ I $.

\par\noindent\rule{\textwidth}{0.4pt}

\textbf{Example 1}
If $ A $ and $ B $ are constants and
$$ y(x) = A\cos(3x) + B\sin(3x) $$
then two successive differentiations yield 
\begin{align*}
    y'(x) & = -3A\sin(3x) + 3B\cos(3x) \\
    y''(x) & = -9A\cos(3x) - 9B\sin(3x) = -9y(x)
\end{align*}
for all $ x $.

\subsection{Slope Fields and Solution Curves}
-

\subsection{Separable Equations and Applications}

\textbf{Video Lecture}:

We have a function:
$$ \frac{dy}{dx} = f(x,y) $$
If $ \frac{dy}{dx} $ can be rewritten, or factored as a product
$$ \phi(x) \psi(y) \text{, where} $$
$$ \frac{1}{\psi(y)}dy = \phi(x)dx $$
then it can be called \textbf{separable}.

\hr

If function $ f(x,y) $ does not involve the variable $ y $, then solving the first-order differential equation
$$ \frac{dy}{dx} = f(x,y) $$
is found by integrating.

\textbf{Example 1}

Find the general solution of
$$ \frac{dy}{dx} = -6x $$

The solution can be found by
\begin{align}
    y(x) & = \int -6xdx \\
         & = -3x^2 + C
\end{align}

\hr

If instead $ f(x,y) $ \textit{does} involve the dependent variable $ y $, then we can no longer solve the equation merely by integrating both sides. For example, the first-order differential equation
\begin{equation}
    \frac{dy}{dx} = -6xy
\end{equation}
differs due to the factor $ y $ appearing on the right-hand side.

Although the differential equation cannot be solved by integrating both sides, it can be solved by methods which are based on the idea of ``integrating both sides." The idea to is to rewrite the given equation in a form that, while equivalent to the given equation, allows both sides to be integrated directly, thus leading to the solution of the original differnetial equation. The method focused on here is the \textit{seperation of variables}.

Using the function (3)
$$ \frac{dy}{dx} = -6xy $$
we can apply the separation of variables by first noting that the right-hand function can be viewed as the product of two expressions; one involving only the independent variable $ x $, and the other involving only the dependent variable $ y $:
\begin{equation}
    \frac{dy}{dx} = \left( -6x \right) \cdot y
\end{equation}
Next, one can informally break up the left-hand side derivative into ``free-floating" differentials $ dx $ and $ dy $ shown as such
\begin{align}
    \frac{dy}{dx} & = \left( -6x \right) \cdot y \\
    \frac{dy}{dx} (dx) & = \left( -6x \right) \cdot y (dx) \text{, multiply $ dx $ on both sides} \\
    dy \left( \frac{1}{y} \right) & = \left( -6x \right) dx \cdot y \left( \frac{1}{y} \right) \text{, multiply $ \frac{1}{y} $ on both sides} \\
    \frac{dy}{y} & = -6xdx
\end{align}
A differential equation with equivalence to (3) has been reached, but with the variables $ x $ and $ y $ seperated, which allows integration on each side. The left-hand side is integrated with respect to $ y $ (with no ``interference" from the variable $ x $), and \textit{vice versa} for the right-hand side.
\begin{align}
    \int \frac{dy}{y} & = \int -6xdx \\
    \ln|y| & = -3x^2 + C
\end{align}
This results in the general solution of (3) \textit{implicitly}. One can continue to solve for $ y $ to give the \textit{explicit} general solution
\begin{align}
    e^{\ln|y|} & = \pm e^{-3x^2 + C} \\
    y(x) & = \pm e^{-3x^2}e^C \\
         & = Ae^{-3x^2}
\end{align}
where $ A $ represents the constant $ \pm e^C $, which can take on any nonzero value. If an initial condition is imposed – $ y(0) = 7 $, then $ A $ can be found as $ A = 7 $, yielding
\begin{equation}
    y(x) = 7e^{-3x^2}
\end{equation}

\hr

In general, the first-order differential equation is called \textbf{separable} provided that $ f(x,y) $ can be written as the product of a function $ x $ and a function $ y $:
$$ \frac{dy}{dx} = f(x,y) = g(x)k(y) = \frac{g(x)}{h(y)} $$
where $ h(y) = \frac{1}{k(y)} $. In this case the variables $ x $ and $ y $ can be \textit{separated} – isoloated on opposite sides of an equation – by writing informally the equation
$$ h(y)dy = g(x)dx $$

\hr

\textbf{Example 2}

Solve the differential equation
\begin{align*}
    \frac{dy}{dx} & = \frac{4 - 2x}{3y^2 - 5} \\
    dy \cdot \frac{1}{dx} & = \frac{4 - 2x}{3y^2 - 5} \\
    \int \left( 3y^2 - 5 \right) dy & = \int \left( 4 - 2x \right) dx \\
    y^3 - 5y & = -x^2 + 4x + C \text{, isolating $ C $} \\
    y^3 - 5y + x^2 - 4x & = C \\
    F(x,y) & = y^3 - 5y + x^2 - 4x
\end{align*}
Find where the function may be discontinuous
\begin{align*}
    \frac{4 - 2x}{3y^2 - 5} \\
    3y^2 - 5 & = 0 \\
    3y^2 & = 5 \\
    y & = \pm \sqrt{\frac{5}{3}}
\end{align*}
This shows that $ y $ is discontinuous on positive and negative $ \sqrt{\frac{5}{3}} $.

\subsection{Linear First-Order Equations}

\textbf{Video Lecture}:

In the form:
$$ \frac{dy}{dx} + \rho(x)y = q(x) $$
Find the integrating factor:
$$ \rho(x) = e^{\int \rho(x)dx} $$
\begin{enumerate}
    \item Integrate $ \int \rho(x)dx $ and let $ c = 0 $
    \item $ \rho(x) = e^{\int \rho(x)dx} $
\end{enumerate}

\hr

Given the differential equation
$$ \frac{dy}{dx} = 2xy \text{ where } (y > 0) $$
This can be turned into
$$ \frac{dy}{y} = 2xdx $$
Which can be rewritten as
$$ \frac{1}{y} \cdot \frac{dy}{dx} = 2x $$
Which can be rewritten as the derivative of the integral of each side
$$ D_x(\ln(y)) = D_x(x^2) $$
Integrating both sides with respect to $ x $ gives the same general solution $ \ln(y) = x^2 + C $ shown before.

The differential equation was able to be solved by first multiplying both of its sides by a factor – known as an \textit{integrating factor} – chosen so that both sides of the resulting equation couldbe recognized as a derivative. More generally, an \textbf{integrating factor} for a differential equation by $ \rho(x,y) $ yields an equation in which each side is recognizable as a derivative. In some cases integrating factors involve both of the variables $ x $ and $ y $; however the second solution of the above equation was based on the integrating factor $ \rho(y) = \frac{1}{y} $, which depends only on $ y $.

\hr

A \textbf{linear first-order equation} is a differential equation of the form
$$ \frac{dy}{dx} + P(x)y = Q(x) $$
It is assumed that the coefficient functions $ P(x) $ and $ Q(x) $ are continuous on some interval on the $ x $-axis. Assuming that the necessary integrals can be found, the general linear equation can always be sovled by multiplying by the integrating factor
$$ \rho(x) = e^{\int P(x)dx} $$
Which is expanded to
$$ e^{\int P(x)dx}\frac{dy}{dx} + P(x)e^{\int P(x)dx}y = Q(x)e^{\int P(x)dx} $$
And as the derivative of the integral of the function $ P(x) $ is equal to the function itself, the equation can be again rewritten as
$$ D_x \left[ y(x) \cdot e^{\int P(x)dx} \right] = Q(x)e^{\int P(x)dx} $$
And when integrating both sides
$$ y(x)e^{\int P(x)dx} = \int \left(Q(x)e^{\int P(x)dx} \right)dx + C $$
Solving for the function $ y $
$$ y(x) = e^{-\int P(x)dx} \left[ \int \left(Q(x)e^{\int P(x)dx} \right)dx + C \right] $$

\subsection{Substitution Methods and Exact Equations}

\end{document}
