\documentclass{article}

% Document extensibility %
%
% Disables native paragraph indentation
\usepackage{parskip} 
%
% Provides further bullet options for lists
\usepackage{enumitem}

% Mathematical symbol and statement packages %
%
% Necessary
\usepackage{amsmath}
\usepackage{amssymb}
%
% Extensive fraction notation
\usepackage{xfrac}
%
% Generic mathematical commands
% Notable: \degree, \celcius
\usepackage{gensymb}
%
% Variable vector notation (arrow above variable)
\usepackage{esvect}
%
% Multiline boxed equations
\usepackage{empheq}
%
% SI Unit
\usepackage{siunitx}

% Graphic packages %
%
% Diagrams and illustrations
\usepackage{tikz}
%
% Image insertion
\usepackage{graphicx}

% Document content %
%
% Change title of table of contents
% \renewcommand{\contentsname}{Title}

\begin{document}

% Command `\hr` to insert horizontal rules
\newcommand{\hr}{\par\noindent\rule{\textwidth}{0.4pt}}

% Command to box and center math equations
\newcommand{\bc}[1]{
	\begin{equation*}
		\begin{boxed}
			{#1}
		\end{boxed}
	\end{equation*}
}

% Command for single line equations with a condition
\newcommand{\cond}[2]{
	\ifmmode
		{#1} \quad {#2}
	\else
		$$ {#1} \quad {#2} $$
	\fi
}

\tableofcontents

\section{Numerical Approximation: Euler's Method}
Given the initial value problem
\cond{\frac{dy}{dx} = f(x,y),}{y(x_0) = y_0,}
Euler's method with step size $ h $ consists of applying the iterative formula
\cond{y_{n+1} = y_n + h \cdot f(x_n,y_n)}{(m \geq 0)}

\hr

\subsection{Example 1} \label{example:1}
Apply Euler's method to approximate the solution of the initial value problem
\begin{enumerate}[label=\textbf{(\alph*)}]
	\item first with step size $ h = 1 $ on the interval $ [0,5], $
	\item then with the step size $ h = 0.2 $ on the interval $ [0,1], $
\end{enumerate}
\begin{equation*}
	\cond{\frac{dy}{dx} = x + \frac{1}{5}y,}{y(0) = -3}
\end{equation*}
\begin{enumerate}[label=\textbf{(\alph*)}]
	\item
		\begin{align*}
			x_0 & = 0 \\
			y_0 & = -3 \\
			f(x,y) & = x + \frac{1}{5}y \\
			h & = 1
		\end{align*}
		\begin{align*}
			y_1 = y_0 + h \cdot \left[ x_0 + \frac{1}{5}y_0 \right] & = (-3) + (1) \left[ 0 + \frac{1}{5}(-3) \right] = -3.6 \\
			y_2 = y_1 + h \cdot \left[ x_1 + \frac{1}{5}y_1 \right] & = (-3.6) + (1) \left[ 1 + \frac{1}{5}(-3.6) \right] = -3.32 \\
			y_3 & = (-3.32) + (1) \left[ 2 + \frac{1}{5}(-3.32) \right] = -1.984 \\
			y_4 & = (-1.984) + (1) \left[ 3 + \frac{1}{5}(-1.984) \right] = 0.6192 \\
			y_5 & = (0.6912) + (1) \left[ 4 + \frac{1}{5}(0.6912) \right] \approx 4.7430
		\end{align*}
	\item
		\begin{align*}
			x_0 & = 0 \\
			y_0 & = -3 \\
			f(x,y) & = x + \frac{1}{5}y \\
			h & = 0.2
		\end{align*}
		\begin{align*}
			y_1 = y_0 + h \cdot \left[ x_0 + \frac{1}{5}y_0 \right] & = (-3) + (0.2) \left[ 0 + \frac{1}{5}(-3) \right] = -3.12 \\
			y_2 & = (-3.12) + (0.2) \left[ 0.2 + \frac{1}{5}(-3.12) \right] \approx -3.205 \\
			y_3 & \approx (-3.205) + (0.2) \left[ 0.4 + \frac{1}{5}(-3.205) \right] \approx -3.253 \\
			y_4 & \approx (-3.253) + (0.2) \left[ 0.6 + \frac{1}{5}(-3.253) \right] \approx -3.263 \\
			y_5 & \approx (-3.263) + (0.2) \left[ 0.8 + \frac{1}{5}(-3.263) \right] \approx -3.234
		\end{align*}
\end{enumerate}

\end{document}
