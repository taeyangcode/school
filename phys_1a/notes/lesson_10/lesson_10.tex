\documentclass{article}

% Document extensibility %
%
% Disables native paragraph indentation
\usepackage{parskip} 
%
% Provides further bullet options for lists
\usepackage{enumitem}

% Mathematical symbol and statement packages %
%
% Necessary
\usepackage{amsmath}
\usepackage{amssymb}
%
% Extensive fraction notation
\usepackage{xfrac}
%
% Generic mathematical commands
% Notable: \degree, \celcius
\usepackage{gensymb}
%
% Variable vector notation (arrow above variable)
\usepackage{esvect}
%
% Multiline boxed equations
\usepackage{empheq}
%
% SI Unit
\usepackage{siunitx}
\DeclareSIUnit\feet{ft}

% Graphic packages %
%
% Diagrams and illustrations
\usepackage{tikz}
%
% Image insertion
\usepackage{graphicx}

% Document content %
%
% Change title of table of contents
% \renewcommand{\contentsname}{Title}

\begin{document}

% Command `\hr` to insert horizontal rules
\newcommand{\hr}{\par\noindent\rule{\textwidth}{0.4pt}}

% Command to box and center math equations
\newcommand{\bc}[1]{
	\begin{equation*}
		\begin{boxed}
			{#1}
		\end{boxed}
	\end{equation*}
}

% Command for single line equations with a condition
\newcommand{\cond}[2]{
	\ifmmode
		{#1} \quad {#2}
	\else
		$$ {#1} \quad {#2} $$
	\fi
}

\tableofcontents

\section{Range Equation}
$$ R = \frac{v_0^2\sin(2\theta)}{g} $$
Only valid of $ \Delta y = 0 $

\subsection{Lab Problem - 676}
\begin{align*}
	R & = \frac{v_0^2\sin(2\theta)}{g} \\
	v_0 & = \sqrt{\frac{Rg}{\sin(2\theta)}} \\
	v_0 & = \sqrt{\frac{(\SI{192}{\feet})(\SI{-32.17}{\feet \per \second \squared})}{\sin(2(\SI{37}{\degree}))}} \\
	v_0 & = \SI{79.18}{\feet \per \second}
\end{align*}
\begin{align*}
	0 & = \SI{160}{\feet} + (\SI{-79.18}{\feet \per \second})(\cos(53\degree))t + \frac{1}{2}(\SI{-32.17}{\feet \per \second \squared})t^2 \\
	t & = \SI{2.003}{\second}
\end{align*}
\begin{align*}
	x & = x_0 + v_{0_x}t \\
	x & = 0 + (\SI{79.18}{\feet \per \second})(\sin(\SI{53}{\degree}))(\SI{2.003}{\second}) \\
	x & = \SI{126.7}{\feet}
\end{align*}
\bc{x & = \SI{126.7}{\feet}}

\end{document}
