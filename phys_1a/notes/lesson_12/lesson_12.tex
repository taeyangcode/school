\documentclass{article}

% Document extensibility %
%
% Disables native paragraph indentation
\usepackage{parskip} 
%
% Provides further bullet options for lists
\usepackage{enumitem}

% Mathematical symbol and statement packages %
%
% Necessary
\usepackage{amsmath}
\usepackage{amssymb}
%
% Extensive fraction notation
\usepackage{xfrac}
%
% Generic mathematical commands
% Notable: \degree, \celcius
\usepackage{gensymb}
%
% Variable vector notation (arrow above variable)
\usepackage{esvect}
%
% Multiline boxed equations
\usepackage{empheq}
%
% SI Unit
\usepackage{siunitx}
%
% More intuitive arrays/matrices
\usepackage{array}

% Graphic packages %
%
% Diagrams and illustrations
\usepackage{tikz}
%
% Image insertion
\usepackage{graphicx}

% Document content %
%
% Change title of table of contents
% \renewcommand{\contentsname}{Title}

\begin{document}

% Command `\hr` to insert horizontal rules
\newcommand{\hr}{\par\noindent\rule{\textwidth}{0.4pt}}

% Command to box and center math equations
\newcommand{\bc}[1]{
	\begin{equation*}
		\begin{boxed}
			{#1}
		\end{boxed}
	\end{equation*}
}

% Command for single line equations with a condition
\newcommand{\cond}[2]{
	\ifmmode
		{#1} \quad {#2}
	\else
		$$ {#1} \quad {#2} $$
	\fi
}

\tableofcontents

\section{Newton's Second Law w/ Acceleration}

\subsection{Example}
\begin{align*}
	v_1 & = v \\
	v_2 & = \frac{1}{2}v \\
	a_1 & = a \\
	a_2 & = \frac{1}{2}a \\
	m_1 & = \SI{8}{\kilogram} \\
	m_2 & = \SI{10}{\kilogram} \\
	a_{m_1} & = ? \\
	a_{m_2} & = ?
\end{align*}
\begin{align*}
	\sum F_y^{(m_1)} & = 0 \\
	N_{m_1} & = w_{m_1}
\end{align*}
\begin{align*}
	\sum F_x^{(m_1)} & = -a_{m_1} \\
	T_1 & = -a_{m_1}
\end{align*}
\begin{align*}
	\sum F_y^{P} & = 0 \\
	T_1 & = 2T_2 \\
	T_2 & = \frac{1}{2}T_1
\end{align*}
\begin{align*}
	\sum F_y^{(m_2)} & = -a_{m_2} \\
	T_2 & = -w_{m_2} - a_{m_2}
\end{align*}
\begin{align*}
	-w_{m_2} - a_{m_2} & = \frac{1}{2}T_1 \\
	T_1 & = -2(w_{m_2} + a_{m_2})
\end{align*}
\begin{align*}
	-2(w_{m_2} + a_{m_2}) & = -a_{m_1} \\
	a_{m_1} & = 2(w_{m_2} + a_{m_2}) \\
	a_{m_2} & = \frac{1}{2}a_{m_1} - w_{m_2}
\end{align*}

\subsection{Example - Lab Manual 592}
\begin{align*}
	m_1 & = \SI{10}{\kilogram} \\
	m_2 & = \SI{5}{\kilogram} \\
	m_3 & = \SI{3}{\kilogram} \\
	\theta & = \SI{25}{\degree} \\
	\phi & = \SI{65}{\degree}
\end{align*}
\begin{align*}
	\sum F_x^{(m_3)} & = m_3a \\
	f_{m_3} & = m_3a + T_2 + m_3g\sin(\theta) \\
	T_2 & = \mu N_{m_3} - m_3a - m_3g\sin(\theta)
\end{align*}
\begin{align*}
	\sum F_y^{(m_3)} & = 0 \\
	N_{g,m_3} & = m_3g\cos(\theta) \\
			  & = (\SI{3}{\kilogram})(\SI{10}{\meter \per \second \squared})\cos(\SI{25}{\degree}) \\
	N_{g,m_3} & = \SI{27.2}{\newton}
\end{align*}
\begin{align*}
	T_2 & = \mu N_{m_3} - m_3a - m_3g\sin(\theta) \\
	T_2 & = \mu \SI{27.2}{\newton} - (\SI{3}{\kilogram})(\SI{2.35}{\meter \per \second \squared}) - (\SI{3}{\kilogram})(\SI{10}{\meter \per \second \squared})\sin(\SI{25}{\degree})
\end{align*}
\begin{align*}
	\sum F_y^{(m_1)} & = -m_1a \\
	T_1 & = m_1g - m_1a \\
		& = (\SI{10}{\kilogram})(\SI{10}{\meter \per \second \squared}) - (\SI{10}{\kilogram})(\SI{2.35}{\meter \per \second \squared}) \\
	T_1 & = \SI{76.5}{\newton}
\end{align*}
\begin{align*}
	\sum F_y^{(m_2)} & = 0 \\
	N_{g,m_2} & = m_2g \\
			  & = (\SI{5}{\kilogram})(\SI{10}{\meter \per \second \squared}) \\
	N_{g,m_2} & = \SI{50}{\newton}
\end{align*}
\begin{align*}
	\sum F_x^{(m_2)} & = -m_2a \\
	T_2 + f_{m_2} & = -m_2a + T_1 \\
	T_2 + \mu N_{m_2} & = -m_2a + T_1 \\
\end{align*}

\end{document}
