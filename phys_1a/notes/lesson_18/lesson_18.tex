\documentclass{article}

% Document extensibility %
%
% Disables native paragraph indentation
\usepackage{parskip} 
%
% Provides further bullet options for lists
\usepackage{enumitem}

% Mathematical symbol and statement packages %
%
% Necessary
\usepackage{amsmath}
\usepackage{amssymb}
%
% Extensive fraction notation
\usepackage{xfrac}
%
% Generic mathematical commands
% Notable: \degree, \celcius
\usepackage{gensymb}
%
% Variable vector notation (arrow above variable)
\usepackage{esvect}
%
% Multiline boxed equations
\usepackage{empheq}
%
% SI Unit
\usepackage{siunitx}
\usepackage{physunits}
%
% More intuitive arrays/matrices
\usepackage{array}
%
% Linear Equations
\usepackage{systeme}
%
% Boxes!
\usepackage{mdframed}

% Graphic packages %
%
% Diagrams and illustrations
\usepackage{tikz}
\usetikzlibrary{positioning}
%
% Image insertion
\usepackage{graphicx}

% Document content %
%
% Change title of table of contents
% \renewcommand{\contentsname}{Title}

\begin{document}

% Command `\hr` to insert horizontal rules
\newcommand{\hr}{\par\noindent\rule{\textwidth}{0.4pt}}

% Command to box and center math equations
\newcommand{\bc}[1]{
	\begin{equation*}
		\begin{boxed}
			{#1}
		\end{boxed}
	\end{equation*}
}

% Command for single line equations with a condition
\newcommand{\cond}[2]{
	\ifmmode
		{#1} \quad {#2}
	\else
		$$ {#1} \quad {#2} $$
	\fi
}

\tableofcontents

\section{Energy}

\subsection{Example}

\begin{align*}
	h_{0_b} & = \SI{1}{\meter} \\
	h_{1_b} & = \SI{0.7}{\meter} \\
	h_{0_t} & = \SI{0.8}{\meter} \\
	h_{1_t} & = \SI{0.4}{\meter}
\end{align*}
\begin{align*}
	\epsilon & = \left| \frac{v_{ref_f}}{v_{ref_i}} \right| \\
	mgh & = \frac{1}{2}mv^2 \\
	v_i & = \sqrt{2gh_0} \\
	v_f & = \sqrt{2gh_1}
\end{align*}
\begin{align*}
	\epsilon & = \sqrt{ \frac{2gh_1}{2gh_0} } \\
	\epsilon & = \sqrt{ \frac{h_1}{h_0} } \\
	\epsilon_b & = 0.84 \\
	\epsilon_t & = 0.71
\end{align*}
A basketball and tennis ball are dropped so that the tennis ball sits on top of the basket ball. $ m_{tb} = \SI{0.0568}{\kilogram} $, $ m_{bb} = \SI{0.4848}{\kilogram} $, $ h_{o_b} = \SI{1.5}{\meter} $, $ D_b = \SI{0.20}{\meter} $.
\begin{enumerate}[label = \textbf{\arabic*)}]
	\item How fast does the basketball hit the ground?
		\begin{align*}
			E_{1.5_m} & = E_{0_m} \\
			m_{bb}gh_0 & = \frac{1}{2}m_{bb}u_A^2 \\
			u_A & = \sqrt{2gh_0} \\
			u_A & = \SI{5.5}{\meter \per \second}
		\end{align*}
	\item How fast does the basketball rebound?
		\begin{align*}
			u_B & = \epsilon u_A \\
			u_B & = (0.84)(\SI{5.5}{\meter \per \second}) \\
			u_B & = \SI{4.6}{\meter \per \second} = u_i \\
			v_i & = \SI{5.5}{\meter \per \second}
		\end{align*}
	\item How fast does the tennis ball hit the basketball?
		\begin{align*}
			v_{CM} & = \frac{m_{TB}v_i + m_{BB}u_i}{m_{TB} + m_{BB}} \\
			v_{CM} & = \frac{(\SI{0.0568}{\kilogram})(\SI{5.5}{\meter \per \second}) - (\SI{0.4848}{\kilogram})(\SI{4.6}{\meter \per \second})}{\SI{0.0568}{\kilogram} + \SI{0.4848}{\kilogram}} \\
			v_{CM} & = \SI{-3.54}{\meter \per \second}
		\end{align*}
		\begin{tabular}{ | c | c | c | c | }
			\hline
			initial & LAB & $ -v_{CM} $ & ZMF \\
			\hline
			$ \vec{v} $ & \SI{5.5}{\meter \per \second} & \SI{3.54}{\meter \per \second} & \SI{9.04}{\meter \per \second} \\
			\hline
			$ \vec{u} $ & \SI{-4.6}{\meter \per \second} & \SI{3.54}{\meter \per \second} & \SI{-1.06}{\meter \per \second} \\
			\hline
		\end{tabular}
		\begin{align*}
			v_f^{ZMF} & = -\epsilon v_i^{ZMF} \\
			v_f^{ZMF} & = -(0.71)(\SI{9.04}{\meter \per \second}) \\
			v_f^{ZMF} & = \SI{-6.42}{\meter \per \second}
		\end{align*}
		\begin{align*}
			v_f^{LAB} & = v_{CM} + v_f^{ZMF} \\
			v_f^{LAB} & = \SI{-6.42}{\meter \per \second} - \SI{3.54}{\meter \per \second} \\
			v_f^{LAB} & = \SI{-9.96}{\meter \per \second}
		\end{align*}
	\item How high does the tennis ball go?
		\begin{align*}
			E_c & = E_2 \\
			\frac{1}{2}mv_c^2 + mgh_c & = \frac{1}{2}mv_2^2 + mgh_2 \\
			\frac{1}{2}v_c^2 + gh_c & = 0 + gh_2 \\
			h_2 & = \frac{v_c^2 + gh_c}{2g} \\
			h_2 & = \frac{(\SI{9.9}{\meter \per \second})^2 + (\SI{10}{\meter \per \second \squared})(\SI{0.2}{\meter})}{2(\SI{10}{\meter \per \second \squared})} \\
			h_2 & = \SI{5.0005}{\meter}
		\end{align*}
\end{enumerate}

\section{Momentum Conservation - Rockets}

\subsection{Tsiolkovsky Rocket}

\begin{equation}
	\sum \vec{F}_{ext} = \frac{d\vec{p}}{dt}
\end{equation}
Momentum of rocket:
\begin{align*}
	P_i & = mv + vdm
\end{align*}
If the rocket speeds up:
\begin{align*}
	P_f & = m(v + dv) + udm
\end{align*}
What is the limit definition of the derivative?
\begin{align*}
	\frac{dp}{dt} \lim_{\Delta t \rightarrow 0} \frac{\Delta p}{\Delta t} & = \frac{P_f - P_i}{dt} \\
	\frac{dp}{dt} & = \frac{mv + mdv + udm - mv - vdm}{dt} \\
	\frac{dp}{dt} & = m\frac{dv}{dt} + (u - v)\frac{dm}{dt} \\
	\frac{dp}{dt} & = m\frac{dv}{dt} + u_{rel}\frac{dm}{dt}
\end{align*}
\begin{equation}
	\sum F = \frac{dp}{dt} = m\frac{dv}{dt} + u_{rel}\frac{dm}{dt}
\end{equation}

\subsection{Free Space Rocket}

\begin{align*}
	\sum F & = \frac{dp}{dt} \\
	0 & = \frac{dp}{dt} \\
	0 & = m\frac{dv}{dt} + u_{rel}\frac{dm}{dt}
\end{align*}
\begin{align*}
	\text{initial mass} & = m_0 + \Delta m \\
	\text{final mass} & = m_0 \\
	v_f - v_i & = \Delta v \quad \text{(Overall change in speed)}
\end{align*}
\begin{align*}
	m\frac{dv}{dt} + u_{rel}\frac{dm}{dt} & = 0 \\
	mdv + u_{rel}dm & = 0 \\
	mdv & = -u_{rel}dm \\
	dv & = -\frac{u_{rel}}{m}dm \\
	\int_{v_i}^{v_f} dv & = -u_{rel} \int_{m_0 + \Delta m}^{m_0} \frac{1}{m} dm \\
	\Delta v & = -u_{rel}\ln \left( \frac{m_0}{m_0 + \Delta m} \right) \\
	\Delta v & = u_{rel}\ln \left( \frac{m_0 + \Delta m}{m_0} \right)
\end{align*}
\begin{equation}
	\Delta v = u_{rel}\ln \left( \frac{m_0 + \Delta m}{m_0} \right)
\end{equation}

\subsection{Rocket Against Gravity}

\begin{align*}
	\sum F & = \frac{dp}{dt} \\
	m\frac{dv}{dt} + u_{rel}\frac{dm}{dt} + mg + gdm & = 0 \\
	mdv + u_{rel}dm + mgdt + gdmdt & = 0 \quad \text{(The $ dmdt $ can be assumed to equal $ 0 $)} \\
	dv + \frac{u_{rel}}{m}dm + gdt & = 0 \\
	dv & = -\frac{u_{rel}}{m}dm - gdt \\
	\int_{v_i}^{v_f} dv & = -u_{re} \int_{m + \Delta m}^{m_0} \frac{1}{m} dm - g\int_{t_i}^{t_f} dt \\
	\Delta v & = u_{rel} \ln \left( \frac{m_0 + \Delta m}{m_0} \right) - g\Delta t
\end{align*}
\begin{equation}
	\Delta v = u_{rel} \ln \left( \frac{m_0 + \Delta m}{m_0} \right) - g\Delta t
\end{equation}

\end{document}
