\documentclass{article}

% Document extensibility %
%
% Disables native paragraph indentation
\usepackage{parskip} 
%
% Provides further bullet options for lists
\usepackage{enumitem}

% Mathematical symbol and statement packages %
%
% Necessary
\usepackage{amsmath}
\usepackage{amssymb}
\usepackage{mathabx}
%
% Extensive fraction notation
\usepackage{xfrac}
%
% Generic mathematical commands
% Notable: \degree, \celcius
\usepackage{gensymb}
%
% Variable vector notation (arrow above variable)
\usepackage{esvect}
%
% Multiline boxed equations
\usepackage{empheq}
%
% SI Unit
\usepackage{siunitx}
\usepackage{physunits}
%
% More intuitive arrays/matrices
\usepackage{array}
%
% Linear Equations
\usepackage{systeme}
%
% Boxes!
\usepackage{mdframed}
%
% Matrix Notation
\usepackage{bm}

% Graphic packages %
%
% Diagrams and illustrations
\usepackage{tikz}
\usetikzlibrary{positioning}
%
% Image insertion
\usepackage{graphicx}

% Document content %
%
% Change title of table of contents
% \renewcommand{\contentsname}{Title}

\begin{document}

% Command `\hr` to insert horizontal rules
\newcommand{\hr}{\par\noindent\rule{\textwidth}{0.4pt}}

% Command to box and center math equations
\newcommand{\bc}[1]{
	\begin{equation*}
		\begin{boxed}
			{#1}
		\end{boxed}
	\end{equation*}
}

% Command for single line equations with a condition
\newcommand{\cond}[2]{
	\ifmmode
		{#1} \quad {#2}
	\else
		$$ {#1} \quad {#2} $$
	\fi
}

% Matrix and Vector notation
\newcommand{\matr}[1]{
	\ifmmode \bm{#1}
	\else \textit{\textbf{#1}}
	\fi
}
\newcommand{\vect}[1]{
	\ifmmode \mathbf{#1}
	\else \textbf{#1}
	\fi
}

\section{Planetary Constants}

\subsection{Kepler's Laws}

\begin{enumerate}[label = \textbf{Kepler's \arabic* Law:}]
	\item Planetary orbits are elliptical
	\item The radial line between a central object and an object in 
\end{enumerate}

\subsection{Constants}

Key Terms
\begin{itemize}
	\item Aphelion - Farthest from sun
	\item Perihelion - Closest approach to sun
\end{itemize}
Astronomical Symbols
\begin{itemize}
	\item $ \Earth $ \textleftarrow earth
	\item $ \Sun $ \textleftarrow sun
	\item $ \Moon $ \textleftarrow moon
\end{itemize}
\begin{align*}
	R_{P_{\Earth}} & = \SI{147e6}{\kilo \meter} \\
	R_{A_{\Earth}} & = \SI{152e6}{\kilo \meter} \\
	V_{P_{\Earth}} & = \SI{30.29}{\kilo \meter \per \second}
\end{align*}
What is $ V_{A_{\Earth}} $?
\begin{align*}
	\vect{L}_A & = \vect{L}_P \\
	R_{A_{\Earth}}M_{\Earth}V_{A_{\Earth}} & = R_{P_{\Earth}}M_{\Earth}V_{P_{\Earth}} \\
	V_{A_{\Earth}} & = \frac{ R_{P_{\Earth}} }{ R_{A_{\Earth}} }v_{P_{\Earth}} = \SI{29.29}{\kilo \meter \per \second}
\end{align*}

\end{document}
