\documentclass{article}

% Document extensibility %
%
% Disables native paragraph indentation
\usepackage{parskip} 
%
% Provides further bullet options for lists
\usepackage{enumitem}

% Mathematical symbol and statement packages %
%
% Necessary
\usepackage{amsmath}
\usepackage{amssymb}
%
% Extensive fraction notation
\usepackage{xfrac}
%
% Generic mathematical commands
% Notable: \degree, \celcius
\usepackage{gensymb}
%
% Variable vector notation (arrow above variable)
\usepackage{esvect}
%
% Multiline boxed equations
\usepackage{empheq}
%
% SI Unit
\usepackage{siunitx}

% Graphic packages %
%
% Diagrams and illustrations
\usepackage{tikz}
%
% Image insertion
\usepackage{graphicx}

% Document content %
%
% Change title of table of contents
% \renewcommand{\contentsname}{Title}

\begin{document}

% Command `\hr` to insert horizontal rules
\newcommand{\hr}{\par\noindent\rule{\textwidth}{0.4pt}}

% Command to box and center math equations
\newcommand{\bc}[1]{
	\begin{equation*}
		\begin{boxed}
			{#1}
		\end{boxed}
	\end{equation*}
}

% Defining inch as a valid unit in SI commands
\let\DeclareUSUnit\DeclareSIUnit
\let\US\SI
\DeclareUSUnit\inch{in}

\tableofcontents

\section{Torque}
\textbf{Torque $ (\vec{\tau}) $}
$$ \vec{\tau} = \vec{r} \times \vec{F} $$
\textbf{Torque} is a measure of a force's ability to rotate an object around a reference point

$ \vec{r} \rightarrow $ ``lever arm" \\
$ \vec{F} \rightarrow $ ``applied force"

\textbf{line of action}: the direction that a vector points along \\
With torque, you are allowed to move vectors along the line of action. Any force whose line of action goes through it contributes no torque.

\hr

\subsection{Two Equations of Static Equilibrium}

\begin{equation}
	\sum F = 0, \left( \frac{d\vec{v}}{dt} = 0 \right)
\end{equation}
\begin{equation}
	\sum \tau = 0, \left( \frac{d\vec{\omega}}{dt} = 0 \right)
\end{equation}

\subsection{Example \ref{example:1}} \label{example:1}

Two children $ m_1 = \SI{20}{\kilogram} $ and $ m_2 = \SI{35}{\kilogram} $ wish to play on an $ M = \SI{200}{\kilogram} $ $ L = \SI{4}{\meter} $ seesaw which is balanced in the middle. Child $ m_1 $ sites on the left end. Where must $ m_2 $ sit to balance the seesaw?
\begin{align*}
	m_1g & = \SI{20}{\kilogram} \cdot \SI{10}{\meter \per \second \squared} \\
		 & = \SI{200}{\newton} \\
	r_{m_1} & = \SI{2}{\meter} \\
	m_2g & = \SI{35}{\kilogram} \cdot \SI{10}{\meter \per \second \squared} \\
		 & = \SI{350}{\newton} \\
	r_{m_2} & = ? \\
	M & = \SI{200}{\kilogram} \cdot \SI{10}{\meter \per \second \squared} \\
	  & = \SI{2000}{N} \\
	N_M & = ?
\end{align*}
\begin{align*}
	\sum \tau & = 0 \\
	(r_{m_1})(m_1g) - (r_{m_2})(m_2g) & = 0 \\
	r_{m_2}m_2g & = r_{m_1}m_1g \\
	r_{m_2} & = \frac{m_1}{m_2}r_{m_2} \\
			& = \left( \frac{\SI{20}{\kilogram}}{\SI{35}{\kilogram}} \right) \left( \SI{2}{\meter} \right) \\
	r_{m_2} & = \SI{1.14}{\meter}
\end{align*}
\bc{r_{m_2} = \SI{1.14}{\meter}}

\hr

\subsection{Example \ref{example:2}} \label{example:2}
\begin{align*}
	r_F & = \SI{12}{\inch} \\
	r_M & = \SI{6}{\inch} \\
	F & = \SI{220}{\newton} \\
	mg & = \SI{1.5}{\newton} \\
	N & = ? \\
	r_N & = \SI{1}{\inch} \\
	F_H & = ?
\end{align*}
\begin{align*}
	\sum \tau & = 0 \\
	(r_{F_\text{app}})(F_\text{app}) + (r_m)(mg) - (r_N)(N) & = 0 \\
	(r_N)(N) & = (r_{F_\text{app}})(F_\text{app}) + (r_m)(mg) \\
	N & = \frac{(r_{F_\text{app}})(F_\text{app}) + (r_m)(mg)}{(r_N)} \\
	  & = \frac{ (\SI{12}{\inch})(\SI{220}{\newton}) + (\SI{6}{\inch})(\SI{1.5}{\newton}) }{\SI{1}{\inch}} \\
	N & = \SI{2650}{\newton}
\end{align*}
\bc{N = \SI{2650}{\newton}}
\begin{align*}
	\sum F_y & = 0 \\
	F + mg - N - F_H & = 0 \\
	F_H & = F + mg - N \\
		& = \SI{220}{\newton} + \SI{1.5}{\newton} - \SI{2650}{\newton} \\
	F_H & = \SI{-2430}{\newton}
\end{align*}
\bc{F_H = \SI{-2430}{\newton}}

\subsection{Example \ref{example:3}} \label{example:3}
\begin{align*}
	\sum \tau & = 0 \\
	\frac{L}{3}T\sin(\theta) - \frac{L}{2}w - Lw & = 0 \\
	\frac{L}{3}T\sin(\theta) & = \frac{L}{2}w + Lw \\
	\frac{T\sin(\theta)}{3} & = \frac{w}{2} + w \\
	T\sin(\theta) & = \frac{3}{2}w + 3w \\
				  & = \frac{9}{2}w
\end{align*}
\begin{align*}
	\sum F_x & = 0 \\
	-T\cos(\theta) + N & = 0 \\
	T\cos(\theta) & = N \\
\end{align*}
$$ f = \mu N = \mu T\cos(\theta) $$
\begin{align*}
	\sum \tau & = 0 \\
	-\frac{fL}{3} - \frac{Lw}{6} - \frac{2Lw}{3} & = 0 \\
	-\frac{f}{3} & = \frac{w}{6} + \frac{2w}{3} \\
	f & = -3 \left( \frac{w}{6} + \frac{2w}{3} \right) \\
	  & = -\frac{w}{2} - 2w \\
	f & = \left| -\frac{5w}{2} \right| = \frac{5w}{2}
\end{align*}
\begin{align*}
	T\cos(\theta) & = \frac{f}{\mu} \\
				  & = \frac{5w}{2\mu}
\end{align*}
\begin{align*}
	\frac{T\sin(\theta)}{T\cos(\theta)} & = \frac{\frac{9w}{2}}{\frac{5w}{2\mu}} \\
	\tan(\theta) & = \frac{18\mu w}{10w} \\
	\tan(\theta) & = \frac{9\mu}{2}
\end{align*}
\bc{\tan(\theta) = \frac{9\mu}{2}}

\end{document}
