\documentclass{article}

% Document extensibility %
%
% Disables native paragraph indentation
\usepackage{parskip} 
%
% Provides further bullet options for lists
\usepackage{enumitem}

% Mathematical symbol and statement packages %
%
% Necessary
\usepackage{amsmath}
\usepackage{amssymb}
%
% Extensive fraction notation
\usepackage{xfrac}
%
% Generic mathematical commands
% Notable: \degree, \celcius
\usepackage{gensymb}
%
% Variable vector notation (arrow above variable)
\usepackage{esvect}
%
% Multiline boxed equations
\usepackage{empheq}
%
% SI Unit
\usepackage{siunitx}
\DeclareSIUnit\pound{lb}
\DeclareSIUnit\foot{ft}

% Graphic packages %
%
% Diagrams and illustrations
\usepackage{tikz}
%
% Image insertion
\usepackage{graphicx}

% Document content %
%
% Change title of table of contents
% \renewcommand{\contentsname}{Title}

\begin{document}

% Command `\hr` to insert horizontal rules
\newcommand{\hr}{\par\noindent\rule{\textwidth}{0.4pt}}

% Command to box and center math equations
\newcommand{\bc}[1]{
	\begin{equation*}
		\begin{boxed}
			{#1}
		\end{boxed}
	\end{equation*}
}

\section{2D Torque}
Verge of tipping location of $ N $ is at an extreme point

\hr

\subsection{Example 1} \label{example:1}
\begin{align}
	\sum F_x & = 0 \\
	F - f & = 0 \\
	F & = f \\
	F & = \mu N
\end{align}
\begin{align}
	\sum F_y & = 0 \\
	N - mg & = 0 \\
	N & = mg
\end{align}
\bc{ F = \mu mg }
\begin{align}
	\sum \tau_\star & = 0 \\
	\frac{w}{2}mg - xN - hF & = 0 \\
	xN & = \frac{w}{2}mg - hF \\
	x & = \frac{ \frac{w}{s}mg - h(\mu mg) }{mg} \\
	  & = \SI{0.125}{\meter} - (0.4)(\SI{0.125}{\meter}) \\
	x & = \SI{0.075}{\meter}
\end{align}
\bc{x = \SI{0.075}{\meter}}
\begin{align}
	\sum \tau_\star & = 0 \\
	\frac{w}{2}mg - hF & = 0 \\
		h & = \frac{ \frac{w}{2}mg }{\mu mg} \\
		  & = \frac{w}{2\mu} \\
		  & = \frac{\SI{0.25}{\meter}}{2(0.4)} \\
		h & = \SI{0.31}{\meter}
\end{align}
\bc{h = \SI{0.31}{\meter}}

\hr

\subsection{Example 2 - Lab Manual 381} \label{example:2}
\begin{enumerate}[label=\textbf{(\alph*)}]
\item
	\begin{align}
		\sum F_y & = 0 \\
		N_B + N_F - mg & = 0 \\
		N_B + N_F & = \SI{80}{\pound}
	\end{align}
	\begin{align}
		\sum F_x & = 0 \\
		F - f_F - f_B & = 0 \\
		F & = f_F + f_B \\
		  & = \mu(N_B + N_F) \\
		  & = (0.2)(\SI{80}{\pound}) \\
		F & = \SI{16}{\pound}
	\end{align}
\item
	\begin{align}
		\sum \tau_\star & = 0 \\
		\SI{2}{\foot}N_B + \SI{2}{\foot}N_F + \SI{3}{\foot}P & = 0
	\end{align}
\end{enumerate}

\hr

\subsection{Example 3} \label{example:3}

\end{document}
