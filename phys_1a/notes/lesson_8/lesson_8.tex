\documentclass{article}

% Document extensibility %
%
% Disables native paragraph indentation
\usepackage{parskip} 
%
% Provides further bullet options for lists
\usepackage{enumitem}

% Mathematical symbol and statement packages %
%
% Necessary
\usepackage{amsmath}
\usepackage{amssymb}
%
% Extensive fraction notation
\usepackage{xfrac}
%
% Generic mathematical commands
% Notable: \degree, \celcius
\usepackage{gensymb}
%
% Variable vector notation (arrow above variable)
\usepackage{esvect}
%
% Multiline boxed equations
\usepackage{empheq}
%
% SI Unit
\usepackage{siunitx}

% Graphic packages %
%
% Diagrams and illustrations
\usepackage{tikz}
%
% Image insertion
\usepackage{graphicx}

% Document content %
%
% Change title of table of contents
% \renewcommand{\contentsname}{Title}

\begin{document}

% Command `\hr` to insert horizontal rules
\newcommand{\hr}{\par\noindent\rule{\textwidth}{0.4pt}}

% Command to box and center math equations
\newcommand{\bc}[1]{
	\begin{equation*}
		\begin{boxed}
			{#1}
		\end{boxed}
	\end{equation*}
}

% Command for single line equations with a condition
\newcommand{\cond}[2]{
	\ifmmode
		{#1} \quad {#2}
	\else
		$$ {#1} \quad {#2} $$
	\fi
}

\tableofcontents

\section{1D Motion}
\begin{align}
	\vec{v} & \equiv \frac{d\vec{x}}{dt} \\
	\vec{a} & \equiv \frac{d\vec{v}}{dt}
\end{align}

\subsection{Kinematics - Constant Acceleration}
When doing calculus you must establish \underline{Boundary Conditions} - A point in time where we know the values of the function.

\subsection{$ v = v_0 + at $ (No $ \Delta x $)}
During the equation of motion:
\begin{align}
	a & = const \\
	t & = 0; x = x_0, v = v_0
\end{align}
We want $ x(t) $. Start by finding velocity:
\begin{align}
	a & = \frac{dv}{dt} \\
	dv & = adt \\
	\int dv & = \int \left( a \right) dt \\
	v & = at + C \quad \text{at } t = 0, v = v_0 \\
	v_0 & = (0) + C \\
	C & = v_0 \\
	v & = v_0 + at
\end{align}

\subsection{Equation of Motion - $ x - x_0 + v_0t + \frac{1}{2}at^2 $ (No $ v $)}
Integrate $ v $ to get $ x(t) $:
\begin{align}
	v & = \frac{dx}{dt} \\
	v_0 + at & = \frac{dx}{dt} \\
	\int_{x_0}^x dx & = \int_0^t \left( v_0 + at \right) dt \\
	x - x_0 & = v_0t + \frac{1}{2}at^2 - 0 - 0 \\
x(t) & = x_0 + v_0t + \frac{1}{2}at^2
\end{align}
$ x $ as a function of $ t $ is called the ``equation of motion"

\subsection{$ x = x_0 + \bar{v}t $ or $x = x_0 + \left( \frac{v_0 + v}{2} \right) t $ (No $ a $)}
\begin{enumerate}[label=\textbf{(\arabic*)}]
	\item $$ v = v_0 + at $$
	\item $$ a = \frac{v - v_0}{t} $$
	\item Plug (1) into (2)
		\begin{align}
			x & = x_0 + v_0t + \frac{1}{2}at^2 \\
			x & = x_0 + v_0t + \frac{1}{2} \left( \frac{v - v_0}{t} \right) t^2 \\
			x & = x_0 + \left( \frac{1}{2}v_0 + \frac{1}{2}v \right) t \\
			x & = x_0 + \left( \frac{v_0 + v}{2} \right) t
		\end{align}
\end{enumerate}

\subsection{$ v^2 = v_0^2 + 2a\Delta x $ (No $ t $)}
\begin{enumerate}[label=\textbf{(\arabic*)}]
	\item $ v = v_0 + at $
	\item $ t = \frac{v - v_0}{a} $
	\item Plug (1) into (3)
		\begin{align}
			x & = x_0 + \left( \frac{v_0 + v}{2} \right) \left( \frac{v - v_0}{a} \right) \\
			2a\Delta x & = (v + v_0)(v - v_0)
		\end{align}
\end{enumerate}

\subsection{$ x = x_0 - vt + \frac{1}{2}at^2 $ (No $ v_0 $)}
No derivation given.

\subsection{Relation}
\begin{itemize}
	\item Slope of velocity $ \rightarrow $ acceleration
	\item Slope of position $ \rightarrow $ velocity
	\item Concavity of position $ \rightarrow $ acceleration
	\item Derivative $ \rightarrow $ slope
	\item Integral $ \rightarrow $ area
\end{itemize}

\end{document}
