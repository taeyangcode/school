\documentclass{article}

% Document extensibility %
%
% Disables native paragraph indentation
\usepackage{parskip} 
%
% Provides further bullet options for lists
\usepackage{enumitem}

% Mathematical symbol and statement packages %
%
% Necessary
\usepackage{amsmath}
\usepackage{amssymb}
%
% Extensive fraction notation
\usepackage{xfrac}
%
% Generic mathematical commands
% Notable: \degree, \celcius
\usepackage{gensymb}
%
% Variable vector notation (arrow above variable)
\usepackage{esvect}
%
% Multiline boxed equations
\usepackage{empheq}
%
% SI Unit
\usepackage{siunitx}
\DeclareSIUnit\atmosphere{atm}
\usepackage{physunits}
%
% More intuitive arrays/matrices
\usepackage{array}
%
% Linear Equations
\usepackage{systeme}
%
% Boxes!
\usepackage{mdframed}
%
% Matrix Notation
\usepackage{bm}

% Graphic packages %
%
% Diagrams and illustrations
\usepackage{tikz}
\usetikzlibrary{positioning}
%
% Image insertion
\usepackage{graphicx}

% LaTeX Commands
%
% Argument Parser
\usepackage{xparse}

% Document content %
%
% Change title of table of contents
% \renewcommand{\contentsname}{Title}

\title{Homework 12}
\date{27 June 2023}
\author{Corey Mostero - 2566652}

\begin{document}

% Command `\hr` to insert horizontal rules
\newcommand{\hr}{\par\noindent\rule{\textwidth}{0.4pt}}

% Command to box and center math equations
\newcommand{\bc}[1]{
	\begin{equation*}
		\begin{boxed}
			{#1}
		\end{boxed}
	\end{equation*}
}

% Command for single line equations with a condition
\newcommand{\cond}[2]{
	\ifmmode
		{#1} \quad {#2}
	\else
		$$ {#1} \quad {#2} $$
	\fi
}

% Matrix and Vector notation
\newcommand{\matr}[1]{
	\ifmmode \bm{#1}
	\else \textit{\textbf{#1}}
	\fi
}
\newcommand{\vect}[1]{
	\ifmmode \mathbf{#1}
	\else \textbf{#1}
	\fi
}

% Laplace
\NewDocumentCommand{\lap}{o}{
	\IfNoValueTF{#1}
		{ \mathcal{L} }
		{ \mathcal{L} \left\{ {#1} \right\} }
}
\NewDocumentCommand{\ilap}{o}{
	\IfNoValueTF{#1}
		{ \mathcal{L}^{-1} }
		{ \mathcal{L}^{-1} \left\{ {#1} \right\} }
}

\maketitle
\newpage

\tableofcontents

\subsection{Question 1}

Density: What is the radius of a sphere that has a density of \SI{5000}{\kilogram \per \meter \cubed} and a mass of \SI{6.00}{\kilogram}?
\begin{align*}
	\rho & = \SI{5000}{\kilogram \per \meter \cubed} \\
	m & = \SI{6.00}{\kilogram} \\
	r & = ?
\end{align*}
\begin{align*}
	\rho & = \frac{ m }{ V } \\
	V & = \frac{ m }{ \rho } \\
	\frac{4}{3}\pi r^3 & = \frac{ m }{ \rho } \\
	r & = \sqrt[^3]{ \frac{ 3m }{ 4\pi \rho } } \\
	r & = \sqrt[^3]{ \frac{ 3(\SI{6.00}{\kilogram}) }{ 4\pi (\SI{5000}{\kilogram \per \meter \cubed}) } } \\
	r & = \SI{0.065922}{\meter} = \SI{6.59}{\centi \meter}
\end{align*}
\bc{ r = \SI{6.59}{\centi \meter} }

\subsection{Question 2}

Pressure in a fluid: A cubical box, \SI{5.00}{\centi \meter} on each side, is immersed in a fluid. The gauge pressure at the top surface of the box is \SI{594}{\pascal} and the gauge pressure on the bottom surface is \SI{1133}{\pascal}. What is the density of the fluid?
\begin{align*}
	h & = \SI{5.00}{\centi \meter} = \SI{0.05}{\meter} \\
	p_0 & = \SI{594}{\pascal} \\
	p_1 & = \SI{1133}{\pascal} \\
	\rho & = ?
\end{align*}
\begin{align*}
	p_1 & = p_0 + \rho gh \\
	\rho & = \frac{ p_1 - p_0 }{ gh } \\
	\rho & = \frac{ \SI{1133}{\pascal} - \SI{594}{\pascal} }{ (\SI{9.80}{\meter \per \second \squared})(\SI{0.05}{\meter}) } \\
	\rho & = \SI{1100}{\kilogram \per \meter \cubed}
\end{align*}
\bc{ \rho = \SI{1100}{\kilogram \per \meter \cubed} }

\subsection{Question 3}

Pressure in a fluid: As shown in the figure, a container has a vertical tube, whose inner radius is \SI{32.00}{\milli \meter}, connected to it at its side. An unknown liquid reaches level $ A $ in the container and level $ B $ in this tube - level $ A $ being \SI{5.0}{\centi \meter} higher than level $ B $. The liquid supports a \SI{20.0}{\centi \meter} high column of oil, between levels $ B $ and $ C $, whose density is \SI{460}{\kilogram \per \meter \cubed}. What is the density of the unknown liquid?
\begin{align*}
	y_{A,B} & = \SI{5.0}{\centi \meter} = \SI{0.05}{\meter} \\
	y_{B,C} & = \SI{20.0}{\centi \meter} = \SI{0.20}{\meter} \\
	\rho_{oil} & = \SI{460}{\kilogram \per \meter \cubed} \\
	\rho_{unknown} & = ?
\end{align*}
\begin{align*}
	\rho_{unk}gy_{A,B} & = \rho_{oil}gy_{B,C} \\
	\rho_{unk} & = \frac{ \rho_{oil}y_{B,C} }{ y_{A,B} } \\
	\rho_{unk} & = \frac{ (\SI{460}{\kilogram \per \meter \cubed})(\SI{0.20}{\meter}) }{ \SI{0.05}{\meter} } \\
	\rho_{unk} & = \SI{1840}{\kilogram \per \meter \cubed}
\end{align*}
\bc{ \rho_{unknown} = \SI{1840}{\kilogram \per \meter \cubed} \approx \SI{1800}{\kilogram \per \meter \cubed} }

\subsection{Question 4}

Pressure in a fluid: In the figure, an open tank contains a layer of oil floating on top of a layer of water (of density \SI{1000}{\kilogram \per \meter \cubed}) that is \SI{3.0}{\meter} thick, as shown. What must be the thickness of the oil layer if the gauge pressure at the bottom of the tank is to be \SI{5.0e4}{\pascal}? The density of the oil is \SI{510}{\kilogram \per \meter \cubed}.
\begin{align*}
	\rho_{water} & = \SI{1000}{\kilogram \per \meter \cubed} \\
	y_{water} & = \SI{3.0}{\meter} \\
	p_0 & = \SI{5.0e4}{\pascal} \\
	\rho_{oil} & = \SI{510}{\kilogram \per \meter \cubed} \\
	y_{oil} & = ?
\end{align*}
\begin{align*}
	\Delta p & = \rho_{water}gy_{water} + \rho_{oil}gy_{oil} \\
	y_{oil} & = \frac{ \Delta p - \rho_{water}gy_{water} }{ \rho_{oil}g } \\
	y_{oil} & = \frac{ \SI{5.0e4}{\pascal} - (\SI{1000}{\kilogram \per \meter \cubed})(\SI{9.80}{\meter \per \second \squared})(\SI{3.0}{\meter}) }{ (\SI{510}{\kilogram \per \meter \cubed})(\SI{9.80}{\meter \per \second \squared}) } \\
	y_{oil} & = \SI{4.12165}{\meter} = \SI{4.12}{\meter}
\end{align*}
\bc{ y_{oil} = \SI{4.12}{\meter} \approx \SI{4.1}{\meter} }

\subsection{Question 5}

Pascal's principle: A \SI{12000}{\newton} car is raised using a hydraulic lift, which consists of a U-tube with arms of unequal areas, filled with incompressible oil and capped at both ends with tight-fitting pistons. The wider arm of the U-tube has a radius of \SI{18.0}{\centi \meter} and the narrower arm has a radius of \SI{5.00}{\centi \meter}. The car rests on the piston on the wider arm of the U-tube. The pistons are initially at the same level. What is the initial force that must be applied to the smaller piston in order to start lifting the car?
\begin{align*}
	w_{car} & = \SI{12000}{\newton} \\
	r_{wide} & = \SI{18.0}{\centi \meter} = \SI{0.18}{\meter} \\
	r_{narrow} & = \SI{5.00}{\centi \meter} = \SI{0.05}{\meter} \\
	F & = ?
\end{align*}
\begin{align*}
	\frac{ F }{ A_{narrow} } & = \frac{ w_{car} }{ A_{wide} } \\
	F & = \frac{ w_{car}\pi r_{narrow}^2 }{ \pi r_{wide}^2 } \\
	F & = \frac{ (\SI{12000}{\newton})(\SI{0.05}{\meter})^2 }{ (\SI{0.18}{\meter})^2 } \\
	F & = \SI{925.926}{\newton} = \SI{926.0}{\newton}
\end{align*}
\bc{ F = \SI{926.0}{\newton} }

\subsection{Question 6}

Buoyancy: A board that is \SI{20.0}{\centi \meter} wide, \SI{5.00}{\centi \meter} thick, and \SI{3.00}{\meter} long has a density \SI{350}{\kilogram \per \meter \cubed}. The board is floating partially submerged in water of density \SI{1000}{\kilogram \per \meter \cubed}. What fraction of the volume is above the surface of the water?
\begin{align*}
	w_{plank} & = \SI{20.0}{\centi \meter} = \SI{0.20}{\meter} \\
	h_{plank} & = \SI{5.00}{\centi \meter} = \SI{0.05}{\meter} \\
	l_{plank} & = \SI{3.00}{\meter} \\
	\rho_{plank} & = \SI{350}{\kilogram \per \meter \cubed} \\
	\rho_{water} & = \SI{1000}{\kilogram \per \meter \cubed}
\end{align*}
\begin{align*}
	\rho_{plank} & = \frac{ m_{plank} }{ V_{plank} } \\
	m_{plank} & = \rho_{plank}V_{plank} \\
	m_{plank} & = (\SI{350}{\kilogram \per \meter \cubed})(\SI{0.20}{\meter})(\SI{0.05}{\meter})(\SI{3.00}{\meter}) \\
	m_{plank} & = \SI{10.5}{\kilogram}
\end{align*}
\begin{align*}
	\sum F_y & = 0 \\
	B & = w_{plank} \\
	\rho_{water}V_{water}g & = m_{plank}g \\
	V_{water} & = \frac{ m_{plank} }{ \rho_{water} } \\
	V_{water} & = \frac{ \SI{10.5}{\kilogram} }{ \SI{1000}{\kilogram \per \meter \cubed} } \\
	V_{water} & = \SI{0.0105}{\meter \cubed}
\end{align*}
Find the volume of the plank, then find its ratio compared with $ V_{water} $.
\begin{align*}
	V_{plank} & : V_{water} \\
	(\SI{0.20}{\meter})(\SI{0.05}{\meter})(\SI{3.00}{\meter}) & : \SI{0.0105}{\meter \cubed} \\
	\SI{0.03}{\meter \cubed} & : \SI{0.0105}{\meter \cubed} \\
	\frac{ \SI{0.03}{\meter \cubed} }{ \SI{0.03}{\meter \cubed} } & : \frac{ \SI{0.0105}{\meter \cubed} }{ \SI{0.03}{\meter \cubed} } \\
	1.00 & : 0.35
\end{align*}
In other words, for every 1.00 (the entirety of the block) block, 0.35 of it is submerged in the water. The fraction of volume that is above the surface of the water would be the difference:
\bc{ 1.00 - 0.35 = 0.650 }

\subsection{Question 7}

Buoyancy: A rock is suspended from a scale reads \SI{20.0}{\newton}. A beaker of water (having a density of \SI{1000}{\kilogram \per \meter \cubed}) is raised up so the rock is totally submerged in the water. The scale now reads \SI{12.5}{\newton}. What is the density of the rock?
\begin{align*}
	w_{rock_0} & = \SI{20.0}{\newton} \\
	w_{rock_1} & = \SI{12.5}{\newton} \\
	\rho_{rock} & = ?
\end{align*}
\begin{align*}
	B & = w_{air} - w_{water} \\
	B & = w_{rock_0} - w_{rock_1} \\
	B & = \SI{20.0}{\newton} - \SI{12.5}{\newton} \\
	B & = \SI{7.50}{\newton}
\end{align*}
\begin{align*}
	\sum F_y & = 0 \\
	B & = w_{rock_1} \\
	B & = \rho_{water}V_{rock}g \\
	V_{rock} & = \frac{ B }{ \rho_{water}g } \\
	V_{rock} & = \frac{ \SI{7.50}{\newton} }{ (\SI{1000}{\kilogram \per \meter \cubed})(\SI{9.80}{\meter \per \second \squared}) } \\
	V_{rock} & = \SI{0.000765}{\meter \cubed} = \SI{7.65e-4}{\meter \cubed}
\end{align*}
\begin{align*}
	\rho_{rock} & = \frac{ m_{rock} }{ V_{rock} } \\
	\rho_{rock} & = \frac{ \frac{ \SI{20.0}{\newton} }{ \SI{9.80}{\meter \per \second \squared} } }{ \SI{7.65e-4}{\meter \cubed} } \\
	\rho_{rock} & = \SI{2667.73}{\kilogram \per \meter \cubed} = \SI{2.67e3}{\kilogram \per \meter \cubed}
\end{align*}
\bc{ \rho_{rock} = \SI{2.67e3}{\kilogram \per \meter \cubed} }

\subsection{Question 8}

Buoyancy: A \SI{7.8}{\kilogram} solid sphere, made of metal whose density is \SI{2500}{\kilogram \per \meter \cubed}, is suspended by a card. When the sphere is immersed in water (of density \SI{1000}{\kilogram \per \meter \cubed}), what is the tension in the cord?
\begin{align*}
	m_{sphere} & = \SI{7.8}{\kilogram} \\
	\rho_{metal} & = \SI{2500}{\kilogram \per \meter \cubed} \\
	\rho_{water} & = \SI{1000}{\kilogram \per \meter \cubed} \\
	T & = ?
\end{align*}
\begin{align*}
	\rho_{sphere} & = \frac{ m_{sphere} }{ V_{sphere} } \\
	V_{sphere} & = \frac{ m_{sphere} }{ \rho_{sphere} }
\end{align*}
\begin{align*}
	\sum F_y & = 0 \\
	T + B & = w_{sphere} \\
	T + \rho_{water}V_{sphere}g & = m_{sphere}g \\
	T & = m_{sphere}g \left( 1 - \frac{ \rho_{water} }{ \rho_{sphere} } \right) \\
	T & = (\SI{7.8}{\kilogram})(\SI{9.80}{\meter \per \second \squared}) \left( 1 - \frac{ \SI{1000}{\kilogram \per \meter \cubed} }{ \SI{2500}{\kilogram \per \meter \cubed} } \right) \\
	T & = \SI{45.9}{\newton}
\end{align*}
\bc{ T = \SI{45.9}{\newton} \approx \SI{46.0}{\newton} }

\subsection{Question 9}

Buoyancy: A circular cylinder of height \SI{1.20}{\meter} having faces of diameter \SI{0.620}{\meter} is immersed in water of density \SI{1.00e3}{\kilogram \per \meter \cubed} with its axis vertical so that its faces are parallel to the surface of the water. The upper face is \SI{2.50}{\meter} below the surface of the water. The net force on this cylinder is observed to be \SI{1120}{\newton} downward. Atmospheric pressure is \SI{1.01e5}{\pascal}. What is the weight of the cylinder?
\begin{align*}
	h & = \SI{1.20}{\meter} \\
	d & = \SI{0.620}{\meter} \\
	\rho_{water} & = \SI{1.00e3}{\kilogram \per \meter \cubed} \\
	\Delta y & = \SI{2.50}{\meter} \\
	F_{total} & = \SI{1120}{\newton} \\
	p_{atm} & = \SI{1.01e5}{\pascal}
\end{align*}
\begin{align*}
	\sum F_y & = 0 \\
	F_{total} + B & = w \\
	w & = F_{total} + \rho_{water}V_{cylidner}g \\
	w & = \SI{1120}{\newton} + (\SI{1.00e3}{\kilogram \per \meter \cubed}) \left( \pi \left( \frac{ \SI{0.620}{\meter} }{ 2 } \right)^2 (\SI{1.20}{\meter}) \right) (\SI{9.80}{\meter \per \second \squared}) \\
	w & = \SI{4670.43}{\newton}
\end{align*}
\bc{ w = \SI{4670.43}{\newton} \approx \SI{4670}{\newton} }

\subsection{Question 10}

Flow rate: Water flowing through a pipe suddenly comes to a section of pipe where the pipe diameter decreases to \SI{86}{\percent} of its previous value. If the speed of the water in the larger section of the pipe was \SI{36}{\meter \per \second}, what is its speed in this smaller section?
\begin{align*}
	d_1 & = (0.86)d_0 \\
	v_0 & = \SI{36}{\meter \per \second} \\
	v_1 & = ?
\end{align*}
\begin{align*}
	A_0v_0 & = A_1v_1 \\
	\pi \left( \frac{ d_0 }{ 2 } \right)^2v_0 & = \pi \left( \frac{ (0.86)d_0 }{ 2 } \right)^2v_1 \\
	v_1 & = \frac{ v_0 }{ 0.86^2 } \\
	v_1 & = \frac{ \SI{36}{\meter \per \second} }{ 0.86^2 } \\
	v_1 & = \SI{48.675}{\meter \per \second}
\end{align*}
\bc{ v_1 = \SI{48.675}{\meter \per \second} \approx \SI{49}{\meter \per \second} }

\subsection{Question 11}

Bernoulli's principle: Incompressible water flows out of a large reservoir through a pipe that opens to the atmosphere \SI{5.70}{\meter} below the level of the water in the reservoir. What is the speed of the water as it comes out of the pipe?
\begin{align*}
	y_0 & = \SI{5.70}{\meter} \\
	y_1 & = 0 \\
	v_0 & = 0 \\
	\rho_{water} & = \SI{1000}{\kilogram \per \meter \cubed} \\
	v_1 & = ?
\end{align*}
\begin{align*}
	\rho_{water}gy_0 & = \frac{1}{2}\rho_{water}v_1^2 \\
	v_1 & = \sqrt{ 2gy_0 } \\
	v_1 & = \sqrt{ 2(\SI{9.80}{\meter \per \second \squared})(\SI{5.70}{\meter}) } \\
	v_1 & = \SI{10.5698}{\meter \per \second} = \SI{10.6}{\meter \per \second}
\end{align*}
\bc{ v_1 = \SI{10.6}{\meter \per \second} }

\subsection{Question 12}

Bernoulli's principle: A horizontal tube consists of a \SI{7.0}{\centi \meter} diameter pipe that narrows to a \SI{2.0}{\centi \meter} diameter throat. In the pipe, the water pressure is twice atmospheric pressure and the water flows with a speed of \SI{0.40}{\meter \per \second}. What is the pressure in the throat, assuming that the water behaves like an ideal fluid? The density of water is \SI{1000}{\kilogram \per \meter \cubed}, and atmospheric pressure is \SI{1.01e5}{\pascal}.
\begin{align*}
	d_0 & = \SI{7.0}{\centi \meter} = \SI{0.07}{\meter} \\
	d_1 & = \SI{2.0}{\centi \meter} = \SI{0.02}{\meter} \\
	p_0 & = 2(\SI{1.01e5}{\pascal}) = \SI{2.02e5}{\pascal} \\
	v_0 & = \SI{0.40}{\meter \per \second} \\
	p_1 & = ?
\end{align*}
\begin{align*}
	A_0v_0 & = A_1v_1 \\
	v_1 & = \frac{ \pi r_0^2v_0 }{ \pi r_1^2 } \\
	v_1 & = \frac{ \left( \frac{ \SI{0.07}{\meter} }{ 2 } \right)^2 (\SI{0.40}{\meter \per \second}) }{ \left( \frac{ \SI{0.02}{\meter} }{ 2 } \right)^2 } \\
	v_1 & = \SI{4.9}{\meter \per \second}
\end{align*}
\begin{align*}
	p_0 + \frac{1}{2}\rho_{water}v_0^2 & = p_1 + \frac{1}{2}\rho_{water}v_1^2 \\
	p_1 & = p_0 + \frac{1}{2}\rho_{water}v_0^2 - \frac{1}{2}\rho_{water}v_1^2 \\
	p_1 & = \SI{2.02e5}{\pascal} + \frac{1}{2}(\SI{1000}{\kilogram \per \meter \cubed}) \left[ (\SI{0.40}{\meter \per \second})^2 - (\SI{4.9}{\meter \per \second})^2 \right] \\
	p_1 & = \SI{190075}{\pascal}
\end{align*}
\bc{ p_1 = \SI{190075}{\pascal} \approx \SI{1.9}{\atmosphere} }

\subsection{Question 13}

Bernoulli's principle: Consider a very small hole in the bottom of a tank \SI{20.0}{\centi \meter} in diameter filled with water to a height of \SI{50.0}{\centi \meter}. Find the speed at which the water exits the tank through the hole.
\begin{align*}
	d & = \SI{20.0}{\centi \meter} = \SI{0.20}{\meter} \\
	r & = \sfrac{ d }{ 2 } = \SI{0.10}{\meter} \\
	h & = \SI{50.0}{\centi \meter} = \SI{0.50}{\meter} \\
	v_1 & = ?
\end{align*}
\begin{align*}
	\rho_{water}gh & = \frac{1}{2}\rho_{water}v_1^2 \\
	v_1 & = \sqrt{ 2gh } \\
	v_1 & = \sqrt{ 2(\SI{9.80}{\meter \per \second \squared})(\SI{0.50}{\meter}) } \\
	v_1 & = \SI{3.1305}{\meter \per \second} = \SI{3.13}{\meter \per \second}
\end{align*}
\bc{ v_1 = \SI{3.13}{\meter \per \second} }

\subsection{Question 14}

Bernoulli's principle: Water flows in the horizontal pipe shown in the figure. At point $ A $ the area is \SI{25.0}{\centi \meter \squared} and the speed of the water is \SI{2.00}{\meter \per \second}. At $ B $ the area is \SI{16.0}{\centi \meter \squared}. The fluid in the manometer is mercury, which has a density of \SI{13600}{\kilogram \per \meter \cubed}. We can treat water as an ideal fluid having a density of \SI{1000}{\kilogram \per \meter \cubed}. What is the manometer reading $ h $?
\begin{align*}
	A_0 & = \SI{25.0}{\centi \meter \squared} = \SI{0.25}{\meter \squared} \\
	v_0 & = \SI{2.00}{\meter \per \second} \\
	A_1 & = \SI{16.0}{\centi \meter \squared} = \SI{0.16}{\meter \squared} \\
	v_1 & = ? \\
	\rho_{mercury} & = \SI{13600}{\kilogram \per \meter \cubed} \\
	\rho_{water} & = \SI{1000}{\kilogram \per \meter \cubed} \\
	h & = ?
\end{align*}
\begin{align*}
	A_0v_0 & = A_1v_1 \\
	v_1 & = \frac{ A_0v_0 }{ A_1 } \\
	v_1 & = \frac{ (\SI{0.25}{\meter \squared})(\SI{2.00}{\meter \per \second}) }{ \SI{0.16}{\meter \squared} } \\
	v_1 & = \SI{3.125}{\meter \per \second}
\end{align*}
\begin{align*}
	p_0 + \frac{1}{2}\rho_{water}v_0^2 & = p_1 + \frac{1}{2}\rho_{water}v_1^2 \\
	\frac{1}{2}\rho_{water}(v_0^2 - v_1^2) & = p_1 - p_0 \\
	\frac{1}{2}\rho_{water}(v_0^2 - v_1^2) & = \rho_{mercury}gh \\
	h & = \frac{ \rho_{water}(v_0^2 - v_1^2) }{ 2\rho_{mercury}g } \\
	h & = \frac{ (\SI{1000}{\kilogram \per \meter \cubed})((\SI{2.00}{\meter \per \second})^2 - (\SI{3.125}{\meter \per \second})^2) }{ 2(\SI{13600}{\kilogram \per \meter \cubed})(\SI{9.80}{\meter \per \second \squared}) } \\
	h & = \SI{-0.02163}{\meter} = \SI{0.02163}{\meter}
\end{align*}
\bc{ h = \SI{0.02163}{\meter} = \SI{2.16}{\centi \meter} }

\subsection{Question 15}

Water stands at a depth $ H $ in a large, open tank whose side walls are vertical, as shown in the figure. A hole is made in one of the walls at a depth $ h $ below the water surface. At what distance $ R $ from the foot of the wall does the emerging stream strike the floor?
Approach: Use conservation of energy to determine the final velocity, then use kinematics to find distance $ R $.
\begin{align*}
	\rho_{water}gh_{hole} & = \frac{1}{2}\rho_{water}v_1^2 \\
	v_1 & = \sqrt{ 2gh_{hole} }
\end{align*}
\begin{align*}
	\Delta y & = v_{0_y}t + \frac{1}{2}gt^2 \\
	h & = 0 + \frac{1}{2}gt^2 \\
	t & = \sqrt{ \frac{ 2h }{ g } }
\end{align*}
\begin{align*}
	\Delta x & = v_1t - \frac{1}{2}a_xt^2 \\
	R & = v_1t \\
	R & = (\sqrt{ 2gh_{hole} }) \left( \sqrt{ \frac{ 2h }{ g } } \right) \\
	R & = \sqrt{ \frac{ 4g(H - h)h }{ g } } \\
	R & = 2\sqrt{ h(H - h) }
\end{align*}
\bc{ R = 2\sqrt{ h(H - h) } }

\end{document}
