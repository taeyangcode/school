\documentclass{article}

% Document extensibility %
%
% Disables native paragraph indentation
\usepackage{parskip} 
%
% Provides further bullet options for lists
\usepackage{enumitem}

% Mathematical symbol and statement packages %
%
% Necessary
\usepackage{amsmath}
\usepackage{amssymb}
%
% Extensive fraction notation
\usepackage{xfrac}
%
% Generic mathematical commands
% Notable: \degree, \celcius
\usepackage{gensymb}
%
% Variable vector notation (arrow above variable)
\usepackage{esvect}
%
% Multiline boxed equations
\usepackage{empheq}
%
% SI Unit
\usepackage{siunitx}
\usepackage{physunits}
%
% More intuitive arrays/matrices
\usepackage{array}
%
% Linear Equations
\usepackage{systeme}
%
% Boxes!
\usepackage{mdframed}
%
% Matrix Notation
\usepackage{bm}

% Graphic packages %
%
% Diagrams and illustrations
\usepackage{tikz}
\usetikzlibrary{positioning}
%
% Image insertion
\usepackage{graphicx}

% LaTeX Commands
%
% Argument Parser
\usepackage{xparse}

% Document content %
%
% Change title of table of contents
% \renewcommand{\contentsname}{Title}

\begin{document}

% Command `\hr` to insert horizontal rules
\newcommand{\hr}{\par\noindent\rule{\textwidth}{0.4pt}}

% Command to box and center math equations
\newcommand{\bc}[1]{
	\begin{equation*}
		\begin{boxed}
			{#1}
		\end{boxed}
	\end{equation*}
}

% Command for single line equations with a condition
\newcommand{\cond}[2]{
	\ifmmode
		{#1} \quad {#2}
	\else
		$$ {#1} \quad {#2} $$
	\fi
}

% Matrix and Vector notation
\newcommand{\matr}[1]{
	\ifmmode \bm{#1}
	\else \textit{\textbf{#1}}
	\fi
}
\newcommand{\vect}[1]{
	\ifmmode \mathbf{#1}
	\else \textbf{#1}
	\fi
}

% Laplace
\NewDocumentCommand{\lap}{o}{
	\IfNoValueTF{#1}
		{ \mathcal{L} }
		{ \mathcal{L} \left\{ {#1} \right\} }
}
\NewDocumentCommand{\ilap}{o}{
	\IfNoValueTF{#1}
		{ \mathcal{L}^{-1} }
		{ \mathcal{L}^{-1} \left\{ {#1} \right\} }
}

\section{F5: Terminal velocity, drag forces}

\begin{equation}
	\frac{ F }{ A } = \eta \frac{ dv }{ dy }
\end{equation}
\begin{equation}
	F_{viscous} = 6\pi \eta Rv
\end{equation}
\begin{equation}
	F_{inertia} = \frac{1}{2}C_D\rho \left( \pi R^2 \right) v^2
\end{equation}
\begin{itemize}
	\item $ C_D $ drag coefficient
	\item $ \rho $ density of fluid
	\item $ R $ radius of sphere
	\item $ \eta $ is the dynamic viscosity
\end{itemize}
\begin{equation}
	\sum F_y = mg - B - 6\pi \eta Rv - \frac{1}{2}C_D \rho \left( \pi R^2 \right) v^2
\end{equation}
\begin{itemize}
	\item If laminal flow ($ v $ is small): $ F_{inertia} < F_{viscosity} $
	\item If turbulant ($ v $ is big) : you may drop $ F_{viscosity} $ term
\end{itemize}
\begin{equation}
	Re = \frac{ \rho v D }{ \eta } \quad D \text{ is diameter}
\end{equation}

\subsection{Specific Gravity}

\begin{equation}
	\text{specific gravity} = \frac{ \rho_{object} }{ \rho_{water} }
\end{equation}
\begin{equation}
	\sum F_y = mg - B - F_{drag} = m(0)
\end{equation}
\begin{equation}
	F_{drag} = mg - B
\end{equation}

\subsection{Deriving Slope Equation}

\begin{align*}
	\Delta \vect{y} & = \vect{v}_{o_y} - \frac{1}{2}gt^2 \\
	y & = \left( \frac{ 1 }{ 2 }g \right)t^2 \\
	\ln(y) & = \ln \left( \frac{ g }{ 2 } \right) + 2\ln(t) \\
	F_{drag} & = Ev_t^k
\end{align*}

\end{document}
