\documentclass{article}

% Document extensibility %
%
% Disables native paragraph indentation
\usepackage{parskip} 
%
% Provides further bullet options for lists
\usepackage{enumitem}

% Mathematical symbol and statement packages %
%
% Necessary
\usepackage{amsmath}
\usepackage{amssymb}
%
% Extensive fraction notation
\usepackage{xfrac}
%
% Generic mathematical commands
% Notable: \degree, \celcius
\usepackage{gensymb}
%
% Variable vector notation (arrow above variable)
\usepackage{esvect}
%
% Multiline boxed equations
\usepackage{empheq}
%
% SI Unit
\usepackage{siunitx}
\DeclareSIUnit\cal{cal}
\usepackage{physunits}
%
% More intuitive arrays/matrices
\usepackage{array}
%
% Linear Equations
\usepackage{systeme}
%
% Boxes!
\usepackage{mdframed}
%
% Matrix Notation
\usepackage{bm}

% Graphic packages %
%
% Diagrams and illustrations
\usepackage{tikz}
\usetikzlibrary{positioning}
%
% Image insertion
\usepackage{graphicx}

% LaTeX Commands
%
% Argument Parser
\usepackage{xparse}

% Document content %
%
% Change title of table of contents
% \renewcommand{\contentsname}{Title}

\begin{document}

% Command `\hr` to insert horizontal rules
\newcommand{\hr}{\par\noindent\rule{\textwidth}{0.4pt}}

% Command to box and center math equations
\newcommand{\bc}[1]{
	\begin{equation*}
		\begin{boxed}
			{#1}
		\end{boxed}
	\end{equation*}
}

% Command for single line equations with a condition
\newcommand{\cond}[2]{
	\ifmmode
		{#1} \quad {#2}
	\else
		$$ {#1} \quad {#2} $$
	\fi
}

% Matrix and Vector notation
\newcommand{\matr}[1]{
	\ifmmode \bm{#1}
	\else \textit{\textbf{#1}}
	\fi
}
\newcommand{\vect}[1]{
	\ifmmode \mathbf{#1}
	\else \textbf{#1}
	\fi
}

% Laplace
\NewDocumentCommand{\lap}{o}{
	\IfNoValueTF{#1}
		{ \mathcal{L} }
		{ \mathcal{L} \left\{ {#1} \right\} }
}
\NewDocumentCommand{\ilap}{o}{
	\IfNoValueTF{#1}
		{ \mathcal{L}^{-1} }
		{ \mathcal{L}^{-1} \left\{ {#1} \right\} }
}

\newcommand{\boldalph}{\textbf{(\alph*)}}

\tableofcontents

\section{Chapter 17 - Temperature and Heat}

To convert Celsius to Fahrenheit:
\begin{equation}
	T_F = \frac{9}{5}T_C + \SI{32}{\degree}
\end{equation}
To convert Fahrenheit to Celsius:
\begin{equation}
	T_C = \frac{5}{9}(T_F - \SI{32}{\degree})
\end{equation}
To convert from Celsius to Kelvin:
\begin{equation}
	T_K = T_C + 273.15
\end{equation}

\subsection{17.3}

\begin{enumerate}[label = \textbf{(\alph*)}]
	\item
		\begin{align*}
			T_{F_0} & = \SI{-4.0}{\degree} \\
			T_{F_1} & = \SI{45.0}{\degree}
		\end{align*}
		\begin{align*}
			\Delta T_F & = T_{F_1} - T_{F_0} \\
			\Delta T_F & = \SI{45.0}{\degree} - \SI{-4.0}{\degree} \\
			\Delta T_F & = \SI{49.0}{\degree}
		\end{align*}
		\begin{align*}
			(100)\Delta T_F & = (180)T_C \\
			T_C & = \SI{27.2}{\celsius}
		\end{align*}
\end{enumerate}

\subsection{17.5}

\begin{align*}
	\Delta T_K & = \SI{10.0}{\kelvin}
\end{align*}
\begin{enumerate}[label = \textbf{(\alph*)}]
	\item
		\begin{align*}
			T_{F_1} & = \frac{9}{5}(\SI{10.0}{\celsius} + \SI{32.0}{\degree}) \\
			T_{F_1} & = \SI{18.0}{\degree}
		\end{align*}
		\begin{align*}
			T_{F_0} & = \frac{9}{5}(0 + \SI{32.0}{\degree}) \\
			T_{F_0} & = \SI{57.6}{\degree}
		\end{align*}
		\begin{align*}
			T_F & = T_{F_1} - T_{F_0} \\
			T_F & = \SI{18.0}{\degree}
		\end{align*}
\end{enumerate}

\section{Linear Thermal Expansion}

\begin{equation}
	\Delta L = \alpha L_0\Delta T
\end{equation}

\subsection{Expanding Holes and Volume Expansion}

\begin{equation}
	\Delta V = \beta V_0\Delta T, \quad \beta = 3\alpha
\end{equation}

\subsection{17.11}

\begin{align*}
	L_0 & = \SI{1410}{\meter} \\
	T_0 & = \SI{-5.0}{\celsius} \\
	T_1 & = \SI{18.0}{\celsius} \\
	\alpha_{steel} & = \SI{1.2e-5}{\per \celsius} \\
	\Delta L & = ?
\end{align*}
\begin{align*}
	\Delta L & = L_0\alpha \Delta T \\
	\Delta L & = (\SI{1410}{\meter})(\SI{1.2e-5}{\per \celsius})(\SI{18.0}{\celsius} - (\SI{-5.0}{\celsius}) \\
	\Delta L & = \SI{0.38916}{\meter}
\end{align*}

\subsection{17.15}

\begin{align*}
	T_0 & = \SI{20.0}{\celsius} \\
	\beta_{copper} & = \SI{5.1e-5}{\per \celsius} \\
	V_1 & = (0.0015)V_0
\end{align*}
\begin{align*}
	\Delta V & = V_0\beta \Delta T \\
	\Delta T & = \frac{ \Delta V }{ V_0\beta } \\
	\Delta T & = \frac{ (0.0015)V_0 }{ V_0\beta } \\
	\Delta T & = \frac{ 0.0015 }{ \SI{5.1e-5}{\per \celsius} } \\
	\Delta T & = \SI{29.4118}{\celsius}
\end{align*}

\subsection{17.16}

\begin{align*}
	d & = \SI{55.0}{\meter} \\
	T_{winter} & = \SI{-15}{\celsius} \\
	T_{summer} & = \SI{35}{\celsius} \\
	\beta_{aluminum} & = \SI{7.2e-5}{\per \celsius} \\
	\Delta V & = ?
\end{align*}
\begin{align*}
	\Delta V & = V_0\beta \Delta T \\
	\Delta V & = \left( \frac{2}{3}\pi \frac{ \SI{55.0}{\meter} }{ 2 } \right) (\SI{7.2e-5}{\per \celsius})(\SI{35}{\celsius} - (\SI{-15}{\celsius})) \\
	\Delta V & = \SI{156.805}{\meter \cubed}
\end{align*}

\subsection{17.19}

\begin{align*}
	d & = \SI{1.35}{\centi \meter} = \SI{0.0135}{\meter} \\
	T_0 & = \SI{25.0}{\celsius} \\
	\alpha_{steel} & = \SI{1.2e-5}{\per \celsius}
\end{align*}
\begin{enumerate}[label = \textbf{(\alph*)}]
	\item
		\begin{align*}
			A_0 & = \pi r^2 \\
			A_0 & = \pi \left( \frac{ \SI{0.0135}{\meter} }{ 2 } \right)^2 \\
			A_0 & = \SI{0.00143}{\meter \squared}
		\end{align*}
	\item
		\begin{align*}
			\Delta A & = 2\alpha A_0\Delta T \\
			\Delta A & = 2(\SI{1.2e-5}{\per \celsius})(\SI{0.00143}{\meter \squared})(\SI{175}{\celsius} - \SI{25.0}{\celsius}) \\
			\Delta A & = \SI{5.148e-6}{\meter \squared}
		\end{align*}
		\begin{align*}
			A & = A_0 + \Delta A \\
			A & = \SI{0.00143}{\meter \squared} + \SI{5.148e-6}{\meter \squared} \\
			A & = \SI{0.001435}{\meter \squared}
		\end{align*}
\end{enumerate}

\section{Thermal Expansion of Water}

\subsection{Thermal Stress}

\begin{equation}
	\frac{ F }{ A } = -Y\alpha \Delta T
\end{equation}

\subsection{17.22}

\begin{align*}
	L_0 & = \SI{185}{\centi \meter} = \SI{1.85}{\meter} \\
	d & = \SI{1.60}{\centi \meter} = \SI{0.016}{\meter} \\
	T_0 & = \SI{120.0}{\celsius} \\
	T_1 & = \SI{10.0}{\celsius} \\
	Y_{brass} & = \SI{9.0e10}{\pascal} \\
	\alpha_{brass} & = \SI{2.0e-5}{\per \celsius} \\
	F & = ?
\end{align*}
\begin{align*}
	\frac{ F }{ A } = -Y\alpha \Delta T \\
	F & = -AY\alpha \Delta T \\
	F & = -\left( \pi \left( \frac{ \SI{0.016}{\meter} }{ 2 } \right)^2 \right)(\SI{9.0e10}{\pascal})(\SI{2.0e-5}{\per \celsius})(\SI{10.0}{\celsius} - \SI{120.0}{\celsius}) \\
	F & = \SI{39810.3}{\newton}
\end{align*}
\begin{align*}
	F_0 = F_1 & = \frac{ F }{ 2 } = \frac{ \SI{39810.3}{\newton} }{ 2 } = \SI{19905.1}{\newton}
\end{align*}

\section{Quantity of Heat}

The quantity of heat $ Q $ required to increase the temperature of a mass $ m $ of a certain  material by $ \Delta T $ is:
\begin{equation}
	Q = mc\Delta T
\end{equation}
\begin{equation}
	\SI{1}{\cal} = \SI{4.186}{\joule}
\end{equation}
Specific heat is found by:
\begin{equation*}
	dQ = mcdT
\end{equation*}
\begin{equation}
	c = \frac{1}{m}\frac{dQ}{dT}
\end{equation}

\subsection{17.29}

\begin{align*}
	w & = \SI{28.4}{\newton} \\
	Q & = \SI{1.25e4}{\joule} \\
	\Delta T & = \SI{18.0}{\celsius} \\
	c & = ?
\end{align*}
\begin{align*}
	Q & = mc\Delta T \\
	c & = \frac{ Q }{ \frac{ w }{ g }\Delta T } \\
	c & = \frac{ \SI{1.25e4}{\joule} }{ \left( \frac{ \SI{28.4}{\newton} }{ \SI{9.80}{\meter \per \second \squared} } \right) (\SI{18.0}{\celsius}) } \\
	c & = \SI{239.632}{\joule \per \kilogram \kelvin}
\end{align*}

\subsection{17.25}

\begin{align*}
	m_{kettle} & = \SI{1.10}{\kilogram} \\
	m_{water} & = \SI{1.80}{\kilogram} \\
	T_0 & = \SI{20.0}{\celsius} \\
	T_1 & = \SI{85.0}{\celsius} \\
	c_{aluminum} & = \SI{910}{\joule \per \kilogram \kelvin} \\
	c_{water} & = \SI{4190}{\joule \per \kilogram \kelvin} \\
	Q & = ?
\end{align*}
\begin{align*}
	Q_{water} & = m_{water}c_{water}\Delta T \\
	Q_{water} & = (\SI{1.80}{\kilogram})(\SI{4190}{\joule \per \kilogram \kelvin})(\SI{85.0}{\celsius} - \SI{20.0}{\celsius}) \\
	Q_{water} & = \SI{490230}{\joule}
\end{align*}
\begin{align*}
	Q_{aluminum} & = m_{aluminum}c_{aluminum}\Delta T \\
	Q_{aluminum} & = (\SI{1.10}{\kilogram})(\SI{910}{\joule \per \kilogram \kelvin})(\SI{85.0}{\celsius} - \SI{20.0}{\celsius}) \\
	Q_{aluminum} & = \SI{65065}{\joule}
\end{align*}
\begin{align*}
	Q & = Q_{water} + Q_{aluminum} \\
	Q & = \SI{490230}{\joule} + \SI{65065}{\joule} \\
	Q & = \SI{555295}{\joule}
\end{align*}

\subsection{17.31}

\begin{align*}
	y_1 & = \SI{225}{\meter} \\
	m & = \SI{1.00}{\liter} \\
	y_0 & = 0 \\
	c_{water} & = \SI{4190}{\joule \per \kilogram \kelvin} \\
	\Delta T & = ?
\end{align*}
\begin{align*}
	U & = Q \\
	mgy_1 & = mc\Delta T \\
	\Delta T & = \frac{ gy_1 }{ c } \\
	\Delta T & = \frac{ (\SI{9.80}{\meter \per \second \squared})(\SI{225}{\meter}) }{ \SI{4190}{\joule \per \kilogram \kelvin} } \\
	\Delta T & = \SI{0.5262}{\celsius}
\end{align*}

\subsection{17.33}

\begin{align*}
	m_{bullet} & = \SI{15.0}{\gram} = \SI{0.015}{\kilogram} \\
	v_0 & = \SI{865}{\meter \per \second} \\
	m_{water} & = \SI{13.5}{\kilogram} \\
	v_1 & = \SI{534}{\meter \per \second} \\
	c_{water} & = \SI{4190}{\joule \per \kilogram \kelvin} \\
	\Delta T & = ?
\end{align*}
\begin{align*}
	E_0 & = E_1 \\
	\frac{1}{2}m_{bullet}v_0^2 & = \frac{1}{2}m_{bullet}v_1^2 + m_{water}c_{water}\Delta T \\
	\Delta T & = \frac{ m_{bullet}(v_0^2 - v_1^2) }{ 2m_{water}c_{water} } \\
	\Delta T & = \frac{ (\SI{0.015}{\kilogram} \left[ (\SI{865}{\meter \per \second})^2 - (\SI{534}{\meter \per \second})^2 \right]) }{ 2(\SI{13.5}{\kilogram})(\SI{4190}{\joule \per \kilogram \kelvin}) } \\
	\Delta T & = \SI{0.61399}{\celsius}
\end{align*}

\section{Molar Heat Capacity}

Total mass $ m $ of material = Mass per mole $ M \times $ Number of moles $ n $:
\begin{equation}
	m = nM
\end{equation}
\begin{equation}
	Q = nMc\Delta T
\end{equation}
The produce $ Mc $ is called the \textbf{molar heat capacity}.
\begin{equation}
	Q = nC\Delta T
\end{equation}

\subsection{Phase Changes}

The \textbf{latent heat}, $ L $, is the heat per unit mass that is transferred in a phase change.
\begin{equation}
	Q = \pm mL
\end{equation}

\subsection{Problem}

\begin{enumerate}[label = \boldalph]
	\item
		\begin{align*}
			m_{water} & = \SI{1}{\kilogram} \\
			T_{water} & = \SI{100}{\celsius} \\
			m_{water\_vapor} & = \SI{1}{\kilogram} \\
			T_{water\_vapor} & = \SI{100}{\celsius}
		\end{align*}
		\begin{align*}
			Q_{water} & = m_{water}L_{water} \\
			Q_{w} & = (\SI{1}{\kilogram})(\SI{2256e3}{\joule \per \kilogram}) \\
			Q_{w} & = \SI{2.256e6}{\joule}
		\end{align*}
\end{enumerate}

\subsection{17.34}

\begin{align*}
	m_{water} & = \SI{750}{\gram} = \SI{0.750}{\kilogram} \\
	T_{water_0} & = \SI{10.0}{\celsius} \\
	T_{water_1} & = \SI{75.0}{\celsius} \\
	T_{boil_0} & = \SI{100.0}{\celsius} \\
	T_{boil_1} & = \SI{75.0}{\celsius}
\end{align*}
\begin{align*}
	m_{water}c_{water}\Delta T_{water} + m_{boil}c_{water}\Delta T_{boil} & = 0 \\
	m_{boil} & = -\frac{ m_{water}\Delta T_{water} }{ \Delta T_{boil} } \\
	m_{boil} & = -\frac{ (\SI{0.750}{\kilogram})(\SI{75.0}{\celsius} - \SI{10.0}{\celsius}) }{ \SI{75.0}{\celsius} - \SI{100.0}{\celsius} } \\
	m_{boil} & = \SI{1.95}{\kilogram}
\end{align*}

\subsection{17.36}

\begin{align*}
	T_1 & = \SI{32.0}{\celsius} \\
	m_{patient} & = \SI{70.0}{\kilogram} \\
	T_{ice} & = \SI{0}{\celsius} \\
	m_{ice} & = ? \\
	c_{human} & = \SI{3480}{\joule \per \kilogram \per \celsius} \\
	T_{human} & = \SI{37.0}{\celsius}
\end{align*}
\begin{align*}
	m_{ice}L_{ice} + m_{human}c_{human}\Delta T & = 0 \\
	m_{ice} & = -\frac{ m_{human}c_{human}\Delta T }{ L_{ice} } \\
	m_{ice} & = -\frac{ (\SI{70.0}{\kilogram})(\SI{3480}{\joule \per \kilogram \per \celsius})(\SI{32.0}{\celsius} - \SI{37.0}{\celsius}) }{ \SI{334e3}{\joule \per \kilogram} } \\
	m_{ice} & = \SI{3.64671}{\kilogram}
\end{align*}

\subsection{17.37}

\begin{align*}
	m_{iron} & = \SI{1.20}{\kilogram} \\
	T_{iron_0} & = \SI{650.0}{\celsius} \\
	T_{water} & = \SI{15.0}{\celsius} \\
	T_{iron_1} & = \SI{120.0}{\celsius} \\
	m_{water} & = ?
\end{align*}
\begin{align*}
	Q_0 + Q_1 + Q_2 & = 0 \\
	m_{iron}c_{iron}\Delta T_{iron} + m_{water}c_{water}\Delta T_{water} + m_{water}L_{water\_vapor} & = 0
\end{align*}
\begin{align*}
	m_{water} & = - \frac{ m_{iron}c_{iron}\Delta T_{iron} }{ c_{water}\Delta T_{water} + L_{water\_vapor} } \\
	m_{water} & = - \frac{ (\SI{1.20}{\kilogram})(\SI{0.47e3}{\joule \per \kilogram \per \kelvin})(\SI{120.0}{\celsius} - \SI{650.0}{\celsius}) }{ (\SI{4190}{\joule \per \kilogram \per \kelvin})(\SI{100.0}{\celsius} - \SI{15.0}{\celsius}) + \SI{2256e3}{\joule \per \kilogram} } \\
	m_{water} & = \SI{0.114}{\kilogram}
\end{align*}

\subsection{17.40}

\begin{align*}
	m_{ice} & = \SI{0.200}{\kilogram} \\
	T_{ice_0} & = \SI{-40.0}{\celsius} \\
	m_{water} & = ? \\
	T_{water_0} & = \SI{80.0}{\celsius} \\
	T_1 & = \SI{28.0}{\celsius}
\end{align*}
\begin{align*}
	Q_0 + Q_1 + Q_2 + Q_3 & = 0 \\
	m_{ice}c_{ice}\Delta T_{ice} + m_{ice}L_{ice} + m_{ice}c_{water}\Delta T_{melted} + m_{water}c_{water}\Delta T_{water} & = 0
\end{align*}
\begin{align*}
	m_{water} & = - \frac{ m_{ice} \left[ c_{ice}\Delta T_{ice} + L_{ice} + c_{water}\Delta T_{melted} \right] }{ c_{water}\Delta T_{water} } \\
	m_{water} & = - \frac{ \SI{0.200}{\kilogram} \left[ (\SI{2100}{\joule \per \kilogram \per \celsius})(0 - \SI{-40.0}{\celsius}) + \SI{334e3}{\joule \per \kilogram} + (\SI{4190}{\joule \per \kilogram \per \kelvin})(\SI{28.0}{\celsius} - 0) \right] }{ (\SI{4190}{\joule \per \kilogram \per \kelvin})(\SI{28.0}{\celsius} - \SI{80.0}{\celsius}) } \\
	m_{water} & = \SI{0.49139}{\kilogram}
\end{align*}

\subsection{17.49}

\begin{align*}
	d_{asteroid} & = \SI{10}{\kilo \meter} = \SI{10000}{\meter} \\
	m_{asteroid} & = \SI{2.60e15}{\kilogram} \\
	v_{asteroid} & = \SI{32.0}{\kilo \meter \per \second} = \SI{32000}{\meter \per \second} \\
	T_{water_0} & = \SI{10.0}{\celsius} \\
	m_{water} & = ?
\end{align*}
\begin{align*}
	KE & = Q_0 + Q_1 \\
	\frac{1}{2}m_{a}v_{a}^2 & = m_{w}c_{w}\Delta T + m_{w}L_{w} \\
	m_{w} & = \frac{ m_{a}v_{a}^2 }{ c_{w}\Delta T + L_{w} } \\
	m_{w} & = \frac{ 0.01(\SI{2.60e15}{\kilogram})(\SI{32000}{\meter \per \second})^2 }{ 2 \left[ (\SI{4190}{\joule \per \kilogram \per \kelvin})(\SI{100.0}{\celsius} - \SI{10.0}{\celsius}) + \SI{2256e3}{\joule \per \kilogram} \right] } \\
	m_{w} & = \SI{5.05564e15}{\kilogram}
\end{align*}

\subsection{17.51}

\begin{align*}
	m_{water} & = \SI{0.250}{\kilogram} \\
	T_{water_0} & = \SI{75.0}{\celsius} \\
	T_{ice} & = \SI{-20.0}{\celsius} \\
	m_{ice} & = ? \\
	T_{1} & = \SI{40.0}{\celsius}
\end{align*}
\begin{align*}
	Q_{ice,water} + Q_{water} + Q_{water,system} & = Q_{water,system} \\
	m_{ice}c_{ice}\Delta T_{ice,water} + m_{ice}L_{water} + m_{ice}c_{water}\Delta T_{water,system} & = m_{water}c_{water}\Delta T_{water,system}
\end{align*}
\begin{align*}
	m_{ice} & = \frac{ m_{water}c_{water}\Delta T_{water,system} }{ c_{ice}\Delta T_{ice,water} + L_{water} + c_{water}\Delta T_{water,system} } \\
	m_{ice} & = \frac{ (\SI{0.250}{\kilogram})(\SI{4190}{\joule \per \kilogram \per \celsius})(\SI{40.0}{\celsius} - \SI{75.0}{\celsius}) }{ (\SI{2100}{\joule \per \kilogram \per \celsius})(0 - \SI{-20.0}{\celsius}) + \SI{334e3}{\joule \per \kilogram} + (\SI{4190}{\joule \per \kilogram \per \kelvin})(\SI{40.0}{\celsius} - 0) } \\
	m_{ice} & = \SI{-0.0674}{\kilogram}
\end{align*}

\section{Mechanisms of Heat Transfer}

Three mechanisms of heat transfer:
\begin{itemize}
	\item \textbf{Conduction} occurs within an object or between two objects in contact. \\
	\item \textbf{Convection} depends on motion of mass from one region of space to another.
	\item \textbf{Radiation} is heat transfer by electromagnetic radiation, such as sunshine, with no need for matter to be present in the space between objects.
\end{itemize}

\begin{equation}
	H = \frac{ dQ }{ dt} = kA\frac{ T_H - T_C }{ L }
\end{equation}

\subsection{17.56}

\begin{align*}
	T_1 & = \SI{100.0}{\celsius} \\
	T_0 & = \SI{0.00}{\celsius} \\
	L & = \SI{60.0}{\centi \meter} = \SI{0.60}{\meter} \\
	A & = \SI{1.25}{\centi \meter \squared} = \SI{0.000125}{\meter \squared} \\
	m_{ice} & = \SI{8.50}{\gram} = \SI{0.00850}{\kilogram} \\
	\Delta t & = \SI{10.0}{\minute} = \SI{600}{\second} \\
	k & = ?
\end{align*}
\begin{align*}
	\frac{ Q }{ \Delta t } & = \frac{ kA(T_1 - T_0) }{ L } \\
	k & = \frac{ mLL }{ \Delta t A(T_1 - T_0) } \\
	k & = \frac{ (\SI{0.00850}{\kilogram})(\SI{334e3}{\joule \per \kilogram})(\SI{0.60}{\meter}) }{ (\SI{600}{\second})(\SI{0.000125}{\meter \squared})(\SI{100.0}{\celsius} - \SI{0.00}{\celsius}) } \\
	k & = \SI{227.12}{\joule \per \meter \per \second \per \celsius} = \SI{227.12}{\watt \per \meter \per \celsius}
\end{align*}

\newpage
\subsection{17.57}

\begin{align*}
	l_{out} & = \SI{3.0}{\centi \meter} = \SI{0.03}{\meter} \\
	l_{in} & = \SI{2.2}{\centi \meter} = \SI{0.022}{\meter} \\
	k_{wood} & = \SI{0.080}{\watt \per \meter \per \kelvin} \\
	k_{styrofoam} & = \SI{0.027}{\watt \per \meter \per \kelvin} \\
	T_{in} & = \SI{19.0}{\celsius} \\
	T_{out} & = \SI{-10.0}{\celsius}
\end{align*}
\begin{enumerate}[label = \boldalph]
	\item
		\begin{align*}
			H_{w} & = k_{w}A \left( \frac{ T - T_{out} }{ l_{out} } \right) \\
			H_{s} & = k_{s}A \left( \frac{ T_{in} - T }{ l_{in} } \right) \\
			k_{w}A \left( \frac{ T - T_{out} }{ l_{out} } \right) & = k_{s}A \left( \frac{ T_{in} - T }{ l_{in} } \right) \\
			k_{w} \left( \frac{ T - T_{out} }{ l_{out} } \right) - k_{s} \left( \frac{ T_{in} - T }{ l_{in} } \right) & = 0 \\
			T \left[ \frac{ k_{w} }{ l_{out} } + \frac{ k_{s} }{ l_{in} } \right] & = \frac{ k_{w}T_{out} }{ l_{out} } + \frac{ k_{s}T_{in} }{ l_{in} } \\
			T & = \frac{ \frac{ k_{w}T_{out} }{ l_{out} } + \frac{ k_{s}T_{in} }{ l_{in} } }{ \frac{ k_{w} }{ l_{out} } + \frac{ k_{s} }{ l_{in} } } \\
			T & = \frac{ \frac{ (\SI{0.080}{\watt \per \meter \per \kelvin})(\SI{-10.0}{\celsius}) }{ \SI{0.03}{\meter} } + \frac{ (\SI{0.027}{\watt \per \meter \per \kelvin})(\SI{19.0}{\celsius}) }{ \SI{0.022}{\meter} } }{ \frac{ \SI{0.080}{\watt \per \meter \per \kelvin} }{ \SI{0.03}{\meter} } + \frac{ \SI{0.027}{\watt \per \meter \per \kelvin} }{ \SI{0.022}{\meter} } } \\
			T & = \SI{-0.860}{\celsius}
		\end{align*}
	\item
		\begin{align*}
			\frac{ H }{ A } & = k_{s} \left( \frac{ T_{in} - T }{ l_{in} } \right) \\
			\frac{ H }{ A } & = (\SI{0.027}{\watt \per \meter \per \kelvin}) \left( \frac{ \SI{19.0}{\celsius} - (\SI{-0.860}{\celsius}) }{ \SI{0.022}{\meter} } \right) \\
			\frac{ H }{ A } & = \SI{24.4}{\watt \per \meter \squared}
		\end{align*}
\end{enumerate}

\subsection{17.60}

\begin{align*}
	l_{copper} & = \SI{1.00}{\meter} \\
	l_{steel} & = ? \\
	A & = \SI{4.00}{\centi \meter \squared} = \SI{4.0e-4}{\meter \squared} \\
	T_{steel,copper} & = \SI{65.0}{\celsius}
\end{align*}
\begin{enumerate}[label = \boldalph]
	\item
		\begin{align*}
			\frac{ Q }{ t } & = \frac{ k_{c}A(T_{boil} - T_{s,c}) }{ l_{c} } \\
			\frac{ Q }{ t } & = \frac{ (\SI{385}{\watt \per \meter \per \kelvin})(\SI{4.0e-4}{\meter \squared})(\SI{100.0}{\celsius} - \SI{65.0}{\celsius}) }{ \SI{1.00}{\meter} } \\
			\frac{ Q }{ t } & = \SI{5.39}{\joule \per \second}
		\end{align*}
		\begin{align*}
			l_{steel} & = \frac{ k_{s}A(T_{s,c} - T_{ice}) }{ \left( \frac{ Q }{ t } \right)^{-1} } \\
			l_{steel} & = \frac{ (\SI{50.2}{\watt \per \meter \per \kelvin})(\SI{4.0e-4}{\meter \squared})(\SI{65.0}{\celsius} - \SI{0.00}{\celsius}) }{ \left( \SI{5.39}{\joule \per \second} \right)^{-1} } \\
			l_{steel} & = \SI{0.242}{\meter}
		\end{align*}
\end{enumerate}

\section{Convection of Heat}

Stefan-Boltzmann law:
\begin{equation}
	H = Ae\sigma T^4
\end{equation}
Stefan-Boltzmann constant:
\begin{equation}
	\sigma = \SI{5.67037442e-8}{\watt \per \meter \squared \per \kelvin \tothe{4}}
\end{equation}

\subsection{17.64}

\begin{enumerate}[label = \boldalph]
	\item
		\begin{align*}
			\frac{ H }{ A } & = e\sigma T^4 \\
			\frac{ H }{ A } & = (1)(\SI{5.67e-8}{\watt \per \meter \squared \per \kelvin \tothe{4}})(\SI{273}{\kelvin})^4 \\
			\frac{ H }{ A } & = \SI{314.944}{\watt \per \meter \squared}
		\end{align*}
	\item
		\begin{align*}
			\frac{ H }{ A } & = e\sigma T^4 \\
			\frac{ H }{ A } & = (1)(\SI{5.67e-8}{\watt \per \meter \squared \per \kelvin \tothe{4}})(\SI{2730}{\kelvin})^4 \\
			\frac{ H }{ A } & = \SI{3.15e6}{\watt \per \meter \squared}
		\end{align*}
\end{enumerate}

\subsection{17.65}

\begin{align*}
	T_{light\_bulb} & = \SI{2450}{\kelvin} \\
	e & = 0.350 \\
	A & = ? \\
	H & = \SI{150}{\watt}
\end{align*}
\begin{align*}
	H & = Ae\sigma T^4 \\
	A & = \frac{ H }{ e\sigma T^4 } \\
	A & = \frac{ \SI{150}{\watt} }{ (0.350)(\SI{5.6e-8}{\watt \per \meter \squared \per \kelvin \tothe{4}})(\SI{2450}{\kelvin})^4 } \\
	A & = \SI{1.96}{\meter \squared}
\end{align*}

\subsection{17.67}

\begin{align*}
	e & = 1 \\
	r & = ?
\end{align*}
\begin{enumerate}[label = \boldalph]
	\item
		\begin{align*}
			H & = \SI{2.7e32}{\watt} \\
			T & = \SI{11000}{\kelvin}
		\end{align*}
		\begin{align*}
			H & = Ae\sigma T^4 \\
			H & = 4\pi r^2e\sigma T^4 \\
			r & = \sqrt{ \frac{ H }{ 4\pi e\sigma T^4 } } \\
			r & = \sqrt{ \frac{ \SI{2.7e32}{\watt} }{ 4\pi (1)(\SI{5.6e-8}{\watt \per \meter \squared \per \kelvin \tothe{4}})(\SI{11000}{\kelvin})^4 } } \\
			r & = \SI{1.62e11}{\meter}
		\end{align*}
	\item
		\begin{align*}
			H & = \SI{2.1e23}{\watt} \\
			T & = \SI{10000}{\kelvin}
		\end{align*}
		\begin{align*}
			r & = \sqrt{ \frac{ H }{ 4\pi e\sigma T^4 } } \\
			r & = \sqrt{ \frac{ \SI{2.1e23}{\watt} }{ 4\pi (1)(\SI{5.6e-8}{\watt \per \meter \squared \per \kelvin \tothe{4}})(\SI{10000}{\kelvin}) } } \\
			r & = \SI{5.46e12}{\meter}
		\end{align*}
\end{enumerate}

\end{document}
