\documentclass{article}

% Document extensibility %
%
% Disables native paragraph indentation
\usepackage{parskip} 
%
% Provides further bullet options for lists
\usepackage{enumitem}

% Mathematical symbol and statement packages %
%
% Necessary
\usepackage{amsmath}
\usepackage{amssymb}
%
% Extensive fraction notation
\usepackage{xfrac}
%
% Generic mathematical commands
% Notable: \degree, \celcius
\usepackage{gensymb}
%
% Variable vector notation (arrow above variable)
\usepackage{esvect}
%
% Multiline boxed equations
\usepackage{empheq}
%
% SI Unit
\usepackage{siunitx}
\DeclareSIUnit\atmosphere{atm}
\usepackage{physunits}
%
% More intuitive arrays/matrices
\usepackage{array}
%
% Linear Equations
\usepackage{systeme}
%
% Boxes!
\usepackage{mdframed}
%
% Matrix Notation
\usepackage{bm}

% Graphic packages %
%
% Diagrams and illustrations
\usepackage{tikz}
\usetikzlibrary{positioning}
%
% Image insertion
\usepackage{graphicx}

% LaTeX Commands
%
% Argument Parser
\usepackage{xparse}

% Document content %
%
% Change title of table of contents
% \renewcommand{\contentsname}{Title}

\begin{document}

% Command `\hr` to insert horizontal rules
\newcommand{\hr}{\par\noindent\rule{\textwidth}{0.4pt}}

% Command to box and center math equations
\newcommand{\bc}[1]{
	\begin{equation*}
		\begin{boxed}
			{#1}
		\end{boxed}
	\end{equation*}
}

% Command for single line equations with a condition
\newcommand{\cond}[2]{
	\ifmmode
		{#1} \quad {#2}
	\else
		$$ {#1} \quad {#2} $$
	\fi
}

% Matrix and Vector notation
\newcommand{\matr}[1]{
	\ifmmode \bm{#1}
	\else \textit{\textbf{#1}}
	\fi
}
\newcommand{\vect}[1]{
	\ifmmode \mathbf{#1}
	\else \textbf{#1}
	\fi
}

\newcommand{\boldalph}{\textbf{(\alph*)}}

% Laplace
\NewDocumentCommand{\lap}{o}{
	\IfNoValueTF{#1}
		{ \mathcal{L} }
		{ \mathcal{L} \left\{ {#1} \right\} }
}
\NewDocumentCommand{\ilap}{o}{
	\IfNoValueTF{#1}
		{ \mathcal{L}^{-1} }
		{ \mathcal{L}^{-1} \left\{ {#1} \right\} }
}

\tableofcontents

\section{Chapter 18 - Thermal Properties of Matter}

Avogadro's number
\begin{equation}
	N_A = \SI{6.02e23}{\mole}
\end{equation}

\subsection{The Ideal Gas Law}

\textbf{Ideal gas}: a collection of atoms or molecules that move randomly and exert no long-range forces on each other.

Number of moles
\begin{equation}
	n = \frac{ N }{ N_A } = \frac{ m_{particle}N }{ m_{particle}N_A } = \frac{ m }{ M }
\end{equation}
The \textbf{molar mass $ M $ (molecular weight)} is the mass per mole. The total mass of $ n $ moles is $ m_{total} = nM $.

Ideal-gas equation
\begin{equation}
	pV = nRT
\end{equation}

Universal gas constant
\begin{equation}
	R = \SI{8.31}{\joule \per \mole \per \kelvin} = \SI{0.0821}{\liter \atmosphere \per \mole \per \kelvin}
\end{equation}
The volume occupied by \SI{1}{\mole} of any ideal gas at atmospheric pressure and at \SI{0}{\celsius} is \SI{22.4}{\liter}.

\subsubsection{Question}

\begin{align*}
	V & = \SI{22.4e-3}{\liter} \\
	T & = \SI{273.15}{\kelvin} \\
	p & = \SI{1.013e5}{\pascal} = \SI{1.0}{\atmosphere} \\
	n & = ?
\end{align*}
\begin{align*}
	pV & = nRT \\
	n & = \frac{ pV }{ RT } \\
	n & = \frac{ (\SI{1.0}{\atm})(\SI{22.4}{\liter}) }{ (\SI{0.0821}{\liter \atmosphere \per \mole \per \kelvin})(\SI{273.15}{\kelvin}) } \\
	n & = \SI{1.000}{\mole}
\end{align*}

\subsubsection{18.3}

\begin{align*}
	V_0 & = \SI{0.110}{\meter \cubed} \\
	p_0 & = \SI{0.355}{\atmosphere} \\
	V_1 & = \SI{0.390}{\meter \cubed} \\
	T & = \text{constant} \\
	p_1 & = ?
\end{align*}
\begin{align*}
	p_0V_0 & = p_1V_1 \\
	p_1 & = \frac{ p_0V_0 }{ V_1 } \\
	p_1 & = \frac{ (\SI{0.355}{\atmosphere})(\SI{0.110}{\meter \cubed}) }{ \SI{0.390}{\meter \cubed} } \\
	p_1 & = \SI{0.1001}{\atmosphere}
\end{align*}

\subsubsection{18.4}

\begin{align*}
	V_0 & = \SI{3.00}{\liter} \\
	p_0 & = \SI{3.00}{\atmosphere} \\
	T_0 & = \SI{20.0}{\celsius} = \SI{293}{\kelvin} \\
	p_1 & = \SI{1.00}{\atmosphere}
\end{align*}
\begin{enumerate}[label = \boldalph]
	\item
		\begin{align*}
			pV & = nRT \\
			\frac{ p }{ T } & = \frac{ nR }{ V } \\
			\frac{ p_0 }{ T_0 } & = \frac{ p_1 }{ T_1 } \\
			T_1 & = \frac{ p_1T_0 }{ p_0 } \\
			T_1 & = \frac{ (\SI{1.00}{\atmosphere})(\SI{293}{\kelvin}) }{ \SI{3.00}{\atmosphere} } \\
			T_1 & = \SI{97.7}{\kelvin} = \SI{-175.3}{\celsius}
		\end{align*}
\end{enumerate}

\subsubsection{18.7}

\begin{align*}
	V_0 & = \SI{499}{\centi \meter \cubed} = \SI{499e-6}{\meter \cubed} \\
	p_0 & = \SI{1.01e5}{\pascal} \\
	T_0 & = \SI{27.0}{\celsius} = \SI{300}{\kelvin} \\
	V_1 & = \SI{46.2}{\centi \meter \cubed} = \SI{46.2e-6}{\meter \cubed} \\
	p_1 & = \SI{2.72e6}{\pascal} + \SI{1}{\atmosphere} = \SI{2.821e6}{\pascal} \\
	T_1 & = ?
\end{align*}
\begin{align*}
	pV & = nR \Delta T \\
	\frac{ p_0V_0 }{ T_0 } & = \frac{ p_1V_1 }{ T_1 } \\
	T_1 & = \frac{ T_0p_1V_1 }{ p_0V_0 } \\
	T_1 & = \frac{ (\SI{300}{\kelvin})(\SI{2.821e6}{\pascal})(\SI{46.2e-6}{\meter \cubed}) }{ (\SI{1.01e5}{\pascal})(\SI{499e-6}{\meter \cubed}) } \\
	T_1 & = \SI{755.79}{\kelvin}
\end{align*}

\subsubsection{18.13}

\begin{align*}
	p_0 & = \SI{1}{\atmosphere}
	V_0 & = V_{earth} \\
	V_1 & = V_{venus} \\
	T_1 & = \SI{1003}{\celsius} = \SI{1276}{\kelvin} \\
	p_1 & = \SI{92}{\atmosphere} \\
	T_0 & = \SI{273}{\kelvin}
\end{align*}
\begin{align*}
	pV & = nR\Delta T \\
	\frac{ p_0V_0 }{ T_0 } & = \frac{ p_1V_1 }{ T_1 } \\
	V_1 & = \frac{ T_1p_0 }{ T_0p_1 } V_0 \\
	V_1 & = \frac{ (\SI{1276}{\kelvin})(\SI{1}{\atmosphere}) }{ (\SI{273}{\kelvin})(\SI{92}{\atmosphere} } \\
	V_1 & = (0.051)V_0
\end{align*}

\subsubsection{18.16}

\begin{align*}
	n & = \SI{3}{\mole} \\
	l & = \SI{0.300}{\meter}
\end{align*}
\begin{enumerate}[label = \boldalph]
	\item
		\begin{align*}
			T & = \SI{20.0}{\celsius} = \SI{293}{\kelvin}
		\end{align*}
		\begin{align*}
			F & = pA \\
			F & = \frac{ nRTA }{ V } \\
			F & = \frac{ (\SI{3}{\mole})(\SI{8.31}{\joule \per \mole \per \kelvin})(\SI{293}{\kelvin})(\SI{0.300}{\meter})^2 }{ (\SI{0.300}{\meter})^3 } \\
			F & = \SI{24348.3}{\newton} = \SI{2.43e4}{\newton}
		\end{align*}
	\item
		\begin{align*}
			T & = \SI{100.0}{\celsius} = \SI{373}{\kelvin}
		\end{align*}
		\begin{align*}
			F & = \frac{ nRTA }{ V } \\
			F & = \frac{ (\SI{3}{\mole})(\SI{8.31}{\joule \per \mole \per \kelvin})(\SI{373}{\kelvin})(\SI{0.300}{\meter})^2 }{ (\SI{0.300}{\meter})^3 } \\
			F & = \SI{30996.3}{\newton} = \SI{3.10e4}{\newton}
		\end{align*}
\end{enumerate}

\subsubsection{18.18}

\begin{align*}
	\Delta y & = \SI{11000}{\meter} \\
	T & = \SI{-56.5}{\celsius} = \SI{216.5}{\kelvin} \\
	\rho & = \SI{0.364}{\kilogram \per \meter \cubed} \\
	p & = ?
\end{align*}
\begin{align*}
	\rho & = \frac{ m }{ V } \\
	m & = \rho V \\
	n & = \frac{ m }{ M } \\
	n & = \frac{ \rho V }{ M }
\end{align*}
\begin{align*}
	pV & = nRT \\
	pV & = \left( \frac{ \rho V }{ M } \right) RT \\
	p & = \frac{ \rho RT }{ M } \\
	p & = \frac{ (\SI{0.364}{\kilogram \per \meter \cubed})(\SI{8.314}{\joule \per \mole \per \kelvin})(\SI{216.5}{\kelvin}) }{ \SI{28.8e-3}{\kilogram \per \mole} } \\
	p & = \SI{22749.8}{\pascal} = \SI{2.27e4}{\pascal}
\end{align*}

\subsubsection{Question}

\begin{align*}
	T & = \SI{0.00}{\celsius} = \SI{273}{\kelvin} \\
	g & = \SI{9.80}{\meter \per \second \squared}
\end{align*}
\begin{align*}
	\frac{ dp }{ dy } & = -\rho g \\
	p & = \rho RT \\
	\rho & = \frac{ p }{ RT } \\
	\frac{ dp }{ dy } & = - \left( \frac{ p }{ RT } \right)g \\
	\frac{ dp }{ dy } & = -\frac{ pg }{ RT } \\
	p' & = -p \cdot \frac{ g }{ RT } \\
	\lap[p'] + \lap[p] & = \frac{ g }{ RT } \\
	sF(s) - f(0) + F(s) & = \frac{ g }{ RT } \\
	F(s)(s - 1) & = \frac{ g }{ RT } \\
	F(s) & = \frac{ g }{ RT } \cdot \frac{ 1 }{ s - 1 } \\
	s & = \frac{ ge^t }{ RT }
\end{align*}

\section{Molecules and Intermolecular Forces}

\subsection{The Van Der Waals Equation}

The model used for the ideal-gas equation ignores the volumes of molecules and the
attractive forces between them.
\begin{equation}
	\left( p + \frac{ an^2 }{ V^2 } \right) (V - nb) = nRT
\end{equation}

\subsection{Kinetic-Molecular Model of an Ideal Gas}

\begin{enumerate}
	\item Molecules are in constant motion and undergo perfectly elastic collisions.
\end{enumerate}

\subsection{Collisions and Gas Pressure}

\begin{align*}
	\Delta \vect{P_y} & = m\Delta \vect{v_y} = 0 \\
	\Delta \vect{P_x} & = m\Delta \vect{v_x} = 2m\vect{v_x}
\end{align*}
The amount of molecules per volume that collide with a given wall area $ A $ in a time interval $ dt $:
\begin{align*}
	V & = Ah \\
	\Delta \vect{x} & = v_{0_y}t + \frac{1}{2}a_y\Delta t^2 \\
	\Delta \vect{x} & = v_{0_y}t \\
	V & = A|v_{x}|dt
\end{align*}
\begin{align*}
	\frac{1}{2} \left( \frac{ N }{ V } \right) (A|v_{x}|dt)
\end{align*}
For all molecules in the gas, the total momentum change $ dP_{x} $ during $ dt $ is the number of collisions multiplied by the momentum change.
\begin{align*}
	dP_{x} & = \frac{1}{2} \left( \frac{ N }{ V } \right) (A|v_{x}|dt) (2m|v_{x}|) \\
		   & = \frac{ NAmv_{x}^2dt }{ V }
\end{align*}
The rate of change of momentum component $ P_{x} $:
\begin{align*}
	\frac{ dP_{x} }{ dt } & = \frac{ NAmv_{x}^2 }{ V }
\end{align*}
Pressure:
\begin{equation}
	p = \frac{ F }{ A } = \frac{ m\frac{ \Delta \vect{p} }{ \Delta t } }{ A } = \frac{ \frac{ NAmv_{x}^2 }{ V } }{ A }  = \frac{ Nmv_{x}^2 }{ V }
\end{equation}

\subsubsection{Pressure and Molecular Kinetic Energies}

The speed $ v $ of a molecule is related to the velocity components by
\begin{align}
	v^2 & = v_x^2 + v_y^2 + v_z^2 \\
	(v^2)_{av} & = (v_x^2)_{av} + (v_y^2)_{av} + (v_z^2)_{av}
\end{align}
As there is no real difference in our model between directions:
\begin{equation}
	(v_x^2)_{av} = \frac{1}{3}(v^2)_{av}
\end{equation}
\begin{align*}
	p & = \frac{ Nmv_{x}^2 }{ V } \\
	pV & = nRT \\
	\left[ \frac{ Nmv_{x}^2 }{ V } \right]V & = nRT \\
	\frac{ N }{ 3 } \left( 2K \right) & = nRT, \quad KE = \frac{1}{2}mv^2, n = \frac{ m }{ M } = \frac{ N }{ N_A } \\
	K & = \frac{3}{2} nRT
\end{align*}
\begin{align*}
	pV & = nRT \\
	nV & = \frac{ N }{ N_A } RT \\
	pV & = Nk_BT
\end{align*}

\subsubsection{18.24}

\begin{align*}
	v_{rmt} & = 2\sqrt{ \frac{ 3RT }{ m } } \\
	v_{rmt} & = \sqrt{ \frac{ 3R }{ m } \cdot 4T }
\end{align*}
\begin{align*}
	p_1V & = nRT_1 \\
	p_1 & = 2p_2 \\
	2p_2V & = n_2R(4T_1) \\
	n_2 & = \frac{ 2 }{ 4 } \cdot \frac{ p_1V }{ RT_1 } \\
	n_2 & = \frac{1}{2}n_1
\end{align*}

\subsubsection{18.28}

\begin{align*}
	V & = \SI{1.64}{\liter} \\
	m & = \SI{0.226}{\kilogram} \\
	v_{rms} & = \SI{182}{\meter \per \second} \\
	p & = ?
\end{align*}
\begin{align*}
	v_{rms} & = \sqrt{ \frac{ 3RT }{ M } } \\
	v_{rms}^2 & = \frac{ 3RT }{ M } \\
	RT & = \frac{ v_{rms}^2M }{ 3 }
\end{align*}
\begin{align*}
	pV & = nRT \\
	pV & = n \left( \frac{ v_{rms}^2M }{ 3 } \right) \\
	p & = \frac{ \frac{ m }{ M } \left( \frac{ v_{rms}^2M }{ 3 } \right) }{ V } \\
	p & = \frac{ mv_{rms}^2 }{ 3V } \\
	p & = \frac{ (\SI{0.226}{\kilogram})(\SI{182}{\meter \per \second})^2 }{ 3(\SI{1.64}{\liter}) } \\
	p & = \SI{1521.55}{\pascal}
\end{align*}

\subsubsection{18.30}

\begin{align*}
	M_{mars} & = \SI{44.0}{\gram \per \mole} = \SI{0.044}{\kilogram \per \mole} \\
	P_{mars} & = \SI{650}{\pascal} \\
	T_0 & = \SI{0.0}{\celsius} = \SI{273}{\kelvin} \\
	T_1 & = \SI{-100.0}{\celsius} = \SI{173}{\kelvin}
\end{align*}
\begin{enumerate}[label = \textbf{(\alph*)}]
	\item
		\begin{align*}
			v_{rms} & = \sqrt{ \frac{ 3RT }{ M } } \\
			v_{rms} & = \sqrt{ \frac{ 3(\SI{8.314}{\joule \per \mole \per \kelvin})(\SI{273}{\kelvin}) }{ \SI{0.044}{\kilogram \per \mole} } } \\
			v_{rms} & = \SI{393.4}{\meter \per \second}
		\end{align*}
		\begin{align*}
			v_{rms} & = \sqrt{ \frac{ 3RT }{ M } } \\
			v_{rms} & = \sqrt{ \frac{ 3(\SI{8.314}{\joule \per \mole \per \kelvin})(\SI{173}{\kelvin}) }{ \SI{0.044}{\kilogram \per \mole} } } \\
			v_{rms} & = \SI{313.2}{\meter \per \second}
		\end{align*}
	\item
		\begin{align*}
			pV & = nRT \\
			pV & = \left( \frac{ m }{ M } \right) RT \\
			m & = \frac{ pVM }{ RT } \\
			\rho & = \frac{ m }{ V } \\
			\rho & = \frac{ \frac{ pVM }{ RT } }{ V } \\
			\rho & = \frac{ pM }{ RT }
		\end{align*}
		\begin{align*}
			\rho_0 & = \frac{ (\SI{650}{\pascal})(\SI{0.044}{\kilogram \per \mole}) }{ (\SI{8.314}{\joule \per \mole \per \kelvin})(\SI{273}{\kelvin}) } \\
			\rho_0 & = \SI{0.0126}{\kilogram \per \meter \cubed} = \SI{12.6}{\gram \per \meter \cubed} \cdot \frac{ 1 }{ \SI{44.0}{\gram \per \mole} } = \SI{0.286}{\mole \per \meter \cubed}
		\end{align*}
		\begin{align*}
			\rho_1 & = \frac{ (\SI{650}{\pascal})(\SI{0.044}{\kilogram \per \mole}) }{ (\SI{8.314}{\joule \per \mole \per \kelvin})(\SI{173}{\kelvin}) } \\
			\rho_1 & = \SI{0.0199}{\kilogram \per \meter \cubed} = \SI{19.9}{\gram \per \meter \cubed} \cdot \frac{ 1 }{ \SI{44.0}{\gram \per \mole} } = \SI{0.452}{\mole \per \meter \cubed}
		\end{align*}
\end{enumerate}

\section{Collisions Between Molecules}

The volume of the cylinder is
\begin{equation}
	V = \pi 4r^2 v dt
\end{equation}
There are $ \frac{ N }{ V } $ molecules per unit volume, so the number $ dN $ with centers in this cylinder is
\begin{align*}
	dN & = 4\pi r^2 v dt \frac{ N }{ V } \\
	\frac{ dN }{ dt } & = \frac{ 4\pi r^2 v N }{ V } \\
	\frac{ dN }{ dt } & = \frac{ 4\pi \sqrt{ 2 }r^2 v N }{ V }
\end{align*}
\begin{equation}
	\lambda = \frac{ V }{ 4\pi \sqrt{ 2 } r^2 N }
\end{equation}

\subsubsection{18.32}

\begin{align*}
	\lambda & = ? \\
	p & = \SI{3.50e-13}{\atmosphere} \\
	T & = \SI{300}{\kelvin}
\end{align*}
\begin{align*}
	\lambda & = \frac{ V }{ 4\pi \sqrt{2}r^2 N } \\
	\frac{ V }{ N } & = \frac{ kT }{ p } \\
	\lambda & = \frac{ kT }{ p } \cdot \frac{ 1 }{ 4\pi \sqrt{2}r^2 } \\
	\lambda & = \frac{ (\SI{1.38e-23}{\joule \per \kelvin})(\SI{300}{\kelvin}) }{ (\SI{3.50e-13}{\atmosphere})(\SI{1.01e5}{\pascal \per \atmosphere}) } \cdot \frac{ 1 }{ 4\pi \sqrt{2}(\SI{2.0e-10}{\meter})^2 }
	\lambda & = \SI{1.65e5}{\meter}
\end{align*}

\subsubsection{18.35}

\begin{align*}
	n & = \SI{3}{\mole} \\
	M & = \SI{4.00}{\gram \per \mole} \\
	V & = \text{constant} \\
	v_{rms_0} & = \SI{900}{\meter \per \second} \\
	v_{rms_1} & = ? \\
	Q & = \SI{2400}{\joule}
\end{align*}
\begin{align*}
	v_{rms} & = \sqrt{ \frac{ 3RT }{ M } } \\
	T & = \frac{ v_{rms}^2M }{ 3R } \\
	T & = \frac{ (\SI{900}{\meter \per \second})^2(\SI{0.004}{\kilogram \per \mole}) }{ 3(\SI{8.314}{\joule \per \mole \per \kelvin}) } \\
	T_0 & = \SI{129.9}{\kelvin}
\end{align*}
\begin{align*}
	Q & = nC_{V}\Delta T \\
	Q & = nC_{V}(T_1 - T_0) \\
	Q & = nC_{V}T_1 - nC_{V}T_0 \\
	T_1 & = \frac{ Q + nC_{V}T_1 }{ nC_{v} } \\
	T_1 & = \frac{ \SI{2400}{\joule} + (\SI{3}{\mole}) \left( \frac{3}{2}(\SI{8.314}{\joule \per \mole \per \kelvin}) \right) (\SI{129.9}{\kelvin}) }{ (\SI{3}{\mole}) \left( \frac{3}{2}(\SI{8.314}{\joule \per \mole \per \kelvin}) \right) } \\
	T_1 & = \SI{194.0}{\kelvin}
\end{align*}
\begin{align*}
	v_{rms} & = \sqrt{ \frac{ 3RT_0 }{ M } } \\
	v_{rms} & = \sqrt{ \frac{ 3(\SI{8.314}{\joule \per \mole \per \kelvin})(\SI{194.0}{\kelvin}) }{ \SI{0.004}{\kilogram \per \mole} } } \\
	v_{rms} & = \SI{1100}{\meter \per \second}
\end{align*}

\subsubsection{18.38}

\begin{align*}
	n & = n_0 = n_1 \\
	Q & = \SI{300}{\joule} \\
	\Delta T_{hydrogen} & = \SI{2.50}{\celsius} \\
	\Delta T_{neon} & = ?
\end{align*}
\begin{align*}
	Q & = nC_{V}\Delta T \\
	C_{V}\Delta T & = \frac{ Q }{ n } \\
	C_{V_0}\Delta T_0 & = C_{V_1}\Delta T_1 \\
	\Delta T_1 & = \frac{ C_{V_0}\Delta T_0 }{ C_{V_1} } \\
	\Delta T_1 & = \frac{ \frac{5}{2}R(\SI{2.5}{\kelvin}) }{ \frac{3}{2}R } \\
	\Delta T_1 & = \SI{4.16}{\celsius}
\end{align*}

\subsubsection{18.40}

\begin{enumerate}[label = \textbf{(\alph*)}]
	\item
		\begin{align*}
			M_{water} & = \SI{18.0}{\gram \per \mole}
		\end{align*}
		\begin{align*}
			C_{V} & = Mc \\
			c & = \frac{ C_{V} }{ M } \\
			c & = \frac{ 6 \left( \frac{1}{2} (\SI{8.314}{\joule \per \mole \per \kelvin}) \right) }{ \SI{18.0}{\kilogram \per \mole} } \\
			c & = \SI{1386.0}{\joule \per \kilogram \per \kelvin}
		\end{align*}
\end{enumerate}

\section{Molecular Speeds}

The function $ f(v) $ describing the actual distribution of molecular speeds is called the \textbf{Maxwell-Boltzmann distribution}.
\begin{equation}
	f(v) = 4\pi \left( \frac{ m }{ 2\pi kT } \right)^{\frac{ 3 }{ 2 }} v^2e^{ \frac{ -mv^2 }{ 2kT } }
\end{equation}
We can also express this function in terms of the translational kinetic energy of a molecule, which we denote by $ \epsilon $; that is $ \epsilon = \frac{1}{2}mv^2 $
\begin{equation}
	f(\epsilon) = \frac{ 8\pi }{ m } \left( \frac{ m }{ 2\pi kT } \right)^{\frac{ 3 }{ 2 }} \epsilon e^{ \frac{ -\epsilon }{ kT } }
\end{equation}

\begin{align*}
	v_{av} & = \int_0^\infty vf(v)dv
\end{align*}

\subsubsection{18.41}

\begin{align*}
	M_{CO_2} & = \SI{44.0}{\gram \per \mole} = \SI{0.044}{\kilogram \per \mole} \\
	T & = \SI{300}{\kelvin}
\end{align*}
\begin{align*}
	pV & = nRT = NkT \\
	nR & = Nk \\
	\frac{ N }{ N_A } R & = \frac{ m }{ M } k \\
\end{align*}
\begin{enumerate}[label = \textbf{(\alph*)}]
	\item
		\begin{align*}
			v_{mp} & = \sqrt{ \frac{ 2kT }{ m } } \\
			v_{mp} & = \sqrt{ \frac{ 2RT}{ M } } \\
			v_{mp} & = \sqrt{ \frac{ 2(\SI{8.314}{\joule \per \mole \per \kelvin})(\SI{300}{\kelvin}) }{ \SI{0.044}{\kilogram \per \mole} } } \\
			v_{mp} & = \SI{336.7}{\meter \per \second}
		\end{align*}
	\item
		\begin{align*}
			v_{av} & = \sqrt{ \frac{ 8kT }{ \pi m } } \\
			v_{av} & = \sqrt{ \frac{ 8RT }{ \pi M } }
		\end{align*}
	\item
		\begin{align*}
			v_{rms} & = \sqrt{ \frac{ 3kT }{ m } } \\
			v_{rms} & = \sqrt{ \frac{ 3RT }{ M } }
		\end{align*}
\end{enumerate}

\end{document}
