\documentclass{article}

% Document extensibility %
%
% Disables native paragraph indentation
\usepackage{parskip} 
%
% Provides further bullet options for lists
\usepackage{enumitem}

% Mathematical symbol and statement packages %
%
% Necessary
\usepackage{amsmath}
\usepackage{amssymb}
%
% Extensive fraction notation
\usepackage{xfrac}
%
% Generic mathematical commands
% Notable: \degree, \celcius
\usepackage{gensymb}
%
% Variable vector notation (arrow above variable)
\usepackage{esvect}
%
% Multiline boxed equations
\usepackage{empheq}
%
% SI Unit
\usepackage{siunitx}
\DeclareSIUnit\atmosphere{atm}
\usepackage{physunits}
%
% More intuitive arrays/matrices
\usepackage{array}
%
% Linear Equations
\usepackage{systeme}
%
% Boxes!
\usepackage{mdframed}
%
% Matrix Notation
\usepackage{bm}

% Graphic packages %
%
% Diagrams and illustrations
\usepackage{tikz}
\usetikzlibrary{positioning}
%
% Image insertion
\usepackage{graphicx}

% LaTeX Commands
%
% Argument Parser
\usepackage{xparse}

% Document content %
%
% Change title of table of contents
% \renewcommand{\contentsname}{Title}

\begin{document}

% Command `\hr` to insert horizontal rules
\newcommand{\hr}{\par\noindent\rule{\textwidth}{0.4pt}}

% Command to box and center math equations
\newcommand{\bc}[1]{
	\begin{equation*}
		\begin{boxed}
			{#1}
		\end{boxed}
	\end{equation*}
}

% Command for single line equations with a condition
\newcommand{\cond}[2]{
	\ifmmode
		{#1} \quad {#2}
	\else
		$$ {#1} \quad {#2} $$
	\fi
}

% Matrix and Vector notation
\newcommand{\matr}[1]{
	\ifmmode \bm{#1}
	\else \textit{\textbf{#1}}
	\fi
}
\newcommand{\vect}[1]{
	\ifmmode \mathbf{#1}
	\else \textbf{#1}
	\fi
}

\newcommand{\boldalph}{\textbf{(\alph*)}}

% Laplace
\NewDocumentCommand{\lap}{o}{
	\IfNoValueTF{#1}
		{ \mathcal{L} }
		{ \mathcal{L} \left\{ {#1} \right\} }
}
\NewDocumentCommand{\ilap}{o}{
	\IfNoValueTF{#1}
		{ \mathcal{L}^{-1} }
		{ \mathcal{L}^{-1} \left\{ {#1} \right\} }
}

\tableofcontents

\section{Chapter 18 - Thermal Properties of Matter}

Avogadro's number
\begin{equation}
	N_A = \SI{6.02e23}{\mole}
\end{equation}

\subsection{The Ideal Gas Law}

\textbf{Ideal gas}: a collection of atoms or molecules that move randomly and exert no long-range forces on each other.

Number of moles
\begin{equation}
	n = \frac{ N }{ N_A } = \frac{ m_{particle}N }{ m_{particle}N_A } = \frac{ m }{ M }
\end{equation}
The \textbf{molar mass $ M $ (molecular weight)} is the mass per mole. The total mass of $ n $ moles is $ m_{total} = nM $.

Ideal-gas equation
\begin{equation}
	pV = nRT
\end{equation}

Universal gas constant
\begin{equation}
	R = \SI{8.31}{\joule \per \mole \per \kelvin} = \SI{0.0821}{\liter \atmosphere \per \mole \per \kelvin}
\end{equation}
The volume occupied by \SI{1}{\mole} of any ideal gas at atmospheric pressure and at \SI{0}{\celsius} is \SI{22.4}{\liter}.

\subsubsection{Question}

\begin{align*}
	V & = \SI{22.4e-3}{\liter} \\
	T & = \SI{273.15}{\kelvin} \\
	p & = \SI{1.013e5}{\pascal} = \SI{1.0}{\atmosphere} \\
	n & = ?
\end{align*}
\begin{align*}
	pV & = nRT \\
	n & = \frac{ pV }{ RT } \\
	n & = \frac{ (\SI{1.0}{\atm})(\SI{22.4}{\liter}) }{ (\SI{0.0821}{\liter \atmosphere \per \mole \per \kelvin})(\SI{273.15}{\kelvin}) } \\
	n & = \SI{1.000}{\mole}
\end{align*}

\subsubsection{18.3}

\begin{align*}
	V_0 & = \SI{0.110}{\meter \cubed} \\
	p_0 & = \SI{0.355}{\atmosphere} \\
	V_1 & = \SI{0.390}{\meter \cubed} \\
	T & = \text{constant} \\
	p_1 & = ?
\end{align*}
\begin{align*}
	p_0V_0 & = p_1V_1 \\
	p_1 & = \frac{ p_0V_0 }{ V_1 } \\
	p_1 & = \frac{ (\SI{0.355}{\atmosphere})(\SI{0.110}{\meter \cubed}) }{ \SI{0.390}{\meter \cubed} } \\
	p_1 & = \SI{0.1001}{\atmosphere}
\end{align*}

\subsubsection{18.4}

\begin{align*}
	V_0 & = \SI{3.00}{\liter} \\
	p_0 & = \SI{3.00}{\atmosphere} \\
	T_0 & = \SI{20.0}{\celsius} = \SI{293}{\kelvin} \\
	p_1 & = \SI{1.00}{\atmosphere}
\end{align*}
\begin{enumerate}[label = \boldalph]
	\item
		\begin{align*}
			pV & = nRT \\
			\frac{ p }{ T } & = \frac{ nR }{ V } \\
			\frac{ p_0 }{ T_0 } & = \frac{ p_1 }{ T_1 } \\
			T_1 & = \frac{ p_1T_0 }{ p_0 } \\
			T_1 & = \frac{ (\SI{1.00}{\atmosphere})(\SI{293}{\kelvin}) }{ \SI{3.00}{\atmosphere} } \\
			T_1 & = \SI{97.7}{\kelvin} = \SI{-175.3}{\celsius}
		\end{align*}
\end{enumerate}

\subsubsection{18.7}

\begin{align*}
	V_0 & = \SI{499}{\centi \meter \cubed} = \SI{499e-6}{\meter \cubed} \\
	p_0 & = \SI{1.01e5}{\pascal} \\
	T_0 & = \SI{27.0}{\celsius} = \SI{300}{\kelvin} \\
	V_1 & = \SI{46.2}{\centi \meter \cubed} = \SI{46.2e-6}{\meter \cubed} \\
	p_1 & = \SI{2.72e6}{\pascal} + \SI{1}{\atmosphere} = \SI{2.821e6}{\pascal} \\
	T_1 & = ?
\end{align*}
\begin{align*}
	p_0V_0 & = nrT \\
	\frac{ p_0V_0 }{ T_0 } & = \frac{ p_1V_1 }{ T_1 } \\
	T_1 & = \frac{ p_1V_1T_0 }{ p_0V_0 }
\end{align*}

\end{document}
