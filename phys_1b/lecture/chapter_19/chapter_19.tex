\documentclass{article}

% Document extensibility %
%
% Disables native paragraph indentation
\usepackage{parskip} 
%
% Provides further bullet options for lists
\usepackage{enumitem}

% Mathematical symbol and statement packages %
%
% Necessary
\usepackage{amsmath}
\usepackage{amssymb}
%
% Extensive fraction notation
\usepackage{xfrac}
%
% Generic mathematical commands
% Notable: \degree, \celcius
\usepackage{gensymb}
%
% Variable vector notation (arrow above variable)
\usepackage{esvect}
%
% Multiline boxed equations
\usepackage{empheq}
%
% SI Unit
\usepackage{siunitx}
\DeclareSIUnit\atmosphere{atm}
\usepackage{physunits}
%
% More intuitive arrays/matrices
\usepackage{array}
%
% Linear Equations
\usepackage{systeme}
%
% Boxes!
\usepackage{mdframed}
%
% Matrix Notation
\usepackage{bm}

% Graphic packages %
%
% Diagrams and illustrations
\usepackage{tikz}
\usetikzlibrary{positioning}
%
% Image insertion
\usepackage{graphicx}

% LaTeX Commands
%
% Argument Parser
\usepackage{xparse}

% Document content %
%
% Change title of table of contents
\renewcommand{\contentsname}{Chapter 19}

\begin{document}

% Command `\hr` to insert horizontal rules
\newcommand{\hr}{\par\noindent\rule{\textwidth}{0.4pt}}

% Command to box and center math equations
\newcommand{\bc}[1]{
	\begin{equation*}
		\begin{boxed}
			{#1}
		\end{boxed}
	\end{equation*}
}

% Command for single line equations with a condition
\newcommand{\cond}[2]{
	\ifmmode
		{#1} \quad {#2}
	\else
		$$ {#1} \quad {#2} $$
	\fi
}

% Matrix and Vector notation
\newcommand{\matr}[1]{
	\ifmmode \bm{#1}
	\else \textit{\textbf{#1}}
	\fi
}
\newcommand{\vect}[1]{
	\ifmmode \mathbf{#1}
	\else \textbf{#1}
	\fi
}

% Laplace
\NewDocumentCommand{\lap}{o}{
	\IfNoValueTF{#1}
		{ \mathcal{L} }
		{ \mathcal{L} \left\{ {#1} \right\} }
}
\NewDocumentCommand{\ilap}{o}{
	\IfNoValueTF{#1}
		{ \mathcal{L}^{-1} }
		{ \mathcal{L}^{-1} \left\{ {#1} \right\} }
}

\tableofcontents

\section{The First Law of Thermodynamics}

In a thermodynamic process, $ Q $ is positive when heat flows \textbf{into} a system, and negative when heat flows \textbf{out} of the system.

Work $ W $ is \textbf{positive} when work is done by the system against its surroundings, and hence corresponds to energy leaving the system. $ W $ is negative when work is done on the system.

\subsection{Work Done During Volume Changes}

\begin{itemize}
	\item Isobaric: $ p $ is constant
		\begin{align*}
			dW & = (F) dx \\
			dW & = (pA) dx, \quad \text{Pressure} = \frac{ \text{Force} }{ \text{Area} } \\
			dW & = (p) dV \\
			\int_{0}^{W} (1) dW & = p \int_{V_0}^{V_1} (1) dV \\
			W & = p \left[ V_1 - V_0 \right]
		\end{align*}
		\begin{equation}
			W = p \left[ V_1 - V_0 \right]
		\end{equation}
	\item Isochoric: $ V $ is constant, $ \Delta U = Q - 0 $
		\begin{align*}
			W & = \int (p) dV = p(0) = 0
		\end{align*}
	\item Isothermal: $ T $ is constant, $ 0 = Q - nRT \ln \frac{ V_1 }{ V_0 } $
		\begin{align*}
			W & = \int (p) dV \\
			W & = \int \left( \frac{ nRT }{ V } \right) dV \\
			W & = nRT \int_{V_0}^{V_1} \left( \frac{ 1 }{ V } \right) dV \\
			W & = nRT \ln \left[ \frac{ V_1 }{ V_0 } \right]
		\end{align*}
	\item Adiabatic: No heat enters or exits, $ Q = 0 $, $ \Delta U = 0 - W $
\end{itemize}

\subsubsection{Question}

\begin{align*}
	dW & = \int \left( \frac{ nRT }{ P } \right) dp \\
	\int_{0}^{W} (1) dW & = nRT \int_{p_0}^{p_1} \left( \frac{ 1 }{ P } \right) dp
\end{align*}

\subsubsection{19.7}

\begin{enumerate}[label = \textbf{(\alph*)}]
	\item
		\begin{align*}
			W_{1,3} & = p_1(V_2 - V_1) \\
			W_{3,2} & = 0 \\
			W_{2,4} & = p_2(V_1 - V_2) \\
			W_{4,1} & = 0
		\end{align*}
		\begin{align*}
			W_{total} & = p_1(V_2 - V_1) + p_2(V_1 - V_2) \\
			W_{total} & = p_1(V_2 - V_1) - p_2(-V_1 + V_2) \\
			W_{total} & = (p_1 - p_2)(V_2 - V_1)
		\end{align*}
\end{enumerate}

\subsubsection{19.1}

\begin{align*}
	n & = \SI{2}{\mole} \\
	T_0 & = \SI{27}{\celsius} = \SI{300}{\kelvin} \\
	T_1 & = \SI{107}{\celsius} = \SI{380}{\kelvin} \\
	W & = ?
\end{align*}
\begin{align*}
	W & = p \int_{V_1}^{V_2} (1) dV \\
	W & = p(V_2 - V_1)
\end{align*}
\begin{align*}
	pV & = nRT \\
	p(V_2 - V_1) & = nR(T_2 - T_1)
\end{align*}
\begin{align*}
	W & = nR(T_2 - T_1) \\
	W & = (\SI{2}{\mole})(\SI{8.314}{\joule \per \mole \per \kelvin})(\SI{380}{\kelvin} - \SI{300}{\kelvin}) \\
	W & = \SI{1330.24}{\joule}
\end{align*}

\subsubsection{19.2}

\begin{align*}
	n & = \SI{6}{\mole} \\
	T_0 & = \SI{27.0}{\celsius} = \SI{300}{\kelvin} \\
	p & = \text{constant} \\
	T_1 & = ? \\
	W & = \SI{2.40e3}{\joule}
\end{align*}
\begin{align*}
	W & = p \left[ V_1 - V_0 \right] \\
	p \left[ V_1 - V_0 \right] & = nR \left[ T_1 - T_0 \right] \\
	W & = nR \left[ T_1 - T_0 \right] \\
	T_1 & = \frac{ W }{ nR } + T_0 \\
	T_1 & = \frac{ \SI{2.40e3}{\joule} }{ (\SI{6}{\mole})(\SI{8.314}{\joule \per \mole \per \kelvin}) } + \SI{300}{\kelvin} \\
	T_1 & = \SI{348.112}{\kelvin}
\end{align*}

\subsubsection{19.3}

\begin{align*}
	n & = \SI{2}{\mole} \\
	T & = \SI{65.0}{\celsius} = \SI{338}{\kelvin} \\
	p_1 & = 3p_0
\end{align*}
\begin{align*}
	W & = \int (p) dV \\
	W & = nRT \ln \left[ \frac{ p_0 }{ p_1 } \right] \\
	W & = nRT \ln \left[ \frac{ p_0 }{ 3p_0 } \right] \\
	W & = (\SI{2}{\mole})(\SI{8.314}{\joule \per \mole \per \kelvin})(\SI{338}{\kelvin}) \ln \left[ \frac{ 1 }{ 3 } \right] \\
	W & = \SI{-6174.49}{\joule}
\end{align*}

\subsubsection{19.5}

\begin{align*}
	n & = \SI{0.305}{\mole} \\
	T & = \SI{22.0}{\celsius} = \SI{295}{\kelvin} \\
	W & = \SI{-392}{\joule} \\
	p_1 & = \SI{1.76}{\atmosphere} \\
	p_0 & = ?
\end{align*}
\begin{align*}
	W & = nRT \ln \left[ \frac{ p_0 }{ p_1 } \right] \\
	e^{W} & = e^{nRT} \cdot \frac{ p_0 }{ p_1 } \\
	p_0 & = \frac{ e^{W} }{ e^{nRT} }p_1 \\
	p_0 & = e^{ \frac{ \SI{-392}{\joule} }{ (\SI{0.305}{\mole})(\SI{8.314}{\joule \per \mole \per \kelvin})(\SI{295}{\kelvin}) } }(\SI{1.76}{\atmosphere}) \\
	p_0 & = \SI{1.04}{\atmosphere}
\end{align*}

\subsubsection{19.6}

\begin{align*}
	V & = \SI{0.200}{\meter \cubed} \\
	p_0 & = \SI{2.00e5}{\pascal} \\
	p_1 & = \SI{5.00e5}{\pascal}
\end{align*}
\begin{align*}
	V_0 & = \SI{0.200}{\meter \cubed} \\
	V_1 & = \SI{0.120}{\meter \cubed} \\
	p & = \SI{5.00e5}{\pascal}
\end{align*}
\begin{align*}
	W & = 0 + p \left[ V_1 - V_0 \right] \\
	W & = (\SI{5.00e5}{\pascal})(\SI{0.120}{\meter \cubed} - \SI{0.200}{\meter \cubed}) \\
	W & = \SI{-40000}{\joule}
\end{align*}

\section{First Law of Thermodynamics}

First Law of Thermodynamics
\begin{equation}
	\Delta U = Q - W
\end{equation}

\subsubsection{19.9}

\begin{align*}
	V_0 & = \SI{0.110}{\meter \cubed} \\
	V_1 & = \SI{0.320}{\meter \cubed} \\
	p & = \SI{1.65e5}{\pascal} \\
	Q & = \SI{1.15e5}{\joule}
\end{align*}
\begin{enumerate}[label = \textbf{(\alph*)}]
	\item
		\begin{align*}
			W & = p\Delta V \\
			W & = (\SI{1.65e5}{\pascal}) \left[ \SI{0.320}{\meter \cubed} - \SI{0.110}{\meter \cubed} \right] \\
			W & = \SI{34650}{\joule}
		\end{align*}
	\item
		\begin{align*}
			\Delta U & = Q - W \\
			\Delta U & = \SI{1.15e5}{\joule} - \SI{34650}{\joule} \\
			\Delta U & = \SI{80350}{\joule}
		\end{align*}
\end{enumerate}

\subsubsection{19.10}

\begin{align*}
	n & = \SI{5}{\mole} \\
	T_0 & = \SI{127}{\celsius} = \SI{400}{\kelvin} \\
	Q & = \SI{1500}{\joule} \\
	W & = \SI{2100}{\joule} \\
	T_1 & = ?
\end{align*}
\begin{align*}
	Q & = nC_V(T_1 - T_0) \\
	T_1 & = \frac{ Q }{ nC_V } + T_0 \\
	T_1 & = \frac{ \SI{1500}{\joule} }{ (\SI{5}{\mole}) \left( \frac{ 3 }{ 2 }(\SI{8.314}{\joule \per \mole \per \kelvin}) \right) } + \SI{400}{\kelvin} \\
	T_1 & = \SI{424.056}{\kelvin}
\end{align*}

\subsubsection{19.11}

\begin{align*}
	n & = \SI{0.0175}{\mole}
\end{align*}
\begin{enumerate}[label = \textbf{(\alph*)}]
	\item
		\begin{align*}
			T_{low} & = ?
		\end{align*}
		\begin{align*}
			T & = \frac{ pV }{ nR } \\
			T_a & = \frac{ p_aV_a }{ nR } \\
			T_a & = \frac{ (\SI{0.20}{\atmosphere})(\SI{2.0}{\liter}) }{ (\SI{0.0175}{\mole})(\SI{0.0821}{\liter \atmosphere \per \mole \per \kelvin}) } \\
			T_a & = \SI{278.406}{\kelvin} \\
			T_b & = \frac{ p_bV_b }{ nR } \\
			T_b & = \frac{ (\SI{0.50}{\atmosphere})(\SI{2.0}{\liter}) }{ (\SI{0.0175}{\mole})(\SI{0.0821}{\liter \atmosphere \per \mole \per \kelvin}) } \\
			T_b & = \SI{696.015}{\kelvin} \\
			T_c & = \frac{ p_cV_c }{ nR } \\
			T_c & = \frac{ (\SI{0.30}{\atmosphere})(\SI{6.0}{\liter}) }{ (\SI{0.0175}{\mole})(\SI{0.0821}{\liter \atmosphere \per \mole \per \kelvin}) } \\
			T_c & = \SI{1252.83}{\kelvin} \\
		\end{align*}
	\item
		\begin{align*}
			W_{a,b} & = 0 \\
			W_{b,c} & = ?
		\end{align*}
		\begin{align*}
			W_{b,c} & = \int \frac{ nR }{ V } dV
		\end{align*}
\end{enumerate}

\subsubsection{19.12}

\begin{align*}
	p & = \SI{1.80e5}{\pascal} \\
	V_0 & = \SI{1.70}{\meter \cubed} \\
	V_1 & = \SI{1.20}{\meter \cubed} \\
	\Delta U & = -\SI{1.40e5}{\joule}
\end{align*}
\begin{enumerate}[label = \textbf{(\alph*)}]
	\item
		\begin{align*}
			W & = p\Delta V \\
			W & = (\SI{1.80e5}{\pascal})(\SI{1.20}{\meter \cubed} - \SI{1.70}{\meter \cubed}) \\
			W & = \SI{-90000}{\joule} = \SI{-9.0e4}{\joule}
		\end{align*}
	\item
		\begin{align*}
			\Delta U & = Q - W \\
			Q & = \Delta U + W \\
			Q & = \SI{-1.40e5}{\joule} - \SI{9.0e4}{\joule} \\
			Q & = \SI{-3.2e5}{\joule}
		\end{align*}
\end{enumerate}

\subsubsection{19.13}

\begin{align*}
	n & = \SI{0.450}{\mole}
\end{align*}
\begin{enumerate}[label = \textbf{(\alph*)}]
	\item
		\begin{align*}
			pV & = nRT \\
			T & = \frac{ pV }{ nR }
		\end{align*}
		\begin{align*}
			T_a & = \frac{ (\SI{2.0e5}{\pascal})(\SI{0.010}{\meter \cubed}) }{ (\SI{0.450}{\mole})(\SI{8.314}{\joule \per \mole \per \kelvin}) } \\
			T_a & = \SI{534.574}{\kelvin} \\
			T_b & = \frac{ (\SI{5.0e5}{\pascal})(\SI{0.070}{\meter \cubed}) }{ (\SI{0.450}{\mole})(\SI{8.314}{\joule \per \mole \per \kelvin}) } \\
			T_b & = \SI{534.574}{\kelvin} \\
			T_b & = \SI{9355.04}{\kelvin} \\
			T_c & = \frac{ (\SI{8.0e5}{\pascal})(\SI{0.070}{\meter \cubed}) }{ (\SI{0.450}{\mole})(\SI{8.314}{\joule \per \mole \per \kelvin}) } \\
			T_c & = \SI{14968.1}{\kelvin} \\
		\end{align*}
	\item
		\begin{align*}
			W & = \frac{ (\SI{5.0e5}{\pascal} - \SI{2.0e5}{\pascal})(\SI{0.070}{\meter \cubed} - \SI{0.010}{\meter \cubed}) }{ 2 }
		\end{align*}
\end{enumerate}

\section{Four Kinds of Thermodynamic Processes}

There are four specific kinds of thermodynamic processes that occur often in practical situations:
\begin{itemize}
	\item
		\textbf{Adiabatic}: No heat is transferred into or out of the system, so $ Q = 0 $, $ \Delta U = 0 - W $.
	\item
		\textbf{Isochoric}: The volume remains constant, so $ W = 0 $, $ \Delta U = Q - 0 $.
	\item
		\textbf{Isobaric}: The pressure remains constant, so $ W = p\Delta V $.
	\item
		\textbf{Isothermal}: The temperature remains constant, so $ \Delta U = 0 $, $ Q = W $.
\end{itemize}
\begin{align*}
	dQ & = nC_PdT \\
	dW & = pdV \\
	dW & = nRdT \\
	dU & = dQ - dW \\
	dU & = nC_PdT - nRdT \\
	nC_VdT & = nC_PdT - nRdT
\end{align*}

\subsubsection{19.16}

\begin{align*}
	\Delta U & = Q - W \\
	W & = Q \\
	W & = \SI{410}{\joule}
\end{align*}

\subsubsection{19.17}

\begin{align*}
	n & = \SI{0.250}{\mole} \\
	T_0 & = \SI{27.0}{\celsius} = \SI{300}{\kelvin} \\
	p & = \SI{1.00}{\atmosphere} \\
	T_1 & = \SI{127.0}{\celsius} = \SI{400}{\kelvin}
\end{align*}
\begin{enumerate}[label = \textbf{(\alph*)}]
	\item
	\item
		\begin{align*}
			W & = p\Delta V = nR\Delta T \\
			W & = (\SI{0.250}{\mole})(\SI{8.314}{\joule \per \mole \per \kelvin})(\SI{400}{\kelvin} - \SI{300}{\kelvin}) \\
			W & = \SI{207.85}{\joule}
		\end{align*}
	\item
	\item
		\begin{align*}
			\Delta U & = nC_V\Delta T \\
			\Delta U & = (\SI{0.250}{\mole})(\SI{28.46}{\joule \per \mole \per \kelvin})(\SI{400}{\kelvin} - \SI{300}{\kelvin}) \\
			\Delta U & = \SI{711.5}{\joule}
		\end{align*}
	\item
		\begin{align*}
			\Delta U & = Q - W \\
			Q & = \Delta U + W \\
			Q & = \SI{711.5}{\joule} + \SI{207.85}{\joule} \\
			Q & = \SI{919.35}{\joule}
		\end{align*}
	\item
		\begin{align*}
			W & = p\Delta V = nR\Delta T \\
			W & = (\SI{0.250}{\mole})(\SI{8.314}{\joule \per \mole \per \kelvin})(\SI{400}{\kelvin} - \SI{300}{\kelvin}) \\
			W & = \SI{207.85}{\joule}
		\end{align*}
\end{enumerate}

\subsubsection{19.18}

\begin{align*}
	n & = \SI{0.0100}{\mole} \\
	T_0 & = \SI{27.0}{\celsius} = \SI{300}{\kelvin}
\end{align*}
\begin{enumerate}[label = \textbf{(\alph*)}]
	\item
		\begin{align*}
			T_1 & = \SI{67.0}{\celsius} = \SI{340}{\kelvin} \\
			\Delta V & = V
		\end{align*}
		\begin{align*}
			Q & = nC_V\Delta T \\
			Q & = (\SI{0.0100}{\mole})(\SI{12.47}{\joule \per \mole \per \kelvin})(\SI{340}{\kelvin} - \SI{300}{\kelvin}) \\
			Q & = \SI{4.988}{\joule}
		\end{align*}
	\item
		\begin{align*}
			Q & = nC_P\Delta T \\
			Q & = n(C_V + R)\Delta T \\
			Q & = (\SI{0.0100}{\mole})(\SI{12.47}{\joule \per \mole \per \kelvin} + \SI{8.314}{\joule \per \mole \per \kelvin})(\SI{340}{\kelvin} - \SI{300}{\kelvin}) \\
			Q & = \SI{8.314}{\joule}
		\end{align*}
	\item
	\item
		\begin{align*}
			\Delta U & = nC_V\Delta T \\
			\Delta U & = (\SI{0.0100}{\mole})(\SI{12.47}{\joule \per \mole \per \kelvin})(\SI{40}{\kelvin}) \\
			\Delta U & = \SI{4.988}{\joule}
		\end{align*}
\end{enumerate}

\end{document}
