\documentclass{article}

% Document extensibility %
%
% Disables native paragraph indentation
\usepackage{parskip} 
%
% Provides further bullet options for lists
\usepackage{enumitem}

% Mathematical symbol and statement packages %
%
% Necessary
\usepackage{amsmath}
\usepackage{amssymb}
%
% Extensive fraction notation
\usepackage{xfrac}
%
% Generic mathematical commands
% Notable: \degree, \celcius
\usepackage{gensymb}
%
% Variable vector notation (arrow above variable)
\usepackage{esvect}
%
% Multiline boxed equations
\usepackage{empheq}
%
% SI Unit
\usepackage{siunitx}
\usepackage{physunits}
%
% More intuitive arrays/matrices
\usepackage{array}
%
% Linear Equations
\usepackage{systeme}
%
% Boxes!
\usepackage{mdframed}
%
% Matrix Notation
\usepackage{bm}

% Graphic packages %
%
% Diagrams and illustrations
\usepackage{tikz}
\usetikzlibrary{positioning}
%
% Image insertion
\usepackage{graphicx}

% LaTeX Commands
%
% Argument Parser
\usepackage{xparse}

% Document content %
%
% Change title of table of contents
\renewcommand{\contentsname}{Chapter 19}

\begin{document}

% Command `\hr` to insert horizontal rules
\newcommand{\hr}{\par\noindent\rule{\textwidth}{0.4pt}}

% Command to box and center math equations
\newcommand{\bc}[1]{
	\begin{equation*}
		\begin{boxed}
			{#1}
		\end{boxed}
	\end{equation*}
}

% Command for single line equations with a condition
\newcommand{\cond}[2]{
	\ifmmode
		{#1} \quad {#2}
	\else
		$$ {#1} \quad {#2} $$
	\fi
}

% Matrix and Vector notation
\newcommand{\matr}[1]{
	\ifmmode \bm{#1}
	\else \textit{\textbf{#1}}
	\fi
}
\newcommand{\vect}[1]{
	\ifmmode \mathbf{#1}
	\else \textbf{#1}
	\fi
}

% Laplace
\NewDocumentCommand{\lap}{o}{
	\IfNoValueTF{#1}
		{ \mathcal{L} }
		{ \mathcal{L} \left\{ {#1} \right\} }
}
\NewDocumentCommand{\ilap}{o}{
	\IfNoValueTF{#1}
		{ \mathcal{L}^{-1} }
		{ \mathcal{L}^{-1} \left\{ {#1} \right\} }
}

\tableofcontents

\section{The First Law of Thermodynamics}

In a thermodynamic process, $ Q $ is positive when heat flows \textbf{into} a system, and negative when heat flows \textbf{out} of the system.

Work $ W $ is \textbf{positive} when work is done by the system against its surroundings, and hence corresponds to energy leaving the system. $ W $ is negative when work is done on the system.

\subsection{Work Done During Volume Changes}

\begin{itemize}
	\item Isobaric: $ p $ is constant
		\begin{align*}
			dW & = (F) dx \\
			dW & = (pA) dx, \quad \text{Pressure} = \frac{ \text{Force} }{ \text{Area} } \\
			dW & = (p) dV \\
			\int_{0}^{W} (1) dW & = p \int_{V_0}^{V_1} (1) dV \\
			W & = p \left[ V_1 - V_0 \right]
		\end{align*}
		\begin{equation}
			W = p \left[ V_1 - V_0 \right]
		\end{equation}
	\item Isochoric: $ V $ is constant, $ \Delta U = Q - 0 $
		\begin{align*}
			W & = \int (p) dV = p(0) = 0
		\end{align*}
	\item Isothermal: $ T $ is constant, $ 0 = Q - nRT \ln \frac{ V_1 }{ V_0 } $
		\begin{align*}
			W & = \int (p) dV \\
			W & = \int \left( \frac{ nRT }{ V } \right) dV \\
			W & = nRT \int_{V_0}^{V_1} \left( \frac{ 1 }{ V } \right) dV \\
			W & = nRT \ln \left[ \frac{ V_1 }{ V_0 } \right]
		\end{align*}
	\item Adiabatic: No heat enters or exits, $ Q = 0 $, $ \Delta U = 0 - W $
\end{itemize}

\subsubsection{Question}

\begin{align*}
	dW & = \int \left( \frac{ nRT }{ P } \right) dp \\
	\int_{0}^{W} (1) dW & = nRT \int_{p_0}^{p_1} \left( \frac{ 1 }{ P } \right) dp
\end{align*}

\subsubsection{19.7}

\begin{enumerate}[label = \textbf{(\alph*)}]
	\item
		\begin{align*}
			W_{1,3} & = p_1(V_2 - V_1) \\
			W_{3,2} & = 0 \\
			W_{2,4} & = p_2(V_1 - V_2) \\
			W_{4,1} & = 0
		\end{align*}
		\begin{align*}
			W_{total} & = p_1(V_2 - V_1) + p_2(V_1 - V_2) \\
			W_{total} & = p_1(V_2 - V_1) - p_2(-V_1 + V_2) \\
			W_{total} & = (p_1 - p_2)(V_2 - V_1)
		\end{align*}
\end{enumerate}

\subsubsection{19.1}

\begin{align*}
	n & = \SI{2}{\mole} \\
	T_0 & = \SI{27}{\celsius} = \SI{300}{\kelvin} \\
	T_1 & = \SI{107}{\celsius} = \SI{380}{\kelvin} \\
	W & = ?
\end{align*}
\begin{align*}
	W & = p \int_{V_1}^{V_2} (1) dV \\
	W & = p(V_2 - V_1)
\end{align*}
\begin{align*}
	pV & = nRT \\
	p(V_2 - V_1) & = nR(T_2 - T_1)
\end{align*}
\begin{align*}
	W & = nR(T_2 - T_1) \\
	W & = (\SI{2}{\mole})(\SI{8.314}{\joule \per \mole \per \kelvin})(\SI{380}{\kelvin} - \SI{300}{\kelvin}) \\
	W & = \SI{1330.24}{\joule}
\end{align*}

\subsubsection{19.2}

\begin{align*}
	n & = \SI{6}{\mole} \\
	T_0 & = \SI{27.0}{\celsius} = \SI{300}{\kelvin} \\
	p & = \text{constant} \\
	T_1 & = ? \\
	W & = \SI{2.40e3}{\joule}
\end{align*}
\begin{align*}
	W & = p \left[ V_1 - V_0 \right] \\
	p \left[ V_1 - V_0 \right] & = nR \left[ T_1 - T_0 \right] \\
	W & = nR \left[ T_1 - T_0 \right] \\
	T_1 & = \frac{ W }{ nR } + T_0 \\
	T_1 & = \frac{ \SI{2.40e3}{\joule} }{ (\SI{6}{\mole})(\SI{8.314}{\joule \per \mole \per \kelvin}) } + \SI{300}{\kelvin} \\
	T_1 & = \SI{348.112}{\kelvin}
\end{align*}

\end{document}
