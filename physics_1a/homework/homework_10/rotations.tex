\documentclass{article}

% Document extensibility %
%
% Disables native paragraph indentation
\usepackage{parskip} 
%
% Provides further bullet options for lists
\usepackage{enumitem}

% Mathematical symbol and statement packages %
%
% Necessary
\usepackage{amsmath}
\usepackage{amssymb}
%
% Extensive fraction notation
\usepackage{xfrac}
%
% Generic mathematical commands
% Notable: \degree, \celcius
\usepackage{gensymb}
%
% Variable vector notation (arrow above variable)
\usepackage{esvect}
%
% Multiline boxed equations
\usepackage{empheq}
%
% SI Unit
\usepackage{siunitx}
\DeclareSIUnit\dyne{dyn}
\usepackage{physunits}
%
% More intuitive arrays/matrices
\usepackage{array}
%
% Linear Equations
\usepackage{systeme}
%
% Boxes!
\usepackage{mdframed}
%
% Matrix Notation
\usepackage{bm}

% Graphic packages %
%
% Diagrams and illustrations
\usepackage{tikz}
\usetikzlibrary{positioning, patterns, angles, quotes, calc}
%
% Image insertion
\usepackage{graphicx}

% LaTeX Commands
%
% Argument Parser
\usepackage{xparse}

% Document content %
%
% Change title of table of contents
% \renewcommand{\contentsname}{Title}

\title{Homework 10 Rotations}
\author{Corey Mostero - 2566652}
\date{6 June 2023}

\begin{document}

% Command `\hr` to insert horizontal rules
\newcommand{\hr}{\par\noindent\rule{\textwidth}{0.4pt}}

% Command to box and center math equations
\newcommand{\bc}[1]{
	\begin{equation*}
		\begin{boxed}
			{#1}
		\end{boxed}
	\end{equation*}
}

% Command for single line equations with a condition
\newcommand{\cond}[2]{
	\ifmmode
		{#1} \quad {#2}
	\else
		$$ {#1} \quad {#2} $$
	\fi
}

% Matrix and Vector notation
\newcommand{\matr}[1]{
	\ifmmode \bm{#1}
	\else \textit{\textbf{#1}}
	\fi
}
\newcommand{\vect}[1]{
	\ifmmode \mathbf{#1}
	\else \textbf{#1}
	\fi
}

% Laplace
\NewDocumentCommand{\lap}{o}{
	\IfNoValueTF{#1}
		{ \mathcal{L} }
		{ \mathcal{L} \left\{ {#1} \right\} }
}
\NewDocumentCommand{\ilap}{o}{
	\IfNoValueTF{#1}
		{ \mathcal{L}^{-1} }
		{ \mathcal{L}^{-1} \left\{ {#1} \right\} }
}

\newcommand{\boldalpha}{\textbf{(\alph*)}}

\maketitle
\newpage

\tableofcontents

\section{Book}

\subsection{10.22}

\begin{align*}
	r & = \SI{8.00}{\centi \meter} \\
	m & = \SI{0.180}{\kilogram} \\
	v_0 & = 0 \\
	\Delta y & = \SI{75.0}{\centi \meter} \\
	I & = mr^2
\end{align*}
\begin{enumerate}[label = \boldalpha]
	\item
		\begin{align*}
			E_{K_0} + E_{P_0} & = E_{K_1} + E_{P_1} \\
			0 + mgh & = \frac{1}{2}I_{cm}\omega^2 + \frac{1}{2}mr^2\omega^2 + 0 \\
			mgh & = \omega^2 \left( \frac{1}{2} \left( mr^2 \right) + \frac{1}{2}mr^2 \right) \\
			\omega & = \frac{ \sqrt{ gh } }{ r } \\
			\omega & = \frac{ \sqrt{ (\SI{10.0}{\meter \per \second \squared})(\SI{0.75}{\meter}) } }{ \SI{0.08}{\meter} } \\
			\omega & = \SI{34.2}{\radian \per \second}
		\end{align*}
	\item
		\begin{align*}
			E_{k_0} + E_{p_0} & = E_{k_1} + E_{p_1} \\
			0 + mgh & = \frac{1}{2}I_{cm}\omega^2 + \frac{1}{2}mv_{cm}^2 + 0 \\
			mgh & = \frac{1}{2}(mr^2) \left( \frac{v_{cm}}{r} \right)^2 + \frac{1}{2}mv_{cm}^2 \\
			v & = \sqrt{ gh } \\
			v & = \sqrt{ (\SI{10.0}{\meter \per \second \squared})(\SI{0.75}{\meter}) } \\
			v & = \SI{2.74}{\meter \per \second}
		\end{align*}
\end{enumerate}

\subsection{10.26}

\begin{align*}
	I_{cm} & = \frac{2}{5}mr^2
\end{align*}
\begin{enumerate}[label = \boldalpha]
	\item
		Velocity for the first half of the bowl:
		\begin{align*}
			E_{K_0} + E_{P_0} & = E_{K_1} + E_{P_1} \\
			0 + mgh & = \frac{1}{2}I_{cm}\omega^2 + \frac{1}{2}mv_{cm}^2 + 0 \\
			mgh & = \frac{1}{2} \left( \frac{2}{5}mr^2 \right) \left( \frac{v_{cm}^2}{r^2} \right) + \frac{1}{2}mv_{cm}^2 \\
			v_{cm} & = \sqrt{ \frac{10gh}{7} }
		\end{align*}
		Since the ball only slides and doesn't rotate, the kinetic energy it experiences it purely linear velocity and \textit{not} angular.
		\begin{align*}
			E_{K_0} + E_{P_0} & = E_{K_1} + E_{P_1} \\
			\frac{1}{2}mv_{cm}^2 + 0 & = 0 + mgh_1 \\
			\left( \sqrt{ \frac{10gh_0}{7} } \right)^2 & = 2gh_1 \\
			h_1 & = \frac{5}{7}h_0
		\end{align*}
		The ball reaches only $ \frac{5}{7} $ of the height of the side of the bowl.
\end{enumerate}

\subsection{10.30}

\begin{enumerate}[label = \boldalpha]
	\item Free-body diagram:
		\begin{center}
			\def\angle{25}
			\begin{tikzpicture}
				\node [circle, draw = black, fill = black, inner sep = 2pt] (origin) {};

				\draw [-stealth, draw = black!80, thick, dashed] (origin) -- ++ (1.5, 0) node (friction_angle) [midway] {};
				\draw [-stealth, thick, draw = black] (origin) -- ++ (\angle:1.5cm) node [midway] (friction) {};
				\draw [-stealth, draw = black!80, thick, dashed] (origin) -- ++ (1.5, 0) node (friction_angle) [midway] {};
				\node [black] at (friction.north west) {$ f $};

				\draw [-stealth, thick, draw = black] (origin) -- ++ (\angle + 90:1.5cm) node [midway] (normal) {};
				\node [black] at (normal.south west) {$ N $};

				\draw [-stealth, thick, draw = black] (origin) -- ++ (-90:1.5cm) node [midway, right] {$ mg $};

				\pic [draw = black!80, thick, stealth-stealth, "$\beta$", angle eccentricity = 1.3, angle radius = 10mm] {angle = friction_angle--origin--friction};

				\draw [-stealth, draw = black!80, thick, dashed] (origin) -- ++ (0, 1.5) node (normal_angle) [midway] {};
				\pic [draw = black!80, thick, stealth-stealth, "$\beta$", angle eccentricity = 1.3, angle radius = 10mm] {angle = normal_angle--origin--normal};
			\end{tikzpicture}
		\end{center}
		The angular velocity of the bowling ball is clockwise $ \circlearrowright $ which the friction has to oppose resulting in the friction going upwards (up the incline).
	\item
		Normal force has no affect as it is directed towards the axis of rotation.
		\begin{align*}
			\sum F_x & = ma_{cm} \\
			-f & = ma_{cm} + mg\sin(\beta) \\
			-\frac{ I_{cm}\alpha }{ r } & = m(a_{cm} + g\sin(\beta)) \\
			-\frac{ \left( \frac{2}{5}mr^2 \right) \left( \frac{ a_{cm} }{ r } \right) }{ r } & = m(a_{cm} + g\sin(\beta)) \\
			-\frac{2}{5}a_{cm} & = a_{cm} + g\sin(\beta) \\
			a_{cm} & = \frac{5g\sin(\beta)}{7}
		\end{align*}
	\item
		\begin{align*}
			\sum F_y & = 0 \\
			N & = mg\cos(\beta)
		\end{align*}
		\begin{align*}
			\sum F_x & = ma_{cm} \\
			-f & = ma_{cm} + mg\sin(\beta) \\
			-\mu(mg\cos(\beta)) & = m \left( \frac{5g\sin(\beta)}{7} + g\sin(\beta) \right) \\
			\mu & = \frac{2\tan(\beta)}{7}
		\end{align*}
\end{enumerate}

\subsection{10.79}

\begin{align*}
	I_{cylinder_{cm}} & = \frac{1}{2}m(2r)^2 \\
	I_{disk_{cm}} & = \frac{1}{2}mr^2
\end{align*}
\begin{align*}
	E_{K_0} + E_{P_0} & = E_{K_1} + E_{P_1} \\
	\left( \frac{1}{2}I_{cylinder_{cm}} \left( \frac{v}{2r} \right)^2 + \frac{1}{2}mv^2 \right) + \left( \frac{1}{2}I_{disk_{cm}} \left( \frac{v}{r} \right)^2 + \frac{1}{2}mv^2 \right) + 0 & = 0 + mgh \\
	\left( \frac{1}{2} \left( \frac{1}{2}m(2r)^2 \right) \left( \frac{v}{2r} \right)^2 + \frac{1}{2}mv^2 \right) + \left( \frac{1}{2} \left( \frac{1}{2}mr^2 \right) \left( \frac{v}{r} \right)^2 + \frac{1}{2}mv^2 \right) & = mgh \\
	v & = \sqrt{ \frac{ 2gh }{ 3 } }
\end{align*}
\begin{align*}
	v_1^2 & = v_0^2 + 2ah \\
	a & = \frac{ v_1^2 }{ 2h } \\
	a & = \frac{ \sqrt{ \frac{ 2gh }{ 3 } }^2 }{ 2h } \\
	a & = \frac{ g }{ 3 }
\end{align*}

\subsection{9.30}

\begin{align*}
	m & = \SI{0.200}{\kilogram} \\
	l & = \SI{0.400}{\meter} \\
\end{align*}
\begin{enumerate}[label = \boldalpha]
	\item
		\begin{align*}
			r & = \sqrt{ (\sfrac{l}{2})^2 + (\sfrac{l}{2})^2 } \\
			r & = \SI{0.283}{\meter}
		\end{align*}
		\begin{align*}
			I & = \sum_{i = 1}^{4} (mr^2) \\
			I & = 4(\SI{0.200}{\kilogram})(\SI{0.283}{\meter})^2 \\
			I & = \SI{0.0641}{\kilogram \meter \squared} = \SI{6.41e-2}{\kilogram \meter \squared}
		\end{align*}
	\item
		\begin{align*}
			r & = \SI{0.200}{\meter}
		\end{align*}
		\begin{align*}
			I & = \sum_{i = 1}^{4} (mr^2) \\
			I & = 4(\SI{0.200}{\kilogram})(\SI{0.200}{\meter})^2 \\
			I & = \SI{0.320}{\kilogram \meter \squared}
		\end{align*}
	\item
		\begin{align*}
			r & = \SI{0.283}{\meter}
		\end{align*}
		\begin{align*}
			I & = \sum_{i = 0}^{2} (mr^2) \\
			I & = 2(\SI{0.200}{\kilogram})(\SI{0.283}{\meter})^2 \\
			I & = \SI{0.320}{\kilogram \meter \squared}
		\end{align*}
\end{enumerate}

\subsection{9.49}

\begin{align*}
	r & = \SI{0.280}{\meter} \\
	m_{block} & = \SI{4.20}{\kilogram} \\
	v_0 & = 0 \\
	\Delta y & = \SI{3.00}{\meter} \\
	t & = \SI{2.00}{\second} \\
	I_{wheel} & = \frac{1}{2}mr^2 \\
	m_{wheel} & = ?
\end{align*}
\begin{enumerate}[labelindent = 0pt, label = \textbf{Method \arabic*: }]
	\item Using Force Equations + Torque
		\begin{align*}
			\sum F_y^{block} & = m_{block}a \\
			m_{block}g & = T + m_{block}a \\
			T & = m_{block}(g - a)
		\end{align*}
		\begin{align*}
			\sum F^{wheel} & = m_{wheel}\alpha \\
			\tau & = I_{wheel}\alpha \\
			Tr & = \left( \frac{1}{2}m_{wheel}r^2 \right) \left( \frac{a}{r} \right) \\
			T & = \frac{ m_{wheel}a }{ 2 }
		\end{align*}
		\begin{align*}
			a & = \frac{ v_1 - v_0 }{ t_1 - t_0 } \\
			a & = \frac{ \SI{3.00}{\meter \per \second} - 0 }{ \SI{2.00}{\second} - 0 } \\
			a & = \SI{1.50}{\meter \per \second \squared}
		\end{align*}
		\begin{align*}
			m_{block}(g - a) & = \frac{ m_{wheel}a }{ 2 } \\
			m_{wheel} & = \frac{ 2m_{block}(g - a) }{ a } \\
			m_{wheel} & = \frac{ 2(\SI{4.20}{\kilogram})(\SI{10.0}{\meter \per \second \squared} - \SI{1.50}{\meter \per \second \squared}) }{ \SI{1.50}{\meter \per \second \squared} } \\
			m_{wheel} & = \SI{47.6}{\kilogram}
		\end{align*}
	\item Using Energy
		\begin{align*}
			\Delta y & = \frac{1}{2}(v_f + v_o)t \\
			v_f & = \frac{ 2\Delta y }{ t } \\
			v_f & = \frac{ 2(\SI{3.00}{\meter}) }{ \SI{2.00}{\second} } \\
			v_f & = \SI{3.00}{\meter \per \second} \\
			\omega & = \frac{ v_f }{ r } \\
			\omega & = \frac{ \SI{3.00}{\meter \per \second} }{ \SI{0.280}{\meter} } \\
			\omega & = \SI{10.7}{\radian \per \second}
		\end{align*}
		\begin{align*}
			E_{K_0} + E_{P_0} & = E_{K_1} + E_{P_1} \\
			0 + E_{P_{block}} & = E_{R_{wheel}} + E_{K_{block}} \\
			m_{block}g\Delta y & = \frac{1}{2}I_{wheel}\omega_{wheel}^2 + \frac{1}{2}m_{block}v_{block}^2 \\
			m_{block}g\Delta y & = \frac{1}{2}\left( \frac{1}{2}m_{wheel}r^2 \right) \omega_{wheel}^2 + \frac{1}{2}m_{block}v_{block}^2 \\
			m_{block} \left( 4g\Delta y - 2v_{block}^2 \right) & = m_{wheel}r^2\omega_{wheel}^2 \\
			m_{wheel} & = \frac{ m_{block} \left( 4g\Delta y - 2v_{block}^2 \right) }{ r^2\omega_{wheel}^2 } \\
			m_{wheel} & = \frac{ (\SI{4.20}{\kilogram}) \left( 4(\SI{10.0}{\meter \per \second \squared})(\SI{3.00}{\meter}) - 2(\SI{3.00}{\meter \per \second})^2 \right) }{ (\SI{0.280}{\meter})^2(\SI{10.7}{\radian \per \second})^2 } \\
			m_{wheel} & = \SI{47.7}{\kilogram}
		\end{align*}
\end{enumerate}

\subsection{9.79}

\begin{align*}
	r_1 & = \SI{2.55}{\centi \meter} = \SI{0.0255}{\meter} \\
	m_1 & = \SI{0.85}{\kilogram} \\
	r_2 & = \SI{5.02}{\centi \meter} = \SI{0.0502}{\meter} \\
	m_2 & = \SI{1.58}{\kilogram}
\end{align*}
\begin{enumerate}[label = \boldalpha]
	\item
		\begin{align*}
			I & = \sum_{i = 1}^{2} I_i \\
			I & = \frac{1}{2}(m_1r_1^2 + m_2r_2^2) \\
			I & = \frac{1}{2}((\SI{0.85}{\kilogram})(\SI{0.0255}{\meter})^2 + (\SI{1.58}{\kilogram})(\SI{0.0502}{\meter})^2) \\
			I & = \SI{0.00227}{\kilogram \meter \squared} = \SI{2.27e-3}{\kilogram \meter \squared}
		\end{align*}
	\item
		\begin{align*}
			v_{block_0} & = 0 \\
			y_1 & = \SI{2.03}{\meter} \\
			y_0 & = 0 \\
			v_{block_1} & = ?
		\end{align*}
		\begin{align*}
			E_{K_0} + E_{P_0} & = E_{K_1} + E_{P_1} \\
			m_{block}g\Delta y + 0 & = \frac{1}{2}I \left( \frac{v_{block_1}}{r_1} \right)^2 + \frac{1}{2}m_{block}v_{block_1}^2 + 0 \\
			2g\Delta y & = v_{block_1}^2 \left( \frac{ I }{ m_{block}(r_1)^2 } + 1 \right) \\
			v_{block_1} & = \sqrt{ \frac{ 2g\Delta y }{ \frac{ I }{ m_{block}(r_1)^2 } + 1 } } \\
			v_{block_1} & = \sqrt{ \frac{ 2(\SI{10.0}{\meter \per \second \squared})(\SI{2.03}{\meter}) }{ \frac{ \SI{0.00277}{\kilogram \meter \squared} }{ (\SI{1.50}{\kilogram})(\SI{0.0255}{\meter})^2 } + 1 } } \\
			v_{block_1} & = \SI{3.25}{\meter \per \second}
		\end{align*}
	\item
		\begin{align*}
			v_{block_1} & = \sqrt{ \frac{ 2g\Delta y }{ \frac{ I }{ m_{block}(r_2)^2 } + 1 } } \\
			v_{block_1} & = \sqrt{ \frac{ 2(\SI{10.0}{\meter \per \second \squared})(\SI{2.03}{\meter}) }{ \frac{ \SI{0.00277}{\kilogram \meter \squared} }{ (\SI{1.50}{\kilogram})(\SI{0.0502}{\meter})^2 } + 1 } } \\
			v_{block_1} & = \SI{4.84}{\meter \per \second}
		\end{align*}
		The final velocity of the block is greater when the string is wrapped around the higher radius metal disk due to how linear velocity increases with longer lengths. Velocity can be calculated from radius and $ \omega $ using the formula $ v = r\omega $ proving that a greater radius will result in a greater velocity.
\end{enumerate}

\subsection{9.86}

\begin{align*}
	P & = \SI{5e31}{\watt} \\
	T & = \SI{0.0331}{\second} \\
	\Delta t & = \SI{4.22e-13}{\second}
\end{align*}
\begin{enumerate}[label = \boldalpha]
	\item
		\begin{align*}
			E & = \frac{1}{2}I\omega^2 \\
			E & = \frac{1}{2}I\left( \frac{ 4\pi^2 }{ T^2 } \right) \\
			\frac{d}{dt}(E) & = \frac{d}{dt} \left( \frac{ 2I\pi^2 }{ T^2 } \right) \\
			P & = -\frac{ 4I\pi^2\Delta t }{ T^3 } \\
			I & = -\frac{ PT^3 }{ 4\pi^2\Delta t } \\
			I & = -\frac{ (\SI{5e31}{\watt})(\SI{0.0331}{\second})^3 }{ 4\pi^2(\SI{4.22e-13}{\second}) } \\
			I & = \SI{1.09e38}{\kilogram \meter \squared}
		\end{align*}
	\item
		\begin{align*}
			m_{sun} & = \SI{1.9891e30}{\kilogram}
		\end{align*}
		\begin{align*}
			I & = \frac{2}{5}mr^2 \\
			r & = \sqrt{ \frac{ 5I }{ 2m } } \\
			r & = \sqrt {\frac{ 5(\SI{1.09e38}{\kilogram \meter \squared}) }{ 2(1.4)(\SI{1.9891e30}{\kilogram}) } } \\
			r & = \SI{9892.16}{\meter} = \SI{9.89e3}{\meter}
		\end{align*}
	\item
		\begin{align*}
			v & = \frac{ 2\pi r }{ T } \\
			v & = \frac{ 2\pi (\SI{9892.16}{\meter}) }{ \SI{0.0331}{\second} } \\
			v & = \SI{1.88e6}{\meter \per \second} = \SI{1877772}{\meter \per \second} < c = \SI{299792458}{\meter \per \second}
		\end{align*}
	\item
		\begin{align*}
			\rho_{rock} & = \SI{3000}{\kilogram \per \meter \cubed} \\
			\rho_{nucleus} & = \SI{10e17}{\kilogram \per \meter \cubed}
		\end{align*}
		\begin{align*}
			\rho_{\text{neutron star}} & = \frac{ m }{ \frac{ 4\pi r^3 }{ 3 } } \\
			\rho_{\text{neutron star}} & = \frac{ (1.4)(\SI{1.99e30}{\kilogram}) }{ \frac{ 4\pi (\SI{9.89e3}{\meter})^3 }{ 3 } } \\
			\rho_{\text{neutron star}} & = \SI{6.88e17}{\kilogram \per \meter \cubed}
		\end{align*}
		The density of a neutron star is much closer to the density of an atomic nuclear when compared with the density of an ordinary rock.
\end{enumerate}

\section{Lab Manual}

\subsection{1170}

\begin{align*}
	I & = \sum_{i = 1}^{\infty} \left[ \left( \frac{ M }{ n } \right) \left( i \frac{ a }{ n } \right)^2 \right] \\
	I & = \frac{ Ma^2 }{ n^3 } \sum_{i = 1}^{\infty} \left[ i^2 \right] \\
	I & = \frac{ Ma^2 }{ n^3 } \left( \frac{ n(n + 1)(2n + 1) }{ 6 } \right) \\
	I & = \frac{ Ma^2 }{ 6 } \left( \frac{ n(n + 1)(2n + 1) }{ n^3 } \right)
\end{align*}
\begin{align*}
	I & = \lim_{n \to \infty} \left( I \right) \\
	I & = \frac{ Ma^2 }{ 6 } \lim_{n \to \infty} \left[ \left( 1 + \frac{1}{n} \right) \left( 2 + \frac{1}{n} \right) \cdots \right] \\
	I & = \frac{ Ma^2 }{ 6 } (3) \\
	I & = \frac{ Ma^2 }{ 3 }
\end{align*}

\subsection{1173}

\begin{enumerate}[label = \boldalpha]
	\item
		\begin{align*}
			A_{small} & = \pi R^2 \\
			A & = \pi \left( \frac{R}{2} \right)^2 \\
			A & = \frac{ \pi R^2 }{ 4 } \\
			m_{small} & = \frac{m_{disk}}{4}
		\end{align*}
		\begin{align*}
			I_{small} & = \frac{1}{2}m_{small}\left(\frac{R}{2}\right)^2 + m_{small}\left(\frac{R}{2}\right)^2 \\
			I_{small} & = \frac{3}{32}m_{disk}R^2
		\end{align*}
		\begin{align*}
			I_{disk} - I_{small} & = I_{crescent} \\
			\frac{1}{2}m_{disk}R^2 - \frac{3}{32}m_{disk}R^2 & = I_{crescent} \\
			I_{crescent} & = \frac{13}{32}m_{disk}R^2
		\end{align*}
	\item
		\begin{align*}
			I & = I_{disk} - I_{small} \\
			I & = \frac{1}{2}m_{disk}r^2 - \left( \frac{1}{2}m_{small} \left( \frac{r}{2} \right)^2 + m_{small} \left( \frac{r}{2} \right)^2 \right) \\
			I & = \frac{1}{2}\left( \frac{4M}{3} \right)r^2 - \left( \frac{1}{2}\left( \frac{M}{3} \right) \left( \frac{r}{2} \right)^2 + \left( \frac{M}{3} \right) \left( \frac{r}{2} \right)^2 \right) \\
			I & = \frac{13Mr^2}{24}
		\end{align*}
\end{enumerate}

\subsection{1175}

\begin{align*}
	r = r_{1,2} & = \SI{10}{\centi \meter} = \SI{0.100}{\meter} \\
	m = m_{1,2} & = \SI{1000}{\gram} = \SI{1.00}{\kilogram} \\
	l & = \SI{30}{\centi \meter} = \SI{0.300}{\meter}
\end{align*}
\begin{enumerate}[label = \boldalpha]
	\item
		\begin{align*}
			I & = I_1 + I_{rod} + I_2 \\
			I & = \left( \frac{2}{5}mr^2 \right) + 0 + \left( \frac{2}{5}mr^2 + ml^2 \right) \\
			I & = \left( \frac{2}{5}(\SI{1.00}{\kilogram})(\SI{0.100}{\meter})^2 \right) + \left( \frac{2}{5}(\SI{1.00}{\kilogram})(\SI{0.100}{\meter})^2 + (\SI{1.00}{\kilogram})(\SI{0.300}{\meter})^2 \right) \\
			I & = \SI{0.0980}{\kilogram \meter \squared}
		\end{align*}
	\item
		\begin{align*}
			\tau & = \SI{98000}{\dyne \centi \meter} = \SI{0.00980}{\newton \meter}
		\end{align*}
		\begin{align*}
			\tau & = I\alpha \\
			\tau & = I \left( \frac{ \omega }{ \frac{ 2\pi }{ \omega } } \right) \\
			\omega & = \sqrt{ \frac{ 2\tau \pi }{ I } } \\
			\omega & = \sqrt{ \frac{ 2(\SI{0.00980}{\newton \meter})\pi }{ \SI{0.0980}{\kilogram \meter \squared} } } \\
			\omega & = \SI{0.793}{\radian \per \second}
		\end{align*}
\end{enumerate}

\subsection{1177}

\subsection{1181}

\subsection{1283}

\subsection{1284}

\section{Problem C: Spherical Symmetry Problem}

Starting with $ I = \int r^2dm $, calculate the moment of inertial for an axis of rotation that goes through the center of a sphere with uniform mass density $ \rho $, and radius $ R $. As discussed in class, you may treat this problem like the integration of a series of concentric spherical shells with thickness $ dr $.

\end{document}
