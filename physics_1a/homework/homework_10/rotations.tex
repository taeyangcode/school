\documentclass{article}

% Document extensibility %
%
% Disables native paragraph indentation
\usepackage{parskip} 
%
% Provides further bullet options for lists
\usepackage{enumitem}

% Mathematical symbol and statement packages %
%
% Necessary
\usepackage{amsmath}
\usepackage{amssymb}
%
% Extensive fraction notation
\usepackage{xfrac}
%
% Generic mathematical commands
% Notable: \degree, \celcius
\usepackage{gensymb}
%
% Variable vector notation (arrow above variable)
\usepackage{esvect}
%
% Multiline boxed equations
\usepackage{empheq}
%
% SI Unit
\usepackage{siunitx}
\usepackage{physunits}
%
% More intuitive arrays/matrices
\usepackage{array}
%
% Linear Equations
\usepackage{systeme}
%
% Boxes!
\usepackage{mdframed}
%
% Matrix Notation
\usepackage{bm}

% Graphic packages %
%
% Diagrams and illustrations
\usepackage{tikz}
\usetikzlibrary{positioning}
%
% Image insertion
\usepackage{graphicx}

% LaTeX Commands
%
% Argument Parser
\usepackage{xparse}

% Document content %
%
% Change title of table of contents
% \renewcommand{\contentsname}{Title}

\title{Homework 10 Rotations}
\author{Corey Mostero - 2566652}
\date{6 June 2023}

\begin{document}

% Command `\hr` to insert horizontal rules
\newcommand{\hr}{\par\noindent\rule{\textwidth}{0.4pt}}

% Command to box and center math equations
\newcommand{\bc}[1]{
	\begin{equation*}
		\begin{boxed}
			{#1}
		\end{boxed}
	\end{equation*}
}

% Command for single line equations with a condition
\newcommand{\cond}[2]{
	\ifmmode
		{#1} \quad {#2}
	\else
		$$ {#1} \quad {#2} $$
	\fi
}

% Matrix and Vector notation
\newcommand{\matr}[1]{
	\ifmmode \bm{#1}
	\else \textit{\textbf{#1}}
	\fi
}
\newcommand{\vect}[1]{
	\ifmmode \mathbf{#1}
	\else \textbf{#1}
	\fi
}

% Laplace
\NewDocumentCommand{\lap}{o}{
	\IfNoValueTF{#1}
		{ \mathcal{L} }
		{ \mathcal{L} \left\{ {#1} \right\} }
}
\NewDocumentCommand{\ilap}{o}{
	\IfNoValueTF{#1}
		{ \mathcal{L}^{-1} }
		{ \mathcal{L}^{-1} \left\{ {#1} \right\} }
}

\newcommand{\boldalpha}{\textbf{(\alph*)}}

\maketitle
\newpage

\tableofcontents

\section{Book}

\subsection{10.22}

\begin{align*}
	r & = \SI{8.00}{\centi \meter} \\
	m & = \SI{0.180}{\kilogram} \\
	v_0 & = 0 \\
	\Delta y & = \SI{75.0}{\centi \meter} \\
	I & = mr^2
\end{align*}
\begin{enumerate}[label = \boldalpha]
	\item
		\begin{align*}
			E_{K_0} + E_{P_0} & = E_{K_1} + E_{P_1} \\
			0 + mgh & = \frac{1}{2}I_{cm}\omega^2 + \frac{1}{2}mr^2\omega^2 + 0 \\
			mgh & = \omega^2 \left( \frac{1}{2} \left( mr^2 \right) + \frac{1}{2}mr^2 \right) \\
			\omega & = \frac{ \sqrt{ gh } }{ r } \\
			\omega & = \frac{ \sqrt{ (\SI{10.0}{\meter \per \second \squared})(\SI{0.75}{\meter}) } }{ \SI{0.08}{\meter} } \\
			\omega & = \SI{34.2}{\radian \per \second}
		\end{align*}
	\item
		\begin{align*}
			E_{k_0} + E_{p_0} & = E_{k_1} + E_{p_1} \\
			0 + mgh & = \frac{1}{2}I_{cm}\omega^2 + \frac{1}{2}mv_{cm}^2 + 0 \\
			mgh & = \frac{1}{2}(mr^2) \left( \frac{v_{cm}}{r} \right)^2 + \frac{1}{2}mv_{cm}^2 \\
			v & = \sqrt{ gh } \\
			v & = \sqrt{ (\SI{10.0}{\meter \per \second \squared})(\SI{0.75}{\meter}) } \\
			v & = \SI{2.74}{\meter \per \second}
		\end{align*}
\end{enumerate}

\subsection{10.26}

\begin{align*}
	I_{cm} & = \frac{2}{5}mr^2
\end{align*}
\begin{enumerate}[label = \boldalpha]
	\item
		Velocity for the first half of the bowl:
		\begin{align*}
			E_{K_0} + E_{P_0} & = E_{K_1} + E_{P_1} \\
			0 + mgh & = \frac{1}{2}I_{cm}\omega^2 + \frac{1}{2}mv_{cm}^2 + 0 \\
			mgh & = \frac{1}{2} \left( \frac{2}{5}mr^2 \right) \left( \frac{v_{cm}^2}{r^2} \right) + \frac{1}{2}mv_{cm}^2 \\
			v_{cm} & = \sqrt{ \frac{10gh}{7} }
		\end{align*}
		Since the ball only slides and doesn't rotate, the kinetic energy it experiences it purely linear velocity and \textit{not} angular.
		\begin{align*}
			E_{K_0} + E_{P_0} & = E_{K_1} + E_{P_1} \\
			\frac{1}{2}mv_{cm}^2 + 0 & = 0 + mgh_1 \\
			\left( \sqrt{ \frac{10gh_0}{7} } \right)^2 & = 2gh_1 \\
			h_1 & = \frac{5}{7}h_0
		\end{align*}
		The ball reaches only $ \frac{5}{7} $ of the height of the side of the bowl.
\end{enumerate}

\subsection{10.30}

\begin{enumerate}[label = \boldalpha]
	\item Free-body diagram:
		\begin{center}
			\def\angle{25}
			\begin{tikzpicture}
				\node [circle, draw = black, fill = black, inner sep = 2pt] (origin) {};

				\node [below right = 3pt and 5pt of origin] (acceleration) {};
				\draw [-stealth, thick, densely dashed, draw = black!80] (acceleration) -- ++ (1, 0) node [midway, below] {$ a $};

				\draw [-stealth, thick, draw = black] (origin) -- ++ (\angle:-1.5cm) node [midway, below right] {$ f $};

				\draw [-stealth, thick, draw = black] (origin) -- ++ (\angle + 90:1.5cm) node [midway, below left] {$ N $};

				\draw [-stealth, thick, draw = black] (origin) -- ++ (-90:1.5cm) node [midway, right] {$ mg $};

				\node [left = 12pt of origin] (angle_start) {};
				\draw [-stealth, draw = black!60, thick, dashed] (angle_start.center) arc(180:180 + \angle:0.5) node [midway, right, black!60, thick] {$ \beta $};
			\end{tikzpicture}
		\end{center}
\end{enumerate}

\subsection{10.79}

\subsection{9.30}

\subsection{9.49}

\subsection{9.79}

\subsection{9.86}

\section{Lab Manual}

\subsection{1170}

\subsection{1173}

\subsection{1175}

\subsection{1177}

\subsection{1181}

\subsection{1283}

\subsection{1284}

\section{Problem C: Spherical Symmetry Problem}

Starting with $ I = \int r^2dm $, calculate the moment of inertial for an axis of rotation that goes through the center of a sphere with uniform mass density $ \rho $, and radius $ R $. As discussed in class, you may treat this problem like the integration of a series of concentric spherical shells with thickness $ dr $.

\end{document}
