\documentclass{article}

% Document extensibility %
%
% Disables native paragraph indentation
\usepackage{parskip} 
%
% Provides further bullet options for lists
\usepackage{enumitem}

% Mathematical symbol and statement packages %
%
% Necessary
\usepackage{amsmath}
\usepackage{amssymb}
%
% Extensive fraction notation
\usepackage{xfrac}
%
% Generic mathematical commands
% Notable: \degree, \celcius
\usepackage{gensymb}
%
% Variable vector notation (arrow above variable)
\usepackage{esvect}
%
% Multiline boxed equations
\usepackage{empheq}
%
% SI Unit
\usepackage{siunitx}

% Graphic packages %
%
% Diagrams and illustrations
\usepackage{tikz}
%
% Image insertion
\usepackage{graphicx}

% Document content %
%
% Change title of table of contents
\renewcommand{\contentsname}{Force Statics}

\title{Homework 2}
\author{Corey Mostero}
\date{Student ID: 256652}

\begin{document}

% Command `\hr` to insert horizontal rules
\newcommand{\hr}{\par\noindent\rule{\textwidth}{0.4pt}}

% Command to box and center math equations
\newcommand{\bc}[1]{
	\begin{equation*}
		\begin{boxed}
			{#1}
		\end{boxed}
	\end{equation*}
}

\maketitle
\newpage

\tableofcontents

\section{Book}

\subsection{5.2}
\begin{enumerate}[label=\textbf{(\alph*)}]
	\item
		\begin{align*}
			\sum F_y & = 0 \\
			T_\text{wall, b} - w_b & = 0 \\
			T_\text{wall, b} & = w_b
		\end{align*}
		\bc{T_\text{wall, b} = w_b}

	\item
		\begin{align*}
			\sum F_y^{(b_1)} & = 0 \\
			T_{b_2,b_1} - w_{b_1} & = 0 \\
			T_{b_2,b_1} & = w_{b_1} \\
			\sum F_y^{(b_2)} & = 0 \\
			T_{b_1,b_2} - w_{b_2} & = 0 \\
			T_{b_1,b_2} & = w_{b_2} \\
			T_{b_2,b_1} + T_{b_1,b_2} & = w_{b_1} + w_{b_2}
		\end{align*}
		where $$ T_{b_1,b_2} = T_{b_2,b_1} \And w_{b_1} = w_{b_2} $$
		\begin{align*}
			T + T & = w + w \\
			2T & = 2w \\
			T & = w
		\end{align*}
		\bc{T = w}

	\item
		\begin{align*}
			\sum F_y^{(b_1)} & = 0 \\
			T_{b_2,b_1} - w & = 0 \\
			T_{b_2,b_1} & = w \\
			\sum F_y^{(b_2)} & = 0 \\
			T_{b_1,b_2} - w & = 0 \\
			T_{b_1,b_2} & = w
		\end{align*}
		where $$ T_{b_1,b_2} = T_{b_2,b_1} $$
		\begin{align*}
			T + T & = w + w \\
			2T & = 2w \\
			T & = w
		\end{align*}
		\bc{T = w}
\end{enumerate}

\subsection{5.6}
\begin{align*}
	b & = \text{ball} \\
	m & = \SI{3620}{\kilogram} \\
	\theta_{T_B,\hat{y}} & = 40\degree
\end{align*}
\begin{enumerate}[label=\textbf{(\alph*)}]
	\item
		\begin{align*}
			T_B & = ? \\
			\cos(\theta) & = \frac{m_bg}{T_B} \\
			T_B & = \frac{m_bg}{\cos(\theta)} \\
				& = \frac{ \SI{3620}{\kilogram} \cdot \SI{10}{\meter \per \second \squared} }{ \cos(40\degree) } \\
			T_B & = \SI{47255.7}{\newton}
		\end{align*}
		\bc{T_B = \SI{47.3e3}{\newton}}
	\item
		\begin{align*}
			T_A & = ? \\
			\theta_{T_B, \hat{x}} & = ? \\
			\theta_{T_B, \hat{x}} & = 90\degree - \theta_{T_B, \hat{y}} \\
								  & = 90\degree - 40\degree \\
			\theta_{T_B, \hat{x}} & = 50\degree \\
			\cos(\theta_{T_B, \hat{x}}) & = \frac{T_{B_x}}{T_B} \\
			T_{B_x} & = \left( T_B \right) \cos \left( \theta_{T_B, \hat{x}} \right) \\
					& = \left( \SI{47.3e3}{\newton} \right) \cos \left( 50\degree \right) \\
			T_{B_x} & = \SI{30403.9}{\newton} \\
			\sum F_x^{(b)} & = 0 \\
			T_{B_x} - T_A & = 0 \\
			T_A & = T_{B_x} \\
			T_A & = \SI{30403.9}{\newton}
		\end{align*}
		\bc{T_A = \SI{30.4e3}{\newton}}
\end{enumerate}

\subsection{5.62}
\begin{align*}
	T_{r,p_1} & = ? \\
	T_{w,p_1} & = ? \\
	w & = m_wg \\
	T_{p_2,p_1} & = ? \\
	T_{r,p_2} & = ? \\
	\vec{F} & = ?
\end{align*}
Based on the free body diagrams, it can be concluded that
\begin{equation} \label{eq:1}
	T_{r,p_1} = T_{p_2,p_1} = \vec{F}
\end{equation}
as they share a common rope.

Therefore the forces of $ p_1 $ in the $ \hat{y} $ direction can be found as
\begin{align*}
	\sum F_y^{(p_1)} & = 0 \\
	T_{r,p_1} + T_{p_2,p_1} - T_{w,p_1} & = 0 \\
	T_{w,p_1} & = 2T
\end{align*}
Finding $ T_{p_1,w} $ from the free body diagram of the weight
\begin{align*}
	\sum F_y^{(\text{weight})} & = 0 \\
	T_{p_1,w} - w & = 0 \\
	T_{p_1,w} & = w
\end{align*}
In order to withhold Newton's third law, the combined tension of $ T_{r,p_1} $ and $ T_{p_2,p_1} $ must equal $ T_{w,p_1} $ (as shown in Equation \ref{eq:1})
\begin{align*}
	2T & = T_{w,p_1} \\
	   & = w \\
	T & = \frac{w}{2}
\end{align*}
It can therefore be concluded that (according to \textit{(1)}) $ \vec{F} $ must equal $ T $, finding the magnitude in terms of $ w $
\bc{\vec{F} = T = \frac{w}{2}}

\subsection{5.64}

\subsection{5.79}

\section{Lab Manuel}

\subsection{270}

\subsection{273}

\subsection{274}

\subsection{287}

\subsection{290}

\end{document}
