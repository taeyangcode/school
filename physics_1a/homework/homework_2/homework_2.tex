\documentclass{article}

% Document extensibility %
%
% Disables native paragraph indentation
\usepackage{parskip} 
%
% Provides further bullet options for lists
\usepackage{enumitem}

% Mathematical symbol and statement packages %
%
% Necessary
\usepackage{amsmath}
\usepackage{amssymb}
%
% Extensive fraction notation
\usepackage{xfrac}
%
% Generic mathematical commands
% Notable: \degree, \celcius
\usepackage{gensymb}
%
% Variable vector notation (arrow above variable)
\usepackage{esvect}
%
% Multiline boxed equations
\usepackage{empheq}
%
% SI Unit
\usepackage{siunitx}

% Graphic packages %
%
% Diagrams and illustrations
\usepackage{tikz}
%
% Image insertion
\usepackage{graphicx}

% Document content %
%
% Change title of table of contents
\renewcommand{\contentsname}{Force Statics}

\title{Homework 2}
\author{Corey Mostero}
\date{Student ID: 256652}

\begin{document}

% Command `\hr` to insert horizontal rules
\newcommand{\hr}{\par\noindent\rule{\textwidth}{0.4pt}}

% Command to box and center math equations
\newcommand{\bc}[1]{
	\begin{equation*}
		\begin{boxed}
			{#1}
		\end{boxed}
	\end{equation*}
}

\maketitle
\newpage

\tableofcontents

\section{Book}

\subsection{5.2}
\begin{enumerate}[label=\textbf{(\alph*)}]
	\item
		\begin{align*}
			\sum F_y & = 0 \\
			T_\text{wall, b} - w_b & = 0 \\
			T_\text{wall, b} & = w_b
		\end{align*}
		\bc{T_\text{wall, b} = w_b}

	\item
		\begin{align*}
			\sum F_y^{(b_1)} & = 0 \\
			T_{b_2,b_1} - w_{b_1} & = 0 \\
			T_{b_2,b_1} & = w_{b_1} \\
			\sum F_y^{(b_2)} & = 0 \\
			T_{b_1,b_2} - w_{b_2} & = 0 \\
			T_{b_1,b_2} & = w_{b_2} \\
			T_{b_2,b_1} + T_{b_1,b_2} & = w_{b_1} + w_{b_2}
		\end{align*}
		where $$ T_{b_1,b_2} = T_{b_2,b_1} \And w_{b_1} = w_{b_2} $$
		\begin{align*}
			T + T & = w + w \\
			2T & = 2w \\
			T & = w
		\end{align*}
		\bc{T = w}

	\item
		\begin{align*}
			\sum F_y^{(b_1)} & = 0 \\
			T_{b_2,b_1} - w & = 0 \\
			T_{b_2,b_1} & = w \\
			\sum F_y^{(b_2)} & = 0 \\
			T_{b_1,b_2} - w & = 0 \\
			T_{b_1,b_2} & = w
		\end{align*}
		where $$ T_{b_1,b_2} = T_{b_2,b_1} $$
		\begin{align*}
			T + T & = w + w \\
			2T & = 2w \\
			T & = w
		\end{align*}
		\bc{T = w}
\end{enumerate}

\subsection{5.6}
\begin{align*}
	b & = \text{ball} \\
	m & = \SI{3620}{\kilogram} \\
	\theta_{T_B,\hat{y}} & = 40\degree
\end{align*}
\begin{enumerate}[label=\textbf{(\alph*)}]
	\item
		\begin{align*}
			T_B & = ? \\
			\cos(\theta) & = \frac{m_bg}{T_B} \\
			T_B & = \frac{m_bg}{\cos(\theta)} \\
				& = \frac{ \SI{3620}{\kilogram} \cdot \SI{10}{\meter \per \second \squared} }{ \cos(40\degree) } \\
			T_B & = \SI{47255.7}{\newton}
		\end{align*}
		\bc{T_B = \SI{47.3e3}{\newton}}
	\item
		\begin{align*}
			T_A & = ? \\
			\theta_{T_B, \hat{x}} & = ? \\
			\theta_{T_B, \hat{x}} & = 90\degree - \theta_{T_B, \hat{y}} \\
								  & = 90\degree - 40\degree \\
			\theta_{T_B, \hat{x}} & = 50\degree \\
			\cos(\theta_{T_B, \hat{x}}) & = \frac{T_{B_x}}{T_B} \\
			T_{B_x} & = \left( T_B \right) \cos \left( \theta_{T_B, \hat{x}} \right) \\
					& = \left( \SI{47.3e3}{\newton} \right) \cos \left( 50\degree \right) \\
			T_{B_x} & = \SI{30403.9}{\newton} \\
			\sum F_x^{(b)} & = 0 \\
			T_{B_x} - T_A & = 0 \\
			T_A & = T_{B_x} \\
			T_A & = \SI{30403.9}{\newton}
		\end{align*}
		\bc{T_A = \SI{30.4e3}{\newton}}
\end{enumerate}

\subsection{5.62}
\begin{align*}
	T_{r,p_1} & = ? \\
	T_{w,p_1} & = ? \\
	w & = m_wg \\
	T_{p_2,p_1} & = ? \\
	T_{r,p_2} & = ? \\
	\vec{F} & = ?
\end{align*}
Based on the free body diagrams, it can be concluded that
\begin{equation} \label{eq:1}
	T_{r,p_1} = T_{p_2,p_1} = \vec{F}
\end{equation}
as they share a common rope.

Therefore the forces of $ p_1 $ in the $ \hat{y} $ direction can be found as
\begin{align*}
	\sum F_y^{(p_1)} & = 0 \\
	T_{r,p_1} + T_{p_2,p_1} - T_{w,p_1} & = 0 \\
	T_{w,p_1} & = 2T
\end{align*}
Finding $ T_{p_1,w} $ from the free body diagram of the weight
\begin{align*}
	\sum F_y^{(\text{weight})} & = 0 \\
	T_{p_1,w} - w & = 0 \\
	T_{p_1,w} & = w
\end{align*}
In order to withhold Newton's third law, the combined tension of $ T_{r,p_1} $ and $ T_{p_2,p_1} $ must equal $ T_{w,p_1} $ (as shown in Equation \ref{eq:1})
\begin{align*}
	2T & = T_{w,p_1} \\
	   & = w \\
	T & = \frac{w}{2}
\end{align*}
It can therefore be concluded that (according to \textit{(1)}) $ \vec{F} $ must equal $ T $, finding the magnitude in terms of $ w $
\bc{\vec{F} = T = \frac{w}{2}}

\subsection{5.64}
\begin{enumerate}[label=\textbf{(\alph*)}]
	\item
	\item
		\begin{align*}
			m_\text{ball} & = ? \\
			\theta_{\hat{x},\text{ramp}} & = 35.0\degree \\
			T_{\text{ramp},\text{ball}} & = ?
		\end{align*}
		Determine the normal force
		\begin{align*}
			\cos(\theta) & = \frac{N_{\text{ball}y}}{N_\text{ball}} \\
			N_\text{ball} & = \frac{N_{\text{ball}y}}{\cos(\theta)} \\
			\text{as well as: } N_{\text{ball}y} & = N_\text{ball}\cos(\theta)
		\end{align*}
		To find $ N_{\text{ball}y} $, utilize the forces in the $ \hat{y} $ direction
		\begin{align*}
			\sum F_y^\text{(ball)} & = 0 \\
			N_{\text{ball}y} - m_\text{ball}g & = 0 \\
			N_{\text{ball}y} & = m_\text{ball}g \\
			N_\text{ball}\cos(\theta) & = m_\text{ball}g \\
			N_\text{ball} & = \frac{m_\text{ball}g}{\cos(\theta)} \\
						  & = \frac{m_\text{ball}\SI{10}{\meter \per \second \squared}}{\cos(35.0\degree)} \\
			N_\text{ball} & = (m_\text{ball})(\SI{12.2}{\meter \per \second \squared})
		\end{align*}
		\bc{N_\text{ball} = (m_\text{ball})(\SI{12.2}{\meter \per \second \squared})}
	\item
		Finding the tension in the wire requires finding the forces in $ \hat{x} $ direction
		\begin{align*}
			\sum F_x^\text{(ball)} & = 0 \\
			T_{\text{ramp},\text{ball}} - N_{\text{ball}x} & = 0
		\end{align*}
		Finding $ N_{\text{ball}x} $
		\begin{align*}
			\sin(\theta) & = \frac{N_{\text{ball}x}}{N_\text{ball}} \\
			N_{\text{ball}x} & = N_\text{ball}\sin(\theta)
		\end{align*}
		And using the value in the force equation above
		\begin{align*}
			T_{\text{ramp},\text{ball}} & = N_\text{ball}\sin(\theta) \\
										& = (m_\text{ball})(\SI{12.2}{\meter \per \second \squared})\sin(35.0\degree) \\
			T_{\text{ramp},\text{ball}} & = (m_\text{ball})(\SI{7.00}{\meter \per \second \squared})
		\end{align*}
		\bc{T_{\text{ramp},\text{ball}} = (m_\text{ball})(\SI{7.00}{\meter \per \second \squared})}
\end{enumerate}

\subsection{5.79}
\begin{enumerate}[label=\textbf{(\alph*)}]
	\item
		\begin{align*}
			N_A & = ? \\
			N_B & = ? \\
			N_A & = N_B \text{, Newton's Third Law} \\
			N & = ? \\
			m_Ag & = \SI{1.20}{\newton} \\
			m_Bg & = \SI{3.60}{\newton} \\
			\mu_k & = 0.300 \\
			f & = \mu_kN \\
			\vec{F} & = ?
		\end{align*}
		In order to find $ \vec{F} $, the normal force is needed which can be found by observing the forces in the $ \hat{y} $ direction
		\begin{align*}
			\sum \vec{F}_{\hat{y}}^\text{(A)} & = 0 \\
			N_A - m_Ag & = 0 \\
			N_A & = m_Ag
		\end{align*}
		\begin{align*}
			\sum \vec{F}_{\hat{y}}^\text{(B)} & = 0 \\
			N - m_Bg - N_B & = ? \\
			N & = m_Bg + N_B \\
			  & = m_Bg + m_Ag \\
			  & = \SI{1.20}{\newton} + \SI{3.60}{\newton} \\
			N & = \SI{4.80}{\newton}
		\end{align*}
		$$ f = \mu_kN = (0.300)(\SI{4.80}{\newton}) = \SI{1.44}{\newton} $$
		Now finding $ \vec{F} $
		\begin{align*}
			\sum \vec{F}_{\hat{x}}^{(B)} & = 0 \\
			-\vec{F} + f & = 0 \\
			\vec{F} & = f \\
					& = \SI{1.44}{\newton} \\
			\vec{F} & = \SI{1.44}{\newton}
		\end{align*}
		\bc{\vec{F} = \SI{1.44}{\newton}}
	\item
		\begin{align*}
			N_A & = ? \\
			N_B & = ? \\
			N_A & = N_B \\
			f_A & = ? \\
			f_B & = ? \\
			f_A & = f_B \text{, Newton's Third Law} \\
			T_{\text{wall},A} & = ? \\
			m_Ag & = \SI{1.20}{\newton} \\
			N & = ? \\
			m_Bg & = \SI{3.60}{\newton} \\
			\vec{F} & = ?
		\end{align*}
		First determine the forces in the $ \hat{x} $ direction of block $ A $ to find tension
		\begin{align*}
			\sum \vec{F}_{\hat{x}}^{(A)} & = 0 \\
			T_{\text{wall},A} - f_A & = 0 \\
			f_A & = T_{\text{wall},A}
		\end{align*}
		Similarly to part \textbf{(a)}, find $ N_A $ using the $ \hat{y} $ forces of block $ A $
		\begin{align*}
			\sum \vec{F}_{\hat{y}}^{(A)} & = 0 \\
			N_A - m_Ag & = 0 \\
			N_A & = m_Ag \\
			N_A & = \SI{1.20}{\newton}
		\end{align*}
		The friction of block $ A $ upon block $ B $ can now be calculated, and further utilized through $ f_B $ due to Newton's Third Law
		$$ f_A = \mu_k N_A = (0.300)(\SI{1.20}{\newton}) = \SI{0.360}{\newton} $$
		Find $ N $ to aid in finding the friction between the ground and block $ B $
		\begin{align*}
			\sum \vec{F}_{\hat{y}}^{(B)} & = 0 \\
			N - m_Bg - N_B & = 0 \\
			N & = m_Bg + N_B \\
			  & = \SI{3.60}{\newton} + \SI{1.20}{\newton} \\
			N & = \SI{4.80}{\newton}
		\end{align*}
		Solve for the forces in the $ \hat{x} $ direction of block $ B $ to finally compute the pulling force
		\begin{align*}
			\sum \vec{F}_{\hat{x}}^{(B)} & = 0 \\
			f + f_B - \vec{F} & = 0 \\
			\vec{F} & = f + f_B \\
					& = \mu_k N + \SI{0.360}{\newton} \\
					& = (0.300)(\SI{4.80}{\newton}) + \SI{0.360}{\newton} \\
			\vec{F} & = \SI{1.80}{\newton}
		\end{align*}
		\bc{\vec{F} = \SI{1.80}{\newton}}
\end{enumerate}

\section{Lab Manual}

\subsection{270}
\begin{enumerate}[label=\textbf{(\alph*)}]
	\item
		\begin{align*}
			r_A & = ? \\
			r_B & = ? \\
			r_C & = ? \\
			r_A & = r_B = r_C \\
			R & = ? \\
			\theta & = ?
		\end{align*}
		The center of mass of all three identical cylinders created $ 60\degree $ angles at each corner. This makes the angle of the top corner of $ A $, $ \frac{60\degree}{2} = 30\degree $ when focusing on only one half of the illustration.
		\begin{align*}
			\cos(30\degree) & = \frac{m_Ag}{2r_A} \\
			r_A & = \frac{m_Ag}{2 \cos(30\degree)} \\
			r_A & = \frac{m_Ag}{\sqrt{3}} \\
			m_Ag & = \sqrt{3}r_A
		\end{align*}
		It can be observed that the opposite side of both $ \theta $ and the angle between $ m_Ag $ and $ 2r_A $ are the same
		\begin{align*}
			\sin(\theta) & = \frac{ \frac{r_B}{2} }{R} \\
			\sin(30\degree) & = \frac{ \frac{r_B}{2} }{r_A} \\
			r_A \sin(30\degree) & = R \sin(\theta) \\
			\frac{m_Ag}{\sqrt{3}} \sin(30\degree) & = R \sin(\theta) \\
			R \sin(\theta) & = \frac{m_Ag}{2\sqrt{3}}
		\end{align*}
		Half of the opposite side of $ \theta $ has been found, now the adjacent is required. The adjacent can be observed as the combined forces in the $ \hat{y} $ direction of the illustration
		\begin{align*}
			\sum F_y & = \cos(30\degree)(N_{B,A} - N_{A,B} + N_{C,A} - N_{A,C}) - m_Ag - m_Bg - m_Cg \\
			\sum F_y & = 3mg \text{, divide by 2 to get the half's weight} \\
			\sum F_y & = \frac{3mg}{2}
		\end{align*}
		\begin{align*}
			\cos(\theta) & = \frac{F_y}{R} \\
						 & = \frac{ \frac{3mg}{2} }{R} \\
			R\cos(\theta) & = \frac{3mg}{2}
		\end{align*}
		Therefore solving for $ \tan(\theta) $
		\begin{align*}
			\tan(\theta) & = \frac{\sin(\theta)}{\cos(\theta)} \\
						 & = \frac{ \frac{mg}{2R\sqrt{3}} }{ \frac{3mg}{2R} } \\
						 & = \frac{(mg)(2R)}{(2R\sqrt{3})(3mg)} \\
			\tan(\theta) & = \frac{1}{3\sqrt{3}}
		\end{align*}
		\bc{\tan(\theta) = \frac{1}{3\sqrt{3}}}
	\item
		Part (b) is otherwise saying that the ratio of $ R $ to $ r $ must be
		$$ R : r(1 + 2\sqrt{7}) $$
		When observing the triangle
		\begin{align*}
			\sin(\theta) & = \frac{\text{opp.}_\text{small}}{r} \\
			\text{opp.}_\text{small} & = r\sin(\theta)
		\end{align*}
		Both the hypotenuse and opposite side have been found for the smaller triangle, which when compared with the triangle made with radius $ R $, the opposite must also be calculated to find the ratio. The opposite side of the larger triangle can be seen to be the sum of the opposite of the smaller triangle, as well as the remaining distance. This remaining distance can be seen as $ r $, due to the remaining length being equivalent to its parallel side (quadrilateral).
		\begin{align*}
			\sin(\theta) & = \frac{r\sin(\theta) + r}{R}
		\end{align*}
		Now the ratio can be found
		\begin{align*}
			\frac{R}{r} & = \frac{r\sin(\theta) + r}{r\sin(\theta)} \\
			\frac{R}{r} & = 1 + \frac{1}{\sin(\theta)} \\
			R & = r \left( 1 + \frac{1}{\sin(\theta)} \right)
		\end{align*}
		Using the previous part to find $ \theta $
		\begin{align*}
			\tan(\theta) & = \frac{1}{3\sqrt{3}} \\
			\theta & = \arctan \left( \frac{1}{3\sqrt{3}} \right) \\
			R & = r \left( 1 + \frac{1}{\sin \left( \arctan \left( \frac{1}{3\sqrt{3}} \right) \right) } \right) \\
			R & = r \left( 1 + 2\sqrt{7} \right)
		\end{align*}
		\bc{R = r \left( 1 + 2\sqrt{7} \right)}
\end{enumerate}

\subsection{273}
\begin{align*}
	\sum F_y & = 0 \\
	dN\sin(\theta) - dw & = 0 \\
	dN\sin(\theta) & = dw
\end{align*}
\begin{align*}
	\sum F_x & = 0 \\
	dN\cos(\theta) - T & = 0 \\
	T & = dN\cos(\theta)
\end{align*}
\begin{align*}
	\frac{dN\cos(\theta)}{dN\sin(\theta)} & = \frac{T}{dW} \\
	T & = \cot(\theta)dW
\end{align*}
\begin{align*}
	& \int_0^{2\pi} T d\phi \\
	& = 2\pi T
\end{align*}
\begin{align*}
	& \int_0^w \frac{gh}{r} dw \\
	& = \frac{wgh}{r}
\end{align*}
\begin{align*}
	2\pi T & = \frac{wgh}{r} \\
	T & = \frac{wgh}{2\pi r}
\end{align*}
\bc{T = \frac{wgh}{2\pi r}}

\subsection{274}
*Each for only a vertical half of the triangle
\begin{align*}
	f_x & = f\cos(\theta) \\
	f_y & = f\sin(\theta) \\
	P_x & = P\sin(\theta) \\
	P_y & = P\sin(\theta) \\
	f & = \mu P
\end{align*}
\begin{enumerate}[label=\textbf{(\alph*)}]
	\item
		\begin{align*}
			\sum f_y & = 0 \\
			2f\cos(\theta) - 2P\sin(\theta) - w & = 0 \\
			2f\cos(\theta) - 2P\sin(\theta) & = w \\
			2\mu P\cos(\theta) - 2P\sin(\theta) & = w \\
			P \left( 2 ( \mu \cos(\theta) - \sin(\theta) ) \right) & = w \\
			P & = \frac{w}{2 ( \mu \cos(\theta) - \sin(\theta) )}
		\end{align*}
		\bc{P = \frac{w}{2(\mu \cos(\theta) - \sin(\theta))}}
	\item
		\begin{align*}
			\mu & = 0.5 \\
			2\theta & = 40\degree \\
			\theta & = 20\degree \\
			P & = \frac{w}{2( \mu \cos(\theta) - \sin(\theta))} \\
			  & = \frac{w}{2( 0.5 \cos(20\degree) - \sin(20\degree) )} \\
			P & = 3.91w
		\end{align*}
		\bc{P = 3.91w}
	\item
		Using the $ \hat{y} $ forces and ignoring insignificant $ w $
		\begin{align*}
			2\mu P\cos(\theta) - 2P\sin(\theta) & = 0 \\
			2(0.5) P\cos(\theta) & = 2P\sin(\theta) \\
			P\cos(\theta) & = 2P\sin(\theta) \\
			\tan(\theta) & = \sfrac{1}{2} \\
			\theta & = \arctan(\sfrac{1}{2}) \\
			\theta & = 26.6\degree
		\end{align*}
		\bc{2\theta = 53.2\degree}
\end{enumerate}

\subsection{287}
\begin{enumerate}[label=\textbf{(\alph*)}]
	\item
		\begin{align*}
			\sum F_y & = 0 \\
			T\sin(\theta) - \frac{m_\text{chain}g}{2} & = 0 \\
			T\sin(\theta) & = \frac{m_\text{chain}g}{2} \\
			T & = \frac{m_\text{chain}g}{2\sin(\theta)}
		\end{align*}
		\bc{T = \frac{m_\text{chain}g}{2\sin(\theta)}}
	\item
		\begin{align*}
			\sum F_x & = 0 \\
			T_\text{low} - T\cos(\theta) & = 0 \\
			T_\text{low} & = T\cos(\theta) \\
						 & = \frac{m_\text{chain}g}{2\sin(\theta)}\cos(\theta) \\
			T_\text{low} & = \frac{m_\text{chain}{g}\cot(\theta)}{2}
		\end{align*}
		\bc{T_\text{low} = \frac{m_\text{chain}{g}\cot(\theta)}{2}}
\end{enumerate}

\subsection{290}
\begin{enumerate}[label=\textbf{(\alph*)}]
	\item
		\begin{align*}
			\sum F_x & = 0 \\
			T_2\cos(50\degree) - T_1\cos(40\degree) & = 0 \\
			T_2\cos(50\degree) & = T_1\cos(40\degree) \\
			T_2 & = \frac{T_1\cos(40\degree)}{\cos(50\degree)} \\
		\end{align*}
		\begin{align*}
			\sum F_y & = 0 \\
			T_1\sin(40\degree) + T_2\sin(50\degree) - m_\text{ball}g & = 0 \\
			T_1\sin(40\degree) + \frac{T_1\cos(40\degree)}{\cos(50\degree)}\sin(50\degree) & = m_\text{ball}g \\
			T_1 & = \frac{\SI{5.0}{\kilogram} \cdot \SI{10}{\meter \per \second \squared}}{\sin(40\degree) + \frac{cos(40\degree)}{\cos(50\degree)}\sin(50\degree)} \\
			T_1 & = \SI{32.1}{\newton}
		\end{align*}
		\begin{align*}
			T_2 & = \frac{T_1\cos(40\degree)}{\cos(50\degree)} \\
				& = \frac{\SI{32.1}{\newton}\cos(40\degree)}{\cos(50\degree)} \\
			T_2 & = \SI{38.3}{\newton}
		\end{align*}
		\bc{T_1 = \SI{32.1}{\newton}, T_2 = \SI{38.3}{\newton}}
	\item
		\begin{align*}
			\sum F_y & = 0 \\
			T_1\sin(60\degree) - m_\text{ball}g & = 0 \\
			T_1\sin(60\degree) & = m_\text{ball}g \\
			T_1 & = \frac{m_\text{ball}g}{\sin(60\degree)} \\
				& = \frac{\SI{10}{\kilogram} \cdot \SI{10}{\meter \per \second \squared}}{\sin(60\degree)} \\
			T_1 & = \SI{115.47}{\newton}
		\end{align*}
		\begin{align*}
			\sum F_x & = 0 \\
			T_2 - T_1\cos(60\degree) & = 0 \\
			T_2 & = T_1\cos(60\degree) \\
				& = \SI{115.47}{\newton}\cos(60\degree) \\
			T_2 & = \SI{57.7}{\newton}
		\end{align*}
		\bc{T_1 = \SI{11.5e1}{\newton},T_2 = \SI{57.7}{\newton}}
\end{enumerate}

\end{document}
