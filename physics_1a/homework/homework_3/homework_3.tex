\documentclass{article}

% Document extensibility %
%
% Disables native paragraph indentation
\usepackage{parskip} 
%
% Provides further bullet options for lists
\usepackage{enumitem}

% Mathematical symbol and statement packages %
%
% Necessary
\usepackage{amsmath}
\usepackage{amssymb}
%
% Extensive fraction notation
\usepackage{xfrac}
%
% Generic mathematical commands
% Notable: \degree, \celcius
\usepackage{gensymb}
%
% Variable vector notation (arrow above variable)
\usepackage{esvect}
%
% Multiline boxed equations
\usepackage{empheq}
%
% SI Unit
\usepackage{siunitx}

% Graphic packages %
%
% Diagrams and illustrations
\usepackage{tikz}
%
% Image insertion
\usepackage{graphicx}

% Document content %
%
% Change title of table of contents
% \renewcommand{\contentsname}{Title}

\renewcommand{\contentsname}{Torque Statics}

\title{Homework 3}
\author{Corey Mostero}
\date{Student ID: 256652}

\begin{document}

% Command `\hr` to insert horizontal rules
\newcommand{\hr}{\par\noindent\rule{\textwidth}{0.4pt}}

% Command to box and center math equations
\newcommand{\bc}[1]{
	\begin{equation*}
		\begin{boxed}
			{#1}
		\end{boxed}
	\end{equation*}
}

% Command for single line equations with a condition
\newcommand{\cond}[2]{
	\ifmmode
		{#1} \quad {#2}
	\else
		$$ {#1} \quad {#2} $$
	\fi
}

\maketitle
\newpage

\tableofcontents

\section{Book}

\subsection{11.14}
\begin{enumerate}[label=\textbf{(\alph*)}]
	\item
	\item
		\begin{align*}
			l_\text{b} & = \SI{9.00}{\meter} \\
			w_\text{b} & = \SI{300}{\newton} \\
			x_\text{B,A} & = \SI{5.00}{\meter} \\
			w_\text{p} & = \SI{600}{\newton}
		\end{align*}
		\begin{align*}
			\sum \tau_\star & = 0 \\
			(F_A)(\SI{5}{\meter}) & = (w_\text{p})(\SI{5}{\meter} - x) + (w_\text{b})(\SI{2.5}{\meter}) \\
			F_A & = \left( \SI{0.2}{\per \meter} \right) \left( (\SI{600}{\newton})(\SI{5}{\meter} - x) + (\SI{300}{\newton})(\SI{2.5}{\meter}) \right) \\
			(0) & = \SI{600}{\newton} - (\SI{120}{\newton \per \meter})x + \SI{150}{\newton} \\
			(\SI{120}{\newton \per \meter})x & = \SI{750}{\newton} \\
			x & = \SI{6.25}{\meter}
		\end{align*}
		$$ x = \SI{6.25}{\meter} - x_\text{B,A} = \SI{1.25}{\meter} $$
		\bc{\SI{1.25}{\meter}}
	\item
		\begin{align*}
			x_\text{p} & = \SI{7.00}{\meter} \\
			w_\text{p} & = \SI{600}{\newton} \\
			x_\text{b} & = \SI{2.5}{\meter} \\
			w_\text{b} & = \SI{300}{\newton} \\
			x_\text{B} & = ? \\
			F_\text{B} & = \SI{900}{\newton} \\
			x_\text{A} & = 0 \\
			F_\text{A} & = 0
		\end{align*}
		\begin{align*}
			\sum \tau_\star & = 0 \\
			(w_\text{b})(x_\text{b}) + (w_\text{p})(x_\text{p}) & = (F_\text{b})(x_\text{b}) \\
			x_\text{b} & = \frac{(\SI{300}{\newton})(\SI{2.5}{\meter}) + (\SI{600}{\newton})(\SI{7.00}{\meter})}{\SI{900}{\newton})} \\
			x_\text{b} & = \SI{1.5}{\meter}
		\end{align*}
		\bc{x_\text{b} = \SI{1.5}{\meter}}
\end{enumerate}

\subsection{11.16}
\begin{align*}
	l_\text{(b)eam} & = \SI{4.00}{\meter} \\
	l_\text{(c)able} & = \SI{5.00}{\meter} \\
	l_\text{(w)all} & = \SI{3.00}{\meter} \\
	w_\text{b} & = \SI{190}{\newton} \\
	w_\text{(o)bject} & = \SI{300}{\newton} \\
	\theta_\text{c,b} & = \SI{36.87}{\degree}
\end{align*}
\begin{enumerate}[label=\textbf{(\alph*)}]
	\item
		\begin{align*}
			T & = ? \\
			T_y & = T\sin(\theta_\text{c,b}) \\
			\sum \tau_\star & = 0 \\
			\left( \frac{l_\text{b}}{2} \right)(w_\text{b}) + (l_\text{b})(w_\text{o}) & = (l_\text{b})(T_y) \\
			T & = \frac{ \left( \frac{\SI{4.00}{\meter}}{2} \right)(\SI{190}{\newton}) + (\SI{4.00}{\meter})(\SI{300}{\newton}) }{(\SI{4.00}{\meter})(\sin(\SI{36.87}{\degree}))} \\
			T & = \SI{658.3}{\newton}
		\end{align*}
		\bc{T = \SI{658.3}{\newton}}
	\item
		\begin{align*}
			F_x & = ? \\
			\sum F_x & = 0 \\
			F_x & = T_x \\
				& = T\cos(\theta_\text{c,b}) \\
				& = (\SI{658.3}{\newton})(\cos(\SI{36.87}{\degree}) \\
			F_x & = \SI{526.6}{\newton}
		\end{align*}
		\begin{align*}
			F_y & = ? \\
			\sum F_y & = 0 \\
			F_y + T_y & = w_\text{b} + w_\text{o} \\
			F_y + (\SI{658.3}{\newton})(\sin(\SI{36.87}{\degree})) & = \SI{190}{\newton} + \SI{300}{\newton} \\
			F_y & = \SI{190}{\newton} + \SI{300}{\newton} - (\SI{658.3}{\newton})(\sin(\SI{36.87}{\degree}) \\
			F_y & = \SI{95.02}{\newton}
		\end{align*}
		\bc{F_x = \SI{526.6}{\newton}, F_y = \SI{95.02}{\newton}}
\end{enumerate}

\subsection{11.23}
\begin{align*}
	F_1 & = F_2 = \SI{6.30}{\newton} \\
	l_{F_1,O} & = \SI{3.00}{\meter}
\end{align*}
\begin{enumerate}[label=\textbf{(\alph*)}]
	\item
		\begin{align*}
			l & = ? \\
			\sum \tau_\star & = \SI{6.50}{\newton \meter} \\
			(F_2)(l_{F_1,0} + l) & = \SI{6.50}{\newton \meter} + (F_1)(l_{F_1,O}) \\
			(\SI{6.30}{\newton})(\SI{3.00}{\meter} + l) & = \SI{6.50}{\newton \meter} + (\SI{6.30}{\newton})(\SI{3.00}{\newton}) \\
			l & = \SI{1.032}{\meter}
		\end{align*}
		\bc{l = \SI{1.032}{\meter}}
	\item
		\bc{\text{clockwise}}
	\item
		\begin{align*}
			l & = ? \\
			F_2 & = 0 \\
			\sum \tau_\star & = (\SI{6.50}{\newton \meter})(\SI{3.00}{\meter} + l) \\
			-(F_1)(l) & = (\SI{6.50}{\newton \meter}) \\
			-(\SI{6.30}{\newton})(l) & = (\SI{6.50}{\newton \meter}) \\
			l & = \SI{-1.032}{\meter}
		\end{align*}
		\bc{l = \SI{-1.032}{\meter}}
\end{enumerate}

\subsection{11.45}
\begin{align*}
	h & = \SI{0.300}{\meter} \\
	x & = \SI{0.080}{\meter} \\
	\theta & = \SI{60}{\degree} \\
	F_1 & = ? \\
	F_2 & = ?
\end{align*}
\begin{align*}
	\sum \tau_\star & = 0 \\
	(F_2)(h) - ({F_1}_y)(x) & = 0 \\
	(F_2)(\SI{0.300}{\meter}) & = (F_1\sin(\SI{60}{\degree}))(\SI{0.080}{\meter}) \\
	F_1 & = F_2(4.330)
\end{align*}
\bc{F_1 = F_2(4.330)}

\subsection{11.49}
\begin{align*}
	\theta & = \SI{25.0}{\degree} \\
	\phi & = \SI{35.0}{\degree} \\
	l_\text{cog} & = \SI{1.1}{\meter} \\
	w_{\text{p}} & = \SI{82.0}{\kilogram} \\
	l_\text{(h)ands} & = \SI{1.40}{\meter} \\
	l_\text{(p)erson} & = \SI{1.90}{\meter}
\end{align*}
\begin{enumerate}[label=\textbf{(\alph*)}]
	\item
		\begin{align*}
			T & = ? \\
			\cos(\phi) & = \frac{l_{\text{cog}_x}}{l_\text{cog}} \\
			l_{\text{cog}_x} & = (\SI{1.1}{\meter})(\cos(\SI{35.0}{\degree})) \\
			l_{\text{cog}_x} & = \SI{0.9011}{\meter} \\
			\psi & = \SI{90}{\degree} - \phi \\
				 & = \SI{90}{\degree} - \SI{35.0}{\degree} \\
			\psi & = \SI{55.0}{\degree} \\
			\sin(\psi) & = \frac{l_{\text{h}_x}}{l_\text{h}} \\
			l_{\text{h}_x} & = (\SI{1.40}{\meter})(\sin(\SI{55.0}{\degree})) \\
			l_{\text{h}_x} & = \SI{1.147}{\meter} \\
			\cos(\theta) & = \frac{T_y}{T} \\
			T_y & = T\cos(\theta) \\
			\sum \tau_\star & = 0 \\
			(w_\text{p})(l_{\text{cog}_x}) & = (l_{h_x})(T\cos(\theta)) \\
			T & = \frac{(\SI{82.0}{\kilogram})(\SI{10}{\meter \per \second \squared})(\SI{0.9011}{\meter})}{(\SI{1.147}{\meter})(\cos(\SI{25.0}{\degree}))} \\
			T & = \SI{710.1}{\newton}
		\end{align*}
\end{enumerate}

\subsection{11.53}

\subsection{11.71}

\subsection{11.75}

\subsection{11.81}

\section{Lab Manual}

\subsection{370}

\subsection{372}

\end{document}
