\documentclass{article}

% Document extensibility %
%
% Disables native paragraph indentation
\usepackage{parskip} 
%
% Provides further bullet options for lists
\usepackage{enumitem}

% Mathematical symbol and statement packages %
%
% Necessary
\usepackage{amsmath}
\usepackage{amssymb}
%
% Extensive fraction notation
\usepackage{xfrac}
%
% Generic mathematical commands
% Notable: \degree, \celcius
\usepackage{gensymb}
%
% Variable vector notation (arrow above variable)
\usepackage{esvect}
%
% Multiline boxed equations
\usepackage{empheq}
%
% SI Unit
\usepackage{siunitx}
%
% More intuitive arrays/matrices
\usepackage{array}

% Graphic packages %
%
% Diagrams and illustrations
\usepackage{tikz}
%
% Image insertion
\usepackage{graphicx}

% Document content %
%
% Change title of table of contents
% \renewcommand{\contentsname}{Title}

\title{Homework 4 - 1D Motion}
\author{Corey Mostero}
\date{Student ID: 256652}

\begin{document}

% Command `\hr` to insert horizontal rules
\newcommand{\hr}{\par\noindent\rule{\textwidth}{0.4pt}}

% Command to box and center math equations
\newcommand{\bc}[1]{
	\begin{equation*}
		\begin{boxed}
			{#1}
		\end{boxed}
	\end{equation*}
}

% Command for single line equations with a condition
\newcommand{\cond}[2]{
	\ifmmode
		{#1} \quad {#2}
	\else
		$$ {#1} \quad {#2} $$
	\fi
}

\maketitle
\newpage

\tableofcontents

\section{Book}

\subsection{2.13}
Turtle's position function
$$ x(t) = \SI{50.0}{\centi \meter} + (\SI{2.00}{\centi \meter \per \second})t - (\SI{0.0625}{\centi \meter \per \second \squared})t^2 $$
\begin{enumerate}[label=\textbf{(\alph*)}]
	\item
		Find the turtle's initial velocity ($ v_0 $), initial position ($ x_0 $), and initial acceleration ($ a_0 $).
		\begin{align*}
			x_0 & = x(0) = \SI{50.0}{\centi \meter} + (\SI{2.00}{\centi \meter \per \second})(0) - (\SI{0.0625}{\centi \meter \per \second \squared})(0)^2 \\
			x_0 & = \SI{50.0}{\centi \meter}
		\end{align*}
		Velocity function $ x'(t) $
		\begin{equation*}
			x'(t) = \SI{2.00}{\centi \meter \per \second} - (\SI{0.125}{\centi \meter \per \second \squared})t
		\end{equation*}
		\begin{align*}
			v_0 & = x'(0) = \SI{2.00}{\centi \meter \per \second} - (\SI{0.125}{\centi \meter \per \second \squared})(0) \\
			v_0 & = \SI{2.00}{\centi \meter \per \second}
		\end{align*}
		Acceleration function $ x''(t) $
		\begin{equation*}
			x''(t) = \SI{-0.125}{\centi \meter \per \second \squared}
		\end{equation*}
		\begin{align*}
			a_0 & = x''(0) = \SI{-0.125}{\centi \meter \per \second \squared} \\
			a_0 & = \SI{-0.125}{\centi \meter \per \second \squared}
		\end{align*}
		\bc{
			x_0 = \SI{50.0}{\centi \meter},
			v_0 = \SI{2.00}{\centi \meter \per \second},
			a_0 = \SI{-0.125}{\centi \meter \per \second \squared}
		}
	\item
		Find $ t $ when $ v = 0 $
		\begin{align*}
			x'(t) & = \SI{2.00}{\centi \meter \per \second} - (\SI{0.125}{\centi \meter \per \second \squared})t \\
			\SI{2.00}{\centi \meter \per \second} - (\SI{0.125}{\centi \meter \per \second \squared})t & = 0 \\
			t & = \SI{16}{\second}
		\end{align*}
		\bc{
			t = \SI{16}{\second}
		}
	\item
		Find $ t $ when $ x = \SI{50.0}{\centi \meter} $
		\begin{align*}
			x(t) & = \SI{50.0}{\centi \meter} + (\SI{2.00}{\centi \meter \per \second})t - (\SI{0.0625}{\centi \meter \per \second \squared})t^2 \\
			\SI{50.0}{\centi \meter} & = \SI{50.0}{\centi \meter} + (\SI{2.00}{\centi \meter \per \second})t - (\SI{0.0625}{\centi \meter \per \second \squared})t^2 \\
			t & = \SI{0}{\second}, \SI{32}{\second}
		\end{align*}
		\bc{t = \SI{32}{\second}}
	\item
		Find $ t $ when $ x = \SI{60.0}{\centi \meter} $ and $ x = \SI{40.0}{\centi \meter} $. Find $ v $ at each time.
		\begin{align*}
			\SI{50.0}{\centi \meter} + (\SI{2.00}{\centi \meter \per \second})t - (\SI{0.0625}{\centi \meter \per \second \squared})t^2 & = \SI{40.0}{\centi \meter} \\
			t & = \SI{-4.40}{\second}, \SI{36.4}{\second} \\
			t & = \SI{36.4}{\second}
		\end{align*}
		\begin{align*}
			\SI{50.0}{\centi \meter} + (\SI{2.00}{\centi \meter \per \second})t - (\SI{0.0625}{\centi \meter \per \second \squared})t^2 & = \SI{60.0}{\centi \meter} \\
			t & = \SI{6.20}{\second}, \SI{25.8}{\second}
		\end{align*}
		\begin{align*}
			x'(\SI{36.4}{\second}) & = \SI{2.00}{\centi \meter \per \second} - (\SI{0.125}{\centi \meter \per \second \squared})(\SI{36.4}{\second}) \\
			v_1 & = \SI{-2.55}{\centi \meter \per \second} \\
			x'(\SI{6.20}{\second}) & = \SI{2.00}{\centi \meter \per \second} - (\SI{0.125}{\centi \meter \per \second \squared})(\SI{6.20}{\second}) \\
			v_2 & = \SI{1.225}{\centi \meter \per \second} \\
			x'(\SI{25.8}{\second}) & = \SI{2.00}{\centi \meter \per \second} - (\SI{0.125}{\centi \meter \per \second \squared})(\SI{25.8}{\second}) \\
			v_3 & = \SI{-1.225}{\centi \meter \per \second}
		\end{align*}
		\bc{
			t = \SI{36.4}{\second}, \SI{6.20}{\second}, \SI{25.8}{\second},
			v = \SI{-2.55}{\centi \meter \per \second}, \SI{1.23}{\centi \meter \per \second}, \SI{-1.23}{\centi \meter \per \second}
		}
	\item
		Graphs
\end{enumerate}

\subsection{2.20}
\begin{align*}
	v_o & = 0 \\
	v_f & = \SI{73.14}{\meter \per \second} \\
	t & = \SI{30.0}{\milli \second} = \SI{0.03}{\second}
\end{align*}
\begin{enumerate}[label=\textbf{(\alph*)}]
	\item
		Find acceleration $ a $ during serve
		\begin{align*}
			v_f & = v_o + at \\
			a & = \frac{v_f - v_o}{t} \\
			a & = \frac{\SI{73.14}{\meter \per \second} - 0}{\SI{0.03}{\second}} \\
			a & = \SI{2438}{\meter \per \second \squared}
		\end{align*}
		\bc{a = \SI{2438}{\meter \per \second \squared}}
	\item
		Find distance $ x $ traveled during serve
		\begin{align*}
			\Delta x & = v_ot + \frac{1}{2}at^2 \\
			\Delta x & = (0)(\SI{0.03}{\second}) + \frac{1}{2}(\SI{2438}{\meter \per \second \squared})(\SI{0.03}{\second})^2 \\
			\Delta x & = \SI{1.0971}{\meter}
		\end{align*}
		\bc{x = \SI{1.097}{\meter}}
\end{enumerate}

\subsection{2.29}
\begin{enumerate}[label=\textbf{(\alph*)}]
	\item
		Find acceleration $ a $ at $ t = \SI{3}{\second}, \SI{7}{\second}, \SI{11}{\second} $
		$$ a_t = \frac{y_1 - y_0}{x_1 - x_0} $$
		\begin{align*}
			a_3 & = \frac{\SI{20}{\meter \per \second} - \SI{20}{\meter \per \second}}{\SI{3}{\second} - 0} \\
			a_3 & = 0 \\
			a_7 & = \frac{\SI{45}{\meter \per \second} - \SI{20}{\meter \per \second}}{\SI{9}{\second} - \SI{5}{\second}} \\
			a_7 & = \SI{6.25}{\meter \per \second \squared} \\
			a_{11} & = \frac{\SI{0}{\meter \per \second} - \SI{45}{\meter \per \second}}{\SI{13}{\second} - \SI{9}{\second}} \\
			a_{11} & = \SI{-11.25}{\meter \per \second \squared}
		\end{align*}
		\bc{a_3 = 0, a_7 = \SI{6.25}{\meter \per \second \squared}, a_{11} = \SI{-11.25}{\meter \per \second \squared}}
	\item
		Find distance traveled $ x $ at $ t = \SI{5}{\second}, \SI{9}{\second}, \SI{13}{\second} $
		\begin{align*}
			v_{5,0} & = \SI{20}{\meter \per \second} \\
			x_{5,0} & = (\SI{5}{\second})(\SI{20}{\meter \per \second}) \\
			x_{5,0} & = \SI{100}{\meter}
		\end{align*}
		\begin{align*}
			v_5 & = \SI{20}{\meter \per \second} \\
			t & = \SI{9}{\second} - \SI{5}{\second} = \SI{4}{\second} \\
			a_7 & = \SI{6.25}{\meter \per \second \squared} \\
			\Delta x_{9,5} & = v_5t + \frac{1}{2}a_7t^2 \\
			\Delta x_{9,5} & = (\SI{20}{\meter \per \second})(\SI{4}{\second}) + \frac{1}{2}(\SI{6.25}{\meter \per \second \squared})(\SI{4}{\second})^2 \\
			\Delta x_{9,5} & = \SI{130}{\meter} \\
			x_{9,0} & = x_{5,0} + x_{9,5} \\
			x_{9,0} & = \SI{230}{\meter}
		\end{align*}
		\begin{align*}
			v_9 & = \SI{45}{\meter \per \second} \\
			t & = \SI{13}{\second} - \SI{9}{\second} = \SI{4}{\second} \\
			a_{11} & = \SI{-11.25}{\meter \per \second \squared} \\
			\Delta x_{13,9} & = v_9t + \frac{1}{2}a_{11}t^2 \\
			\Delta x_{13,9} & = (\SI{45}{\meter \per \second})(\SI{4}{\second}) + \frac{1}{2}(\SI{-11.25}{\meter \per \second \squared})(\SI{4}{\second})^2 \\
			\Delta x_{13,9} & = \SI{90}{\meter} \\
			x_{13,0} & = x_{9,0} + x{13,9} \\
			x_{13,0} & = \SI{320}{\meter}
		\end{align*}
		\bc{x_{5,0} = \SI{100}{\meter}, x_{9,0} = \SI{230}{\meter}, x_{13,0} = \SI{320}{\meter}}
\end{enumerate}

\subsection{2.39}
\begin{align*}
	\Delta y & = y_f - y_o \\
	y_f & = 0 \quad \text{(floor)} \\
	y_o & = d = ? \\
	v_{0_y} & = 0 \\
	t & = ? \\
	a_y & = g = \SI{-10}{\meter \per \second \squared}
\end{align*}
\begin{enumerate}[label=\textbf{(\alph*)}]
	\item
		\begin{align*}
			\Delta y & = v_{o_y}t + \frac{1}{2}a_yt^2 \\
			0 - y_o & = (0)t + \frac{1}{2}(\SI{-10}{\meter \per \second \squared})t^2 \\
			-y_o & = (\SI{-5}{\meter \per \second \squared})t^2 \\
			d & = (\SI{5}{\meter \per \second \squared})t^2
		\end{align*}
		\bc{d = (\SI{5}{\meter \per \second \squared})t^2}
	\item
		\begin{align*}
			d & = \SI{17.6}{\centi \meter} = \SI{0.176}{\meter} \\
			\SI{0.176}{\meter} & = (\SI{5}{\meter \per \second \squared})t^2 \\
			t & = \SI{0.188}{\second}
		\end{align*}
		\bc{t = \SI{0.188}{\second}}
\end{enumerate}

\subsection{2.51}
Acceleration of motorcycle
$$ a_x(t) = At - Bt^2 $$
\begin{align*}
	A & = \SI{1.50}{\meter \per \second \cubed} \\
	B & = \SI{0.120}{\meter \per \second \tothe{4}} \\
	v_o & = 0
\end{align*}
\begin{enumerate}[label=\textbf{(\alph*)}]
	\item
		\begin{align*}
			v_x(t) & = \int a_x(t) dt \\
				   & = \int \left( At - Bt^2 \right) dt \\
			v_x(t) & = \frac{1}{2}At^2 - \frac{1}{3}Bt^3 + C
		\end{align*}
		Assume constant is of value $ 0 $
		\bc{v_x(t) = (\SI{0.75}{\meter \per \second \cubed})t^2 - (\SI{0.04}{\meter \per \second \tothe{4}})t^3}
		\begin{align*}
			x(t) & = \int v_x(t) dt \\
				 & = \int \left( \frac{1}{2}At^2 - \frac{1}{3}Bt^3 + C \right) \\
			x(t) & = \frac{1}{6}At^3 - \frac{1}{12}Bt^4 + Ct + D
		\end{align*}
		Assume constants to be of value $ 0 $
		\bc{x(t) = (\SI{0.25}{\meter \per \second \cubed})t^3 - (\SI{0.01}{\meter \per \second \tothe{4}})t^4}
	\item
		Find $ v $ when $ a = 0 $ (maximum velocity)
		\begin{align*}
			a_x(t) & = (\SI{1.50}{\meter \per \second \cubed})t - (\SI{0.120}{\meter \per \second \tothe{4}})t^2 = 0 \\
			t & = 0, \SI{12.5}{\second} \\
			v_x(\SI{12.5}{\second}) & = (\SI{0.75}{\meter \per \second \cubed})(\SI{12.5}{\second})^2 - (\SI{0.04}{\meter \per \second \tothe{4}})(\SI{12.5}{\second})^3 \\
			v & = \SI{39.1}{\meter \per \second}
		\end{align*}
		\bc{v_\text{max} = \SI{39.1}{\meter \per \second}}
\end{enumerate}

\subsection{2.71}
Acceleration of particle
$$ a_x(t) = \SI{-2.00}{\meter \per \second \squared} + (\SI{3.00}{\meter \per \second \cubed})t $$
\begin{enumerate}[label=\textbf{(\alph*)}]
	\item
		Find $ v_{o_x} $ so that $ x $ at $ t = 0, \SI{4.00}{\second} $ are the same
		\begin{align*}
			v_x(t) & = \int a_x(t) dt \\
				   & = \int \left( \SI{-2.00}{\meter \per \second \squared} + (\SI{3.00}{\meter \per \second \cubed})t \right) dt \\
			v_x(t) & = (\SI{-2.00}{\meter \per \second \squared})t + (\SI{1.5}{\meter \per \second \cubed})t^2 + C \\
			x(t) & = \int v_x(t) dt \\
				 & = \int \left( (\SI{-2.00}{\meter \per \second \squared})t + (\SI{1.5}{\meter \per \second \cubed})t^2 + C \right) dt \\
			x(t) & = (\SI{-1.00}{\meter \per \second \squared})t^2 + (\SI{0.5}{\meter \per \second \cubed})t^3 + Ct + D
		\end{align*}
		\begin{align*}
			x(0) & = D \\
			x(\SI{4.00}{\second}) & = (\SI{-1.00}{\meter \per \second \squared})(\SI{4.00}{\second})^2 + (\SI{0.5}{\meter \per \second \cubed})(\SI{4.00}{\second})^3 + C(\SI{4.00}{\second}) + D \\
			x(\SI{4.00}{\second}) & = \SI{16.0}{\meter} + (\SI{4.00}{\second})C + D \\
			x(0) & = x(\SI{4.00}{\second}) \\
			D & = \SI{16.0}{\meter} + (\SI{4.00}{\second})C + D \\
			C & = \SI{-4.00}{\meter \per \second}
		\end{align*}
		\bc{v_{0_x} = \SI{-4.00}{\meter \per \second}}
	\item
		Find velocity at $ t = \SI{4.00}{\second} $
		\begin{align*}
			v_x(\SI{4.00}{\second}) & = (\SI{-2.00}{\meter \per \second \squared})(\SI{4.00}{\second}) + (\SI{1.5}{\meter \per \second \cubed})(\SI{4.00}{\second})^2 + (\SI{4.00}{\second}) \\
			v & = \SI{12.0}{\meter \per \second}
		\end{align*}
		\bc{v_{\SI{4.00}{\second}} = \SI{12.0}{\meter \per \second}}
\end{enumerate}

\section{Lab Manual}

\subsection{471}
\begin{enumerate}[label=\textbf{(\alph*)}]
	\item
		\begin{align*}
			v & = at(1 + 2 + 3 + \cdots + n) \\
			\sum_{t = 1}^n t & = \frac{n(n + 1)}{2} \\
			v & = at \left( \frac{n(n + 1)}{2} \right)
		\end{align*}
		\bc{v = at \left( \frac{n(n + 1)}{2} \right)}
	\item
		\begin{align*}
			x & = \frac{1}{2}at^2(1 + 4 + 9 + 16 + \cdots + n^2) \\
			\sum_{t = 1}^n t^2 & = \frac{n(n + 1)(2n + 1)}{6} \\
			x & = \frac{1}{2}at^2 \left( \frac{n(n + 1)(2n + 1)}{6} \right) \\
			x & = \frac{n(2n^2 + 3n + 1)at^2}{12}
		\end{align*}
		\bc{x = \frac{n(2n^2 + 3n + 1)at^2}{12}}
\end{enumerate}

\subsection{474}
\begin{align*}
	v_{o_A} & = 0 \\
	a_A & = \SI{8.00}{\meter \per \second \squared} \\
	x_{0_{B,A}} & = \SI{30}{\meter} \\
	v_B & = \SI{40}{\meter \per \second}
\end{align*}
Find distance traveled within the first two seconds using data related to car $ A $. (At $ t = \SI{2}{\second} $ car $ A $ and $ B $ are at the same position, so using either car's data is valid)
\begin{align*}
	\Delta x & = v_{o_A}t + \frac{1}{2}a_At^2 \\
			 & = (0)(\SI{2}{\second}) + \frac{1}{2}(\SI{8.00}{\meter \per \second \squared})(\SI{2}{\second})^2 \\
	\Delta x & = \SI{16.0}{\meter}
\end{align*}
Now find the when car $ A $ and $ B $ meet
\begin{align*}
	v_{f_A} & = v_{o_A} + a_At \\
			& = 0 + (\SI{8.00}{\meter \per \second \squared})(\SI{2}{\second}) \\
	v_{f_A} & = \SI{16.0}{\meter \per \second} \\
	\Delta x_A & = v_{o_A}t + \frac{1}{2}a_At^2 \\
	\Delta x_A & = (\SI{16.0}{\meter \per \second})t + (\SI{4.0}{\meter \per \second \squared})t^2
\end{align*}
\begin{align*}
	\Delta x_B & = v_{o_B}t + \frac{1}{2}b_At^2 \\
	\Delta x_B & = (\SI{40}{\meter \per \second})t + (\SI{3.0}{\meter \per \second \squared})t^2
\end{align*}
Set the distance equations equal to each other to find $ t $ (the time the cars meet)
\begin{align*}
	\Delta x_A & = \Delta x_B \\
	(\SI{16.0}{\meter \per \second})t + (\SI{4.0}{\meter \per \second \squared})t^2 & = (\SI{40}{\meter \per \second})t + (\SI{3.0}{\meter \per \second \squared})t^2 \\
	t & = 0, \SI{24}{\second}
\end{align*}
It is found that car $ A $ and $ B $ meet at \SI{24}{\second} after car $ A $ initially overtakes car $ B $. Use $ t = \SI{24}{\second} $ to find the distance that they meet at.
\begin{align*}
	\Delta x_A(\SI{24}{\second}) & = (\SI{16.0}{\meter \per \second})(\SI{24}{\second}) + (\SI{4.0}{\meter \per \second \squared})(\SI{24}{\second})^2 \\
	x & = \SI{2688}{\meter}
\end{align*}
Combine these values with the initial distance / time to find the total distance / time.
\begin{align*}
	x & = \SI{16.0}{\meter} + \SI{2688}{\meter} = \SI{2704}{\meter} \\
	t & = \SI{2}{\second} + \SI{24}{\second} = \SI{26}{\second}
\end{align*}
\bc{x = \SI{2704}{\meter}, t = \SI{26}{\second}}

\end{document}
