\documentclass{article}

% Document extensibility %
%
% Disables native paragraph indentation
\usepackage{parskip} 
%
% Provides further bullet options for lists
\usepackage{enumitem}

% Mathematical symbol and statement packages %
%
% Necessary
\usepackage{amsmath}
\usepackage{amssymb}
%
% Extensive fraction notation
\usepackage{xfrac}
%
% Generic mathematical commands
% Notable: \degree, \celcius
\usepackage{gensymb}
%
% Variable vector notation (arrow above variable)
\usepackage{esvect}
%
% Multiline boxed equations
\usepackage{empheq}
%
% SI Unit
\usepackage{siunitx}
\usepackage{physunits}
\DeclareSIUnit\mile{mi}
%
% More intuitive arrays/matrices
\usepackage{array}
%
% Linear Equations
\usepackage{systeme}

% Graphic packages %
%
% Diagrams and illustrations
\usepackage{tikz}
%
% Image insertion
\usepackage{graphicx}

% Document content %
%
% Change title of table of contents
% \renewcommand{\contentsname}{Title}

\begin{document}

% Command `\hr` to insert horizontal rules
\newcommand{\hr}{\par\noindent\rule{\textwidth}{0.4pt}}

% Command to box and center math equations
\newcommand{\bc}[1]{
	\begin{equation*}
		\begin{boxed}
			{#1}
		\end{boxed}
	\end{equation*}
}

% Command for single line equations with a condition
\newcommand{\cond}[2]{
	\ifmmode
		{#1} \quad {#2}
	\else
		$$ {#1} \quad {#2} $$
	\fi
}

\tableofcontents

\section{Energy Ext.}

In 1D
\begin{equation}
	\vec{F} = - \frac{\partial \text{PE}}{\partial x} \hat{x}
\end{equation}
\begin{equation}
	\text{PE}_g = mgy \rightarrow \vec{F} = -mg \hat{y}
\end{equation}

Potential Energy is only based on position.

\hr

\underline{871:}
\begin{itemize}
	\item $ k $ = Kinetic Energy
	\item $ u $ = Potential Energy
	\item $ H $ = Total Energy
\end{itemize}

\section{Hooke's Law}

\begin{equation}
	\vec{F}_s = -k\vec{x}
\end{equation}

\subsection{Example}

\begin{align*}
	\sum F & = 0 \\
	kx - mg & = 0 \\
	k & = \frac{mg}{x} \\
	k & = \frac{(\SI{17}{\gram})(\SI{1000}{\centi \meter \per \second \squared})}{\SI{10}{\centi \meter}} \\
	k & = \SI{1700}{\dyne \per \centi \meter}
\end{align*}
\bc{k = \SI{1700}{\dyne \per \centi \meter}}

\subsection{Series and Parallel Springs}

Series
\begin{align*}
	F_1 & = F_2 \\
	\Delta x_1 & \neq \Delta x_2
\end{align*}

Parallel:
\begin{align*}
	\Delta x_1 & = \Delta x_2 \\
	F_1 \neq F_2
\end{align*}

Replace Parallel Springs with $ k_{eq} $
\begin{align*}
	\sum F_{eq} & = 0 \\
	mg & = k_{eq} \Delta x
\end{align*}
\begin{align*}
	\sum F_{1, 2} & = 0 \\
	mg & = k_1 \Delta x_1 + k_2 \Delta x_2 \\
	k_{eq} \Delta x_{eq} & = k_1 \Delta x_1 + k_2 \Delta x_2 \\
	\Delta x_{eq} & = \Delta x_1 = \Delta x_2
\end{align*}
\begin{align*}
	k_{\parallel} & = k_1 + k_2 \\
	\frac{1}{k_s} & = \frac{1}{k_1} + \frac{1}{k_2}
\end{align*}

\hr

\underline{Find out the work done against a spring}

Recall:
\begin{align*}
	W & = \int \vec{F} \cdot d\vec{x} \\
	\text{PE} & = -W
\end{align*}
\begin{align*}
	\text{PE}_s & = - \int_0^x (-kx) dx \\
	\text{PE}_s & = \frac{1}{2}kx^2
\end{align*}
Potential Energy: \bc{\text{PE}_s = \frac{1}{2}kx^2}
$ x = 0 $ at equilibrium

The only time the final height depends on the shape of the path is when there is friction.

\subsection{Example}

Variables
\begin{align*}
	k & = \SI{200}{\newton \per \meter} \\
	x_1 & = \SI{0.5}{\meter} \\
	m & = \SI{1}{\kilogram} \\
	v_1 & = 0 \\
	v_2 & = ? \\
	v_3 & = 0 \\
	h_3 & = ?
\end{align*}
\begin{align*}
	E_1 & = E_3 \\
	\frac{1}{2}kx_1^2 & = mgh_3 \\
	h_3 & = \frac{kx^2}{2mg} \\
	h_3 & = \frac{(\SI{200}{\newton \per \meter})(\SI{0.5}{\meter})^2}{2(\SI{10}{\newton})} \\
	h_3 & = \SI{2.5}{\meter}
\end{align*}
\bc{h_3 = \SI{2.5}{\meter}}

\section{Power}

\begin{equation}
	\text{Power} = \frac{\text{Energy}}{\text{Time}}
\end{equation}
\bc{P = \frac{dE}{dt}}
Definition of Power
\begin{align*}
	P & = \frac{d \left( \int \vec{F} \cdot d\vec{x} \right)}{dt} \\
	P & = \frac{d \left( \frac{1}{2}mv^2 \right)}{dt} \\
	P & = \frac{1}{2} \frac{d}{dt} \left[ mv^2 \right] \\
	P & = \frac{1}{2} (\dot{m}v)v + \frac{1}{2}m(2v\dot{v})
\end{align*}
In the case that $ m $ doesn't change ($ \dot{m} = 0 $)
\begin{align*}
	P & = (ma)v \\
	P & = \vec{F} \cdot \vec{v}
\end{align*}
\bc{P = \vec{F} \cdot \vec{v}}

\section{Momentum}

Recall:
\begin{equation}
	\vec{F} = \frac{d\vec{p}}{dt}
\end{equation}
\begin{equation}
	\vec{P} \equiv m\vec{v}
\end{equation}
Momentum - ``how hard is it to reduce $ \vec{v} $ to zero?"

$ \vec{P} $ is a vector that points in the same direction as $ \vec{v} $. $ \vec{P} $ is conserved.
$$ KE = \frac{p^2}{2m}; F = \frac{d\vec{P}}{dt} $$

Collision \textrightarrow Momentum
\begin{align}
	\sum \vec{P}_i & = \sum \vec{P}_f \\
	\sum \vec{P}_{i_x} & = \sum \vec{P}_{f_x} \\
	\sum \vec{P}_{i_y} & = \sum \vec{P}_{f_y} \\
	\sum \vec{P}_{i_z} & = \sum \vec{P}_{f_z}
\end{align}

\subsection{2D Momentum Conservation:}

Situation - Two masses (vehicles) are colliding
\begin{align*}
	M & = \SI{3000}{\kilogram} \\
	u_i & = \SI{90}{\mile \per \hour} \\
	m & = \SI{2500}{\kilogram} \\
	v_i & = \SI{75}{\mile \per \hour} \\
	v_f & = ? \\
	\phi & = ? \\
	\theta_f & = \SI{30}{\degree} \\
	u_f & = \SI{60}{\mile \per \hour}
\end{align*}
\begin{align*}
	\sum \vec{P}_{i_x} & = \sum \vec{P}_{f_x} \\
	Mu_i - mv_i & = Mu_f\cos(\theta_f) - mv_f\cos(\phi_f) \\
	mv_f\cos(\phi_f) & = Mu_f\cos(\theta_f) - Mu_i + mv_i
\end{align*}
\begin{align*}
	\sum \vec{P}_{i_y} & = \sum \vec{P}_{f_y} \\
	0 & = mv_f\sin(\phi_f) - Mu_f\sin(\theta_f) \\
	mv_f\sin(\phi_f) & = Mu_f\sin(\theta_f)
\end{align*}
\begin{align*}
	\tan(\phi_f) & = \frac{Mu_f\sin(\theta_f)}{Mu_f\cos(\theta_f) - Mu_i + mv_i} \\
	\phi_f & = \arctan \left[ \frac{(\SI{3000}{\kilogram})(\SI{60}{\mile \per \hour})(\sin(\SI{30}{\degree}))}{(\SI{3000}{\kilogram})(\SI{60}{\mile \per \hour})(\cos(\SI{30}{\degree})) - (\SI{3000}{\kilogram})(\SI{90}{\mile \per \hour}) + (\SI{2500}{\kilogram})(\SI{75}{\mile \per \hour})} \right] \\
	\phi_f & = \SI{51}{\degree}
\end{align*}
\bc{\phi_f = \SI{51}{\degree}}
\begin{align*}
	mv_f\sin(\phi_f) & = Mu_f\sin(\theta_f) \\
	v_f & = \left[ \frac{M}{m} \frac{\sin(\theta_f)}{\sin(\phi_f)} \right] u_f \\
	v_f & = \frac{(\SI{3000}{\kilogram})(\sin(\SI{30}{\degree}))(\SI{60}{\mile \per \hour})}{(\SI{2500}{\kilogram})(\sin(\SI{51}{\degree}))} \\
	v_f & = \SI{46}{\mile \per \hour}
\end{align*}
\bc{v_f = \SI{46}{\mile \per \hour}}

\section{Collisions}

\underline{Perfectly Elastic ($ \epsilon = 1 $):}

Energy is conserved.

\underline{Partially Elastic ($ 0 < \epsilon <  1 $):}

Bounre; Energy not conserved.

\underline{Inelastic ($ \epsilon = 0 $):}

Two objects stick together or explode.

\hr

Is the car collision elastic?
\begin{align*}
	\frac{1}{2}mv_i^2 + \frac{v}{2}Mu_i^2 & = \frac{1}{2}mv_f^2 + \frac{1}{2}Mu_f^2 \\
	(\SI{2500}{\kilogram})(\SI{75}{\mile \per \hour})^2 + (\SI{3000}{\kilogram})(\SI{90}{\mile \per \hour})^2 & = (\SI{2500}{\kilogram})(\SI{46}{\mile \per \hour})^2 + (\SI{3000}{\kilogram})(\SI{60}{\mile \per \hour})^2 \\
	\SI{3.8e7}{\kilogram \mile \per \hour} & = \SI{1.6e7}{\kilogram \mile \per \hour}
\end{align*}
\bc{\SI{3.8e7}{\kilogram \mile \per \hour} \neq \SI{1.6e7}{\kilogram \mile \per \hour} \therefore \text{ Partially Elastic}}

\end{document}
