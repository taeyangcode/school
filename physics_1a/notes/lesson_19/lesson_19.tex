\documentclass{article}

% Document extensibility %
%
% Disables native paragraph indentation
\usepackage{parskip} 
%
% Provides further bullet options for lists
\usepackage{enumitem}

% Mathematical symbol and statement packages %
%
% Necessary
\usepackage{amsmath}
\usepackage{amssymb}
%
% Extensive fraction notation
\usepackage{xfrac}
%
% Generic mathematical commands
% Notable: \degree, \celcius
\usepackage{gensymb}
%
% Variable vector notation (arrow above variable)
\usepackage{esvect}
%
% Multiline boxed equations
\usepackage{empheq}
%
% SI Unit
\usepackage{siunitx}
\usepackage{physunits}
%
% More intuitive arrays/matrices
\usepackage{array}
%
% Linear Equations
\usepackage{systeme}
%
% Boxes!
\usepackage{mdframed}

% Graphic packages %
%
% Diagrams and illustrations
\usepackage{tikz}
\usetikzlibrary{positioning}
%
% Image insertion
\usepackage{graphicx}

% Document content %
%
% Change title of table of contents
% \renewcommand{\contentsname}{Title}

\begin{document}

% Command `\hr` to insert horizontal rules
\newcommand{\hr}{\par\noindent\rule{\textwidth}{0.4pt}}

% Command to box and center math equations
\newcommand{\bc}[1]{
	\begin{equation*}
		\begin{boxed}
			{#1}
		\end{boxed}
	\end{equation*}
}

% Command for single line equations with a condition
\newcommand{\cond}[2]{
	\ifmmode
		{#1} \quad {#2}
	\else
		$$ {#1} \quad {#2} $$
	\fi
}

\tableofcontents

\section{Non-inertial Reference Frame}

\begin{itemize}
	\item Inertial - No acceleration
	\item Non Inertial - Yes acceleration
		\begin{itemize}
			\item Fictitious forces
		\end{itemize}
\end{itemize}

\section{Centripetal Forces}

\begin{itemize}
	\item Centripetal Acceleration - $ a_c = v\omega $
	\item Angular Acceleration (Angular, Linear) - $ R\omega^2 \leftrightarrows \frac{v^2}{R} $
\end{itemize}

\subsection{Example}

\begin{align*}
	R & = \SI{40}{\meter} \\
	v & = \SI{7}{\meter \per \second} \\
	\mu_{min} & = ?
\end{align*}
\begin{align*}
	\sum F_z & = 0 \\
	N & = mg
\end{align*}
\begin{align*}
	\sum F_c & = \frac{mv^2}{R} \\
	f & = \frac{mv^2}{R} \\
	\mu g & = \frac{v^2}{R} \\
	\mu & = \frac{v^2}{gR} \\
	\mu & = \frac{(\SI{7}{\meter \per \second})^2}{(\SI{10}{\meter \per \second \squared})(\SI{40}{\meter})} \\
	\mu & = 0.12
\end{align*}

\subsection{Example}

\begin{align*}
	R & = \SI{10}{\meter} \\
	\theta & = \SI{36}{\degree} \\
	\mu & = 0 \\
	v & = 0
\end{align*}
\begin{align*}
	\sum F_z & = 0 \\
	N\sin(\theta) & = mg \\
	N & = \frac{mg}{\cos(\theta)}
\end{align*}
\begin{align*}
	\sum F_c & = ma_c \\
	-f\cos(\theta) & = ma_c \\
	-\mu mg\tan(\theta) + N\sin(\theta) & = \frac{mv^2}{R} \\
	\left( \frac{mg}{\cos(\theta)} \right) \sin(\theta) & = \frac{mv^2}{R} \\
	v & = \sqrt{ Rg\tan(\theta) }
\end{align*}

\subsection{Example}

\begin{align*}
	D & = \SI{4}{\meter} \\
	\mu_{max} & = 0.4 \\
	\omega & = ? \\
	m & = \SI{60}{\kilogram}
\end{align*}
\begin{align*}
	\sum F_z^{(m)} & = 0 \\
	f & = mg \\
	\mu N & = mg \\
	N & = \frac{mg}{\mu}
\end{align*}

\end{document}
