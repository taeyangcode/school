\documentclass{article}

% Document extensibility %
%
% Disables native paragraph indentation
\usepackage{parskip} 
%
% Provides further bullet options for lists
\usepackage{enumitem}

% Mathematical symbol and statement packages %
%
% Necessary
\usepackage{amsmath}
\usepackage{amssymb}
%
% Extensive fraction notation
\usepackage{xfrac}
%
% Generic mathematical commands
% Notable: \degree, \celcius
\usepackage{gensymb}
%
% Variable vector notation (arrow above variable)
\usepackage{esvect}
%
% Multiline boxed equations
\usepackage{empheq}
%
% SI Unit
\usepackage{siunitx}
\usepackage{physunits}
%
% More intuitive arrays/matrices
\usepackage{array}
%
% Linear Equations
\usepackage{systeme}
%
% Boxes!
\usepackage{mdframed}
%
% Matrix Notation
\usepackage{bm}

% Graphic packages %
%
% Diagrams and illustrations
\usepackage{tikz}
\usetikzlibrary{positioning}
%
% Image insertion
\usepackage{graphicx}

% Document content %
%
% Change title of table of contents
% \renewcommand{\contentsname}{Title}

\begin{document}

% Command `\hr` to insert horizontal rules
\newcommand{\hr}{\par\noindent\rule{\textwidth}{0.4pt}}

% Command to box and center math equations
\newcommand{\bc}[1]{
	\begin{equation*}
		\begin{boxed}
			{#1}
		\end{boxed}
	\end{equation*}
}

% Command for single line equations with a condition
\newcommand{\cond}[2]{
	\ifmmode
		{#1} \quad {#2}
	\else
		$$ {#1} \quad {#2} $$
	\fi
}

% Matrix and Vector notation
\newcommand{\matr}[1]{\bm{#1}}
\newcommand{\vect}[1]{
	\ifmmode \mathbf{#1}
	\else \textbf{#1}
	\fi
}

\tableofcontents

\section{Circular Motion}

\subsection{Roller coaster Example}

\begin{align*}
	R & = \SI{15}{\meter} \\
	m & = \SI{2000}{\kilogram}
\end{align*}
Find $ v_B $ in terms of $ R $:
\begin{align*}
	\sum F_c & = ma_c \\
	N_t + mg & = ma_c \\
	g & = \frac{v_T^2}{R} \\
	v_T & = \sqrt{gR}
\end{align*}
\begin{align*}
	E_T & = E_B \\
	\frac{1}{2}mv_T^2 + mgh_T & = \frac{1}{2}mv_B^2 \\
	v_B^2 & = v_T^2 + 2gh_T \\
	v_B & = \sqrt{(gR) + 2g(2R)} = \sqrt{5gR}
\end{align*}
Find $ N_B $
\begin{align*}
	\sum F & = ma_c \\
	N_B - mg & = m\frac{v_B^2}{R} \\
	N_B & = mg + m \frac{(5gR)}{R} \\
	N_B & = 6mg
\end{align*}

\section{Angular Derivations}

\begin{align*}
	S & = R\theta \\
	v & = R\omega \\
	a & = R\alpha
\end{align*}
\begin{align*}
	\Delta x & = v_0t + \frac{1}{2}at^2 \\
	(R(\Delta \theta)) & = (R\omega_0)t + \frac{1}{2}(R\alpha)t^2 \\
	\Delta \theta & = \omega_0 t + \frac{1}{2}\alpha t^2
\end{align*}

\subsection{Angular Kinematics}

\begin{align*}
	\omega & = \omega_0 + \alpha t \\
	\Delta \theta & = \omega_0 t + \frac{1}{2}\alpha t^2 \\
	\Delta \theta & = \frac{1}{2}(\omega + \omega_0)t \\
	\omega^2 & = \omega_0^2 + 2\alpha \Delta \theta \\
	\Delta \theta & = \omega t - \frac{1}{2}\alpha t^2
\end{align*}
\begin{align*}
	\Delta x & \rightarrow \theta \\
	v & \rightarrow \omega \\
	a & \rightarrow \alpha \\
	v = \frac{dx}{dt} & \rightarrow \omega = \frac{d\theta}{dt}
\end{align*}
\begin{align*}
	F & \rightarrow \tau \\
	m^{!} & = I^{@} \\
\end{align*}
\begin{itemize}
	\item ! - Resistance to change in $ v $
	\item @ - Resistance to change in $ \omega $
\end{itemize}
\begin{align*}
	KE_T = \frac{1}{2}mv^2 & \rightarrow KE_R = \frac{1}{2}I\omega^2 \\
	\vect{P} = m\vect{v} & \rightarrow \vect{L}^! = I\vect{\omega} \\
	\vect{P} = \frac{d\vect{p}}{dt} & \rightarrow \vect{\tau} = \frac{d\vect{L}}{dt}
\end{align*}
\begin{itemize}
	\item ! - Angular Momentum
\end{itemize}
\begin{align*}
	\sum \vect{F}_i & = dm\vect{a} \\
	\vect{r} \times \sum F_i & = dm(\vect{r} \times \vect{a}) \\
	\sum \vect{r} \times \vect{F}_i & = ra_{\perp}dm \\
	\tau_i & = r^2\alpha dm \\
	\sum \tau & = \int (r^2 dm)\alpha \\
	\sum \tau_{ext} + \sum \tau_{i} & = I\vect{\alpha}, \quad \sum \tau_{i} = 0 \\
	\sum \vect{\tau}_{ext} & = I\vect{\alpha}
\end{align*}

\section{Moments of Inertia}

\begin{itemize}
	\item Volume Density \textrightarrow $ \rho = \frac{dm}{dV} $
	\item Surface Density \textrightarrow $ \sigma = \frac{dm}{dA}, \quad A = $ \text{Area}
	\item Linear Density \textrightarrow $ \lambda = \frac{dm}{dx} $
\end{itemize}
\begin{align*}
	dm & = \rho dv = \rho(h(2\pi r)dr) \\
	I_{cylinder} & = \int_0^R r^2 dm \\
	I & = 2\pi \rho h \int_0^R r^3 dr \\
	I & = \left. \frac{1}{2}\pi \rho hr^4 \right|_0^R \\
	I & = \frac{\pi \rho hR^4}{2} \\
	\rho & = \frac{M}{\pi R^2h} \\
	I & = \frac{1}{2}\pi \left( \frac{M}{\pi R^2h} \right) R^4 \\
	I_{cylinder} & = \frac{1}{2}MR^2
\end{align*}
\begin{align*}
	I & = \int x^2 dm \\
	I & = \int_0^L x^2\lambda dx \\
	I & = \frac{1}{3}L^3\lambda \\
	\lambda & = \frac{M}{L} \\
	I & = \frac{1}{3}L^3 \left( \frac{M}{L} \right) \\
	I & = \frac{1}{3}ML^2
\end{align*}
\begin{align*}
	I_{cylinder} = I_{disk} & = \frac{1}{2}MR^2 \\
	I_{hoop} & = MR^2 \\
	I_{sphere} & = \frac{2}{5}MR^2 \\
	I_{spherial shell} & = \frac{2}{3}MR^2
\end{align*}

\end{document}
