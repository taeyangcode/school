\documentclass{article}

% Document extensibility %
%
% Disables native paragraph indentation
\usepackage{parskip} 
%
% Provides further bullet options for lists
\usepackage{enumitem}

% Mathematical symbol and statement packages %
%
% Necessary
\usepackage{amsmath}
\usepackage{amssymb}
%
% Extensive fraction notation
\usepackage{xfrac}
%
% Generic mathematical commands
% Notable: \degree, \celcius
\usepackage{gensymb}
%
% Variable vector notation (arrow above variable)
\usepackage{esvect}
%
% Multiline boxed equations
\usepackage{empheq}
%
% SI unit display
\usepackage{siunitx}

% Graphic packages %
%
% Diagrams and illustrations
\usepackage{tikz}
%
% Image insertion
\usepackage{graphicx}

% Document content %
%
% Change title of table of contents
% \renewcommand{\contentsname}{Title}

\begin{document}

% Command `\hr` to insert horizontal rules
\newcommand{\hr}{\par\noindent\rule{\textwidth}{0.4pt}}

\section{A Review of 2A Error Theory}

If
$$ x = x_{\text{best}} \pm \delta x $$
– Where $ x_{\text{best}} $ is th ebest guess at the measurement and $ \delta x $ is the uncertainty – on a measurement device uncertainty is the the resolution

\begin{enumerate}[label=\text{Rule} \arabic*:]
    \item $ \delta x $ has same units as $ x_{\text{best}} $
    \item $ \delta x $ always is rounded to one sig. fig.
    \item $ x_\text{best} $ \underline{must} have the same precision as $ \delta x $
\end{enumerate}
For example \SI{6.87415}{\cm} $ \pm $ \SI{0.5}{\mm}, turn this into properly reported value:
\begin{enumerate}[label=\text{Step} \arabic*:]
    \item Same units \SI{68.7415}{\mm} $ \pm $ \SI{0.5}{\mm}
    \item Round $ x_{\text{best}} $ to match $ \delta x $
        $$ \SI{68.7 \pm 0.5}{\mm} $$
\end{enumerate}

\hr

$$ x_{\text{best}} \pm \delta x $$
$$ y_{\text{best}} \pm \delta y $$
for $ x_{\text{best}} \pm y_{\text{best}} $
$$ \delta_{x + y} = \delta_x + delta_y $$
$$ \delta{x - y} = \delta_x + delta_y $$

for $ x_{\text{best}} \cdot y_{\text{best}} $ or $ \frac{x_{\text{best}}}{y_{\text{best}}} $
$$ \frac{\delta_{xy}}{xy} = \frac{\delta_x}{x} + \frac{\delta_y}{y} $$
where $ \frac{\delta_y}{y} $ is the fractional error, relative error, or percent error

$$ \delta f = \frac{\partial f}{\partial x} \Biggr|_{x_{\text{best}}} \delta x $$
$$ \frac{df}{dx} = \frac{df}{dx} $$
$$ df = \frac{df}{dx}dx $$

For determinate errors:
$$ \delta f = \sum \frac{\partial f}{\partial x_i}\delta{x_i} $$
in 2D:
$$ \delta f = \left| \frac{\partial f}{\partial x} \right| \delta x + \left| \frac{\partial f}{\partial y} \right| \delta y $$

\hr

Given
$$ f = x + y $$
Taking the partial derivative in respect to $ x $
$$ \frac{\partial f}{\partial x} = 1 $$
Taking the partial derivative in respect to $ y $
$$ \frac{\partial f}{\partial y} = 1 $$
$$ \delta f = (1)(\delta x) + (1)(\delta y) = \delta x + \delta y $$

\hr

Given
$$ f = xy $$
$$ \frac{\partial f}{\partial x} = y $$
$$ \frac{\partial f}{\partial y} = x $$
$$ \delta f = y \delta x + x \delta y $$
Now dividing by $ f $ (which is equal to $ xy $)
$$ \frac{\delta f}{f} = \frac{y\delta x}{xy} + \frac{x\delta y}{xy} = \frac{\delta x}{x} + \frac{\delta y}{y} $$

\hr

Given
$$ f = ax^n $$
$$ \frac{\partial f}{\partial x} = \left| anx^{n-1} \right| $$
$$ \frac{\partial f}{f} = \frac{\frac{\partial f}{\partial x}\delta x}{f} $$
$$ \frac{\partial f}{f} = \left| \frac{anx^{n-1}}{ax^n}\delta x \right| $$
$$ \frac{\partial f}{f} = \left| n\frac{\delta x}{x} \right| $$

\hr

deviation = $ \bar{x} - x_i $

Root Mean Square
$$ \sigma_x - \sqrt{\frac{\sum (\bar{x}-x_i)^2}{N}} $$
Where $ \sigma_x $ is standard deviation

The best value for statistical data is given by the average
$$ x_{\text{best}} = \bar{x} = \frac{\sum x_i}{N} $$

$ \delta_x $ for a statistical spread is:
$$ \delta_x = \frac{\sigma_x}{\sqrt{N}} $$

\begin{align*}
	x_\text{best} & = \bar{x} \pm \frac{\sigma_x}{\sqrt{N}} \\
				  & = \SI{1.960462}{\second} \pm \SI{0.069127}{\second} \\
				  & = \SI{1.96 \pm 0.07}{\second}
\end{align*}
\begin{equation*}
	\boxed{
		x_\text{best} = \SI{1.96 \pm 0.07}{\second}
	}
\end{equation*}

\hr

$$ \omega = \sqrt{\frac{g}{l}} $$
$$ \frac{2\pi}{T} = \sqrt{\frac{g}{l}} $$
$$ T = 2\pi l^{\sfrac{1}{2}} g^{-\sfrac{1}{2}} $$
$$ g^{\sfrac{1}{2}} = 2\pi l^{\sfrac{1}{2}} T^{-1} $$
$$ g = 4pi^2 l T^{-2} $$

\hr

For \underline{statistical} uncertainties
$$ \delta f = \sqrt{ \left( \frac{\partial f}{\partial x}\delta x \right)^2 + \left( \frac{\partial f}{\partial y}\delta y \right)^2 } $$
$$ \frac{\partial g}{\partial l} = 4\pi^2T^{-2} $$
$$ \frac{\partial g}{\partial T} = (-2)4\pi^2lT^{-3} $$

\hr

\begin{align*}
	\frac{\delta g}{g} & = \frac{\sqrt{ \left( \frac{\partial g}{\partial l}\delta l \right)^2 + \left( \frac{\partial g}{\partial T}\delta T \right)^2 }}{g} \\
	\frac{\delta g}{g} & = \sqrt{ \left( \frac{\partial g}{\partial l}\frac{\delta l}{g} \right)^2 + \left( \frac{\partial g}{\partial l}\frac{\delta T}{g} \right)^2 } \\
	\frac{\partial g}{g} & = \sqrt{\left( \frac{\partial l}{l} \right)^2 + 4 \left(\frac{\partial T}{T} \right)^2} \\
	& = \sqrt{\left( \frac{2}{100} \right)^2 + 4 \left( \frac{0.07}{1.96} \right)^2} \\
	& = 7%
\end{align*}
\begin{align*}
	g & = 4\pi^2lT^{-2} \\
	  & = 4\pi^2(\SI{1.00}{\meter})(\SI{1.96}{\second}^{-2} \\
	g & = \SI{10.27655602}{\meter \per \second^2}
\end{align*}
\begin{align*}
	\delta g & = (0.07)(\SI{10.27655}{\meter \per \second^2} \\
			 & = \SI{0.7}{\meter \per \second^2} \\
	g & = \SI{10.3 \pm 0.7}{\meter \per \second^2}
\end{align*}

\end{document}
