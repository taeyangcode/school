\documentclass{article}

% Document extensibility %
%
% Disables native paragraph indentation
\usepackage{parskip} 
%
% Provides further bullet options for lists
\usepackage{enumitem}

% Mathematical symbol and statement packages %
%
% Necessary
\usepackage{amsmath}
\usepackage{amssymb}
%
% Extensive fraction notation
\usepackage{xfrac}
%
% Generic mathematical commands
% Notable: \degree, \celcius
\usepackage{gensymb}
%
% Variable vector notation (arrow above variable)
\usepackage{esvect}
%
% Multiline boxed equations
\usepackage{empheq}
%
% SI Unit
\usepackage{siunitx}

% Graphic packages %
%
% Diagrams and illustrations
\usepackage{tikz}
%
% Image insertion
\usepackage{graphicx}

% Document content %
%
% Change title of table of contents
% \renewcommand{\contentsname}{Title}

\begin{document}

% Command `\hr` to insert horizontal rules
\newcommand{\hr}{\par\noindent\rule{\textwidth}{0.4pt}}

% Command to box and center math equations
\newcommand{\bc}[1]{
	\begin{equation*}
		\begin{boxed}
			{#1}
		\end{boxed}
	\end{equation*}
}

% Command for single line equations with a condition
\newcommand{\cond}[2]{
	\ifmmode
		{#1} \quad {#2}
	\else
		$$ {#1} \quad {#2} $$
	\fi
}

\tableofcontents

\section{One More 1D Motion Problem}

\subsection{Finding Drag}

\begin{equation}
	F_\text{drag} = -bv
\end{equation}
Find the equation of motion if:
\begin{align*}
	\sum F & = ma \\
	-v & = ma
\end{align*}
\begin{equation}
	a = -\frac{b}{m}v
\end{equation}
\begin{align*}
	\frac{dv}{dt} & = -\frac{b}{m}v \\
	\int_{v_0}^v \frac{dv}{v} & = \int_0^t -\frac{b}{m}dt \\
	\ln\left(\frac{v}{v_0}\right) & = -\frac{b}{m}(t - 0) \\
	\frac{v}{v_0} & = e^{-\frac{b}{m}t} \\
	v & = v_0e^{-\frac{b}{m}t}
\end{align*}
\begin{equation}
	v = \frac{dx}{dt}
\end{equation}
\begin{align*}
	\frac{dx}{dt} & = v_0e^{-\frac{b}{m}t} \\
	\int dx = \int v_0e^{-\frac{b}{m}t}dt \\
	x + C & = v_0 \int e^{-\frac{b}{m}t}dt \\
	x + C & = -\frac{v_0m}{be^{\frac{b}{m}}}, \quad \text{at} \quad t = 0; x = 0 \\
	C & = -\frac{mv_0}{b}
\end{align*}
\begin{align*}
	x + C & = -\frac{mv_0}{b}e^{-\frac{b}{m}t} \\
	x + \left( -\frac{mv_0}{b} \right) & = -\frac{mv_0}{b}e^{-\frac{b}{m}t} \\
	x & = \frac{mv_0}{b} \left( 1 - e^{-\frac{b}{m}t} \right)
\end{align*}
\begin{equation}
	x = \frac{mv_0}{b} \left( 1 - e^{-\frac{b}{m}t} \right)
\end{equation}

\section{2D (Projectile) Motion}

Previously Known Kinematic Equations:
\begin{enumerate}
	\item
		$$ v = v_0 + at $$
	\item
		$$ x = x_0 + v_0t + \frac{1}{2}at^2 $$
	\item
		$$ x = x_0 + \frac{1}{2}(v + v_0)t $$
	\item
		$$ v^2 = v_0^2 + 2a\Delta x $$
	\item
		$$ x = x_0 + vt - \frac{1}{2}at^2 $$
\end{enumerate}

Converted to 2D Kinematics:
\begin{enumerate}
	\item
		$$ \vec{v} = \vec{v}_0 + \vec{a}t $$
	\item
		$$ \vec{x} = \vec{x_0} + \vec{v}_0t + \frac{1}{2}\vec{a}t^2 $$
	\item
		$$ \vec{x} = \vec{x_0} + \frac{1}{2}(\vec{v} + \vec{v}_0)t $$
	\item
		$$ \vec{v}^2 = \vec{v_0}^2 + 2\vec{a} \cdot \Delta \vec{x} $$
	\item
		$$ \vec{x} = \vec{x_0} + \vec{v}t - \frac{1}{2}\vec{a}t^2 $$
\end{enumerate}

\subsection{Assumptions}

\begin{enumerate}
	\item
		An object in free fall is subject only to gravity
		$$ a_x = 0 $$
		$$ a_y = \pm g (down) $$
	\item
		Vector components act completely independently

		\begin{tabular}{ c | c }
			\textbf{x (horizontal)} & \textbf{y (vertical)} \\
			\hline
			$ v_x = v_{0_x} = C $ & $ v_y = v_{0_y} + a_yt $ \\
			$ x = x_0 + v_{0_x}t $ & $ y = y_0 + v_{0_y}t + \frac{1}{2}a_yt^2 $ \\
									  & $ y = y_0 + \frac{1}{2}(v_{0_y} + v_y)t $ \\
									  & $ v_y^2 = v_{0_y}^2 + 2a\Delta y $ \\
									  & $ y = y_0 + v_yt - \frac{1}{2}a_yt^2 $
		\end{tabular}
	\item
		Any object in free fall is a projectile
\end{enumerate}

\subsection{Homework Sketches}

Required Elements
\begin{enumerate}
	\item
		Any variable
	\item
		Any location of interest should be indexed
	\item
		Path of motion
	\item
		Any vectors need arrows
	\item
		Origin \& positive directions
\end{enumerate}

\subsection{Example One}

\begin{align*}
	\theta & = \SI{30.0}{\degree} \\
	v_0 & = \SI{45}{\meter \per \second} \\
	a_y & = \SI{-10}{\meter \per \second \squared} \\
	x_0 & = 0 \\
	y_0 & = 0
\end{align*}
Find total time off the ground $ t $
\begin{align*}
	\cos(\theta) & = \frac{v_{0_x}}{v_0} \\
	v_{0_x} & = (\SI{45}{\meter \per \second})(\cos(\SI{30.0}{\degree})) \\
	v_{0_x} & = \SI{38.97}{\meter \per \second}
\end{align*}
\begin{align*}
	\sin(\theta) & = \frac{x_{0_y}}{v_0} \\
	v_{0_y} & = (\SI{45}{\meter \per \second})(\sin(\SI{30.0}{\degree})) \\
	v_{0_y} & = \SI{22.5}{\meter \per \second}
\end{align*}
\begin{align*}
	v_y & = v_{0_y} + a_yt_1 \\
	t_1 & = \frac{v_{1_y} - v_{0_y}}{a_y} \\
	t_1 & = \frac{\SI{-22.5}{\meter \per \second} - \SI{22.5}{\meter \per \second}}{\SI{-10}{\meter \per \second \squared}} \\
	t_1 & = \SI{4.5}{\second}
\end{align*}
\bc{t_1 = \SI{4.5}{\second}}
Find total distance traveled $ x_1 $
\begin{align*}
	x_1 & = x_0 + v_{0_x}t \\
	x_1 & = (0) + (\SI{39.0}{\meter \per \second})(\SI{4.5}{\second}) \\
	x_1 & = \SI{176}{\meter}
\end{align*}
\bc{x_1 = \SI{176}{\meter}}

Find the maximum height $ y_2 $
\begin{align*}
	v_{2_y}^2 & = v_{0_y}^2 + 2a_y(y_2 - y_0) \\
	y_2 & = -\frac{v_{0_y}^2}{2a_y} \\
	y_2 & = -\frac{(\SI{22.5}{\meter \per \second})^2}{2(\SI{-10}{\meter \per \second \squared})} \\
	y_2 & = \SI{25.3}{\meter}
\end{align*}
\bc{y_2 = \SI{25.3}{\meter}}

\subsection{Example Two}
\begin{align*}
	\Delta x & = \SI{88}{\meter} \\
	t & = \SI{4.0}{\second} \\
	\theta_0 & = \SI{30}{\degree}
\end{align*}
Find $ v_0, v_{0_x}, v_{0_y} $
\begin{align*}
	x & = x_0 + v_{0_x}t \\
	v_{0_x} & = \frac{x - x_0}{t} \\
			& = \frac{\SI{88}{\meter} - 0}{\SI{4.0}{\second}} \\
	v_{0_x} & = \SI{22}{\meter \per \second}
\end{align*}
\begin{align*}
	\tan(\theta) & = \frac{v_{0_y}}{v_{0_x}} \\
	v_{0_y} & = (v_{0_x})(\tan(\theta)) \\
			& = (\SI{22}{\meter \per \second})(\tan(\SI{30}{\degree})) \\
	v_{0_y} & = \SI{12.70}{\meter \per \second}
\end{align*}
\begin{align*}
	v_0 & = \sqrt{v_{0_x}^2 + v_{0_y}^2} \\
		& = \sqrt{(\SI{22}{\meter \per \second})^2 + (\SI{12.70}{\meter \per \second})^2} \\
	v_0 & = \SI{25.40}{\meter \per \second}
\end{align*}
\bc{v_{0_x} = \SI{22}{\meter \per \second}, v_{0_y} = \SI{12.70}{\meter \per \second}, v_0 = \SI{25.40}{\meter \per \second}}
Find $ y_0, y $
\begin{align*}
	v_y & = v_{0_y} + a_yt \\
	v_y & = (\SI{12.70}{\meter \per \second}) + (\SI{-10}{\meter \per \second \squared})(\SI{4.0}{\second}) \\
	v_y & = \SI{-27.3}{\meter \per \second}
\end{align*}
\begin{align*}
	y & = y_0 + v_yt - \frac{1}{2}a_yt^2 \\
	y & = (0) + (\SI{-27.3}{\meter \per \second})(\SI{4.0}{\second}) - \frac{1}{2}(\SI{-10}{\meter \per \second \squared})(\SI{4.0}{\second})^2 \\
	y & = \SI{-29.2}{\meter}
\end{align*}
\bc{y_0 = 0, y = \SI{-29.2}{\meter}}
Find time at maximum height
\begin{align*}
	v_y & = v_{0_y} + a_yt \\
	(0) & = \SI{12.70}{\meter \per \second} + (\SI{-10}{\meter \per \second \squared})t \\
	t & = \SI{1.27}{\second}
\end{align*}
\bc{t = \SI{1.27}{\second}}
Find maximum height
\begin{align*}
	y & = y_0 + v_{0_y}t + \frac{1}{2}a_yt^2 \\
	y & = (0) + (\SI{12.70}{\meter \per \second})(\SI{1.27}{\second}) + \frac{1}{2}(\SI{-10}{\meter \per \second \squared})(\SI{1.27}{\second})^2 \\
	y & = \SI{8.065}{\meter}
\end{align*}
\bc{y = \SI{8.065}{\meter}}

\end{document}
